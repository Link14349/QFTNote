\section{绪论}
这是我在上2025年秋季学期复旦大学物理系周洋老师开设的量子场论课程产生的课程笔记, 主要记录了这学期我学习量子场论期间学会的所有内容以及思考过程. 我补充了一些周老师没有提到或者没有讲清楚的内容, 并融合了历次作业中部分有启发性的题目, 从而形成了这一份笔记.
\subsection{为什么我们需要QFT?}
量子场论是一个框架,用于纳入无限数量的量子自由度,这些自由度在时空中排列,其相互作用受局域性、对称性、幺正性和因果性约束。它是描述自然规律的语言。\cite{qft:lecture1}
\begin{itemize}
    \item 局域性: 相互作用发生在时空的单个点上
    \item 对称性: 物理定律在特定变换下保持不变
    \item 幺正性: 量子事件所有可能结果的总概率为1
    \item 因果性: 原因先于结果,且信息传播速度不超过光速
\end{itemize}
\subsection{符号约定}
\begin{itemize}
    \item 采用自然Heaviside-Lorentz单位制: $\hbar = c = \mu_0 = \epsilon_0 = 1$
    \item 度规: $g_{\mu\nu} = \mathrm{diag}(1,-1,-1,-1)$
    \item 爱因斯坦求和约定: 重复指标隐含求和, 并且希腊字母如$\mu,\nu,\lambda\cdots$取0,1,2,3, 拉丁字母如$i,j,k\cdots$取1,2,3
    \item 在不影响歧义的情况下, 对于标量场$\phi$, 其偏导$\mu$分量记为$\phi_\mu=\partial_\mu\phi$
    \item 在不影响歧义的情况下, $p_\mu x^\mu$类的缩并可简记为$px$
    \item $\phi_\mathbf x\equiv\phi(\mathbf x)$
    \item $\phi_i\equiv\phi(x_i)$
    \item $\delta_{xy}\equiv\delta^4(x-y)$
    \item $\delta_{ij}\equiv\delta^4(x_i-x_j)$
    \item $D_{xy}\equiv D_F(x-y)$
    \item $D_{ij}\equiv D_F(x_i-x_j)$
    \item $\braket{\phi_1\phi_2\cdots\phi_n}\equiv\braket{\Omega|\mathcal T\phi_1\phi_2\cdots\phi_n|\Omega}$
\end{itemize}
\begin{equation}
    \Theta(x)=\begin{cases}
        1, & x>0\\
        0, & x<0
    \end{cases}=\lim_{\epsilon\to0^+}\int\frac{i\d\omega}{2\pi}\frac{\exp{-i\omega t}}{\omega+i\epsilon}\label{Theta-definition}
\end{equation}
单位元电荷
\begin{equation}
    e=-|e|
\end{equation}
精细结构常数
\begin{equation}
    \alpha=\frac{e^2}{4\pi}\approx \frac{1}{137}
\end{equation}
作用量
\begin{equation}
    S = \int \mathcal{L} \mathrm d^4x=\int (-m\mathrm d\tau-qA_\mu\mathrm dx^\mu)-\int \mathrm d^4x \frac14 F_{\mu\nu}F^{\mu\nu}
\end{equation}
其中,$F_{\mu\nu}$为电磁场张量
\begin{equation}
    F_{\mu\nu} = \mathrm dA_{\mu\nu}
\end{equation}
$\mathrm d$为外微分算符,$A_\mu$为电磁四势.\\
展开为分量形式有: 
\begin{subequations}
    \begin{align}
        F_{\mu\nu} &= \begin{bmatrix}
            0 & E_x & E_y & E_z \\
            -E_x & 0 & -B_z & B_y \\
            -E_y & B_z & 0 & -B_x \\
            -E_z & -B_y & B_x & 0
        \end{bmatrix}\\
        F^{\mu\nu} &= \begin{bmatrix}
            0 & -E_x & -E_y & -E_z \\
            E_x & 0 & -B_z & B_y \\
            E_y & B_z & 0 & -B_x \\
            E_z & -B_y & B_x & 0
        \end{bmatrix}
    \end{align}
\end{subequations}
以及有粒子动力学方程:
\begin{equation}
    m\frac{\mathrm d^2x^\mu}{\mathrm d\tau^2}=qF^\mu_{~~\nu}\frac{\mathrm dx^\nu}{\mathrm d\tau}
\end{equation}
还有Maxwell方程组的协变形式:
\begin{equation}
    \begin{cases}
        \partial_\mu F^{\mu\nu} = J^\nu\\
        \partial_{[\mu} F_{\nu\lambda]} = 0
    \end{cases}
\end{equation}
化为矢量方程组:
\begin{equation}
    \begin{cases}
        \nabla \cdot \mathbf{E} &= \rho \\
        \nabla \cdot \mathbf{B} &= 0 \\
        \nabla \times \mathbf{E} &= -\frac{\partial \mathbf{B}}{\partial t} \\
        \nabla \times \mathbf{B} &= \mathbf{J} + \frac{\partial \mathbf{E}}{\partial t}
    \end{cases}
\end{equation}
spin$1/2$ Pauli矩阵:
\begin{equation}
    \sigma^0=\begin{pmatrix}
        1 & 0 \\
        0 & 1
    \end{pmatrix}
\end{equation}
\begin{equation}
    \sigma^1=\begin{pmatrix}
        0 & 1 \\
        1 & 0
    \end{pmatrix}
\end{equation}
\begin{equation}
    \sigma^2=\begin{pmatrix}
        0 & -i \\
        i & 0
    \end{pmatrix}
\end{equation}
\begin{equation}
    \sigma^3=\begin{pmatrix}
        1 & 0 \\
        0 & -1
    \end{pmatrix}
\end{equation}

\begin{equation}
    [\frac{\sigma^i}2, \frac{\sigma^j}2]=i\epsilon_{ijk}\frac{\sigma^k}2
\end{equation}
\begin{equation}
    \{\frac{\sigma^i}2, \frac{\sigma^j}2\}=\frac12\delta_{ij}
\end{equation}
\begin{equation}
    \sigma^i\sigma^j=\delta_{ij}+i\epsilon_{ijk}\sigma^k
\end{equation}
spin$1$ Pauli矩阵
\begin{equation}
    S^1=\frac{1}{\sqrt{2}}\begin{pmatrix}
        0 & 1 & 0\\
        1 & 0 & 1\\
        0 & 1 & 0
    \end{pmatrix}
\end{equation}
\begin{equation}
    S^2=\frac{1}{\sqrt{2}}\begin{pmatrix}
        0 & -i & 0\\
        i & 0 & -i\\
        0 & i & 0
    \end{pmatrix}
\end{equation}
\begin{equation}
    S^3=\begin{pmatrix}
        1 & 0 & 0\\
        0 & 0 & 0\\
        0 & 0 & -1
    \end{pmatrix}
\end{equation}
\begin{equation}
    [S^i, S^j]=i\epsilon_{ijk}S^k
\end{equation}
Lorentz群变换(见\ref{Lorentz}节)
\begin{equation}
    \Lambda=\exp{-\frac i2\omega_{\mu\nu}M^{\mu\nu}}
\end{equation}
其中
\begin{equation}
    \delta\omega_{\mu\nu}=\begin{bmatrix}
        0 & v^1 & v^2 & v^3 \\
        -v^1 & 0 & -\theta^3 & \theta^2 \\
        -v^2 & \theta^3 & 0 & -\theta^1\\
        -v^3 & -\theta^2 & \theta^1 & 0
    \end{bmatrix}
\end{equation}
Lorentz群生成元
\begin{align}
    & J^i=\frac12\epsilon_{ijk}M^{jk}\\
    & K^i=M^{i0}
\end{align}
\begin{equation}
    M^{\mu\nu}=\begin{bmatrix}
        0 & -K^1 & -K^2 & -K^3\\
        K^1 & 0 & J^3 & -J^2\\
        K^2 & -J^3 & 0 & J^1\\
        K^3 & J^2 & -J^1 & 0
    \end{bmatrix}
\end{equation}


\newpage
\section{泛函基础简述}
\subsection{泛函}
\begin{definition}[泛函]
    泛函$f: \mathscr{F} \to \mathbb{R}$是定义在某个函数空间上的映射, 全体泛函组成的集合记为$\mathcal{F}$
\end{definition}
\begin{example}
    设$\mathcal{F}$是定义在区间$[a,b]$上的所有实值连续函数的空间,则
    \begin{equation}
        f[\phi] = \int_a^b \phi(x) \, \mathrm{d}x
    \end{equation}
    是$\mathcal{F}$上的一个泛函
\end{example}
\begin{example}\label{functional-delta}
    \begin{equation}
        \delta_x[\phi]=\phi(x)
    \end{equation}
\end{example}

\begin{definition}[$\delta$函数]
    $\delta(x)$是满足
    \begin{equation}
        \int f(x) \delta(x) \mathrm dx = f(0)
    \end{equation}
    的函数
\end{definition}
\kaishu 讨论: 数学人经常argue的一点是$\delta$函数并不是一个良定义的函数, 实际上如果要严格定义函数, 应当是将其定义为例\ref{functional-delta}中的那个泛函. \songti
\begin{theorem}
    \begin{equation}
        \int \mathrm dx \mathrm e^{ikx}=2\pi \delta(k)\label{matheq1}
    \end{equation}
\end{theorem}
\begin{theorem}
    \begin{equation}
        \int_a^b f(x) \delta(g(x)) \mathrm dx = \sum_{i\in \{i~|~g(x_i)=0, x_i\in[a, b]\}}\frac{f(x_i)}{|g'(x_i)|}= \sum_{i\in \{i~|~g(x_i)=0\}}\frac{f(x_i)}{|g'(x_i)|}\Theta(b-x)\Theta(x-a)
    \end{equation}
\end{theorem}

\subsection{变分}
\begin{definition}[变分]
    设泛函$f: \mathcal{F}\times\mathbb{R} \to \mathcal{F}$\\
    其变分
    \begin{equation}
        \frac{\delta f}{\delta \phi(x_0)} = \lim_{\epsilon \to 0} \frac{f[\phi(x) + \epsilon \delta(x-x_0)] - f[\phi]}{\epsilon}
    \end{equation}
\end{definition}
\begin{example}
    设泛函
    \begin{equation}
        f_x[\phi] = \phi(x)
    \end{equation}
    则
    \begin{equation}
        \frac{\delta f_x}{\delta \phi(y)} = \delta(x-y)
    \end{equation}
\end{example}
\begin{example}
    设泛函
    \begin{equation}
        f[\phi] = g(\phi), \text{其中} g: \mathbb{R} \to \mathbb{R}
    \end{equation}
    则
    \begin{equation}
        \frac{\delta f}{\delta \phi(x_0)} = \frac{\partial g}{\partial \phi} \delta(x-x_0)
    \end{equation}
\end{example}
\begin{example}
    设泛函
    \begin{equation}
        f[\phi] = \int g(\phi) \mathrm dx, \text{其中} g: \mathbb{R} \to \mathbb{R}
    \end{equation}
    则
    \begin{equation}
        \frac{\delta f}{\delta \phi(x_0)} = \left.\frac{\partial g}{\partial \phi}\right|_{\phi(x_0)}
    \end{equation}
\end{example}
\begin{example}
    设泛函
    \begin{equation}
        f[\phi] = \int g(\phi, \nabla\phi) \mathrm d^3x, \text{其中} g: \mathbb{R} \times \mathbb{R}^3 \to \mathbb{R}
    \end{equation}
    则
    \begin{equation}
        \frac{\delta f}{\delta \phi(x_0)} = \left.\left(\frac{\partial g}{\partial \phi} - \partial_i\frac{\partial g}{\partial \phi_i}\right)\right|_{\nabla\phi(x_0)}
    \end{equation}
\end{example}
参考: \cite{functionals}

% \subsection{群\&群表示论}

\newpage

\section{量子力学基础}
\subsection{Hilbert空间}
量子力学研究生活在Hilbert空间\cite{LiangQMMath}的矢量, 所谓Hilbert空间是完备的内积空间.
\begin{definition}[内积空间]
    复矢量空间$V$称为内积空间, 若存在内积$(\cdot, \cdot): V\times V\to \mathbb{C}$, $\forall f, g, h\in V, c\in\mathbb C$:
    \begin{enumerate}
        \item $(f, g+h)=(f, g)+(f, h)$
        \item $(f, cg)=c(f, g)$
        \item $(f, g)=(g, f)^*$
        \item $(f, f)\ge0$, 且$(f, f)=0$当且仅当$f=0$
    \end{enumerate}
\end{definition}
根据定义我们有推论
\begin{equation}
    (cf, g)=c^*(f, g)
\end{equation}

我们可以利用内积定义任意两元素的距离
\begin{definition}[元素距离]
    \begin{equation}
        d(f, g)\equiv\sqrt{(f-g, f-g)}
    \end{equation}
\end{definition}

然后可以定义矢量序列$\{f_n\}$的极限
\begin{definition}[矢量序列极限]
    若$\exists f\in V, \forall \epsilon>0, \exists N\in\mathbb{N}, \forall n>N, d(f_n, f)<\epsilon$, 则称$\{f_n\}$收敛, 并且有
    \begin{equation}
        \lim_{n\to\infty}f_n=f
    \end{equation}
\end{definition}

\begin{definition}[柯西序列]
    类似高等数学中我们学习的柯西数列, 若对于序列$\{f_n\}$, $\forall \epsilon>0, \exists N\in\mathbb N, \forall n, m\ge N, d(f_m, f_n)<\epsilon$
\end{definition}
在一般的内积空间中, 收敛的序列一定是柯西序列, 但是柯西序列不一定是收敛的: 其极限不一定在矢量空间内. 因此我们需要扩大矢量空间包含的范围, 使其完备
\begin{definition}[完备性]
    若内积空间$V$中任意柯西序列都收敛, 则$V$完备
\end{definition}

而我们研究的Hilbert空间就是完备的.
\begin{definition}[Hilbert空间]
    Hilbert空间是完备的内积空间, 记为$\mathscr H$
\end{definition}

和我们在线性代数中学习的一样, 任意矢量空间都可以定义其对偶空间.
\begin{definition}[对偶空间]
    内积空间$V$的对偶空间
    \begin{equation}
        V^*\equiv\mathcal L(V, \mathbb C)
    \end{equation}
\end{definition}

利用内积我们可以自然诱导出一个从Hilbert空间到对偶空间的反线性同构映射
\begin{definition}[$\mathscr H\to\mathscr H^*$的反线性同构映射]
    \begin{equation}
        \eta: \mathscr H\to\mathscr H^*, \eta_h\equiv(h, \cdot)
    \end{equation}
\end{definition}
在这里我们之所以要求Hilbert空间而不是内积空间就是要求$\mathscr H$与$\mathscr H^*$是一样大的: $\eta$是反线性同构而不是同态, 若内积空间不完备, 则$\mathscr H^*$比$\mathscr H$大.

\subsection{线性算符}
\begin{definition}[线性算符]\label{linear-operator}
    $A: \mathscr H\to\mathscr H$是线性算符若
    \begin{equation}
        A(c_1f_1+c_2f_2)=c_1A(f_1)+c_2A(f_2)\label{linear-operator-eq1}
    \end{equation}
\end{definition}
显然可以定义线性算符的加法与数乘, 从而构成一个矢量空间.

并且我们还可以定义线性算符间的乘法
\begin{definition}[线性算符的乘法]
    对于线性算符$A, B$其乘法
    \begin{equation}
        (AB)(f)\equiv A(B(f))
    \end{equation}
\end{definition}

和对偶空间结合我们可以定义对偶线性算符
\begin{definition}[对偶线性算符]
    线性算符$A$的对偶线性算符$A^*: \mathscr H^* \to \mathscr H^*$定义为
    \begin{equation}
        A^*(\nu)(f)\equiv\nu(Af)
    \end{equation}
\end{definition}
\begin{theorem}
    \begin{equation}
        (A+B)^*=A^*+B^*
    \end{equation}
\end{theorem}
\begin{proof}
    \begin{equation}
        (A+B)^*(\nu)(f)=\nu(A(f)+\nu(B(f)))=\nu(Af)+\nu(Bf)=A^*(\nu)(f)+B^*(\nu)(f)
    \end{equation}
\end{proof}

然后我们可以定义$A$的伴随算符$A^\dagger: \mathscr H\to\mathscr H$
\begin{definition}[]
    \begin{equation}
        A^\dagger: \mathscr H\to\mathscr H, \eta^{-1}\circ A^*\circ\eta
    \end{equation}
\end{definition}

\begin{theorem}[伴随算符的性质]\label{linear-adjoint}
    伴随算符满足
    \begin{equation}
        (f, Ag)=(A^\dagger f, g)\label{linear-adjoint-eq1}
    \end{equation}
    \begin{equation}
        (A^\dagger)^\dagger=A
    \end{equation}
    \begin{equation}
        (cA)^\dagger=c^*A^\dagger
    \end{equation}
    \begin{equation}
        (A+B)^\dagger=A^\dagger+B^\dagger
    \end{equation}
    \begin{equation}
        (AB)^\dagger=B^\dagger A^\dagger\label{linear-adjoint-eq2}
    \end{equation}
\end{theorem}
\begin{proof}
    \begin{equation}
        (f, Ag)=\eta(f)(Ag)=(A^*\circ\eta(f))(g)=\eta[(\eta^{-1}A^*\circ\eta)(f)](g)=\eta[(A^\dagger)(f)](g)=(A^\dagger f, g)
    \end{equation}
    \begin{equation}
        (f, (A^\dagger)^\dagger g)=((A^\dagger)^\dagger g, f)^*=(g, A^\dagger f)^*=(A^\dagger f, g)=(f, Ag)
    \end{equation}

    剩下的几个证明是显然的.
\end{proof}

\subsection{Dirac记号}
\begin{definition}[左右矢]
    我们记$f\in\mathscr H$为右矢, ket
    \begin{equation}
        \ket f
    \end{equation}
    我们记$\eta_f\in\mathscr H^*$为左矢, bra
    \begin{equation}
        \bra{\eta_f}
    \end{equation}

    因为$\eta$是由内积确定的自然映射, 所以是不会引起歧义的, 所以一般我们把$\bra{\eta_f}$直接记为$\bra f$
\end{definition}

\begin{definition}[Dirac记号中的算符]
    \begin{equation}
        A\ket\psi\equiv\ket{A\psi}
    \end{equation}
\end{definition}

如何理解$\bra\psi A$呢? 我们将其定义为
\begin{definition}
    \begin{equation}
        \bra\psi A\equiv (A^*\bra\psi)
    \end{equation}
\end{definition}

于是我们可以看到, 诸如
\begin{equation}
    \braket{\phi|A|\psi}
\end{equation}
的式子是有两种理解方式的, 并且这两个方式结果是自洽的.
\begin{equation}
    \braket{\phi|A|\psi}=\bra\phi\left(A\ket\psi\right)
\end{equation}
\begin{equation}
    \braket{\phi|A|\psi}=\left(\bra\phi A\right)\ket\psi=(A*\bra\phi)\ket\psi=\bra\phi\left(A\ket\psi\right)
\end{equation}

定理\ref{linear-adjoint}的式子\ref{linear-adjoint-eq1}可以重新写为
\begin{equation}
    \braket{\phi|A\psi}=\braket{A^\dagger\phi|\psi}
\end{equation}

\begin{definition}[算符的逆]
    若$\forall\ket\psi$, 有且仅有一$\ket\phi\in\mathscr H, A\ket\phi=\ket\psi$, 则称算符$A$可逆, 并且记
    \begin{equation}
        \phi=A^{-1}\ket\psi
    \end{equation}
\end{definition}
不难证明$A^{-1}$是线性的.

\begin{definition}[幺正算符]
    若算符$U$满足
    \begin{equation}
        UU^\dagger=U^\dagger U=1
    \end{equation}
    则称$U$是幺正算符.
\end{definition}

\subsection{反线性算符}
一般在量子力学中我们考虑的都是线性算符, 但是同样存在反线性算符, 比如时间反演算符$T$. 所谓反线性就是指将定义\ref{linear-operator}中的式\eqref{linear-operator-eq1}替换为反线性性:
\begin{definition}[反线性算符]\label{antilinear-operator}
    $S: \mathscr H\to\mathscr H$是反线性算符若
    \begin{equation}
        S(c_1f_1+c_2f_2)=c_1^*S(f_1)+c_2^*S(f_2)\label{antilinear-operator-eq1}
    \end{equation}
\end{definition}
同样可以定义反线性算符之间的加法、数乘, 构成一个矢量空间.

我们还可以定义反线性算符间的乘法或者与线性算符的乘法:
\begin{definition}[(反)线性算符的乘法]
    对于(反)线性算符$A, B$($A$, $B$的线性性或者反线性性是任意的)其乘法
    \begin{equation}
        (AB)(f)\equiv A(B(f))
    \end{equation}
\end{definition}

而对于复合算符的线性性与反线性有如下定理
\begin{theorem}[复合算符的(反)线性性]\label{anti-linear-multiplication}
    若$A, B$都是反线性, 则$AB$线性; 若$A, B$中有一个反线性一个线性, 则$AB$反线性; 若$A, B$都是线性则$AB$线性.
\end{theorem}
\begin{proof}
    对于$A, B$都是反线性, $\forall c\in\mathbb C$
    \begin{equation}
        ABc=Ac^*B=cAB
    \end{equation}

    对于$A$线性, $B$反线性, $\forall c\in\mathbb C$
    \begin{equation}
        ABc=Ac^*B=c^*AB
    \end{equation}

    对于$AB$线性$\forall c\in\mathbb C$
    \begin{equation}
        ABc=AcB=cAB
    \end{equation}
\end{proof}

但是我们不能定义线性算符与反线性算符间的乘法, 因为若我们这么定义线性算符$A$与反线性算符$S$的乘法
\begin{equation}
    (A+S)f\equiv A(f)+S(f)
\end{equation}
那么考虑其作用到$cf$上
\begin{equation}
    (A+S)(cf)=A(cf)+S(cf)=cA(f)+c^*S(f)
\end{equation}
我们发现这个新的映射既不是线性的, 也不是反线性的: 它根本不满足任何线性性! 所以这个定义是不良的.

反线性算符相比于线性算符有很多不一样的性质, 我们需要小心对待. 比如与线性算符的对偶算符不同, 我们需要定义其对偶算符$S^*$为
\begin{definition}[对偶线性算符]
    反线性算符$S$的对偶反线性算符$S^*: \mathscr H^* \to \mathscr H^*$定义为
    \begin{equation}
        S^*(\nu)(f)\equiv\nu(Sf)^{\textcolor{red}{\mathbf*}}
    \end{equation}
    
    读者不难验证$S^*$也是一个反线性算符.
\end{definition}
如何理解这个定义? 为什么在这里我们需要引入一次共轭? 这是为了让$S^*$作用到$\nu$上后的结果$S^*\nu$仍然是一个线性映射而不是反线性映射: 对偶矢量的定义就是线性映射, 而不是反线性映射. 即
\begin{equation}
    S^*(\nu)(cf)=\nu(Scf)^*=(\nu(c^*Sf))^*=(c^*\nu(Sf))^*=c(\nu(Sf))^*=cS^*(\nu)(f)
\end{equation}

然后我们可以定义反线性算符的伴随$S^\dagger$
\begin{definition}[伴随算符]
    反线性算符$S^\dagger: \mathscr H\to\mathscr H, \eta^{-1}\circ S^*\circ\eta$
\end{definition}
可以看到这个定义是完全一样的. 但是其带来的伴随算符的性质是不同的:
\begin{theorem}[反线性算符的伴随算符的性质]\label{antilinear-adjoint}
    反线性算符的伴随算符同样是反线性算符, 并且满足
    \begin{equation}
        (f, Sg)=(g, S^\dagger f)\label{antilinear-adjoint-eq1}
    \end{equation}
    \begin{equation}
        (S^\dagger)^\dagger=S
    \end{equation}
    \begin{equation}
        (cS)^\dagger=cS^\dagger
    \end{equation}
\end{theorem}
\begin{proof}
    首先证明反线性算符的伴随算符同样是反线性算符:
    \begin{equation}
        S^\dagger(cf)=\eta^{-1}\circ S^*\circ\eta(cf)=\eta^{-1}[S^*(c\eta(f))]=\eta^{-1}[c^*S^*(\eta(f))]=c^*\eta^{-1}[S^*(\eta(f))]=c^*S^\dagger(f)
    \end{equation}

    然后直接计算可以证明后面三式
    \begin{align}
        (f, Sg)&=\eta_f(Sg)=\left[(S*\eta(f))(g)\right]^*=\left\{\eta\left[\eta^{-1}\circ S^*\circ\eta(f)\right](g)\right\}^*\\&
        =(S^\dagger f, g)^*=(g, S^\dagger f)
    \end{align}
    \begin{align}
        (f, (S^\dagger)^\dagger g)=(g, S^\dagger f)=(f, S^\dagger g)
    \end{align}
    \begin{align}
        (g, (cS)^\dagger f)=(f, cSg)=c(f, Sg)=c(g, S^\dagger f)=(g, cS^\dagger f)
    \end{align}
\end{proof}

对于$AB$这样的复合算符, 是否有与线性算符一样的取$\dagger$方式? 下面的定理指出与式\eqref{linear-adjoint-eq2}对反线性算符间或者线性算符与反线性算符的乘积依然是成立的.
\begin{theorem}[反线性算符的复合]
    对于线性算符或者反线性算符$A, B$, 其乘法$AB$的伴随
    \begin{equation}
        (AB)^\dagger=B^\dagger A^\dagger
    \end{equation}
\end{theorem}
\begin{proof}
    若$A, B$都是反线性的, 根据定理\ref{anti-linear-multiplication}, 那么$AB$就是线性的, 从而$(AB)^\dagger$也是线性的, 则
    \begin{equation}
        ((AB)^\dagger f, g)=(f, ABg)=(Bg, A^\dagger f)=(B^\dagger A^\dagger f, g)
    \end{equation}
    
    若$A, B$中$A$是线性的, $B$是反线性的, 那么$AB, (AB)^\dagger$也是反线性, 于是
    \begin{equation}
        ((AB)^\dagger f, g)=(ABg, f)=(Bg, A^\dagger f)=(B^\dagger A^\dagger f, g)
    \end{equation}
\end{proof}

然后我们同样可以定义幺正的反线性算符
\begin{definition}[反线性幺正算符]
    若反线性算符$U$满足
    \begin{equation}
        UU^\dagger=U^\dagger U=1
    \end{equation}
    则称$U$是幺正算符.
\end{definition}


\newpage
\section{狭义相对论简述}
\subsection{狭义相对论动力学技巧}
一些常用技巧\cite{griffthsPPSRTrick}:
\begin{enumerate}
    \item $\vec v=\frac{\vec p}{E}$
    \item 使用四矢量以及不变量点积
    \item 使用质心系简化计算
\end{enumerate}
\begin{definition}[Mandelstam变量]
    对于以$m_1, p_1$, $m_2, p_2$的粒子入射, 以$m_3, p_3$ $m_4, p_4$的粒子出射的动力学系统
    \begin{equation}
        s=(p_1+p_2)^2=(p_3+p_4)^2, \quad t=(p_1-p_3)^2=(p_2-p_4)^2, \quad u=(p_1-p_4)^2=(p_2-p_3)^2
    \end{equation}
\end{definition}
\begin{theorem}[s+t+u守恒]
    \begin{equation}
        s+t+u=m_1^2+m_2^2+m_3^2+m_4^2
    \end{equation}
\end{theorem}
\begin{proof}
    我们取质心系, 可以写出$p_1, p_2, p_3, p_4$
    \begin{align}
        p_1=(E_1, \vec p), \quad p_2=(E_2, -\vec p)\\
        p_3=(E_3, \vec k), \quad p_4=(E_4, -\vec k)
    \end{align}

    于是
    \begin{equation}
        s=(E_1+E_2)^2, \quad t=(E_3-E_1)^2-(\vec k-\vec p)^2, \quad u=(E_4-E_1)^2-(\vec k+\vec p)^2
    \end{equation}

    计算可得:
    \begin{align}
        s+t+u&=E_1^2+E_2^2+E_3^2+E_4^2-2\vec p^2-2\vec k^2+2E_1^2+2E_1E_2-2E_1E_3-2E_1E_4\\
        &=m_1^2+m_2^2+m_3^2+m_4^2+2E_1(E_1+E_2-E_3-E_4)\\
        &=m_1^2+m_2^2+m_3^2+m_4^2
    \end{align}
\end{proof}

\begin{theorem}[质心系中$m_1$的能量]
    \begin{equation}
        E^{\mathbf{CM}}_1=\frac{s+m_1^2-m_2^2}{2\sqrt s}
    \end{equation}
\end{theorem}
\begin{theorem}[实验室系($m_2$静止)中$m_1$的能量]
    \begin{equation}
        E^{\mathbf{lab}}_1=\frac{s-m_1^2-m_2^2}{2m_2}
    \end{equation}
\end{theorem}
\begin{theorem}[质心系总能量]
    \begin{equation}
        E^{\mathbf{CM}}_{\mathbf{TOT}}=\sqrt s
    \end{equation}
\end{theorem}
