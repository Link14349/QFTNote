\section{经典场论}
\subsection{拉氏量、作用量与Euler-Lagrange方程}
\begin{definition}[拉氏量\&拉氏量密度]
    拉氏量
    \begin{equation}
        L(t)=\int \mathrm d^3x \, \mathcal{L}(\phi, \partial_\mu \phi)
    \end{equation}
    其中$\mathcal L$即拉氏量密度
\end{definition}
\begin{definition}[作用量]
    作用量$S$是拉氏量密度在时空上的积分
    \begin{equation}
        S = \int L \, \mathrm{d}t=\int\mathcal L\,\mathrm d^4x
    \end{equation}
\end{definition}
\theorem[标量的Euler-Lagrange方程]
\begin{equation}
    \frac{\partial\mathcal L}{\partial\phi}-\partial_\mu\frac{\partial\mathcal L}{\partial\phi_\mu}=0
\end{equation}
\begin{proof}
    设$\phi\to\phi+\delta\phi$,则
    \begin{equation}
        \delta S = \int \mathrm d^4x \left(\frac{\partial\mathcal L}{\partial\phi}\delta\phi + \frac{\partial\mathcal L}{\partial\phi_\mu}\delta\phi_\mu\right)
    \end{equation}
    对第二项分部积分,忽略边界项,有
    \begin{equation}
        \delta S = \int \mathrm d^4x \left(\frac{\partial\mathcal L}{\partial\phi} - \partial_\mu\frac{\partial\mathcal L}{\partial\phi_\mu}\right)\delta\phi
    \end{equation}
    由$\delta S=0$可得Euler-Lagrange方程
\end{proof}
\begin{example}
    自由粒子的拉氏量密度
    \begin{equation}
        \mathcal L = \frac12\partial_\mu\phi\partial^\mu\phi - \frac12 m^2\phi^2
    \end{equation}
    代入Euler-Lagrange方程, 有
    \begin{equation}
        (\partial_\mu\partial^\mu + m^2)\phi = 0
    \end{equation}
    即Klein-Gordon方程
\end{example}
\begin{theorem}[矢量场的Euler-Lagrange方程]
    设矢量场为$A_\mu$, 
    定义$F_{\mu\nu}=\mathrm dA_{\mu\nu}$, $\Pi^{\mu\nu}=\frac{\partial\mathcal L}{\partial F_{\mu\nu}}, B^\nu=\frac{\partial \mathcal L}{\partial A_\nu}$, 那么
    \begin{equation}
        B^\mu-2\partial_\nu\Pi^{[\nu\mu]}=0
    \end{equation}
\end{theorem}
\begin{proof}
    \begin{equation}
        \delta\mathcal L=\Pi^{\mu\nu}\delta \mathrm dA_{\mu\nu}+\frac{\partial\mathcal L}{\partial A_\mu}\delta A_\mu
    \end{equation}
    交换$\delta$与外微分算子$\mathrm d$, 并利用乘法法则有:
    \begin{subequations}
        \begin{align}
            \delta\mathcal L&=2\partial_{[\mu}(\Pi^{\mu\nu}\delta A_{\nu]})-2(\partial_{[\mu}\Pi^{\mu\nu})\delta A_\nu+\frac{\partial\mathcal L}{\partial A_\mu}\delta A_\mu\\
            &=2\partial_{[\mu}(\Pi^{\mu\nu}\delta A_{\nu]})-2(\partial_\nu\Pi^{[\nu\mu]})\delta A_\mu+\frac{\partial\mathcal L}{\partial A_\mu}\delta A_\mu
        \end{align}
    \end{subequations}
    忽略边界项, 由$\delta S=0$可得Euler-Lagrange方程.
\end{proof}
\begin{example}
    电磁场的拉氏量密度
    \begin{equation}
        \mathcal L = -\frac14 F_{\mu\nu}F^{\mu\nu}-J^\mu A_\mu
    \end{equation}
    代入Euler-Lagrange方程, 有
    \begin{equation}
        \partial_\mu F^{\mu\nu} = J^\nu
    \end{equation}
    即Maxwell方程.
\end{example}
\subsection{对称性与守恒量}
\songti 在这里我们考虑连续对称性.
\begin{definition}[连续对称性]
    连续对称性是指在某个参数$\alpha$下,场的变换
    \begin{equation}
        \phi(x) \to \phi'(x) = \phi(x) + \alpha \Delta\phi(x)
    \end{equation}
    其中, $\Delta$为某一算符.\\
    若系统具有该变换的对称性, 则运动方程应该不变.\\
    即:
    \begin{equation}
        \mathcal L \to \mathcal L+\alpha\partial_{\mu} J^\mu
    \end{equation}
\end{definition}
\theorem[Noether定理]
对于存在某对称性的系统, 存在守恒流
\begin{equation}
    j^\mu=\Pi^\mu\Delta\phi - J^\mu
\end{equation}
\begin{proof}
    \begin{subequations}
        \begin{align}
            \delta\mathcal L 
            &= \alpha\frac{\partial\mathcal L}{\partial\phi}\Delta\phi + \alpha\frac{\partial\mathcal L}{\partial\phi_\mu}\partial_\mu\Delta\phi\\
            &= \alpha\left(\frac{\partial\mathcal L}{\partial\phi} - \alpha\partial_\mu\frac{\partial\mathcal L}{\partial\phi_\mu}\right)\Delta\phi + \partial_\mu\left(\frac{\partial\mathcal L}{\partial\phi_\mu}\Delta\phi\right)\\
            &= \alpha\partial_\mu\left(\Pi^\mu\Delta\phi\right)
        \end{align}
    \end{subequations}
    又因为系统存在对称性:
    \begin{equation}
        \delta\mathcal L = \alpha\partial_\mu J^\mu
    \end{equation}
    因此:
    \begin{equation}
        \partial_\mu(\Pi^\mu\Delta\phi-J^\mu)=\partial_\mu j^\mu = 0
    \end{equation}
\end{proof}
\begin{example}[复标量场的$U(1)$对称性]\label{csU1}
    考虑复标量场
    \begin{equation}
        \mathcal L = \partial^\mu\psi^*\partial_\mu\psi - m^2\psi^*\psi
    \end{equation}
    其对称性为
    \begin{equation}
        \psi \to \psi' = e^{i\alpha}\psi, \quad \psi^* \to \psi'^* = e^{-i\alpha}\psi^*
    \end{equation}
    不难发现, $\mathcal L$与$\psi$的相位无关,因此:
    \begin{equation}
        \mathcal L \to \mathcal L + 0\Rightarrow J^\mu = 0
    \end{equation}
    于是, 由Noether定理可得守恒流
    \begin{equation}
        j^\mu = i(\psi\partial^\mu\psi^* - \psi^*\partial^\mu\psi)
    \end{equation}
    我们可以代入验证其守恒:
    \begin{equation}
        \partial_\mu j^\mu = i(\psi\partial^2\psi^*-\psi^*\partial^2\psi)
    \end{equation}
    代入运动方程$(\partial^2+m^2)\psi=0$
    \begin{equation}
        \partial_\mu j^\mu = i(\psi(-m^2\psi^*) - \psi^*(-m^2\psi)) = 0
    \end{equation}
\end{example}
\begin{example}[标量场的时空平移对称性]\label{ex:ct_scalar_translation}
    时空中有Killing矢量场$\xi^\mu$, 其满足Killing方程
    \begin{equation}
        \nabla_{(\mu}\xi_{\nu)} = 0
    \end{equation}
    缩并有:
    \begin{equation}
        \nabla_\mu\xi^\mu = 0
    \end{equation}
    在这里, $\Delta$算符即Lie导数$\mathscr L_\xi$.\\
    对$\mathcal L$沿着$-\xi^\mu$的方向进行平移, 有
    \begin{equation}
        \mathcal L\to \mathcal L + \alpha \mathscr L_\xi\mathcal L=\mathcal L + \alpha \xi^\mu\nabla_\mu\mathcal L=\mathcal L + \alpha\nabla_\mu(\xi^\mu\mathcal L)
    \end{equation}
    因此, $J^\mu = \xi^\mu\mathcal L$. 由Noether定理可得守恒流
    \begin{equation}
    \begin{split}
        j^\mu &= \Pi^\mu\Delta\phi - J^\mu = \Pi^\mu\xi^\nu\nabla_\nu\phi-\xi^\mu\mathcal L\\
        &= \xi^\nu(\nabla^\mu\phi\nabla_\nu\phi-\mathcal L\delta^\mu_{~~\nu})
    \end{split}
    \end{equation}
    于是有能动张量
    \begin{equation}
        T_{\mu\nu}=\nabla_\mu\phi\nabla_\nu\phi - \mathcal L g_{\mu\nu}
    \end{equation}
    其满足:
    \begin{equation}
        \nabla_\mu T^{\mu\nu} = 0
    \end{equation}
    由此可见, 能动量守恒完全是时空平移不变性的体现.
\end{example}
\begin{example}[矢量场的时空平移对称性]
    类似\ref{ex:ct_scalar_translation}, 时空中有Killing矢量场$\xi^\mu$.
    我们首先设两个辅助场:
    \begin{align}
        &\Pi^{\mu\nu}=2\pa{\mathcal L}{F_{\mu\nu}}\\
        &B^\mu=\pa{\mathcal L}{A_\mu}
    \end{align}
    沿着$-\xi^\mu$变换, 同样有:
    \begin{equation}
        \mathcal L\to \mathcal L + \alpha \xi^\mu\nabla_\mu\mathcal L=\mathcal L + \alpha\nabla_\mu(\xi^\mu\mathcal L)
    \end{equation}
    \begin{equation}
        A_\mu \to A_\mu + \alpha\mathscr{L}_\xi A_\mu
    \end{equation}
    \begin{equation}
        F_{\mu\nu} \to F_{\mu\nu} + \alpha\mathscr{L}_\xi F_{\mu\nu}
    \end{equation}
    需要注意的是, $\mathscr L_\xi$与外微分算子$\mathrm d$不对易, 因此
    \begin{equation}
        \mathscr L_\xi F_{\mu\nu} = \mathscr L_\xi \mathrm d A_{\mu\nu} \neq \mathrm d(\mathscr L_\xi A)_{\mu\nu}
    \end{equation}
    而
    \begin{equation}
        \begin{split}
            \mathscr L_\xi F_{\mu\nu} &= \xi^\lambda\nabla_\lambda F_{\mu\nu} + F_{\lambda\nu}\nabla_\mu\xi^\lambda + F_{\mu\lambda}\nabla_\nu\xi^\lambda\\
            &= \xi^\lambda\nabla_\lambda F_{\mu\nu}+\nabla_{\mu}(F_{\lambda\nu\xi^\lambda})-\xi^\lambda\nabla_\mu F_{\lambda\nu}\\
            &\quad+\nabla_\nu(F_{\mu\lambda}\xi^\lambda)-\xi^\lambda\nabla_\nu F_{\mu\lambda}\\
            &= \xi^\lambda\nabla_\lambda F_{\mu\nu}+\xi^\lambda\nabla_\mu F_{\nu\lambda}+\xi^\lambda\nabla_\nu F_{\lambda\mu}+\nabla_{\mu}(F_{\lambda\nu}\xi^\lambda)+\nabla_\nu(F_{\mu\lambda}\xi^\lambda)
        \end{split}
    \end{equation}
    由外微分算子性质有$\nabla_{[\mu}F_{\nu\lambda]}=0$, 可知前三项为零.\\
    因此
    \begin{equation}
        \mathscr L_\xi F_{\mu\nu}=\nabla_{\mu}(F_{\lambda\nu}\xi^\lambda)+\nabla_\nu(F_{\mu\lambda}\xi^\lambda)
    \end{equation}
    于是
    \begin{equation}
        \begin{split}
            \mathscr L_\xi\mathcal L &= \nabla_\mu(2\Pi^{\mu\nu F_{\lambda\nu}\xi^\lambda})-F_{\mu\nu}\xi^\lambda B^\nu+B^\nu(\xi^\lambda\nabla_\lambda A_\nu+A_\lambda\nabla_\nu\xi^\lambda)\\
            &= \nabla_\mu(2\Pi^{\mu\nu F_{\lambda\nu}\xi^\lambda})-F_{\mu\nu}\xi^\lambda B^\nu+B^\nu\xi^\lambda\nabla_\lambda A_\nu-B^\nu\xi^\lambda\nabla_\nu A_\lambda\\
            &\quad+B^\nu\nabla_\nu(A_\lambda\xi^\lambda)
        \end{split}
    \end{equation}
    这里我们取规范
    \begin{equation}
        \nabla_\nu B^\nu=0
    \end{equation}
    故:
    \begin{equation}
        \mathscr L_\xi\mathcal L=\nabla_\mu(2\Pi^{\mu\nu}F_{\lambda\nu}\xi^\lambda+B^\mu A_\nu\xi^\nu)
    \end{equation}
    类似地, 可以得到能动张量:
    \begin{equation}
        T_{\mu\nu} = -2\Pi_{\mu}^{~~\lambda}F_{\lambda\nu}+B_\mu A_\nu-g_{\mu\nu}\mathcal L
    \end{equation}
    参考:本人25年首考后写的笔记\cite{LinkZhihu}(很惭愧, 现在尝试重新推导的时候卡了好久...无奈看了当时的笔记才推出来, 真是奇怪, 明明当时也是我自己推出来的, 怎么现在一点都推不出来了呢= =).
\end{example}
\subsection{哈密顿量}
\begin{definition}[共轭动量]
    共轭动量
    \begin{equation}
        \Pi = \frac{\partial\mathcal L}{\partial\dot\phi}
    \end{equation}
\end{definition}
做Legendre变换, 有Hamiltonian:
\begin{equation}
    H(\phi, \Pi, \nabla\phi) = \int \mathrm d^3x \, \Pi\dot\phi-L=\int \mathrm d^3x \, (\Pi\dot\phi-\mathcal L)
\end{equation}
定义Hamiltonian密度:
\begin{equation}
    \mathcal H(\phi, \Pi, \nabla\phi) = \Pi\dot\phi - \mathcal L
\end{equation}
定义Poisson括号$\{, \}: \mathcal F\times\mathcal F\to\mathbb{R}$:
\begin{equation}
    \{F, G\}=\displaystyle\int \mathrm d^3x \left(\frac{\delta F}{\delta\phi(x)}\frac{\delta G}{\delta\Pi(y)}-\frac{\delta G}{\delta\phi(y)}\frac{\delta F}{\delta\Pi(x)}\right)
\end{equation}
\begin{example}
    \begin{equation}
        \{\phi(\mathbf x), \Pi(\mathbf y)\} = \delta^3(\mathbf x - \mathbf y)
    \end{equation}
\end{example}
\begin{proof}
    设$F=\phi(\mathbf x), G=\Pi(\mathbf y)$, 则
    \begin{equation}
        \frac{\delta F}{\delta\phi(x)} = \delta^3(\mathbf x - \mathbf x'), \quad \frac{\delta F}{\delta\Pi(x)} = 0
    \end{equation}
    \begin{equation}
        \frac{\delta G}{\delta\phi(x)} = 0, \quad \frac{\delta G}{\delta\Pi(x)} = \delta^3(\mathbf y - \mathbf x')
    \end{equation}
    代入Poisson括号定义, 有
    \begin{equation}
        \{\phi(\mathbf x), \Pi(\mathbf y)\} = \int \mathrm d^3x' (\delta^3(\mathbf x - \mathbf x')\delta^3(\mathbf y - \mathbf x') - 0) = \delta^3(\mathbf x - \mathbf y)
    \end{equation}
\end{proof}
\theorem[哈密顿正则方程]
\begin{equation}
    \begin{cases}
        \dot\phi(\mathbf x) = \{\phi_{\mathbf x}, H\}=\frac{\delta H}{\delta\Pi}=\frac{\partial\mathcal H}{\partial\Pi}\\
        \dot\Pi(\mathbf x) = \{\Pi_{\mathbf x}, H\}=-\frac{\delta H}{\delta\phi}=-\frac{\partial\mathcal H}{\partial\phi}+\partial_i\frac{\partial\mathcal H}{\partial(\phi_i)}
    \end{cases}
\end{equation}
\begin{proof}
    \begin{equation}
        \begin{split}
            \delta H &= \int \mathrm d^3x \left(\delta\Pi_\mathbf x\dot\phi_\mathbf x+\Pi_\mathbf x\delta\dot\phi_\mathbf x-\frac{\partial\mathcal L}{\partial\phi}\delta\phi-\Pi_\mathbf x\delta\dot\phi-\frac{\partial\mathcal L}{\partial\phi_i}\delta\phi_i\right)\\
            &= \int \mathrm d^3x \left(\delta\Pi_\mathbf x\dot\phi_\mathbf x - \frac{\partial\mathcal L}{\partial\phi}\delta\phi - \frac{\partial\mathcal L}{\partial\phi_i}\delta\phi_i\right)\\
        \end{split}
    \end{equation}
    分部积分并消去边缘项有:
    \begin{equation}
        \delta H= \int \mathrm d^3x \left(\delta\Pi_\mathbf x\dot\phi_\mathbf x - \frac{\partial\mathcal L}{\partial\phi}\delta\phi + \partial_i\frac{\partial\mathcal L}{\partial\phi_i}\delta\phi\right)\\
    \end{equation}
    于是可得:
    \begin{equation}
        \begin{cases}
            \dot\phi(\mathbf x) = \frac{\partial\mathcal H}{\partial\Pi}\\
            \dot\Pi(\mathbf x) = -\frac{\partial\mathcal H}{\partial\phi}+\partial_i\frac{\partial\mathcal H}{\partial(\phi_i)}
        \end{cases}
    \end{equation}
\end{proof}
讨论: 在转换到Hamilton力学的过程中, Legendre变换给予了$\dot\phi$特殊的地位, 使得我们选定的参考系的时间轴$t$具有了特殊的地位, 因此破坏了洛伦兹协变性.

\newpage
\section{二次量子化}
\songti 二次量子化是将场(如电磁场、电子场)本身进行量子化的框架,它将描述单粒子概率幅的经典场提升为场算符,其激发则对应粒子的产生与湮灭,从而自然地描述了粒子数可变的多粒子系统. 相比之下, 一次量子化中粒子是给定的,其运动(波函数)是量子的.
\subsection{自由实标量场的量子化}
% \begin{enumerate}
%     \item 将场的动力学方程转换为算符方程
%     \item 找到动力学方程的一般解
%     \item 将一般解的积分常数升级为常算符
%     \item 施加量子化条件
%     \item 用常算符构造Hilbert空间
% \end{enumerate}
\begin{enumerate}
    \item 写下Lagrangian
    \item 得到Hamiltonian
    \item 做正则变换解耦
    \item 施加量子化条件
    \item 得到产生湮灭算符
    \item 得到场方程
\end{enumerate}
\kaishu 注: 这里流程没按周洋讲的来, 因为我略微感觉他那样子做有一点点奇怪, 有些地方存在神秘的天降系数, 按照这样从Lagrangian出发的流程会更清晰一些.\songti
写下Lagrangian:
\begin{equation}
    \mathcal L=\frac12\partial_\mu\phi\partial^\mu\phi - \frac12 m^2\phi^2
\end{equation}
我们有EoM:
\begin{equation}
    (\partial_\mu\partial^\mu + m^2)\phi = 0
\end{equation}
或者写成
\begin{equation}
    (\Box + m^2)\phi = 0
\end{equation}
并有色散关系:
\begin{equation}
    \omega^2=\vec p^2+m^2
\end{equation}
满足这个关系的称为on-shell.

得到Hamiltonian:
\begin{equation}
    \mathcal H=\frac12(\pi^2+m^2\phi^2+(\nabla\phi)^2)
\end{equation}

做正则变换, 有母函数
\begin{equation}
    U=-\int\Pi(\vec k)\pi(\vec x)\mathrm e^{i\vec k\cdot\vec x}\mathrm d^3x\mathrm d^3k
\end{equation}

因此
\begin{align}
    &\phi(\vec x)=\int\Phi_{\vec k}\mathrm e^{i\vec k\cdot\vec x}\mathrm d^3k\\
    &\pi(\vec x)=\frac1{(2\pi)^3}\int\Pi_{\vec k}\mathrm e^{-i\vec k\cdot\vec x}\mathrm d^3k
\end{align}
\begin{align}
    &\Phi_{\vec k}=\frac1{(2\pi)^3}\int\phi(\vec x)\mathrm e^{-i\vec k\cdot\vec x}\mathrm d^3x\\
    &\Pi_{\vec k}=\int\pi(\vec x)\mathrm e^{i\vec k\cdot\vec x}\mathrm d^3x
\end{align}
并且有:
\begin{equation}
    \Phi_{\vec k}^\dagger=\Phi_{-\vec k}, \quad \Pi_{\vec k}^\dagger=\Pi_{-\vec k}
\end{equation}
可以得到解耦后的Hamiltonian:
\begin{equation}
    H=\frac12\int \mathrm d^3k (\frac1{(2\pi)^3}\Pi_{\vec k}\Pi_{-\vec k}+(2\pi)^3\omega^2\Phi_{\vec k}\Phi_{-\vec k})
\end{equation}
其中$\omega^2=m^2+k^2$

添加量子化条件:
\begin{equation}
    [\phi(\vec x, t),\pi(\vec y, t)]=i\delta^3(x-y)
\end{equation}
于是有:
\begin{equation}
    [\Phi_{\vec k}, \Pi_{\vec p}]=i\delta^3(k-p)
\end{equation}
令:
\begin{align}
    &a_{\vec k}=(2\pi)^3\sqrt{\frac{\omega}{2}}\left(\Phi_{\vec k}+\frac i{\omega}\frac1{(2\pi)^3}\Pi_{\vec k}^\dagger\right)=\frac1{\sqrt{2\om k}}\int\d^3x\exp{ikx}(\omega\phi+i\pi)\\
    &a^\dagger_{\vec k}=(2\pi)^3\sqrt{\frac{\omega}{2}}\left(\Phi_{\vec k}^\dagger-\frac i{\omega}\frac1{(2\pi)^3}\Pi_{\vec k}\right)=\frac1{\sqrt{2\om k}}\int\d^3x\exp{-ikx}(\omega\phi-i\pi)
\end{align}

有对易子:
\begin{equation}
    [a_{\vec k}, a_{\vec p}^\dagger]=(2\pi)^3\delta^3(\vec k-\vec p)
\end{equation}
更准确地说, 对于$[a_{\vec k}, a_{\vec k}^\dagger]$:
\begin{equation}
    [a_{\vec k}, a_{\vec k}^\dagger]=\mathcal V
\end{equation}
$\mathcal V$为系统的总体积.

于是Hamiltonian可以被对角化:
\begin{equation}
    H=\int\frac{\mathrm d^3p}{(2\pi)^3}\omega_{\vec p}(a^\dagger_{\vec p} a_{\vec p}+\frac12\mathcal V)
\end{equation}

Fourier逆变换可以得到:
\begin{align}
    &\phi(\vec x)=\int\frac{\mathrm d^3p}{(2\pi)^3}\frac1{\sqrt{2\omega_{\vec p}}}(a^\dagger_{\vec p}\mathrm e^{-i\vec p\cdot\vec x}+a_{\vec p}\mathrm e^{i\vec p\cdot\vec x})\\
    &\pi(\vec x)=\int\frac{\mathrm d^3p}{(2\pi)^3} i\sqrt{\frac{\omega_{\vec p}}{2}}(a^\dagger_{\vec p} e^{-i\vec p\cdot\vec x}-a_{\vec p} e^{i\vec p\cdot\vec x})
\end{align}

最后利用时间演化算符得到场方程:
\begin{important}
    \begin{equation}\label{ch4freephi}
        \phi(x)=\mathrm e^{iHt}\phi(\vec x)\mathrm e^{-iHt}=\int\frac{\mathrm d^3p}{(2\pi)^3}\frac1{\sqrt{2\omega_{\vec p}}}(a_{\vec p}^\dagger\mathrm e^{ipx}+a_{\vec p}\mathrm e^{-ipx})
    \end{equation}
    \begin{equation}\label{ch4freepi}
        \pi(x)=\mathrm e^{iHt}\pi(\vec x)\mathrm e^{-iHt}=\int\frac{\mathrm d^3p}{(2\pi)^3} i\sqrt{\frac{\omega_{\vec p}}{2}}(a^\dagger_{\vec p} \e^{ipx}-a_{\vec p} \e^{-ipx})
    \end{equation}
\end{important}

我们可以计算所谓的真空零点能密度:
\begin{equation}
    \frac{\bra0 H\ket0}{\mathcal V}=\int\frac{\mathrm d^3p}{(2\pi)^3}\frac{\omega_{\vec p}}{2}=\int\frac{\mathrm d^3p}{(2\pi)^3}\frac{\sqrt{p^2+m^2}}{2}=+\infty
\end{equation}

我们还可以定义动量算子:
\begin{equation}
    P^i=\int\mathrm\partial^0\phi\partial^i\pi\mathrm d^3x=-\int\phi_t\phi_i\mathrm d^3x
\end{equation}
即:
\begin{align}
    \vec P&=-\int\phi_t\nabla\phi\mathrm d^3x\\
    &=-\int\mathrm d^3x\int\ldsq{p}[i\omega_{\vec p}](a^\dagger_{\vec p} e^{ipx}-a_{\vec p} e^{-ipx})\\
    &\quad\int\ldsq{q}[i\vec q](-a^\dagger_{\vec q}\exp{iqx}+a_{\vec q}\exp{-iqx})\\
    &=\int\ld{p}[\vec p\omega_{\vec p}](a^\dagger_{\vec p} a_{\vec p}+a_{\vec p}a^\dagger_{\vec p})\\
    &=\int\ddd p\vec p a^\dagger_{\vec p} a_{\vec p}
\end{align}
% 动力学方程:
% \begin{equation}
%     (\partial^2+m^2)\phi=0
% \end{equation}
% 利用Fourier变换, 得:
% \begin{equation}
%     \omega^2=p^2+m^2
% \end{equation}
% 于是, 有
% \begin{align}
%     &\phi(x)=\int \frac{\mathrm d^3p}{(2\pi)^3} \frac{1}{\sqrt{2\omega_{\vec p}}}(a_{\vec p} e^{-ipx}+a_{\vec p}^\dagger e^{ipx})\\
%     &\pi(x)=\dot\phi(x)=\int \frac{\mathrm d^3p}{(2\pi)^3} i\sqrt{\frac{\omega}{2}}(-a_{\vec p} e^{-ipx}+a_{\vec p}^\dagger e^{ipx})
% \end{align}
% 做Fourier逆变换, 有:
% \begin{align}
%     &\Phi(\vec p)=\int \mathrm d^3x \, \phi(x)e^{-ipx}=
% \end{align}
% 施加量子化条件:
% \begin{equation}
%     [\phi(\vec x, t),\pi(\vec y, t)]=i\delta^3(x-y)
% \end{equation}

这里需要补充一点:
\begin{theorem}[Lorentz不变的体元]
    \begin{equation}
        \int\frac{\mathrm d^3p}{(2\pi)^3}\frac1{2E}
    \end{equation}
    是一个Lorentz不变量
\end{theorem}
\begin{proof}
    考虑
    \begin{equation}
        \int\frac{\mathrm d^4p}{(2\pi)^4}\Theta(p^0) 2\pi\delta(p^2-m^2)
    \end{equation}
    可以发现其等于
    \begin{equation}
        \int\frac{\mathrm d^3p}{(2\pi)^3}\frac1{2E}
    \end{equation}
\end{proof}
然后我们就可以定义真空态以及Fock空间:
\begin{definition}[真空态]
    真空态$\ket0$满足
    \begin{equation}
        a_{\vec p}\ket0=0, \quad \forall \vec p
    \end{equation}
\end{definition}
\begin{definition}[Fock空间]
    Fock空间为
    \begin{equation}
        \mathcal H=\bigoplus_{n=1}\mathcal H_n
    \end{equation}
    其中$\mathcal H_n$为$n$粒子空间, 即:
    \begin{equation}
        \mathcal H_n=\mathrm{span}\{a_{\vec p_1}^\dagger a_{\vec p_2}^\dagger \cdots a_{\vec p_n}^\dagger \ket0 \,|\, \forall \vec p_i\}
    \end{equation}
\end{definition}
接下来我们检查自由标量场的二次量子化结果与我们的经典一次量子化结果相一致.

\begin{definition}[动量本征态]
    \begin{equation}
        \ket{\vec p}=\sqrt{2\omega_{\vec p}}a^\dagger_{\vec p}\ket0
    \end{equation}
    \begin{equation}
        \ket{\vec p\vec q}=\sqrt{4\omega_{\vec p}\omega_{\vec q}}a^\dagger_{\vec p}a^\dagger_{\vec q}\ket0
    \end{equation}
\end{definition}
我们不难验证:
\begin{equation}
    \braket{\vec p|\vec q}=2\omega_{\vec p}(2\pi)^3\delta^3(\vec p-\vec q)
\end{equation}
\begin{equation}
    \braket{\vec p'\vec q'|\vec p\vec q}=4\omega_{\vec p}\omega_{\vec q}(2\pi)^6\left(\delta^3(\vec p-\vec p')\delta^3(\vec q-\vec q')+\delta^3(\vec p'-\vec q)\delta^3(\vec q'-\vec p)\right)
\end{equation}
\begin{equation}
    \vec P\ket{\vec p}=\vec p\ket{\vec p}
\end{equation}

\begin{definition}[位置本征态]
    定义位置产生算符:
    \begin{align}
        &\psi^\dagger(x)=\int\ldsq pa^\dagger_{\vec p}\mathrm e^{ipx}\\
        &\psi(x)=\int\ldsq pa_{\vec p}\mathrm e^{-ipx}
    \end{align}
    以及位置本征态
    \begin{equation}
        \ket{\vec x}=\psi^\dagger(x)\ket0
    \end{equation}
\end{definition}
我们可以发现:
\begin{equation}
    \ket{\vec x}=\psi^\dagger(x)\ket0=\int\ld{p}\exp{ipx}\ket{\vec p}
\end{equation}
\begin{equation}
    \braket{\vec p|\vec x}=\exp{ipx}
\end{equation}

可以发现与我们在QM里学的一致.

\subsection{自由复标量场量子化}
\begin{equation}
    \mathcal L=\partial_\mu\psi^*\partial^\mu\psi-m^2\psi^*\psi
\end{equation}
其中$\psi=\phi_1+i\phi_2$

\begin{align}
    \pi^\mu=\partial^\mu\psi^*
\end{align}

正则变换:
\begin{align}
    \begin{cases}
        &\Psi_{\vec k}=\frac1{\dpi3}\int\d^3x\psi_{\vec x}\exp{-i\vec k\cdot\vec x}\\
        &\Pi_{\vec k}=\int\d^3x\pi_{\vec x}\exp{i\vec k\cdot\vec x}
    \end{cases}
\end{align}

于是得到:
\begin{equation}
    H=\int\d^3k\left(\frac1{\dpi3}\Pi^\dagger_{-\vec k}\Pi_{\vec k}+\dpi3\om p^2\Psi_{-\vec k}^\dagger\Psi_\vec k\right)
\end{equation}

添加正则量子化条件:
\begin{equation}
    [\psi_{\vec x}, \pi_{\vec y}]=i\delta^3(\vec x-\vec y), \quad [\psi_\vec x, \psi_\vec y]=[\pi_\vec x, \pi_\vec y]=0
\end{equation}

于是我们得到:
\begin{align}
    [\Psi_\vec p, \Pi_\vec q]=[\Psi^\dagger_\vec p, \Pi^\dagger_\vec q]=i\delta^3(\vec p-\vec q), \quad [\Psi_\vec p, \Psi_\vec q]=[\Pi_\vec p, \Pi_\vec q]=0
\end{align}

设
\begin{align}
    &\a p=\dpi3\sqrt{\frac{\om p}2}(\Psi_\vec p+\frac i{\dpi3\om p}\Pi^\dagger_{-\vec p})\\
    &\b p=\dpi3\sqrt{\frac{\om p}2}(\Psi_{-\vec p}^\dagger+\frac i{\dpi3\om p}\Pi_{\vec p})
\end{align}

我们有:
\begin{align}
    [\a p, \a q^\dagger]&=i\dpi3\delta^3(\vec p-\vec q), \quad [\a p, \a q]=[\a p^\dagger, \a q^\dagger]=0\\
    [\b p, \b q^\dagger]&=i\dpi3\delta^3(\vec p-\vec q), \quad [\b p, \b q]=[\b p^\dagger, \b q^\dagger]=0
\end{align}

然后可以对角化Hamiltonian
\begin{equation}
    H=\int\ddd p\om p\left(\a p^\dagger\a p+\b p^\dagger\b p+\mathcal V\right)
\end{equation}


然后有:
\begin{important}
    \begin{align}
        &\psi(x)=\int\ldsq p\left(\a p\exp{-ipx}+\b p^\dagger\exp{ipx}\right)\\
        &\pi(x)=\int\ddd pi\sqrt{\frac{\om p}2}\left(\a p^\dagger\exp{ipx}-\b p\exp{-ipx}\right)
    \end{align}
\end{important}

例\ref{csU1}指出复标量场的$U(1)$对称性导致有守恒荷$Q$
\begin{equation}
    Q\equiv J_N^0=i\left[(\partial^0\psi^\dagger)\psi-(\partial^0\psi)\psi^\dagger\right]
\end{equation}

经过计算我们可以得到:
\begin{equation}
    Q_N=\int\ddd p\left(\b p^\dagger\b p-\a p^\dagger\a p\right)
\end{equation}

\kaishu 讨论: 复标量场是双自由度的, 因此它有两个产生湮灭算子$a, b$. $a, b$分别代表正粒子和反粒子的运动模式. 而从其的$U(1)$对称性导出的守恒量中可以得出结论, 正粒子与反粒子数目之差为常数.\songti

\subsection{两点关联函数}
\begin{definition}[关联函数$D(x-y)$\cite{peskinCausality}]
    \begin{equation}
        D(x-y)\equiv\braket{0|\phi(x)\phi(y)|0}
    \end{equation}
\end{definition}
计算可得:
\begin{theorem}
    \begin{align}
        D(x-y)&=\int\ldsq{p}\int\ldsq{q}\exp{iqy-ipx}\braket{0|\a p\ad q|0}\\
        &=\int\ld{p}\exp{-ip(x-y)}
    \end{align}
\end{theorem}
可以发现, 这是一个Lorentz不变量(更准确来说, 是$\mathcal P\times\rm{SO}(1,3)$的不变量, 由于存在$\Theta$它不能在$\mathcal T$下不变).

\kaishu 讨论: 两点关联函数的意义是什么? 不难证明这就是上一节中的$\braket{x|y}$, 也就是说, 它表示在时空点$y$处(假设y更早发生)激发一个粒子, 在时空点$x$处测量到它的概率密度. \songti

我们分类讨论类时、类空间隔的关联函数$D(x-y)$:\\
类时: 我们不妨假设$x^0>y^0$, 那么我们可以做Lorentz变换, 使得$\vec x'=\vec y'=0$. 设$t={x'}^0-{y'}^0$, 则:
\begin{equation}
    D(x-y)=\int\ld p\exp{-i\omega_{\vec p}t'}=\int_m^{+\infty}\sqrt{\omega^2-m^2}\exp{-i\omega t'}\frac{\d\omega}{\dpi2}
\end{equation}
对于$t'\rightarrow+\infty$
\begin{equation}
    D(x-y)\sim\exp{-imt'}
\end{equation}
类空: 我们可以做Lorentz变换, 使得$x^0=y^0=0$, 并设$\vec r=\vec x-\vec y$, 则:
\begin{align}
    D(x-y)&=\int\ld p\exp{i\vec p\cdot(\vec x-\vec y)}\\
    &=\int_0^{+\infty}\frac{2\pi p^2\d p}{\dpi3}\frac1{2\omega_{\vec p}}\int_0^\pi\exp{ipr\cos\theta}\sin\theta\d\theta\\
    &=\int_0^{+\infty}\frac{-i}{2\dpi2}\frac{\exp{ipr}-\exp{-ipr}}r\frac{p\d p}{\sqrt{p^2+m^2}}\\
    &=-\frac i{2\dpi2r}\int_{-\infty}^{+\infty}\frac{p}{\sqrt{p^2+m^2}}\exp{ipr}\d p
\end{align}
注意到, $p=\pm im$是被积分函数的两个支点, 于是将积分路径改为图\ref{fig:q2Dspace}中沿着上半部分支割线的路径. 然后可得:
\begin{equation}
    D(x-y)=\frac1{\dpi2 r}\int_m^{+\infty}\frac{\omega\exp{-\omega r}}{\sqrt{\omega^2-m^2}}\d\omega
\end{equation}
对于$r\rightarrow+\infty$
\begin{equation}
    D(x-y)\sim\exp{-mr}
\end{equation}
我们发现, 即使是类空间隔, $D(x-y)$仍不为0, 这说明$\phi(x), \phi(y)$在空间中存在重叠(overlap). 
\begin{figure}
    \centering
    \begin{tikzpicture}[x=0.75pt,y=0.75pt,yscale=-.7,xscale=.7]
        \draw  (186,155.64) -- (402,155.64)(296.16,39) -- (296.16,255) (395,150.64) -- (402,155.64) -- (395,160.64) (291.16,46) -- (296.16,39) -- (301.16,46)  ;
        %Shape: Circle [id:dp009460912696354407] 
        \draw  [fill={rgb, 255:red, 0; green, 0; blue, 0 }  ,fill opacity=1 ] (294.83,126) .. controls (294.83,125.17) and (295.5,124.5) .. (296.33,124.5) .. controls (297.16,124.5) and (297.83,125.17) .. (297.83,126) .. controls (297.83,126.83) and (297.16,127.5) .. (296.33,127.5) .. controls (295.5,127.5) and (294.83,126.83) .. (294.83,126) -- cycle ;
        %Shape: Circle [id:dp9972292459262887] 
        \draw  [fill={rgb, 255:red, 0; green, 0; blue, 0 }  ,fill opacity=1 ] (294.83,185) .. controls (294.83,184.17) and (295.5,183.5) .. (296.33,183.5) .. controls (297.16,183.5) and (297.83,184.17) .. (297.83,185) .. controls (297.83,185.83) and (297.16,186.5) .. (296.33,186.5) .. controls (295.5,186.5) and (294.83,185.83) .. (294.83,185) -- cycle ;
        %Straight Lines [id:da906582585507226] 
        \draw    (296,40.17) .. controls (297.67,41.83) and (297.68,43.5) .. (296.02,45.17) .. controls (294.36,46.84) and (294.37,48.51) .. (296.04,50.17) .. controls (297.71,51.83) and (297.72,53.5) .. (296.06,55.17) .. controls (294.4,56.84) and (294.41,58.51) .. (296.08,60.17) .. controls (297.75,61.83) and (297.76,63.5) .. (296.1,65.17) .. controls (294.44,66.84) and (294.45,68.51) .. (296.12,70.17) .. controls (297.79,71.83) and (297.8,73.5) .. (296.14,75.17) .. controls (294.48,76.84) and (294.49,78.51) .. (296.16,80.17) .. controls (297.83,81.84) and (297.83,83.5) .. (296.17,85.17) .. controls (294.51,86.84) and (294.52,88.51) .. (296.19,90.17) .. controls (297.86,91.83) and (297.87,93.5) .. (296.21,95.17) .. controls (294.55,96.84) and (294.56,98.51) .. (296.23,100.17) .. controls (297.9,101.83) and (297.91,103.5) .. (296.25,105.17) .. controls (294.59,106.84) and (294.6,108.51) .. (296.27,110.17) .. controls (297.94,111.83) and (297.95,113.5) .. (296.29,115.17) .. controls (294.63,116.84) and (294.64,118.51) .. (296.31,120.17) .. controls (297.98,121.83) and (297.99,123.5) .. (296.33,125.17) -- (296.33,126) -- (296.33,126) ;
        %Straight Lines [id:da6403683789941244] 
        \draw    (296.33,183.5) .. controls (298,185.17) and (298,186.83) .. (296.33,188.5) .. controls (294.66,190.17) and (294.66,191.83) .. (296.33,193.5) .. controls (298,195.17) and (298,196.83) .. (296.33,198.5) .. controls (294.66,200.17) and (294.66,201.83) .. (296.33,203.5) .. controls (298,205.17) and (298,206.83) .. (296.33,208.5) .. controls (294.66,210.17) and (294.66,211.83) .. (296.33,213.5) .. controls (298,215.17) and (298,216.83) .. (296.33,218.5) .. controls (294.66,220.17) and (294.66,221.83) .. (296.33,223.5) .. controls (298,225.17) and (298,226.83) .. (296.33,228.5) .. controls (294.66,230.17) and (294.66,231.83) .. (296.33,233.5) .. controls (298,235.17) and (298,236.83) .. (296.33,238.5) .. controls (294.66,240.17) and (294.66,241.83) .. (296.33,243.5) .. controls (298,245.17) and (298,246.83) .. (296.33,248.5) .. controls (294.66,250.17) and (294.66,251.83) .. (296.33,253.5) -- (296.33,255) -- (296.33,255) ;
        %Straight Lines [id:da20632335279729674] 
        \draw    (288.17,41) -- (287.83,131) ;
        %Straight Lines [id:da03237870466308923] 
        \draw    (304,40.17) -- (303.67,130.17) ;
        %Shape: Arc [id:dp3060537526358036] 
        \draw  [draw opacity=0] (303.67,129.25) .. controls (303.67,129.25) and (303.67,129.25) .. (303.67,129.25) .. controls (303.67,129.25) and (303.67,129.25) .. (303.67,129.25) .. controls (303.67,133.62) and (300.12,137.17) .. (295.75,137.17) .. controls (291.41,137.17) and (287.88,133.67) .. (287.83,129.34) -- (295.75,129.25) -- cycle ; \draw   (303.67,129.25) .. controls (303.67,129.25) and (303.67,129.25) .. (303.67,129.25) .. controls (303.67,129.25) and (303.67,129.25) .. (303.67,129.25) .. controls (303.67,133.62) and (300.12,137.17) .. (295.75,137.17) .. controls (291.41,137.17) and (287.88,133.67) .. (287.83,129.34) ;  
        %Straight Lines [id:da32051585297040974] 
        \draw    (288.33,78.83) -- (288.14,83.01) ;
        \draw [shift={(288,86)}, rotate = 272.66] [fill={rgb, 255:red, 0; green, 0; blue, 0 }  ][line width=0.08]  [draw opacity=0] (8.93,-4.29) -- (0,0) -- (8.93,4.29) -- cycle    ;
        %Straight Lines [id:da6503474180961808] 
        \draw    (304,88.5) -- (303.78,82.83) ;
        \draw [shift={(303.67,79.83)}, rotate = 87.8] [fill={rgb, 255:red, 0; green, 0; blue, 0 }  ][line width=0.08]  [draw opacity=0] (8.93,-4.29) -- (0,0) -- (8.93,4.29) -- cycle    ;    
        \draw (306.33,123.9) node [anchor=north west][inner sep=0.75pt]  [font=\small]  {$C$};
    \end{tikzpicture}
    \caption{$f(p)=\frac{p}{\sqrt{p^2+m^2}}\exp{ipr}$}\label{fig:q2Dspace}
\end{figure}

\kaishu 讨论: 类空间隔的关联函数非零能说明这违反了因果律吗?并不能, 因为所谓的因果律需要是指类空间隔的测量之间互相不影响, 也就是说交换这两个算子作用在态上的顺序不会影响结果. 这暗示我们或许应当计算两个算符的对易子来检验我们的理论是否违背了因果律, 而最直接的检验就是计算$[\phi(x), \phi(y)]$. \songti

因此我们考虑$\phi(x), \phi(y)$的对易子:
\begin{equation}
    [\phi(x), \phi(y)]=\braket{0|[\phi(x), \phi(y)]|0}=D(x-y)-D(y-x)
\end{equation}

再次尝试对类时类空间隔分类讨论:\\
类空: 首先将换参考系, 使得$x$, $y$在同一三维空间中, 然后利用Parity算符$\mathcal P$, 显然可以使得$D(x-y)\rightarrow D(y-x)$, 而$\mathcal P$操作不会改变结果, 于是我们有:
\begin{equation}
    [\phi(x), \phi(y)]=0
\end{equation}
类时: 由于$[\phi(x), \phi(y)]$不在$\mathcal T$中保持不变, 因此我们无法如类空般做到交换$x$, $y$, 因此我们不能得到$[\phi(x), \phi(y)]=0$.

可以发现, 对易子对类空间隔一定为0而对类时间隔则不一定, 这正是我们想要的因果性!

然后我们尝试进一步计算对易子
\begin{align}
    [\phi(x), \phi(y)]&=D(x-y)-D(y-x)\\
    &=\int\ld p(\exp{-ip(x-y)}-\exp{ip(x-y)})\\
    &=\int\ddd p(\frac1{2\omega_p}\exp{-ip(x-y)}+\frac1{-2\omega_p}\exp{ip(x-y)})
\end{align}
注意到对第二项做$\vec p\rightarrow-\vec p$换元结果不变, 于是:
\begin{align}
    [\phi(x), \phi(y)]&=\int\ddd p\exp{i\vec p\cdot(\vec x-\vec y)}\left(\frac{\exp{-i\omega_p(x^0-y^0)}}{2\omega_p}+\frac{\exp{i\omega_p(x^0-y^0)}}{-2\omega_p}\right)
\end{align}

观察这个形式, 令我们想到留数定理, 这两项就是留数的相加, 我们可以将其化为:
\begin{align}
    \frac{\exp{-i\omega_p(x^0-y^0)}}{2\omega_p}+\frac{\exp{i\omega_p(x^0-y^0)}}{-2\omega_p}&=\frac{1}{-2\pi i}\left(-2\pi i\rm{Res}(...)-2\pi i\rm{Res}(...)\right)\label{q2eq1}\\
    &=\frac i{2\pi}\int_C\d\omega\frac{\exp{-i\omega(x^0-y^0)}}{(\omega+\omega_{\vec p})(\omega-\omega_{\vec p})}\\
    &=\frac i{2\pi}\int_C\d\omega\frac{\exp{-i\omega(x^0-y^0)}}{\omega^2-\vec p^2-m^2}\\
    &=\int_C\frac{\d\omega}{2\pi}\frac{i}{p^2-m^2}\exp{-i\omega(x^0-y^0)}\label{q2eq2}
\end{align}
\begin{figure}
    \begin{subfigure}[b]{0.45\textwidth}
        \centering
        % \tikzset{every picture/.style={line width=0.75pt}} %set default line width to 0.75pt        
        \begin{tikzpicture}[x=0.75pt,y=0.75pt,yscale=-.7,xscale=.7]
            \draw  (108,169.55) -- (470,169.55)(285.38,21.5) -- (285.38,303.5) (463,164.55) -- (470,169.55) -- (463,174.55) (280.38,28.5) -- (285.38,21.5) -- (290.38,28.5)  ;
            \draw  [fill={rgb, 255:red, 0; green, 0; blue, 0 }  ,fill opacity=1 ] (246,170) .. controls (246,168.9) and (246.9,168) .. (248,168) .. controls (249.1,168) and (250,168.9) .. (250,170) .. controls (250,171.1) and (249.1,172) .. (248,172) .. controls (246.9,172) and (246,171.1) .. (246,170) -- cycle ;
            \draw  [fill={rgb, 255:red, 0; green, 0; blue, 0 }  ,fill opacity=1 ] (322,170) .. controls (322,168.9) and (322.9,168) .. (324,168) .. controls (325.1,168) and (326,168.9) .. (326,170) .. controls (326,171.1) and (325.1,172) .. (324,172) .. controls (322.9,172) and (322,171.1) .. (322,170) -- cycle ;
            \draw    (137,169.75) -- (203.5,169.27) ;
            \draw [shift={(206.5,169.25)}, rotate = 179.59] [fill={rgb, 255:red, 0; green, 0; blue, 0 }  ][line width=0.08]  [draw opacity=0] (8.93,-4.29) -- (0,0) -- (8.93,4.29) -- cycle    ;
            \draw    (374.5,170.25) -- (380.5,169.92) ;
            \draw [shift={(383.5,169.75)}, rotate = 176.82] [fill={rgb, 255:red, 0; green, 0; blue, 0 }  ][line width=0.08]  [draw opacity=0] (8.93,-4.29) -- (0,0) -- (8.93,4.29) -- cycle    ;
            \draw  [draw opacity=0] (152.02,170.24) .. controls (152.21,96.38) and (211.78,36.2) .. (285.85,35.47) .. controls (360.49,34.74) and (421.59,94.66) .. (422.32,169.3) .. controls (422.32,169.61) and (422.32,169.93) .. (422.33,170.24) -- (287.17,170.62) -- cycle ; \draw   (152.02,170.24) .. controls (152.21,96.38) and (211.78,36.2) .. (285.85,35.47) .. controls (360.49,34.74) and (421.59,94.66) .. (422.32,169.3) .. controls (422.32,169.61) and (422.32,169.93) .. (422.33,170.24) ;  
            \draw    (252,40.25) -- (246.75,42.55) ;
            \draw [shift={(244,43.75)}, rotate = 336.37] [fill={rgb, 255:red, 0; green, 0; blue, 0 }  ][line width=0.08]  [draw opacity=0] (8.93,-4.29) -- (0,0) -- (8.93,4.29) -- cycle    ;
            \draw  [draw opacity=0] (241.23,170) .. controls (241.23,166.32) and (244.19,163.3) .. (247.89,163.24) .. controls (251.62,163.17) and (254.7,166.15) .. (254.76,169.89) .. controls (254.77,169.94) and (254.77,170) .. (254.77,170.06) -- (248,170) -- cycle ; \draw   (241.23,170) .. controls (241.23,166.32) and (244.19,163.3) .. (247.89,163.24) .. controls (251.62,163.17) and (254.7,166.15) .. (254.76,169.89) .. controls (254.77,169.94) and (254.77,170) .. (254.77,170.06) ;  
            \draw  [draw opacity=0] (317.23,170) .. controls (317.23,166.32) and (320.19,163.3) .. (323.89,163.24) .. controls (327.62,163.17) and (330.7,166.15) .. (330.76,169.89) .. controls (330.77,169.94) and (330.77,170) .. (330.77,170.06) -- (324,170) -- cycle ; \draw   (317.23,170) .. controls (317.23,166.32) and (320.19,163.3) .. (323.89,163.24) .. controls (327.62,163.17) and (330.7,166.15) .. (330.76,169.89) .. controls (330.77,169.94) and (330.77,170) .. (330.77,170.06) ;  
            \draw (239,172.9) node [anchor=north west][inner sep=0.75pt]  [font=\footnotesize]  {$-\omega _{\vec{p}}$};
            \draw (317.76,173.4) node [anchor=north west][inner sep=0.75pt]  [font=\footnotesize]  {$\omega _{\vec{p}}$};
        \end{tikzpicture}
        \caption{$x^0<y^0$的围道}
        \label{fig:q1f1a}
    \end{subfigure}
    \hfill
    \begin{subfigure}[b]{0.45\textwidth}
        \centering
        \begin{tikzpicture}[x=0.75pt,y=0.75pt,yscale=-.7,xscale=.7]
            \draw  (108,169.55) -- (470,169.55)(285.38,21.5) -- (285.38,303.5) (463,164.55) -- (470,169.55) -- (463,174.55) (280.38,28.5) -- (285.38,21.5) -- (290.38,28.5)  ;
            \draw  [fill={rgb, 255:red, 0; green, 0; blue, 0 }  ,fill opacity=1 ] (246,170) .. controls (246,168.9) and (246.9,168) .. (248,168) .. controls (249.1,168) and (250,168.9) .. (250,170) .. controls (250,171.1) and (249.1,172) .. (248,172) .. controls (246.9,172) and (246,171.1) .. (246,170) -- cycle ;
            \draw  [fill={rgb, 255:red, 0; green, 0; blue, 0 }  ,fill opacity=1 ] (322,170) .. controls (322,168.9) and (322.9,168) .. (324,168) .. controls (325.1,168) and (326,168.9) .. (326,170) .. controls (326,171.1) and (325.1,172) .. (324,172) .. controls (322.9,172) and (322,171.1) .. (322,170) -- cycle ;
            \draw    (137,169.75) -- (203.5,169.27) ;
            \draw [shift={(206.5,169.25)}, rotate = 179.59] [fill={rgb, 255:red, 0; green, 0; blue, 0 }  ][line width=0.08]  [draw opacity=0] (8.93,-4.29) -- (0,0) -- (8.93,4.29) -- cycle    ;
            \draw    (374.5,170.25) -- (380.5,169.92) ;
            \draw [shift={(383.5,169.75)}, rotate = 176.82] [fill={rgb, 255:red, 0; green, 0; blue, 0 }  ][line width=0.08]  [draw opacity=0] (8.93,-4.29) -- (0,0) -- (8.93,4.29) -- cycle    ;
            \draw  [draw opacity=0] (422.32,169.25) .. controls (422.32,169.5) and (422.32,169.75) .. (422.32,170.01) .. controls (422.66,244.65) and (362.43,305.43) .. (287.79,305.77) .. controls (213.15,306.11) and (152.36,245.87) .. (152.02,171.23) .. controls (152.02,170.92) and (152.02,170.6) .. (152.02,170.29) -- (287.17,170.62) -- cycle ; \draw   (422.32,169.25) .. controls (422.32,169.5) and (422.32,169.75) .. (422.32,170.01) .. controls (422.66,244.65) and (362.43,305.43) .. (287.79,305.77) .. controls (213.15,306.11) and (152.36,245.87) .. (152.02,171.23) .. controls (152.02,170.92) and (152.02,170.6) .. (152.02,170.29) ;  
            \draw    (379,270.25) -- (375.3,273.33) ;
            \draw [shift={(373,275.25)}, rotate = 320.19] [fill={rgb, 255:red, 0; green, 0; blue, 0 }  ][line width=0.08]  [draw opacity=0] (8.93,-4.29) -- (0,0) -- (8.93,4.29) -- cycle    ;
            \draw  [draw opacity=0] (241.23,170) .. controls (241.23,166.32) and (244.19,163.3) .. (247.89,163.24) .. controls (251.62,163.17) and (254.7,166.15) .. (254.76,169.89) .. controls (254.77,169.94) and (254.77,170) .. (254.77,170.06) -- (248,170) -- cycle ; \draw   (241.23,170) .. controls (241.23,166.32) and (244.19,163.3) .. (247.89,163.24) .. controls (251.62,163.17) and (254.7,166.15) .. (254.76,169.89) .. controls (254.77,169.94) and (254.77,170) .. (254.77,170.06) ;  
            \draw  [draw opacity=0] (317.23,170) .. controls (317.23,166.32) and (320.19,163.3) .. (323.89,163.24) .. controls (327.62,163.17) and (330.7,166.15) .. (330.76,169.89) .. controls (330.77,169.94) and (330.77,170) .. (330.77,170.06) -- (324,170) -- cycle ; \draw   (317.23,170) .. controls (317.23,166.32) and (320.19,163.3) .. (323.89,163.24) .. controls (327.62,163.17) and (330.7,166.15) .. (330.76,169.89) .. controls (330.77,169.94) and (330.77,170) .. (330.77,170.06) ;  
            \draw (239,172.9) node [anchor=north west][inner sep=0.75pt]  [font=\footnotesize]  {$-\omega _{\vec{p}}$};
            \draw (317.76,173.4) node [anchor=north west][inner sep=0.75pt]  [font=\footnotesize]  {$\omega _{\vec{p}}$};
        \end{tikzpicture}
        \caption{$x^0>y^0$的围道}
        \label{fig:q1f1a}
    \end{subfigure}
    \caption{$D_R(x-y)$围道示意图}
    \label{fig:q2f1}
\end{figure}

为了让大圆弧不会对积分结果产生贡献, 对于$x^0<y^0$我们取围道如图\ref{fig:q1f1a}; 对于$x^0>y^0$我们取围道: \ref{fig:q1f1a}

于是我们可以定义新的关联函数
\begin{theorem}[推迟关联函数$D_R(x-y)$]
    取围道$C$如图\ref{fig:q2f1}
    \begin{equation}
        D_R(x-y)\equiv\int\dddd p \frac i{p^2-m^2}\exp{-ip(x-y)}\label{q2eq3}
    \end{equation}
    或者等价于取一个无穷小正数$\epsilon$:
    \begin{equation}
        D_R(x-y)\equiv\lim_{\epsilon\rightarrow0^+}\int\dddd p \frac i{(p+i\epsilon)^2-m^2}\exp{-ip(x-y)}
    \end{equation}
\end{theorem}
不难发现, 对于$x^0<y^0$, 围道内不含有任何pole, 故$D_R(x-y)=0$, 而对于$x^0>y^0$, 围道内含有pole, 故$D_R(x-y)=D(x-y)-D(y-x)$

于是我们发现
\begin{equation}
    D_R(x-y)=\Theta(x^0-y^0)[\phi(x), \phi(y)]
\end{equation}
这令人想起电动力学中我们学过的推迟势, 故而得名.

类似地, 我们可以定义提前关联函数
\begin{theorem}[提前关联函数$D_A(x-y)$]
    取围道$C$如图\ref{fig:q2f2}
    \begin{equation}
        D_A(x-y)\equiv\int\dddd p \frac i{p^2-m^2}\exp{-ip(x-y)}\label{q2eq4}
    \end{equation}
    或者等价于取一个无穷小正数$\epsilon$:
    \begin{equation}
        D_A(x-y)\equiv\lim_{\epsilon\rightarrow0^+}\int\dddd p \frac i{(p-i\epsilon)^2-m^2}\exp{-ip(x-y)}
    \end{equation}
\end{theorem}
我们可以发现:
\begin{equation}
    D_A(x-y)=-\Theta(y^0-x^0)[\phi(x), \phi(y)]
\end{equation}

这正与电动力学中所谓的提前势相对应.
\begin{figure}
    \begin{subfigure}[b]{0.45\textwidth}
        \centering
        \begin{tikzpicture}[x=0.75pt,y=0.75pt,yscale=-.7,xscale=.7]
            \draw  (108,169.55) -- (470,169.55)(285.38,21.5) -- (285.38,303.5) (463,164.55) -- (470,169.55) -- (463,174.55) (280.38,28.5) -- (285.38,21.5) -- (290.38,28.5)  ;
            %Flowchart: Connector [id:dp1477182047051221] 
            \draw  [fill={rgb, 255:red, 0; green, 0; blue, 0 }  ,fill opacity=1 ] (246,170) .. controls (246,168.9) and (246.9,168) .. (248,168) .. controls (249.1,168) and (250,168.9) .. (250,170) .. controls (250,171.1) and (249.1,172) .. (248,172) .. controls (246.9,172) and (246,171.1) .. (246,170) -- cycle ;
            %Flowchart: Connector [id:dp1451620126132538] 
            \draw  [fill={rgb, 255:red, 0; green, 0; blue, 0 }  ,fill opacity=1 ] (322,170) .. controls (322,168.9) and (322.9,168) .. (324,168) .. controls (325.1,168) and (326,168.9) .. (326,170) .. controls (326,171.1) and (325.1,172) .. (324,172) .. controls (322.9,172) and (322,171.1) .. (322,170) -- cycle ;
            %Straight Lines [id:da554770276238623] 
            \draw    (137,169.75) -- (203.5,169.27) ;
            \draw [shift={(206.5,169.25)}, rotate = 179.59] [fill={rgb, 255:red, 0; green, 0; blue, 0 }  ][line width=0.08]  [draw opacity=0] (8.93,-4.29) -- (0,0) -- (8.93,4.29) -- cycle    ;
            %Straight Lines [id:da2352474286511015] 
            \draw    (374.5,170.25) -- (380.5,169.92) ;
            \draw [shift={(383.5,169.75)}, rotate = 176.82] [fill={rgb, 255:red, 0; green, 0; blue, 0 }  ][line width=0.08]  [draw opacity=0] (8.93,-4.29) -- (0,0) -- (8.93,4.29) -- cycle    ;
            %Shape: Arc [id:dp2545574686176101] 
            \draw  [draw opacity=0] (152.03,171.94) .. controls (152.03,171.69) and (152.02,171.44) .. (152.02,171.19) .. controls (151.71,96.54) and (211.97,35.78) .. (286.61,35.47) .. controls (361.25,35.16) and (422.01,95.41) .. (422.32,170.05) .. controls (422.33,170.37) and (422.33,170.68) .. (422.33,171) -- (287.17,170.62) -- cycle ; \draw   (152.03,171.94) .. controls (152.03,171.69) and (152.02,171.44) .. (152.02,171.19) .. controls (151.71,96.54) and (211.97,35.78) .. (286.61,35.47) .. controls (361.25,35.16) and (422.01,95.41) .. (422.32,170.05) .. controls (422.33,170.37) and (422.33,170.68) .. (422.33,171) ;  
            %Straight Lines [id:da07283350213642492] 
            \draw    (217,55.25) -- (212.52,58.13) ;
            \draw [shift={(210,59.75)}, rotate = 327.26] [fill={rgb, 255:red, 0; green, 0; blue, 0 }  ][line width=0.08]  [draw opacity=0] (8.93,-4.29) -- (0,0) -- (8.93,4.29) -- cycle    ;
            %Shape: Arc [id:dp1734533961545065] 
            \draw  [draw opacity=0] (241.24,170.23) .. controls (241.36,174.31) and (244.42,177.55) .. (248.13,177.49) .. controls (251.86,177.42) and (254.83,174.02) .. (254.76,169.89) .. controls (254.76,169.83) and (254.76,169.77) .. (254.76,169.71) -- (248,170) -- cycle ; \draw   (241.24,170.23) .. controls (241.36,174.31) and (244.42,177.55) .. (248.13,177.49) .. controls (251.86,177.42) and (254.83,174.02) .. (254.76,169.89) .. controls (254.76,169.83) and (254.76,169.77) .. (254.76,169.71) ;  
            %Shape: Arc [id:dp1937528136867961] 
            \draw  [draw opacity=0] (317.27,169.22) .. controls (317.24,169.52) and (317.23,169.81) .. (317.24,170.11) .. controls (317.31,174.25) and (320.39,177.55) .. (324.13,177.49) .. controls (327.86,177.43) and (330.84,174.02) .. (330.76,169.89) .. controls (330.76,169.83) and (330.76,169.77) .. (330.76,169.71) -- (324,170) -- cycle ; \draw   (317.27,169.22) .. controls (317.24,169.52) and (317.23,169.81) .. (317.24,170.11) .. controls (317.31,174.25) and (320.39,177.55) .. (324.13,177.49) .. controls (327.86,177.43) and (330.84,174.02) .. (330.76,169.89) .. controls (330.76,169.83) and (330.76,169.77) .. (330.76,169.71) ;  

            % Text Node
            \draw (236,178.9) node [anchor=north west][inner sep=0.75pt]  [font=\footnotesize]  {$-\omega _{\vec{p}}$};
            % Text Node
            \draw (316.73,180.9) node [anchor=north west][inner sep=0.75pt]  [font=\footnotesize]  {$\omega _{\vec{p}}$};
        \end{tikzpicture}
        \caption{$x^0<y^0$的围道}
        \label{fig:q1f2a}
    \end{subfigure}
    \hfill
    \begin{subfigure}[b]{0.45\textwidth}
        \centering
        \begin{tikzpicture}[x=0.75pt,y=0.75pt,yscale=-.7,xscale=.7]
            \draw  (108,169.55) -- (470,169.55)(285.38,21.5) -- (285.38,303.5) (463,164.55) -- (470,169.55) -- (463,174.55) (280.38,28.5) -- (285.38,21.5) -- (290.38,28.5)  ;
            %Flowchart: Connector [id:dp1477182047051221] 
            \draw  [fill={rgb, 255:red, 0; green, 0; blue, 0 }  ,fill opacity=1 ] (246,170) .. controls (246,168.9) and (246.9,168) .. (248,168) .. controls (249.1,168) and (250,168.9) .. (250,170) .. controls (250,171.1) and (249.1,172) .. (248,172) .. controls (246.9,172) and (246,171.1) .. (246,170) -- cycle ;
            %Flowchart: Connector [id:dp1451620126132538] 
            \draw  [fill={rgb, 255:red, 0; green, 0; blue, 0 }  ,fill opacity=1 ] (322,170) .. controls (322,168.9) and (322.9,168) .. (324,168) .. controls (325.1,168) and (326,168.9) .. (326,170) .. controls (326,171.1) and (325.1,172) .. (324,172) .. controls (322.9,172) and (322,171.1) .. (322,170) -- cycle ;
            %Straight Lines [id:da554770276238623] 
            \draw    (137,169.75) -- (203.5,169.27) ;
            \draw [shift={(206.5,169.25)}, rotate = 179.59] [fill={rgb, 255:red, 0; green, 0; blue, 0 }  ][line width=0.08]  [draw opacity=0] (8.93,-4.29) -- (0,0) -- (8.93,4.29) -- cycle    ;
            %Straight Lines [id:da2352474286511015] 
            \draw    (374.5,170.25) -- (380.5,169.92) ;
            \draw [shift={(383.5,169.75)}, rotate = 176.82] [fill={rgb, 255:red, 0; green, 0; blue, 0 }  ][line width=0.08]  [draw opacity=0] (8.93,-4.29) -- (0,0) -- (8.93,4.29) -- cycle    ;
            %Shape: Arc [id:dp2545574686176101] 
            \draw  [draw opacity=0] (422.32,169.25) .. controls (422.32,169.5) and (422.32,169.75) .. (422.32,170.01) .. controls (422.66,244.65) and (362.43,305.43) .. (287.79,305.77) .. controls (213.15,306.11) and (152.36,245.87) .. (152.02,171.23) .. controls (152.02,170.92) and (152.02,170.6) .. (152.02,170.29) -- (287.17,170.62) -- cycle ; \draw   (422.32,169.25) .. controls (422.32,169.5) and (422.32,169.75) .. (422.32,170.01) .. controls (422.66,244.65) and (362.43,305.43) .. (287.79,305.77) .. controls (213.15,306.11) and (152.36,245.87) .. (152.02,171.23) .. controls (152.02,170.92) and (152.02,170.6) .. (152.02,170.29) ;  
            %Straight Lines [id:da07283350213642492] 
            \draw    (379,270.25) -- (375.3,273.33) ;
            \draw [shift={(373,275.25)}, rotate = 320.19] [fill={rgb, 255:red, 0; green, 0; blue, 0 }  ][line width=0.08]  [draw opacity=0] (8.93,-4.29) -- (0,0) -- (8.93,4.29) -- cycle    ;
            %Shape: Arc [id:dp1734533961545065] 
            \draw  [draw opacity=0] (241.24,170.23) .. controls (241.36,174.31) and (244.42,177.55) .. (248.13,177.49) .. controls (251.86,177.42) and (254.83,174.02) .. (254.76,169.89) .. controls (254.76,169.83) and (254.76,169.77) .. (254.76,169.71) -- (248,170) -- cycle ; \draw   (241.24,170.23) .. controls (241.36,174.31) and (244.42,177.55) .. (248.13,177.49) .. controls (251.86,177.42) and (254.83,174.02) .. (254.76,169.89) .. controls (254.76,169.83) and (254.76,169.77) .. (254.76,169.71) ;  
            %Shape: Arc [id:dp1937528136867961] 
            \draw  [draw opacity=0] (317.27,169.22) .. controls (317.24,169.52) and (317.23,169.81) .. (317.24,170.11) .. controls (317.31,174.25) and (320.39,177.55) .. (324.13,177.49) .. controls (327.86,177.43) and (330.84,174.02) .. (330.76,169.89) .. controls (330.76,169.83) and (330.76,169.77) .. (330.76,169.71) -- (324,170) -- cycle ; \draw   (317.27,169.22) .. controls (317.24,169.52) and (317.23,169.81) .. (317.24,170.11) .. controls (317.31,174.25) and (320.39,177.55) .. (324.13,177.49) .. controls (327.86,177.43) and (330.84,174.02) .. (330.76,169.89) .. controls (330.76,169.83) and (330.76,169.77) .. (330.76,169.71) ;  
            % Text Node
            \draw (235.5,151.4) node [anchor=north west][inner sep=0.75pt]  [font=\footnotesize]  {$-\omega _{\vec{p}}$};
            % Text Node
            \draw (316.73,152.9) node [anchor=north west][inner sep=0.75pt]  [font=\footnotesize]  {$\omega _{\vec{p}}$};    
        \end{tikzpicture}
        \caption{$x^0>y^0$的围道}
        \label{fig:q1f2b}
    \end{subfigure}
    \caption{$D_A(x-y)$围道示意图}
    \label{fig:q2f2}
\end{figure}
\begin{figure}
    \begin{subfigure}[b]{0.45\textwidth}
        \centering
        \begin{tikzpicture}[x=0.75pt,y=0.75pt,yscale=-.7,xscale=.7]
            \draw  (108,169.55) -- (470,169.55)(285.38,21.5) -- (285.38,303.5) (463,164.55) -- (470,169.55) -- (463,174.55) (280.38,28.5) -- (285.38,21.5) -- (290.38,28.5)  ;
            %Flowchart: Connector [id:dp1477182047051221] 
            \draw  [fill={rgb, 255:red, 0; green, 0; blue, 0 }  ,fill opacity=1 ] (246,170) .. controls (246,168.9) and (246.9,168) .. (248,168) .. controls (249.1,168) and (250,168.9) .. (250,170) .. controls (250,171.1) and (249.1,172) .. (248,172) .. controls (246.9,172) and (246,171.1) .. (246,170) -- cycle ;
            %Flowchart: Connector [id:dp1451620126132538] 
            \draw  [fill={rgb, 255:red, 0; green, 0; blue, 0 }  ,fill opacity=1 ] (322,170) .. controls (322,168.9) and (322.9,168) .. (324,168) .. controls (325.1,168) and (326,168.9) .. (326,170) .. controls (326,171.1) and (325.1,172) .. (324,172) .. controls (322.9,172) and (322,171.1) .. (322,170) -- cycle ;
            %Straight Lines [id:da554770276238623] 
            \draw    (137,169.75) -- (203.5,169.27) ;
            \draw [shift={(206.5,169.25)}, rotate = 179.59] [fill={rgb, 255:red, 0; green, 0; blue, 0 }  ][line width=0.08]  [draw opacity=0] (8.93,-4.29) -- (0,0) -- (8.93,4.29) -- cycle    ;
            %Straight Lines [id:da2352474286511015] 
            \draw    (374.5,170.25) -- (380.5,169.92) ;
            \draw [shift={(383.5,169.75)}, rotate = 176.82] [fill={rgb, 255:red, 0; green, 0; blue, 0 }  ][line width=0.08]  [draw opacity=0] (8.93,-4.29) -- (0,0) -- (8.93,4.29) -- cycle    ;
            %Shape: Arc [id:dp2545574686176101] 
            \draw  [draw opacity=0] (152.03,169.25) .. controls (152.75,95.5) and (212.61,35.78) .. (286.61,35.47) .. controls (360.98,35.16) and (421.58,94.98) .. (422.32,169.25) -- (287.17,170.62) -- cycle ; \draw   (152.03,169.25) .. controls (152.75,95.5) and (212.61,35.78) .. (286.61,35.47) .. controls (360.98,35.16) and (421.58,94.98) .. (422.32,169.25) ;  
            %Straight Lines [id:da07283350213642492] 
            \draw    (217,55.25) -- (212.52,58.13) ;
            \draw [shift={(210,59.75)}, rotate = 327.26] [fill={rgb, 255:red, 0; green, 0; blue, 0 }  ][line width=0.08]  [draw opacity=0] (8.93,-4.29) -- (0,0) -- (8.93,4.29) -- cycle    ;
            %Shape: Arc [id:dp1734533961545065] 
            \draw  [draw opacity=0] (241.24,170.23) .. controls (241.36,174.31) and (244.42,177.55) .. (248.13,177.49) .. controls (251.86,177.42) and (254.83,174.02) .. (254.76,169.89) .. controls (254.76,169.83) and (254.76,169.77) .. (254.76,169.71) -- (248,170) -- cycle ; \draw   (241.24,170.23) .. controls (241.36,174.31) and (244.42,177.55) .. (248.13,177.49) .. controls (251.86,177.42) and (254.83,174.02) .. (254.76,169.89) .. controls (254.76,169.83) and (254.76,169.77) .. (254.76,169.71) ;  
            %Shape: Arc [id:dp1937528136867961] 
            \draw  [draw opacity=0] (330.76,169.72) .. controls (330.6,165.64) and (327.52,162.42) .. (323.82,162.51) .. controls (320.22,162.6) and (317.35,165.79) .. (317.24,169.72) -- (324,170) -- cycle ; \draw   (330.76,169.72) .. controls (330.6,165.64) and (327.52,162.42) .. (323.82,162.51) .. controls (320.22,162.6) and (317.35,165.79) .. (317.24,169.72) ;  

            % Text Node
            \draw (236,178.9) node [anchor=north west][inner sep=0.75pt]  [font=\footnotesize]  {$-\omega _{\vec{p}}$};
            % Text Node
            \draw (317.73,179.4) node [anchor=north west][inner sep=0.75pt]  [font=\footnotesize]  {$\omega _{\vec{p}}$};
        \end{tikzpicture}
        \caption{$x^0<y^0$的围道}
        \label{fig:q1f3a}
    \end{subfigure}
    \hfill
    \begin{subfigure}[b]{0.45\textwidth}
        \centering
        \begin{tikzpicture}[x=0.75pt,y=0.75pt,yscale=-.7,xscale=.7]
            %Shape: Axis 2D [id:dp2814236532414889] 
            \draw  (108,169.55) -- (470,169.55)(285.38,21.5) -- (285.38,303.5) (463,164.55) -- (470,169.55) -- (463,174.55) (280.38,28.5) -- (285.38,21.5) -- (290.38,28.5)  ;
            %Flowchart: Connector [id:dp1477182047051221] 
            \draw  [fill={rgb, 255:red, 0; green, 0; blue, 0 }  ,fill opacity=1 ] (246,170) .. controls (246,168.9) and (246.9,168) .. (248,168) .. controls (249.1,168) and (250,168.9) .. (250,170) .. controls (250,171.1) and (249.1,172) .. (248,172) .. controls (246.9,172) and (246,171.1) .. (246,170) -- cycle ;
            %Flowchart: Connector [id:dp1451620126132538] 
            \draw  [fill={rgb, 255:red, 0; green, 0; blue, 0 }  ,fill opacity=1 ] (322,170) .. controls (322,168.9) and (322.9,168) .. (324,168) .. controls (325.1,168) and (326,168.9) .. (326,170) .. controls (326,171.1) and (325.1,172) .. (324,172) .. controls (322.9,172) and (322,171.1) .. (322,170) -- cycle ;
            %Straight Lines [id:da554770276238623] 
            \draw    (137,169.75) -- (203.5,169.27) ;
            \draw [shift={(206.5,169.25)}, rotate = 179.59] [fill={rgb, 255:red, 0; green, 0; blue, 0 }  ][line width=0.08]  [draw opacity=0] (8.93,-4.29) -- (0,0) -- (8.93,4.29) -- cycle    ;
            %Straight Lines [id:da2352474286511015] 
            \draw    (374.5,170.25) -- (380.5,169.92) ;
            \draw [shift={(383.5,169.75)}, rotate = 176.82] [fill={rgb, 255:red, 0; green, 0; blue, 0 }  ][line width=0.08]  [draw opacity=0] (8.93,-4.29) -- (0,0) -- (8.93,4.29) -- cycle    ;
            %Shape: Arc [id:dp2545574686176101] 
            \draw  [draw opacity=0] (422.32,172.01) .. controls (421.58,245.76) and (361.71,305.47) .. (287.72,305.77) .. controls (213.34,306.07) and (152.76,246.23) .. (152.03,171.96) -- (287.17,170.62) -- cycle ; \draw   (422.32,172.01) .. controls (421.58,245.76) and (361.71,305.47) .. (287.72,305.77) .. controls (213.34,306.07) and (152.76,246.23) .. (152.03,171.96) ;  
            %Straight Lines [id:da07283350213642492] 
            \draw    (361.5,283.75) -- (357.02,286.63) ;
            \draw [shift={(354.5,288.25)}, rotate = 327.26] [fill={rgb, 255:red, 0; green, 0; blue, 0 }  ][line width=0.08]  [draw opacity=0] (8.93,-4.29) -- (0,0) -- (8.93,4.29) -- cycle    ;
            %Shape: Arc [id:dp1734533961545065] 
            \draw  [draw opacity=0] (241.24,170.23) .. controls (241.36,174.31) and (244.42,177.55) .. (248.13,177.49) .. controls (251.86,177.42) and (254.83,174.02) .. (254.76,169.89) .. controls (254.76,169.83) and (254.76,169.77) .. (254.76,169.71) -- (248,170) -- cycle ; \draw   (241.24,170.23) .. controls (241.36,174.31) and (244.42,177.55) .. (248.13,177.49) .. controls (251.86,177.42) and (254.83,174.02) .. (254.76,169.89) .. controls (254.76,169.83) and (254.76,169.77) .. (254.76,169.71) ;  
            %Shape: Arc [id:dp1937528136867961] 
            \draw  [draw opacity=0] (330.76,169.72) .. controls (330.6,165.64) and (327.52,162.42) .. (323.82,162.51) .. controls (320.22,162.6) and (317.35,165.79) .. (317.24,169.72) -- (324,170) -- cycle ; \draw   (330.76,169.72) .. controls (330.6,165.64) and (327.52,162.42) .. (323.82,162.51) .. controls (320.22,162.6) and (317.35,165.79) .. (317.24,169.72) ;  

            % Text Node
            \draw (236,178.9) node [anchor=north west][inner sep=0.75pt]  [font=\footnotesize]  {$-\omega _{\vec{p}}$};
            % Text Node
            \draw (317.73,179.4) node [anchor=north west][inner sep=0.75pt]  [font=\footnotesize]  {$\omega _{\vec{p}}$};
        \end{tikzpicture}
        \caption{$x^0>y^0$的围道}
        \label{fig:q1f3b}
    \end{subfigure}
    \caption{$D_F(x-y)$围道示意图}
    \label{fig:q2f3}
\end{figure}
\begin{figure}
    \begin{subfigure}[b]{0.45\textwidth}
        \centering
        \begin{tikzpicture}[x=0.75pt,y=0.75pt,yscale=-.7,xscale=.7]
            \draw  (108,169.55) -- (470,169.55)(285.38,21.5) -- (285.38,303.5) (463,164.55) -- (470,169.55) -- (463,174.55) (280.38,28.5) -- (285.38,21.5) -- (290.38,28.5)  ;
            %Flowchart: Connector [id:dp1477182047051221] 
            \draw  [fill={rgb, 255:red, 0; green, 0; blue, 0 }  ,fill opacity=1 ] (246,170) .. controls (246,168.9) and (246.9,168) .. (248,168) .. controls (249.1,168) and (250,168.9) .. (250,170) .. controls (250,171.1) and (249.1,172) .. (248,172) .. controls (246.9,172) and (246,171.1) .. (246,170) -- cycle ;
            %Flowchart: Connector [id:dp1451620126132538] 
            \draw  [fill={rgb, 255:red, 0; green, 0; blue, 0 }  ,fill opacity=1 ] (322,170) .. controls (322,168.9) and (322.9,168) .. (324,168) .. controls (325.1,168) and (326,168.9) .. (326,170) .. controls (326,171.1) and (325.1,172) .. (324,172) .. controls (322.9,172) and (322,171.1) .. (322,170) -- cycle ;
            %Straight Lines [id:da554770276238623] 
            \draw    (137,169.75) -- (203.5,169.27) ;
            \draw [shift={(206.5,169.25)}, rotate = 179.59] [fill={rgb, 255:red, 0; green, 0; blue, 0 }  ][line width=0.08]  [draw opacity=0] (8.93,-4.29) -- (0,0) -- (8.93,4.29) -- cycle    ;
            %Straight Lines [id:da2352474286511015] 
            \draw    (374.5,170.25) -- (380.5,169.92) ;
            \draw [shift={(383.5,169.75)}, rotate = 176.82] [fill={rgb, 255:red, 0; green, 0; blue, 0 }  ][line width=0.08]  [draw opacity=0] (8.93,-4.29) -- (0,0) -- (8.93,4.29) -- cycle    ;
            %Shape: Arc [id:dp2545574686176101] 
            \draw  [draw opacity=0] (152.03,169.39) .. controls (152.68,95.64) and (212.47,35.86) .. (286.47,35.47) .. controls (360.84,35.08) and (421.5,94.84) .. (422.32,169.11) -- (287.17,170.62) -- cycle ; \draw   (152.03,169.39) .. controls (152.68,95.64) and (212.47,35.86) .. (286.47,35.47) .. controls (360.84,35.08) and (421.5,94.84) .. (422.32,169.11) ;  
            %Straight Lines [id:da07283350213642492] 
            \draw    (216,56.25) -- (211.52,59.13) ;
            \draw [shift={(209,60.75)}, rotate = 327.26] [fill={rgb, 255:red, 0; green, 0; blue, 0 }  ][line width=0.08]  [draw opacity=0] (8.93,-4.29) -- (0,0) -- (8.93,4.29) -- cycle    ;
            %Shape: Arc [id:dp1734533961545065] 
            \draw  [draw opacity=0] (254.77,170.05) .. controls (254.81,165.97) and (251.88,162.6) .. (248.18,162.52) .. controls (244.44,162.43) and (241.33,165.7) .. (241.24,169.84) .. controls (241.23,169.9) and (241.23,169.95) .. (241.23,170.01) -- (248,170) -- cycle ; \draw   (254.77,170.05) .. controls (254.81,165.97) and (251.88,162.6) .. (248.18,162.52) .. controls (244.44,162.43) and (241.33,165.7) .. (241.24,169.84) .. controls (241.23,169.9) and (241.23,169.95) .. (241.23,170.01) ;  
            %Shape: Arc [id:dp1937528136867961] 
            \draw  [draw opacity=0] (317.24,170.3) .. controls (317.41,174.38) and (320.51,177.59) .. (324.21,177.49) .. controls (327.81,177.39) and (330.66,174.19) .. (330.76,170.26) -- (324,170) -- cycle ; \draw   (317.24,170.3) .. controls (317.41,174.38) and (320.51,177.59) .. (324.21,177.49) .. controls (327.81,177.39) and (330.66,174.19) .. (330.76,170.26) ;  
            % Text Node
            \draw (236,178.9) node [anchor=north west][inner sep=0.75pt]  [font=\footnotesize]  {$-\omega _{\vec{p}}$};
            % Text Node
            \draw (317.73,179.4) node [anchor=north west][inner sep=0.75pt]  [font=\footnotesize]  {$\omega _{\vec{p}}$};
        \end{tikzpicture}
        \caption{$x^0<y^0$的围道}
        \label{fig:q1f4a}
    \end{subfigure}
    \hfill
    \begin{subfigure}[b]{0.45\textwidth}
        \centering
        \begin{tikzpicture}[x=0.75pt,y=0.75pt,yscale=-.7,xscale=.7]
            %Shape: Axis 2D [id:dp2814236532414889] 
            \draw  (108,169.55) -- (470,169.55)(285.38,21.5) -- (285.38,303.5) (463,164.55) -- (470,169.55) -- (463,174.55) (280.38,28.5) -- (285.38,21.5) -- (290.38,28.5)  ;
            %Flowchart: Connector [id:dp1477182047051221] 
            \draw  [fill={rgb, 255:red, 0; green, 0; blue, 0 }  ,fill opacity=1 ] (246,170) .. controls (246,168.9) and (246.9,168) .. (248,168) .. controls (249.1,168) and (250,168.9) .. (250,170) .. controls (250,171.1) and (249.1,172) .. (248,172) .. controls (246.9,172) and (246,171.1) .. (246,170) -- cycle ;
            %Flowchart: Connector [id:dp1451620126132538] 
            \draw  [fill={rgb, 255:red, 0; green, 0; blue, 0 }  ,fill opacity=1 ] (322,170) .. controls (322,168.9) and (322.9,168) .. (324,168) .. controls (325.1,168) and (326,168.9) .. (326,170) .. controls (326,171.1) and (325.1,172) .. (324,172) .. controls (322.9,172) and (322,171.1) .. (322,170) -- cycle ;
            %Straight Lines [id:da554770276238623] 
            \draw    (137,169.75) -- (203.5,169.27) ;
            \draw [shift={(206.5,169.25)}, rotate = 179.59] [fill={rgb, 255:red, 0; green, 0; blue, 0 }  ][line width=0.08]  [draw opacity=0] (8.93,-4.29) -- (0,0) -- (8.93,4.29) -- cycle    ;
            %Straight Lines [id:da2352474286511015] 
            \draw    (374.5,170.25) -- (380.5,169.92) ;
            \draw [shift={(383.5,169.75)}, rotate = 176.82] [fill={rgb, 255:red, 0; green, 0; blue, 0 }  ][line width=0.08]  [draw opacity=0] (8.93,-4.29) -- (0,0) -- (8.93,4.29) -- cycle    ;
            %Shape: Arc [id:dp2545574686176101] 
            \draw  [draw opacity=0] (422.32,172.01) .. controls (421.58,245.76) and (361.71,305.47) .. (287.72,305.77) .. controls (213.34,306.07) and (152.76,246.23) .. (152.03,171.96) -- (287.17,170.62) -- cycle ; \draw   (422.32,172.01) .. controls (421.58,245.76) and (361.71,305.47) .. (287.72,305.77) .. controls (213.34,306.07) and (152.76,246.23) .. (152.03,171.96) ;  
            %Straight Lines [id:da07283350213642492] 
            \draw    (361.5,283.75) -- (357.02,286.63) ;
            \draw [shift={(354.5,288.25)}, rotate = 327.26] [fill={rgb, 255:red, 0; green, 0; blue, 0 }  ][line width=0.08]  [draw opacity=0] (8.93,-4.29) -- (0,0) -- (8.93,4.29) -- cycle    ;
            %Shape: Arc [id:dp1734533961545065] 
            \draw  [draw opacity=0] (254.77,170.05) .. controls (254.81,165.97) and (251.88,162.6) .. (248.18,162.52) .. controls (244.44,162.43) and (241.33,165.7) .. (241.24,169.84) .. controls (241.23,169.9) and (241.23,169.95) .. (241.23,170.01) -- (248,170) -- cycle ; \draw   (254.77,170.05) .. controls (254.81,165.97) and (251.88,162.6) .. (248.18,162.52) .. controls (244.44,162.43) and (241.33,165.7) .. (241.24,169.84) .. controls (241.23,169.9) and (241.23,169.95) .. (241.23,170.01) ;  
            %Shape: Arc [id:dp1937528136867961] 
            \draw  [draw opacity=0] (317.24,170.3) .. controls (317.41,174.38) and (320.51,177.59) .. (324.21,177.49) .. controls (327.81,177.39) and (330.66,174.19) .. (330.76,170.26) -- (324,170) -- cycle ; \draw   (317.24,170.3) .. controls (317.41,174.38) and (320.51,177.59) .. (324.21,177.49) .. controls (327.81,177.39) and (330.66,174.19) .. (330.76,170.26) ;  
            % Text Node
            \draw (236,178.9) node [anchor=north west][inner sep=0.75pt]  [font=\footnotesize]  {$-\omega _{\vec{p}}$};
            % Text Node
            \draw (317.73,179.4) node [anchor=north west][inner sep=0.75pt]  [font=\footnotesize]  {$\omega _{\vec{p}}$};
        \end{tikzpicture}
        \caption{$x^0>y^0$的围道}
        \label{fig:q1f4b}
    \end{subfigure}
    \caption{$\widetilde{D}_F(x-y)$围道示意图}
    \label{fig:q2f4}
\end{figure}

此外, 我们还有两种取围道的方式, 由此定义两种关联函数, 也就是所谓的Feynman传播子.
% \newpage
\begin{theorem}[Feynman传播子$D_F(x-y)$]
    取围道$C$如图\ref{fig:q2f3}
    \begin{equation}
        D_F(x-y)\equiv\int\dddd p \frac i{p^2-m^2}\exp{-ip(x-y)}\label{q2eq5}
    \end{equation}
    或者等价于取一个无穷小正数$\epsilon$:
    \begin{equation}
        D_F(x-y)\equiv\lim_{\epsilon\rightarrow0^+}\int\dddd p \frac i{p^2-m^2+i\epsilon}\exp{-ip(x-y)}
    \end{equation}
\end{theorem}
\begin{theorem}[共轭Feynman传播子$\widetilde{D}_F(x-y)$]
    取围道$C$如图\ref{fig:q2f4}
    \begin{equation}
        \widetilde D_F(x-y)\equiv\int\dddd p \frac i{p^2-m^2}\exp{-ip(x-y)}\label{q2eq6}
    \end{equation}
    或者等价于取一个无穷小正数$\epsilon$:
    \begin{equation}
        \widetilde D_F(x-y)\equiv\lim_{\epsilon\rightarrow0^+}\int\dddd p \frac i{p^2-m^2-i\epsilon}\exp{-ip(x-y)}
    \end{equation}
\end{theorem}

不难发现, 对于$x^0<y^0$或者$x^0>y^0$, 这俩关联函数都存在一个pole, 并且可以计算发现:
\begin{align}
    D_F(x-y)&=\Theta(x^0-y^0)D(x-y)+\Theta(y^0-x^0)D(y-x)\\
    &=\Theta(x^0-y^0)\braket{0|\phi(x)\phi(y)|0}+\Theta(y^0-x^0)\braket{0|\phi(y)\phi(x)|0}\\
    &=\braket{0|\mathcal T\phi(x)\phi(y)|0}
\end{align}
\begin{align}
    \widetilde D_F(x-y)&=-\Theta(x^0-y^0)D(x-y)-\Theta(y^0-x^0)D(y-x)\\
    &=-\Theta(x^0-y^0)\braket{0|\phi(x)\phi(y)|0}-\Theta(y^0-x^0)\braket{0|\phi(y)\phi(x)|0}\\
    &=-\braket{0|\mathcal T\phi(x)\phi(y)|0}
\end{align}
其中$\mathcal T$为时间顺序算符
\begin{definition}[时间顺序算符$\mathcal T$]
    对于一串$\phi(x_1,x_2,\cdots,x_n)$的算符, 时间顺序算符即起到排序作用, 将时间上在后面发生的算符放在左边, 时间上在前面发生的算符放在右边, 相当于一个冒泡排序.
\end{definition}

\kaishu 注意: 从\eqref{q2eq1}到\eqref{q2eq2}只有在一定条件下取正确的围道才成立, 并不能认为是一个任何条件下都恒等的关系. 它的作用在于启发性得导出传播子$\frac i{p^2-m^2}$, 并由此根据不同的围道取法定义出四种不同的关联函数. 因此\eqref{q2eq3}、\eqref{q2eq4}、\eqref{q2eq5}、\eqref{q2eq6}并不能视为是从\eqref{q2eq2}直接推导得到的.\songti

\begin{theorem}[关联函数的性质]
    这四个关联函数$D_R(x-y)$, $D_A(x-y)$, $D_F(x-y)$, $\widetilde D_F(x-y)$都是EoM的Green函数, 即:
    \begin{equation}
        \begin{split}
            &(\Box+m^2)D_R(x-y)=-i\delta^4(x-y)\\
            &(\Box+m^2)D_A(x-y)=-i\delta^4(x-y)\\
            &(\Box+m^2)D_F(x-y)=-i\delta^4(x-y)\\
            &(\Box+m^2)\widetilde D_F(x-y)=-i\delta^4(x-y)
        \end{split}
    \end{equation}

    而$D(x-y)$则没有这个性质.
\end{theorem}
\begin{proof}
    记这四种中的某个关联函数为$D_X(x-y)$
    我们从定义出发
    \begin{equation}
        D_X(x-y)=\int\dddd p \frac i{p^2-m^2}\exp{-ip(x-y)}
    \end{equation}

    因此
    \begin{equation}
        \Box D_X(x-y)=\int\dddd p \frac i{p^2-m^2}(-p^2)\exp{-ip(x-y)}
    \end{equation}
    \begin{equation}
        m^2 D_X(x-y)=\int\dddd p \frac i{p^2-m^2}(m^2)\exp{-ip(x-y)}
    \end{equation}
    直接相加, 利用\eqref{matheq1}即得:
    \begin{equation}
        (\Box+m^2)D_X(x-y)=-i\int\dddd p\exp{-ip(x-y)}=-i\delta^4(x-y)
    \end{equation}

    而对于$D(x-y)$
    \begin{align}
        (\Box+m^2)D(x-y)&=\int\ld p(p^2+m^2)\exp{-ip(x-y)}\\
        &=m^2\int\ddd p\frac1{\omega_{\vec p}}\exp{-ip(x-y)}\neq-i\delta^4(x-y)
    \end{align}
\end{proof}
\kaishu 讨论: 这些传播子在经典场论中中对应的就是EoM的Green函数, 这就说明了它们的物理意义正是标量粒子的传播. 而$\dddd p$则暗示着在qft中它的传播是off-shell的, 存在不满足色散关系的所谓"内线"粒子. 反之, 若考虑满足色散关系的实粒子, 则我们需要在$\dddd p$后加上$\Theta(p^0)\delta(p^2-m^2)$, 或者使用三维体元$\ld p$以使其on-shell.\songti

\subsection{Wick定理}
\begin{definition}[$\mathcal N$算符]
    $\mathcal N$算符可以将输入它的一串$\a{p}$, $\a{p}^\dagger$强行变成$\a{p}^\dagger$在前$\a{p}$在后的顺序. 例如:
    \begin{equation}
        \mathcal N\left\{\a{p}\a{p}^\dagger\a{q}\a{k}^\dagger\right\}=\a{p}^\dagger\a{k}^\dagger\a{q}\a{p}
    \end{equation}
\end{definition}
\begin{definition}[Wick收缩(Wick contract)]
    \begin{equation}
        \overline{\phi(x)\phi(y)}\equiv D_F(x-y)
    \end{equation}
\end{definition}

\begin{theorem}[Wick定理]\label{wicksTheorem}
    \begin{equation}
        \mathcal T\left[\phi(x_1)\phi(x_2)\cdots\phi(x_n)\right]=\mathcal N\left\{\phi(x_1)\phi(x_2)\cdots\phi(x_n)+\text{所有的Wick收缩}\right\}
    \end{equation}
    其中, 所谓的"所有的Wick"收缩指, 任意的一对两两收缩、任意的两对两两收缩、任意的三对两两收缩等等等\\
    比如对于$n=4$
    \begin{equation}
        \begin{split}
            \mathcal T[\phi(x_1)&\phi(x_2)\phi(x_3)\phi(x_4)]
            =\mathcal N\{\phi(x_1)\phi(x_2)\phi(x_3)\phi(x_4)\\
            &+\overline{\phi(x_1)\phi(x_2)}\phi(x_3)\phi(x_4)+\overline{\phi(x_1)\phi(x_3)}\phi(x_2)\phi(x_4)\\
            &+\overline{\phi(x_1)\phi(x_4)}\phi(x_2)\phi(x_3)+\overline{\phi(x_2)\phi(x_3)}\phi(x_1)\phi(x_4)\\
            &+\overline{\phi(x_2)\phi(x_4)}\phi(x_1)\phi(x_3)+\overline{\phi(x_3)\phi(x_4)}\phi(x_1)\phi(x_2)\\
            &+\overline{\phi(x_1)\phi(x_2)}\cdot\overline{\phi(x_3)\phi(x_4)}+\overline{\phi(x_1)\phi(x_3)}\cdot\overline{\phi(x_2)\phi(x_4)}\\
            &+\overline{\phi(x_1)\phi(x_4)}\cdot\overline{\phi(x_2)\phi(x_3)}\}
        \end{split}
    \end{equation}
\end{theorem}
\begin{proof}
    \begin{lemma}\label{lemma:N_commute}
        $\mathcal N$算符可以与$[,]$换序:
        \begin{equation}
            [\a{p}, N(\cdots)]=N([\a{p}, \cdots])
        \end{equation}
    \end{lemma}
    \begin{lemma}\label{lemma:psi_commute}
        对于$x^0>y^0$
        \begin{equation}
            [\psi(x), \phi(y)]=[\psi(x), \psi^\dagger(y)]=D(x-y)=D_F(x-y)
        \end{equation}
    \end{lemma}
    不妨假设$x_1^0\geq x_2^0 \geq x_3^0\geq\cdots x_n^0$
    首先, 对于$n=2$:
    \begin{equation}
        \mathcal T[\phi(x_1)\phi(x_2)]=\mathcal N\{\phi(x_1)\phi(x_2)+\overline{\phi(x_1)\phi(x_2)}\}
    \end{equation}
    显然成立.

    考虑数学归纳法, 若对$n=k-1$成立, 则对$n=k$:
    \begin{align*}
        &\mathcal T\left[\phi(x_1)\phi(x_2)\cdots\phi(x_k)\right]=\phi(x_1)\left[\phi(x_1)\phi(x_2)\cdots\phi(x_k)\right]\\
        &=(\psi(x_1)+\psi^\dagger(x_1))\mathcal N\left\{\phi(x_2)\phi(x_3)\cdots\phi(x_n)+\text{所有的}(k-1)\text{Wick收缩}\right\}\\
        &=\mathcal N\left\{\psi^\dagger(x_1)(\phi(x_2)\phi(x_3)\cdots\phi(x_n)+\text{所有的}(k-1)\text{Wick收缩})\right\}\\
        &~~~~+\mathcal N\left\{(\phi(x_2)\phi(x_3)\cdots\phi(x_n)+\text{所有的}(k-1)\text{Wick收缩})\psi(x_1)\right\}+[\psi(x_1), \mathcal N\{\cdots\}]\\
        &=\mathcal N\left\{\phi(x_1)\phi(x_2)\phi(x_3)\cdots\phi(x_n)+\text{所有的}(k-1)\text{Wick收缩}\right\}+\mathcal N\{[\psi(x_1), \cdots]\}\\
        &=\mathcal N\left\{\phi(x_1)\phi(x_2)\phi(x_3)\cdots\phi(x_n)+\text{所有的}(k-1)\text{Wick收缩}\right\}\\
        &~~~~+\mathcal N\{\text{与}\psi(x_1)\text{收缩后的所有}(k-1)\text{Wick收缩}\}
    \end{align*}

    最终我们合并得到:
    \begin{equation}
        \mathcal T\left[\phi(x_1)\phi(x_2)\cdots\phi(x_k)\right]=\mathcal N\left\{\phi(x_1)\phi(x_2)\phi(x_3)\cdots\phi(x_n)+\text{所有的}k\text{Wick收缩}\right\}
    \end{equation}
    于是根据归纳公理我们证明得到了Wick定理.

    \kaishu 最后补充一点, 对于Wick定理的证明, 我们并没有利用到任何Free Field的性质, 因此Wick定理也是可以适用于接下来的Interaction Theory的. \songti
\end{proof}