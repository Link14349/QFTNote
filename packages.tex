\usepackage{array}
\usepackage{amssymb,amsfonts}
\usepackage{amsmath}
\usepackage{mathrsfs}
\usepackage{braket}
\usepackage{tcolorbox}
\usepackage{xparse}
\usepackage{tikz}
\usepackage{subcaption}
\usepackage{makecell}
\usepackage{tikz-feynman}
\usepackage{slashed}
\usepackage{cancel}
\usepackage{geometry}
\usepackage{feynmp-auto}
\makeatletter
\def\input@path{{./texmf/}}
\makeatother
\usepackage{tikz-feynman}
\tikzfeynmanset{compat=1.1.0} % 推荐设置兼容性版本

% 定义通用的样式设置,减少重复代码
\tikzfeynmanset{
    my_diagram/.style={
        small, 
        inline=(a.center), 
        % baseline=-0.5ex, % 垂直居中对齐
        horizontal=a to b,
        dot, % <-- 添加这一行:将所有顶点渲染为实心黑点
        % /tikz/dot/minimum size=4pt, % 可选:调整点的大小 (默认值通常是 4pt 左右)
    }
}

\usetikzlibrary{calc}
\usetikzlibrary{arrows.meta}
\usetikzlibrary{bending}

% 定义重要结论环境
\newtcolorbox{important}{
    colback=white,    % 背景色
    colframe=black,   % 边框色
    sharp corners,    % 直角
    boxrule=1.5pt,    % 边框粗细
    coltitle=black
}

\numberwithin{equation}{subsection}
\numberwithin{figure}{section}
\numberwithin{table}{section}
\newcommand{\de}[2]{\frac{\mathrm d#1}{\mathrm d#2}}
\newcommand{\pa}[2]{\frac{\partial #1}{\partial #2}}
\renewcommand{\vec}[1]{\mathbf{#1}}
\renewcommand{\rm}[1]{\mathrm{#1}}
\renewcommand{\bf}[1]{\mathbf{#1}}
\renewcommand{\exp}[1]{\mathrm e^{#1}}
\newcommand{\ddd}[1]{\frac{\mathrm d^3#1}{(2\pi)^3}}
\newcommand{\dddd}[1]{\frac{\mathrm d^4#1}{(2\pi)^4}}
\renewcommand{\a}[1]{a_{\vec #1}}
\renewcommand{\b}[1]{b_{\vec #1}}
\newcommand{\om}[1]{\omega_{\vec #1}}
\newcommand{\ad}[1]{a^\dagger_{\vec #1}}
\renewcommand{\d}{\mathrm d}
\newcommand{\dx}{\mathrm dx}
\newcommand{\dy}{\mathrm dy}
\newcommand{\dz}{\mathrm dz}
\newcommand{\dw}{\mathrm dw}
\newcommand{\dt}{\mathrm dt}
% \renewcommand{\dp}{\mathrm dp}
\newcommand{\dpi}[1]{(2\pi)^{#1}}
\renewcommand{\vec}[1]{\mathbf{#1}}
\newcommand{\Tr}{\mathrm{Tr}}
\newcommand{\uvec}[1]{\mathbf{\hat{#1}}}
\newcommand{\e}{\mathrm e}
\newcommand{\tensor}[1]{\overleftrightarrow{\vec{#1}}}
\NewDocumentCommand{\ld}{m O{1}}{\frac{\mathrm d^3#1}{(2\pi)^3}\frac{#2}{2\omega_{\vec #1}}}
\NewDocumentCommand{\ldsq}{m O{1}}{\frac{\mathrm d^3#1}{(2\pi)^3}\frac{#2}{\sqrt{2\omega_{\vec #1}}}}
\NewDocumentCommand{\lips}{m O{1}}{\frac{\mathrm d^3#1}{(2\pi)^3}\frac{#2}{2\omega_{\vec #1}}}