\section{绪论}
\subsection{Why QFT?}
QFT is a framework for incorporating an infinite number of quantum degrees of
freedom, arranged in spacetime, with the interactions constrained by locality, symmetry,
unitarity, and causality. And it is the language of the laws of nature.\cite{qft:lecture1}
\begin{itemize}
    \item Locality: interactions occur at single points in spacetime
    \item Symmetry: the laws of physics are invariant under certain transformations
    \item Unitarity: the total probability of all possible outcomes of a quantum event is 1
    \item Causality: cause precedes effect, and information cannot travel faster than light
\end{itemize}
\subsection{符号约定}
\begin{itemize}
    \item 采用自然Heaviside-Lorentz单位制: $\hbar = c = \mu_0 = \epsilon_0 = 1$
    \item 度规: $g_{\mu\nu} = \mathrm{diag}(1,-1,-1,-1)$
    \item 爱因斯坦求和约定: 重复指标隐含求和, 并且希腊字母如$\mu,\nu,\lambda\cdots$取0,1,2,3, 拉丁字母如$i,j,k\cdots$取1,2,3
    \item 在不影响歧义的情况下, 对于标量场$\phi$, 其偏导$\mu$分量记为$\phi_\mu=\partial_\mu\phi$
    \item 在不影响歧义的情况下, $p_\mu x^\mu$类的缩并可简记为$px$
    \item $\phi_\mathbf x\equiv\phi(\mathbf x)$
    \item $\phi_i\equiv\phi(x_i)$
    \item $\delta_{xy}\equiv\delta^4(x-y)$
    \item $\delta_{ij}\equiv\delta^4(x_i-x_j)$
    \item $D_{xy}\equiv D_F(x-y)$
    \item $D_{ij}\equiv D_F(x_i-x_j)$
    \item $\braket{\phi_1\phi_2\cdots\phi_n}\equiv\braket{\Omega|\mathcal T\phi_1\phi_2\cdots\phi_n|\Omega}$
\end{itemize}
\begin{equation}
    \Theta(x)=\begin{cases}
        1, & x>0\\
        0, & x<0
    \end{cases}
\end{equation}
单位元电荷
\begin{equation}
    e=-|e|
\end{equation}
精细结构常数
\begin{equation}
    \alpha=\frac{e^2}{4\pi}\approx \frac{1}{137}
\end{equation}
作用量
\begin{equation}
    S = \int \mathcal{L} \mathrm d^4x=\int (-m\mathrm d\tau-qA_\mu\mathrm dx^\mu)-\int \mathrm d^4x \frac14 F_{\mu\nu}F^{\mu\nu}
\end{equation}
其中,$F_{\mu\nu}$为电磁场张量
\begin{equation}
    F_{\mu\nu} = \mathrm dA_{\mu\nu}
\end{equation}
$\mathrm d$为外微分算符,$A_\mu$为电磁四势.\\
展开为分量形式有: 
\begin{subequations}
    \begin{align}
        F_{\mu\nu} &= \begin{bmatrix}
            0 & E_x & E_y & E_z \\
            -E_x & 0 & -B_z & B_y \\
            -E_y & B_z & 0 & -B_x \\
            -E_z & -B_y & B_x & 0
        \end{bmatrix}\\
        F^{\mu\nu} &= \begin{bmatrix}
            0 & -E_x & -E_y & -E_z \\
            E_x & 0 & -B_z & B_y \\
            E_y & B_z & 0 & -B_x \\
            E_z & -B_y & B_x & 0
        \end{bmatrix}
    \end{align}
\end{subequations}
以及有粒子动力学方程:
\begin{equation}
    m\frac{\mathrm d^2x^\mu}{\mathrm d\tau^2}=qF^\mu_{~~\nu}\frac{\mathrm dx^\nu}{\mathrm d\tau}
\end{equation}
还有Maxwell方程组的协变形式:
\begin{equation}
    \begin{cases}
        \partial_\mu F^{\mu\nu} = J^\nu\\
        \partial_{[\mu} F_{\nu\lambda]} = 0
    \end{cases}
\end{equation}
化为矢量方程组:
\begin{equation}
    \begin{cases}
        \nabla \cdot \mathbf{E} &= \rho \\
        \nabla \cdot \mathbf{B} &= 0 \\
        \nabla \times \mathbf{E} &= -\frac{\partial \mathbf{B}}{\partial t} \\
        \nabla \times \mathbf{B} &= \mathbf{J} + \frac{\partial \mathbf{E}}{\partial t}
    \end{cases}
\end{equation}

\newpage
\section{数学基础}
\subsection{泛函}
\begin{definition}[泛函]
    泛函$f: \mathscr{F} \to \mathbb{R}$是定义在某个函数空间上的映射, 全体泛函组成的集合记为$\mathcal{F}$
\end{definition}
\begin{example}
    设$\mathcal{F}$是定义在区间$[a,b]$上的所有实值连续函数的空间,则
    \begin{equation}
        f[\phi] = \int_a^b \phi(x) \, \mathrm{d}x
    \end{equation}
    是$\mathcal{F}$上的一个泛函
\end{example}

\subsection{$\delta$函数}
\begin{definition}[$\delta$函数]
    $\delta(x)$是满足
    \begin{equation}
        \int f(x) \delta(x) \mathrm dx = f(0)
    \end{equation}
    的函数
\end{definition}
\begin{theorem}
    \begin{equation}
        \int \mathrm dx \mathrm e^{ikx}=2\pi \delta(x)\label{matheq1}
    \end{equation}
\end{theorem}
\begin{theorem}
    \begin{equation}
        \int_a^b f(x) \delta(g(x)) \mathrm dx = \sum_{i\in \{i~|~g(x_i)=0, x_i\in[a, b]\}}\frac{f(x_i)}{|g'(x_i)|}= \sum_{i\in \{i~|~g(x_i)=0\}}\frac{f(x_i)}{|g'(x_i)|}\Theta(b-x)\Theta(x-a)
    \end{equation}
\end{theorem}

\subsection{变分}
\begin{definition}[变分]
    设泛函$f: \mathcal{F} \to \mathbb{R}$\\
    其变分
    \begin{equation}
        \frac{\delta f}{\delta \phi(x_0)} = \lim_{\epsilon \to 0} \frac{f[\phi(x) + \epsilon \delta(x-x_0)] - f[\phi]}{\epsilon}
    \end{equation}
\end{definition}
\begin{example}
    设泛函
    \begin{equation}
        f[\phi] = \phi
    \end{equation}
    则
    \begin{equation}
        \frac{\delta f}{\delta \phi(x_0)} = \delta(x-x_0)
    \end{equation}
\end{example}
\begin{example}
    设泛函
    \begin{equation}
        f[\phi] = g(\phi), \text{其中} g: \mathbb{R} \to \mathbb{R}
    \end{equation}
    则
    \begin{equation}
        \frac{\delta f}{\delta \phi(x_0)} = \frac{\partial g}{\partial \phi} \delta(x-x_0)
    \end{equation}
\end{example}
\begin{example}
    设泛函
    \begin{equation}
        f[\phi] = \int g(\phi) \mathrm dx, \text{其中} g: \mathbb{R} \to \mathbb{R}
    \end{equation}
    则
    \begin{equation}
        \frac{\delta f}{\delta \phi(x_0)} = \left.\frac{\partial g}{\partial \phi}\right|_{\phi(x_0)} \delta(x-x_0)
    \end{equation}
\end{example}
\begin{example}
    设泛函
    \begin{equation}
        f[\phi] = \int g(\phi, \nabla\phi) \mathrm d^3x, \text{其中} g: \mathbb{R} \times \mathbb{R}^3 \to \mathbb{R}
    \end{equation}
    则
    \begin{equation}
        \frac{\delta f}{\delta \phi(x_0)} = \left.\left(\frac{\partial g}{\partial \phi} - \partial_i\frac{\partial g}{\partial \phi_i}\right)\right|_{\nabla\phi(x_0)}
    \end{equation}
\end{example}
参考: \cite{functionals}

% \subsection{群\&群表示论}


\subsection{Lorentz群}
\begin{definition}[Lorentz变换]
    Lorentz变换为一种保内积的变换:
    \begin{equation}
        \bar x^\mu=\Lambda^\mu_{~~\nu}x^\nu
    \end{equation}
    使得
    \begin{equation}
        \bar x^\mu\bar x_\mu=x^\mu x_\mu
    \end{equation}
\end{definition}
\begin{theorem}[Lorentz变换的性质]\label{theorem:lorentz_property}
    \begin{equation}
        \Lambda^\mu_{~~\sigma}g_{\mu\nu}\Lambda^\nu_{~~\rho}=g_{\sigma\rho}
    \end{equation}
\end{theorem}
\begin{proof}
    \begin{equation}
        \bar{x}^2=g_{\mu\nu}\bar{x}^\mu\bar x^\nu=x^\sigma(\Lambda^\mu_{~~\sigma}g_{\mu\nu}\Lambda^\nu_{~~\rho})x^\rho=x^\sigma g_{\sigma\rho} x^\rho
    \end{equation}
\end{proof}
于是我们可以有如下推论:
\begin{theorem}[Lorentz变换的逆矩阵]\label{theorem:Lorentz_inverse}
    \begin{equation}
        (\Lambda^{-1})^{\mu}_{~~\nu}=\Lambda_\nu^{~~\mu}
    \end{equation}
\end{theorem}
\begin{proof}
    由\ref{theorem:lorentz_property}可得
    \begin{equation}
        g^{\rho\beta}\Lambda^\mu_{~~\rho}\Lambda^\nu_{~~\sigma}g_{\mu\nu}=g_{\rho\sigma}g^{\rho\beta}=\delta^\beta_{~~\sigma}
    \end{equation}
    即:
    \begin{equation}
        \Lambda_\nu^{~~\beta}\Lambda^\nu_{~~\sigma}=\delta^\beta_{~~\sigma}
    \end{equation}
    于是可以得证.
\end{proof}
\begin{definition}[$\delta\omega^\mu_{~~\nu}$]\label{def:deltaomega}
    对于无穷小Lorentz变换$\Lambda^\mu_{~~\nu}$, 定义
    \begin{equation}
        \Lambda^\mu_{~~\nu}=\delta^\mu_{~~\nu}+\delta\omega^\mu_{~~\nu}
    \end{equation}
\end{definition}
通过\eqref{theorem:lorentz_property}可以发现$\delta\omega_{\mu\nu}$是反称的.
\begin{theorem}[$\delta\omega_{\mu\nu}$的性质]
    \begin{equation}
        \delta\omega_{\mu\nu}=\delta\omega_{[\mu\nu]}
    \end{equation}
\end{theorem}
\begin{definition}[对$\phi$的Unitary变换$U(\Lambda)$]
    \begin{equation}
        U(\mathbf 1+\delta\omega)=1+\frac i2\delta\omega_{\mu\nu}M^{\mu\nu}
    \end{equation}
    其中,$M^{\mu\nu}=M^{[\mu\nu]}$
\end{definition}
\begin{theorem}[结合律]\label{theorem:U_combine}
    我们要求$U$满足:
    \begin{equation}
        U(\Lambda\Lambda')=U(\Lambda)U(\Lambda')
    \end{equation}
\end{theorem}
根据定理\ref{theorem:Lorentz_inverse}, 定理\ref{theorem:U_combine}, 定义\ref{def:deltaomega}, 我们要求$U(\Lambda^{-1}\Lambda'\Lambda)=U(\Lambda^{-1})U(\Lambda')U(\Lambda)$, 于是有\ref{theorem:UMU}:
\begin{theorem}\label{theorem:UMU}
    \begin{equation}
        U^{-1}_\Lambda M^{\mu\nu}U_\Lambda=\Lambda^\mu_{~~\rho}\Lambda^\nu_{~~\sigma}M^{\rho\sigma}
    \end{equation}
\end{theorem}
\begin{proof}
    \begin{equation}\label{2eq1}
        U_\Lambda^{-1}U_{\Lambda'}U_\Lambda=1+\frac i2\delta{\omega'}_{\mu\nu}U_\Lambda^{-1}M^{\mu\nu}U_\Lambda
    \end{equation}
    \begin{equation}
        U(\Lambda^{-1}\Lambda'\Lambda)=U(1+\Lambda^{-1}\omega'\Lambda)=1+\frac i2(\Lambda^{-1}\delta\omega'\Lambda)_{\mu\nu}M^{\mu\nu}
    \end{equation}
    计算$(\Lambda^{-1}\delta\omega'\Lambda)^{\mu}_{~~\nu}$
    \begin{equation}
        (\Lambda^{-1}\delta\omega'\Lambda)^{\mu}_{~~\nu}=\Lambda_\sigma^{~~\mu}\delta{\omega'}^\sigma_{~~\rho}\Lambda^\rho_{~~\nu}
    \end{equation}
    于是
    \begin{equation}
        (\Lambda^{-1}\delta\omega'\Lambda)_{\mu\nu}=\Lambda^{\sigma}_{~~\mu}\delta{\omega'}_{\sigma\rho}\Lambda^\rho_{~~\nu}
    \end{equation}
    因此
    \begin{equation}\label{2eq2}
        U(\Lambda^{-1}\Lambda'\Lambda)=U(1+\Lambda^{-1}\omega'\Lambda)=1+\frac i2\Lambda^{\sigma}_{~~\mu}\delta{\omega'}_{\sigma\rho}\Lambda^\rho_{~~\nu}M^{\mu\nu}
    \end{equation}
    将\eqref{2eq1}与\eqref{2eq2}取等我们有
    \begin{equation}
        \delta{\omega'}_{\rho\sigma}U_\Lambda^{-1}M^{\rho\sigma}U_\Lambda=\Lambda^{\sigma}_{~~\mu}\delta{\omega'}_{\sigma\rho}\Lambda^\rho_{~~\nu}M^{\mu\nu}
    \end{equation}
    于是
    \begin{equation}
        U^{-1}_\Lambda M^{\mu\nu}U_\Lambda=\Lambda^\mu_{~~\rho}\Lambda^\nu_{~~\sigma}M^{\rho\sigma}
    \end{equation}
\end{proof}
进一步展开我们可以得到
\begin{theorem}[$M^{\mu\nu}$的对易子]
    \begin{equation}
        [M^{\mu\nu}, M^{\rho\sigma}]=i(g^{\mu\rho}M^{\nu\sigma}+g^{\sigma\nu}M^{\mu\rho}-g^{\mu\sigma}M^{\nu\rho}-g^{\rho\nu}M^{\mu\sigma})
    \end{equation}
\end{theorem}
\begin{proof}
    展开
    \begin{equation}
        (1-\frac i 2\delta \omega_{\alpha\beta}M^{\alpha\beta})M^{\mu\nu}(1+\frac i 2\delta \omega_{\rho\sigma}M^{\rho\sigma})=(\delta^\mu_{~~\rho}+\delta\omega^\mu_{~~\rho})(\delta^\nu_{~~\sigma}+\delta\omega^\nu_{~~\sigma})M^{\rho\sigma}
    \end{equation}
    化简整理得到
    \begin{equation}
        -\frac i2\delta\omega_{\rho\sigma}[M^{\rho\sigma}, M^{\mu\nu}]=\delta\omega_{\rho\sigma}(M^{\mu\sigma}g^{\rho\nu}-M^{\rho\nu}g^{\mu\sigma})
    \end{equation}
    于是
    \begin{equation}\label{eq3}
        [M^{\mu\nu}, M^{\rho\sigma}]=2i(-g^{\mu\sigma}M^{\nu\rho}-g^{\rho\nu}M^{\mu\sigma})+A^{\mu\nu\rho\sigma}
    \end{equation}
    其中$A^{\mu\nu\rho\sigma}=A^{\nu\mu\rho\sigma}$, $A^{\mu\nu\rho\sigma}=A^{\mu\nu\sigma\rho}$.\\
    交换$\mu$, $\nu$:
    \begin{equation}\label{eq4}
        [M^{\nu\mu}, M^{\rho\sigma}]=2i(-g^{\nu\sigma}M^{\mu\rho}-g^{\rho\mu}M^{\nu\sigma})+A^{\mu\nu\rho\sigma}
    \end{equation}
    注意到$M^{\mu\nu}$反称, \eqref{eq3}+\eqref{eq4}得到:
    \begin{equation}
        A^{\mu\nu\rho\sigma}=i(g^{\mu\sigma}M^{\nu\rho}+g^{\nu\rho}M^{\mu\sigma}+g^{\mu\sigma}M^{\nu\sigma}+g^{\nu\sigma}M^{\mu\rho})
    \end{equation}
    于是可得
    \begin{important}
        \begin{equation}
            [M^{\mu\nu}, M^{\rho\sigma}]=i(g^{\mu\rho}M^{\nu\sigma}+g^{\sigma\nu}M^{\mu\rho}-g^{\mu\sigma}M^{\nu\rho}-g^{\rho\nu}M^{\mu\sigma})
        \end{equation}
    \end{important}
\end{proof}
\begin{definition}[(伪)旋转生成元]
    \begin{equation}
        J_i = \frac12\epsilon_{ijk}M^{jk}
    \end{equation}
    \begin{equation}
        K_i = M_{i0}
    \end{equation}
\end{definition}
\begin{theorem}[(伪)旋转生成元的对易关系]
    \begin{equation}
        [J_i, J_j]=i\epsilon_{ijk}J^k
    \end{equation}
    \begin{equation}
        [J_i, K_j]=i\epsilon_{ijk}K^k
    \end{equation}
    \begin{equation}
        [K_i, K_j]=-i\epsilon_{ijk}J^k
    \end{equation}
\end{theorem}
\begin{proof}
    注意到$g=\mathrm{diag}(1,-1,-1,-1)$
    \begin{equation}
        \begin{split}
            [J_i, J_j]&=-\frac i2\epsilon_{ikl}\epsilon_{jmn}(g^{nk}M^{lm}+g^{ml}M^{kn})\\
            &=-\frac i2\epsilon_{ikl}\epsilon_{jmn}(\textcolor{red}{-\delta^{nk}}M^{lm}\textcolor{red}{-\delta^{ml}}M^{kn})\\
            &=\frac i2(\epsilon_{kli}\epsilon_{kjm}M^{lm}+\epsilon_{lik}\epsilon_{lnj}M^{kn})\\
            &=\frac i2\left((\delta_{lj}\delta_{im}-\delta_{lm}\delta_{ij}M^{lm}+(\delta_{in}))\right)\\
            &=-iM_{ij}
        \end{split}
    \end{equation}
    又因为
    \begin{equation}
        \epsilon_{kij}\epsilon^{kmn}=-(\delta_i^m\delta_j^n-\delta_i^n\delta_j^m)
    \end{equation}
    \textcolor{red}{(不要忘记我们升指标的时候我们度规三次, 而(+ - - -)度规在三维空间的诱导度规为$\mathrm{diag}(-1,-1,-1)$, 因此总体上我们乘了三次-1,产生一个额外的负号)}\\
    因此
    \begin{equation}\label{eq5}
        \epsilon_{ijk}J^k=\frac12\epsilon_{kij}\epsilon^{kmn}M_{mn}=-M_{ij}
    \end{equation}
    然后即得
    \begin{equation}
        [J_i, J_j]=i\epsilon_{ijk}J^k
    \end{equation}
    接着尝试证明$[J_i,K_j]$.
    \begin{equation}
        [J_i,K_j]=[\frac12\epsilon_{imn}M^{mn}, M_{j0}]=\frac12\epsilon_{imn}[M^{mn}, M_{j0}]
    \end{equation}
    计算$[M^{\mu\nu}, M_{\rho\sigma}]$
    \begin{equation}
        [M^{\mu\nu}, M_{\rho\sigma}]=-2i(\delta_\sigma^{~~\mu}M^\nu_{~~\rho}+\delta_\rho^{~~\nu}M^\mu_{~~\sigma})
    \end{equation}
    于是
    \begin{equation}
        [J_i, K_j]=\frac12\epsilon_{ikl}(-2i)\left(\delta_0^{~~k}M^l_{~~j}+\delta_j^{~~l}M^k_{~~0}\right)=i\epsilon_{ijk}M^k_{~~0}
    \end{equation}
    又因为$g_{\mu\nu}$升$0$指标不会改变符号, 因此我们可以直接升指标然后得到
    \begin{equation}
        [J_i, K_j]=i\epsilon_{ijk}M^{k0}=i\epsilon_{ijk}K^k
    \end{equation}
    最后,考虑证明:
    \begin{equation}
        [K_i, K_j]=[M_{i0}, M_{j0}]
    \end{equation}
    因为
    \begin{equation}
        [M_{\mu\nu}, M_{\rho\sigma}]=i(g_{\mu\rho}M_{\nu\sigma}+g_{\sigma\nu}M_{\mu\rho}-g_{\mu\sigma}M_{\nu\rho}-g_{\rho\nu}M_{\mu\sigma})
    \end{equation}
    于是
    \begin{equation}
        [K_i, K_j]=i(g_{ij}M_{00}+g_{00}M_{ij}-g_{i0}M_{0j}-g_{j0}M_{i0})=iM_{ij}
    \end{equation}
    利用\eqref{eq5}我们得到
    \begin{equation}
        [K_i, K_j]=-i\epsilon_{ijk}J^k
    \end{equation}
\end{proof}

\newpage
\section{狭义相对论动力学技巧}
一些常用技巧\cite{griffthsPPSRTrick}:
\begin{enumerate}
    \item $\vec v=\frac{\vec p}{E}$
    \item 使用四矢量以及不变量点积
    \item 使用质心系简化计算
\end{enumerate}
\begin{definition}[Mandelstam变量]
    对于以$m_1, p_1$, $m_2, p_2$的粒子入射, 以$m_3, p_3$ $m_4, p_4$的粒子出射的动力学系统
    \begin{equation}
        s=(p_1+p_2)^2=(p_3+p_4)^2, \quad t=(p_1-p_3)^2=(p_2-p_4)^2, \quad u=(p_1-p_4)^2=(p_2-p_3)^2
    \end{equation}
\end{definition}
\begin{theorem}[s+t+u守恒]
    \begin{equation}
        s+t+u=m_1^2+m_2^2+m_3^2+m_4^2
    \end{equation}
\end{theorem}
\begin{proof}
    我们取质心系, 可以写出$p_1, p_2, p_3, p_4$
    \begin{align}
        p_1=(E_1, \vec p), \quad p_2=(E_2, -\vec p)\\
        p_3=(E_3, \vec k), \quad p_4=(E_4, -\vec k)
    \end{align}

    于是
    \begin{equation}
        s=(E_1+E_2)^2, \quad t=(E_3-E_1)^2-(\vec k-\vec p)^2, \quad u=(E_4-E_1)^2-(\vec k+\vec p)^2
    \end{equation}

    计算可得:
    \begin{align}
        s+t+u&=E_1^2+E_2^2+E_3^2+E_4^2-2\vec p^2-2\vec k^2+2E_1^2+2E_1E_2-2E_1E_3-2E_1E_4\\
        &=m_1^2+m_2^2+m_3^2+m_4^2+2E_1(E_1+E_2-E_3-E_4)\\
        &=m_1^2+m_2^2+m_3^2+m_4^2
    \end{align}
\end{proof}

\begin{theorem}[质心系中$m_1$的能量]
    \begin{equation}
        E^{\mathbf{CM}}_1=\frac{s+m_1^2-m_2^2}{2\sqrt s}
    \end{equation}
\end{theorem}
\begin{theorem}[实验室系($m_2$静止)中$m_1$的能量]
    \begin{equation}
        E^{\mathbf{lab}}_1=\frac{s-m_1^2-m_2^2}{2m_2}
    \end{equation}
\end{theorem}
\begin{theorem}[质心系总能量]
    \begin{equation}
        E^{\mathbf{CM}}_{\mathbf{TOT}}=\sqrt s
    \end{equation}
\end{theorem}
