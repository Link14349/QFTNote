\section{矢量场}
\subsection{有质量矢量场(Massive)}
我们有拉格朗日量
\begin{equation}
    \mathcal L=-\frac14 F_{\mu\nu}F^{\mu\nu}+\frac12 m^2 A_\mu A^\mu
\end{equation}
其中$A^\mu=(\phi, \vec A)\Rightarrow A_\mu=(\phi, -\vec A)$

于是我们得到EoM:
\begin{equation}
    \partial_\mu F^{\mu\nu}+m^2 A^\nu=0
\end{equation}

将$\partial_\nu$作用到EoM, 并且因为$F^{\mu\nu}$反称, $\partial_\mu\partial_\nu F^{\mu\nu}=0$, 我们得到:
\begin{equation}
    m^2\partial_\nu A^\nu=0
\end{equation}

于是我们可以得到Proca方程
\begin{equation}
    (\partial^2+m^2)A^\nu=0
\end{equation}

正则共轭
\begin{equation}
    \Pi^{\mu\nu}=\pa{\mathcal L}{(\partial_\mu A_\nu)}=-F^{\mu\nu}=\begin{bmatrix}
        0 & E^1 & E^2 & E^3\\
        -E^1 & 0 & B^3 & -B^2\\
        -E^2 & -B^3 & 0 & B^1\\
        -E^3 & B^2 & -B^1 & 0
    \end{bmatrix}
\end{equation}

即$\Pi^{00}=0$, $\Pi^{0i}=E^i$

然后我们有
\begin{equation}
    \mathcal L=-\frac12(\vec B^2-\vec E^2)+\frac12m^2A_{\mu}A^{\mu}
\end{equation}

做Legendre变换, 得到哈密顿密度
\begin{align}
    \mathcal H&=\Pi^0_{~~i}\dot A^i-\mathcal L=-\Pi^{0i}\partial_t A^i-\mathcal L\\
    &=-\vec E\cdot\partial_t\vec A+\frac12 \vec  B^2-\frac12 \vec  E^2-\frac12m^2A_{\mu}A^{\mu}\\
    &=-\vec E\cdot\partial_t\vec A+\frac12 \vec B^2-\frac12 \vec  E^2-\frac12m^2\phi^2+\frac12m^2\vec A^2
\end{align}

根据
\begin{equation}
    \vec E\cdot\nabla\phi=\nabla\cdot(\phi\vec E)-\phi\nabla\cdot\vec E
\end{equation}
以及EoM
\begin{equation}
    \nabla\cdot\vec E=-m^2\phi
\end{equation}
我们最终得到
\begin{equation}
    \mathcal H=\frac12\vec B^2+\frac12\vec E^2+\frac1{2m^2}(\nabla\cdot\vec E)^2+\frac12m^2\vec A^2
\end{equation}

然后我们尝试量子化, 利用正则量子化关系
\begin{equation}
    [A^i_{\vec x}, \Pi^{0j}_{\vec y}]=[A^i_{\vec x}, E^j_{\vec y}]=i\delta^{ij}\delta^3(\vec x-\vec y)=-ig^{ij}\delta^3(\vec x-\vec y)
\end{equation}

我们做Fourier变换, 得到
\begin{align}
    &A^\mu=\int\ldsq{p}\sum_{\lambda=1}^3\left(\epsilon_\lambda^\mu a_{\lambda\vec p}\exp{-ipx}+\epsilon_\lambda^{\mu*}a^\dagger_{\lambda\vec p}\exp{ipx}\right)\\
    &E^\mu=\int\ldsq{p}\sum_{\lambda=1}^3\left((p^\mu\epsilon_\lambda^{0*}-p^0\epsilon_\lambda^{\mu*})a^\dagger_{\lambda\vec p}\exp{ipx}-(p^\mu\epsilon^0_\lambda-p^0\epsilon_\lambda^\mu)a_{\lambda\vec p}\exp{-ipx}\right)
\end{align}

于是有对易子:
\begin{equation}
    [a_{\lambda\vec p}, a^\dagger_{\lambda'\vec p'}]=\dpi3\delta^3(\vec p-\vec p')\delta_{\lambda\lambda'}
\end{equation}

此外, 我们还可以得到
\begin{align}
    &\nabla\cdot\vec E=-m^2\int\ldsq p \sum_\lambda\left(\epsilon_\lambda^0a^\dagger_{\lambda\vec p}\exp{ipx}+\epsilon_\lambda^0a_{\lambda\vec p}\exp{-ipx}\right)\\
    &\vec B=\nabla\times\vec A=\int\ldsq p i\sum_\lambda\left((\vec p\times\vec\epsilon_\lambda)a_{\lambda\vec p}^\dagger\exp{ipx}-(\vec p\times\vec\epsilon_\lambda)a_{\lambda\vec p}\exp{-ipx}\right)
\end{align}

于是经过艰苦卓绝的爆算, 我们得到
\begin{align}
    H&=\int\d^3x\frac12(E^2+B^2+m^2A^2+\frac1{m^2}(\nabla\cdot\vec E)^2)\\
    &=\frac12\int\ld p\sum_\lambda\sum_{\lambda'}(\cdots)\\
    &=\frac12\int\ld p\sum_\lambda\sum_{\lambda'}(-m^2\epsilon_\lambda^0\epsilon_{\lambda'}^0+\omega^2\delta_{\lambda\lambda'}-m^2\epsilon^0_{\lambda}\epsilon^0_{\lambda'}\notag\\
    &\quad\quad+\vec p^2\delta_{\lambda\lambda'}+2m^2\epsilon_{\lambda'}^0\epsilon_\lambda^0+m^2\delta_{\lambda\lambda'})(a^\dagger_{\lambda\vec p}a_{\lambda'\vec p}+a_{\lambda'\vec p}a^\dagger_{\lambda\vec p})\\
    &=\int\ddd p\om p\sum_\lambda\left(a^\dagger_{\lambda\vec p}a_{\lambda\vec p}+\frac12\mathcal V\right)
\end{align}

% (感兴趣的可以见图\ref{fig:massiveEDQ}的手动具体计算过程)

% \begin{figure}[htbp!]
%     \centering
%     \includegraphics[width=0.8\textwidth]{image/massiveEM1.jpg}
%     \includegraphics[width=0.8\textwidth]{image/massiveEM2.jpg}
%     \caption{手算过程}
%     \label{fig:massiveEDQ}
% \end{figure}

然后我们计算它的传播子, 根据
\begin{align}
    \braket{0|A^\mu(x)A^\nu(y)|0}&=\int\ld p\exp{-ip(x-y)}\left(\sum_\lambda^3\epsilon^\mu_\lambda\epsilon^\nu_\lambda\right)\\
    &=\int\ld p\exp{-ip(x-y)}\left(-g^{\mu\nu}+p^\mu p^\nu/p^2\right)
\end{align}
注意到, 此时$p$还是on shell的, 因此$p^2=m^2$, 从而有
\begin{equation}
    \braket{0|A^\mu(x)A^\nu(y)|0}=\int\ld p\exp{-ip(x-y)}\left(-g^{\mu\nu}+p^\mu p^\nu/m^2\right)
\end{equation}
从而不难计算得到
\begin{equation}
    \braket{0|\mathcal TA_\mu(x)A_\nu(y)|0}=\int\dddd p\exp{-ip(x-y)}\frac{-i(g_\mu\nu-p_\mu p_\nu/m^2)}{p^2-m^2+i\epsilon}.
\end{equation}
因此传播子为:
\begin{equation}
    \frac{-i(g_{\mu\nu}-p_\mu p_\nu/m^2)}{p^2-m^2+i\epsilon}
\end{equation}

\subsection{无质量的规范矢量场: 电磁场(Massless)}
为了得到电磁场的二次量子化结果, 自然的想法就是对电磁场取$m\rightarrow0$的结果. 然而这会面临一个问题: 电磁场的$A^\mu$具有规范不变性, 即$A^\mu\rightarrow A^\mu+\partial^\mu\Lambda$不改变其物理意义, 进过规范变换$\partial_\mu A^\mu$也不一定为0. 这使得我们对电磁场的描述存在冗余自由度, 这的一个直接结果就是电磁场只有两个极化方向而不是重电磁场的三个. 

\subsubsection{Lorentz规范}
于是, 为了能够正确得处理自由度, 消除冗余, 我们引入Lorenz规范:
\begin{equation}
    \nabla_\mu A^\mu=0
\end{equation}

但是故事并没有结束, 我们仍然可以通过满足$\partial^2\Lambda=0$的$\Lambda$来进行规范, 因此我们可以进一步地取$\partial_0\Lambda=-A_0=-\varphi$, 从而使得$\varphi=0$.

这样Lorenz规范就退化成了Coulomb规范:
\begin{equation}
    \nabla\cdot\vec A=0
\end{equation}

引入正则量子化条件:
\begin{equation}
    [a_{\vec pr}, a^\dagger_{\vec qs}]=\dpi3\delta_{rs}\delta^3(\vec p-\vec q)
\end{equation}

于是我们就可以二次量子化$\vec A$了
\begin{equation}
    \vec A=\int\ldsq p\sum_{r=1}^2\left(\vec\epsilon_ra_{\vec pr}\exp{-ipx}+\vec\epsilon_r^*a_{\vec pr}^\dagger\exp{ipx}\right)
\end{equation}

从而有
\begin{equation}
    \vec E=-i\int\ddd p\sqrt{\frac{\om p}2}\sum_{r=1}^2\left(\vec\epsilon_ra_{\vec pr}\exp{-ipx}-\vec\epsilon_r^*a_{\vec pr}^\dagger\exp{ipx}\right)
\end{equation}

然后有对易子
\begin{align}
    [A^i_{\vec x}, E^j_{\vec y}]&=i\int\ddd p\exp{i\vec p\cdot(\vec x-\vec y)}\sum_r\epsilon_r^i(\vec p)\epsilon^j_r(\vec p)\\
    &=i\int\ddd p\exp{i\vec p\cdot(\vec x-\vec y)}(\delta^{ij}-\frac{p^ip^j}{\vec p^2})\\
    &=i\delta^3_{\bf{tr}}(\vec x-\vec y)
\end{align}

\kaishu 这里我们可能会有疑问, 为什么这不能按照我们一般的正则量子化的方法, 让
\begin{equation}
    [A^i_{\vec x}, E^j_{\vec y}]=i\delta^{ij}\delta^3(\vec x-\vec y)
\end{equation}

这是因为根据我们的Coulomb规范, $\nabla\cdot\vec A=0$, 因此
\begin{equation}
    [\partial_i A^i, E^j]=0
\end{equation}

然而代入上面的对易关系, 我们会发现
\begin{equation}
    [\partial_iA^i_{\vec x}, E^j_{\vec y}]=i\partial^j\delta^3(\vec x-\vec y)\neq0
\end{equation}

这个正则量子化条件是不自洽的! 这里的原因还是因为无质量的电磁场存在规范冗余, 导致我们丢失了一个"物理的"极化方向.

而由$a, a^\dagger$写出的正则量子化条件, 我们可以验证它可以保证
\begin{align}
    [\partial_i A^i, E^j]&=i\ddd p\exp{i\vec p\cdot(\vec x-\vec y)}ip^i(\delta^{ij}-\frac{p^ip^j}{\vec p^2})\\
    &=-\ddd p\exp{i\vec p\cdot(\vec x-\vec y)}(p^j-\frac{\vec p^2 p^j}{\vec p^2})\\
    &=0
\end{align}
\songti

接着下一个任务就是给出光子的传播子了. 如果不做规范选取, 我们从EoM开始:
\begin{equation}
    (g_{\mu\nu}\partial^2-\partial_\mu\partial_\nu)A^\nu=J_\mu
\end{equation}

换到傅里叶空间, 我们有
\begin{equation}
    (-p^2g_{\mu\nu}+p_\mu p_\nu)\tilde A^\nu=\tilde J_\mu
\end{equation}

似乎只要求出$-p^2g_{\mu\nu}+p_\mu p_\nu$的逆矩阵就好了...?

然而不难发现, $-p^2g_{\mu\nu}+p_\mu p_\nu$是奇异的: $g_\mu\nu-\frac{p_\mu p_\nu}{p^2}$就是在度规张量在类光面上的诱导度规, 它的秩仅有2, 根本不可能找到逆.

这个原因在于电磁场具有规范冗余, 而规范冗余会导致这个矩阵奇异. 因此我们需要做规范的选取. 最直接的想法就是我们在Lorentz规范下计算传播子. 那么我们首先需要计算
\begin{equation}
    \braket{0|A_i(x)A_j(y)|0}=\int\ld p\sum_s\epsilon_{si}(\vec p)\epsilon_{sj}(\vec p)\exp{-ip(x-y)}, 
\end{equation}
根据$\epsilon_i p^i=0$, 我们有
\begin{equation}
    \sum_s\epsilon_{si}(\vec p)\epsilon_{sj}(\vec p)=\delta_{ij}+\frac{p_ip_j}{\vec p^2}=-g_{ij}++\frac{p_ip_j}{\vec p^2}, 
\end{equation}
于是
\begin{equation}
    \braket{0|A_i(x)A_j(y)|0}=\int\ld p\left(g_{ij}+\frac{p_ip_j}{\vec p^2}\right)\exp{-ip(x-y)}.
\end{equation}
所以
\begin{equation}
    \Theta(t_x-t_y)\braket{0|A_i(x)A_j(y)|0}=\int i\dddd p\frac{\exp{-ip(x-y)}}{2|\vec p|(\omega-|\vec p|+i\epsilon)}\left(-g_{ij}+\frac{p_ip_j}{|\vec p|^2}\right)
\end{equation}
于是有Feynman传播子(但是并非下小节的Feynman规范的传播子)
\begin{align}
    \braket{0|\mathcal TA_i(x)A_j(y)|0}&=\int\dddd p\frac{-i(g_{ij}-p_i p_j/|\vec p|^2)}{p^2+i\epsilon}\exp{-ip(x-y)}
\end{align}
, 并且还有
\begin{align}
    \braket{0|\mathcal TA_0(x)A_\mu(y)|0}=\braket{0|\mathcal TA_\mu(x)A_0(y)|0}=0.
\end{align}

这个传播子实在有点丑, 而且很不方便. 这一问题来源于Lorentz规范下$A^0$并没有动力学, 这导致我们的传播子没有$00, 0i, i0$元素. 同时电磁场是没有纵模的, 只有横模的两个极化方向, 这导致传播子中需要将纵模剔除, 这导致$g_{ij}+p_ip_j/|\vec p|^2$中第二项的出现. 尽管理论上来说我们完全可以在Lorentz规范下进行计算, 但一般情况下我们并不希望使用这么复杂的传播子, 因此我们试图寻找一种新的规范固定方式, 使得其传播子拥有一个简洁的形式.

\subsubsection{$\xi$规范}
我们希望我们的传播子能够更加漂亮且方便, 我们引入一个所谓的$\xi$规范来进行规范固定(注意, 规范固定Gauge Fixing和规范条件Gauge Condition并不是一个概念, 我们上面的Lorentz规范是一个规范条件, 它用一个额外的方程强行约束了$A$, 但是接下来我们会看到, 规范固定采取一种不同的做法). 我们的Motivation是引入一个参数$\xi$来使得它有逆, 即:
\begin{equation}
    (-p^2g_{\mu\nu}+p_\mu p_\nu)\to(-p^2g_{\mu\nu}+(1-\frac1\xi)p_\mu p_\nu)
\end{equation}

这样我们就能对$(-p^2g_{\mu\nu}+(1-\frac1\xi)p_\mu p_\nu)$求逆了:
\begin{equation}
    -\frac{g^{\mu\lambda}+(\xi-1)p^\mu p^\lambda/p^2}{p^2}(-p^2g_{\lambda\nu}+(1-\frac1\xi)p_\lambda p_\nu)
\end{equation}

也就是说我们想要EoM变为:
\begin{equation}
    (g_{\mu\nu}\partial^2-(1-\frac1\xi)\partial_\mu\partial_\nu)A^\nu=J_\mu
\end{equation}

为了得到这样的EoM, 我们可以在拉氏量中加入一个规范项:
\begin{equation}
    \mathcal L=-\frac14F^2-J_\mu A^\mu-\frac1{2\xi}(\partial_\mu A^\mu)^2, 
\end{equation}
从而得到我们想要的EoM. 并且注意到, 做规范变换$A'^\mu=A^\mu+\partial^\mu\Lambda$, 其中
\begin{equation}
    \partial^2\partial_\nu\Lambda=\frac{\xi'-\xi}{\xi}\partial_\nu\partial_\mu A^\mu
\end{equation}
或者写成
\begin{equation}
    \partial^2\partial_\nu\Lambda=\frac{\xi'-\xi}{\xi'}\partial_\nu\partial_\mu A'^\mu
\end{equation}
$A'^\mu$的EoM就变成:
\begin{equation}
    (g_{\mu\nu}\partial^2+(\frac1{\xi'}-1)\partial_\mu\partial_\nu)A'^\mu=J_\nu
\end{equation}
这说明$\xi$的不同取值其实就对应不同的规范.

\kaishu
这里我们会质疑, 规范变换不应当改变EoM, 为什么$\xi$规范会改变EoM? 我们可以证明, $\xi$规范带来的额外非物理贡献在涉及计算具体物理量, 比如$S$矩阵时是可以严格消掉的. (具体证明待续)

进一步的理解就是, 这里的规范固定后的场已经不是我们经典意义下的电磁场了: 显而易见, 它的动力学和经典电磁场完全不同, 它并不是Maxwell方程所描述的电磁场. 我们回到经典情况下考虑, 这个$\xi$"规范"下电磁场的EoM为
\begin{align}
    \nabla\cdot\vec E=-\rho-\frac1\xi\partial_t\left(\partial_t A^0+\nabla\cdot\vec A\right)\\
    \nabla\times\vec B-\partial_t\vec E=\vec J+\frac1\xi\nabla\left(\partial_t A^0+\nabla\cdot\vec A\right)
\end{align}
无源的两个Maxwell仍然满足:
\begin{align}
    \nabla\times\vec E+\partial_t\vec B=0\\
    \nabla\cdot\vec B=0
\end{align}
粒子的受力仍然为
\begin{equation}
    \vec F=q(\vec E+\vec v\times\vec B).
\end{equation}

有源的两个方程组的修改导致$\vec E, \vec B$显然完全不是经典电磁场的值. 所以说经过这么一个规范固定过程, 我们相当于找到了一个并不满足经典电磁场运动学, 但是能够保持$S$矩阵不变的"辅助场", 在某个确定$\xi$的Lagrangian下, 它就是一个不具有规范不变性的场: 没有Lorentz规范之类的规范条件强行约束. 它不具有规范冗余, 单纯的一个EoM
\begin{equation}
    (g_{\mu\nu}\partial^2-(1-\frac1\xi)\partial_\mu\partial_\nu)A^\nu=0
\end{equation}
就足以描述它的全部动力学性质.
\songti

% 我们将EoM改写为
% \begin{equation}
%     \partial_\mu F^{\mu\nu}+\frac1\xi\partial^\nu\partial_\mu A^\mu=0,
% \end{equation}
% 利用$F^{\mu\nu}$的反称性, 对等式施加$\partial_\nu$, 我们有
% \begin{equation}
%     (\partial^2)\partial_\mu A^\mu=0.
% \end{equation}
% 在动量空间可以写为
% \begin{equation}
%     p^2 p\cdot A=0, 
% \end{equation}

% 由此可见, $\xi$的不同取值其实就对应不同的规范, 在不同规范下传播子是不同的:
% % 并且Lorenz规范就是$\xi=0$: 这时$\frac1\xi$变为无穷大, 为了满足EoM, 我们必须要求$\partial_\mu A^\mu=0$
我们直接写出Green函数
\begin{equation}
    G_{\mu\nu}(x-y)=\int\dddd p\left(-\frac{g_{\mu\nu}+(\xi-1)p_\mu p_\nu/p^2}{p^2}\right)\exp{-ip(x-y)}
\end{equation}
从而得到光子传播子:
\begin{equation}
    \frac{-i(g_{\mu\nu}+(\xi-1)p_\mu p_\nu/p^2)}{p^2+i\epsilon}
\end{equation}
取$\xi=1$, 就是Feynman规范, 得到其传播子
\begin{equation}
    \frac{-ig_{\mu\nu}}{p^2+i\epsilon}
\end{equation}

\newpage
\section{标量QED}
\subsection{规范变换}
我们知道, 复标量场$\psi$有$U(1)$对称性, 这是一个全局变换. 但是在一个局域的$U(1)$变换$\psi\to\psi\exp{i\alpha}$, 即我们新定义的一个规范变换下并不具有不变性. 但我们希望能够通过某些构造使其有这一不变性.

于是我们让$A_\mu$作为联络, 新定义对$\psi$的微分算符:
\begin{equation}
    D_\mu=\partial_\mu+ieA_\mu
\end{equation}

并且在$U(1)$规范变换下, $A_\mu, \psi$以如下方式变换:
\begin{align}
    &A'_\mu=A_\mu-\frac1e\partial_\mu\alpha\\
    &\psi'=\psi\exp{i\alpha}
\end{align}

不难发现
\begin{equation}
    D'_\mu\psi'=(\partial_\mu+ieA_\mu-i\partial_\mu\alpha)(\psi\exp{i\alpha})=\exp{i\alpha}(\partial_\mu+ieA_\mu)\psi=\exp{i\alpha}D_\mu\psi
\end{equation}

所以如果构造一个电磁场与标量场$\phi$耦合的拉氏密度
\begin{equation}
    \mathcal L=-\frac14 F^2+D_\mu\psi(D^\mu\psi)^*-m^2\psi\psi^*
\end{equation}
,则其有$U(1)$规范不变性.

我们可以给他写为更显式的形式
\begin{equation}
    \mathcal L=-\frac14F^2+\partial_\mu\psi\partial^\mu\psi^*-m^2\psi\psi^*+ieA_\mu(\psi\partial^\mu\psi^*-\psi^*\partial^\mu\psi-ie\psi\psi^*A^\mu)
\end{equation}

我们可以写出EoM
\begin{align}
    \begin{cases}
        &(\partial^2+m^2)\psi=-2ieA_\mu\partial^\mu\psi+e^2\psi A_\mu A^\mu\\
        &\partial_\mu F^{\mu\nu}=ie(\psi\partial^\nu\psi^*-\psi^*\partial^\nu\psi-2ie\psi\psi^*A^\nu)
    \end{cases}
\end{align}

于是我们可以得到电流:
\begin{equation}
    J^\nu=ie(\psi\partial^\nu\psi^*-\psi^*\partial^\nu\psi-2ie\psi\psi^*A^\nu)
\end{equation}

并且, 如果我们再次做全局$U(1)$变换, 我们可以再次得到这一守恒流:
\begin{align}
    J^\nu&=\pa{\mathcal L}{\partial_\mu\psi}\frac{\delta\psi}{\delta\alpha}+\pa{\mathcal L}{\partial_\mu\psi^*}\frac{\delta\psi^*}{\delta\alpha}\\
    &=i(\psi\partial^\nu\psi^*-\psi^*\partial^\nu\psi-2e\psi\psi^*A^\nu)
\end{align}

于是, 就从对规范不变性的追求中, 我们得到了标量QED. 

\subsection{标量QED的Feynman规则}
我们不难写出Dyson级数
\begin{align}
    \braket{0|\mathcal T\exp{-i\int\d^4x H_I'}|0}&=\braket{0|\mathcal T\exp{-e\int\d^4xA_\mu\left(\psi\partial^\mu\psi^\dagger-\psi^\dagger\partial^\mu\psi-ie\psi^\dagger\psi A^\mu\right)}|0}\\
    &=1-e\int\d^4xA_\mu\left(\psi\partial^\mu\psi^\dagger-\psi^\dagger\partial^\mu\psi-ie\psi^\dagger\psi A^\mu\right)+\cdots
\end{align}

这里主要的难点在于这个$\partial_\mu\psi, \partial_\mu\psi^\dagger$怎么处理. 我们可以首先将$\partial_\mu$提出缩并外, 考虑到
\begin{equation}
    \braket{0|\mathcal T\psi_x\psi_y^\dagger|0}=\int\dddd p\frac{i}{p^2-m^2+i\epsilon}\exp{-ip(x-y)}
\end{equation}
所以说$\partial_\mu\psi$就相当于一个$-ip_\mu\psi$, $\partial_\mu\psi^\dagger$就相当于$ip_\mu\psi^\dagger$, 我们可以将$p_\mu$给放在相互作用顶点中, 于是我们可以总结Feynman规则
\begin{enumerate}
    \item 复标量传播子
    $$
        \begin{tikzpicture}[baseline=(current bounding box.center)]
            \begin{feynman}
                \vertex (a);
                \vertex [right=2cm of a] (b);
                \diagram* {
                    % dashed, arrow=... 添加了方向箭头
                    (a) -- [scalar arrow, momentum'=\(p\)] (b),
                };
            \end{feynman}
        \end{tikzpicture}
        = \frac{i}{p^2 - m^2 + i\epsilon}
    $$
    \item 光子传播子
    $$
        \begin{tikzpicture}[baseline=(current bounding box.center)]
            \begin{feynman}
                \vertex (a) {\(\mu\)};
                \vertex [right=2cm of a] (b) {\(\nu\)};
                \diagram* {
                    (a) -- [photon, momentum'=\(p\)] (b),
                };
            \end{feynman}
        \end{tikzpicture}
        = \frac{-i g_{\mu\nu}}{p^2 + i\epsilon}
    $$
    \item 标量-光子三点顶角
    $$
        \begin{tikzpicture}[baseline=(current bounding box.center)]
            \begin{feynman}
                \vertex (a);
                \vertex [above right=1.5cm of a] (b); % 出射标量 p'
                \vertex [below right=1.5cm of a] (c); % 入射标量 p
                \vertex [left=1.5cm of a] (d) {\(\mu\)}; % 光子
                \diagram* {
                    % 注意箭头的方向:从 c 到 a,从 a 到 b
                    (c) -- [scalar arrow, momentum=\(p\)] (a),
                    (a) -- [scalar arrow, momentum'=\(p'\)] (b),
                    (a) -- [photon] (d),
                };
            \end{feynman}
        \end{tikzpicture}
        = -ie(p+p')^\mu
    $$
    \item 标量-双光子四点顶角
    $$
        \begin{tikzpicture}[baseline=(current bounding box.center)]
            \begin{feynman}
                \vertex (a);
                \vertex [above left=1.5cm of a] (b) {\(\mu\)};
                \vertex [below left=1.5cm of a] (c) {\(\nu\)};
                \vertex [above right=1.5cm of a] (d);
                \vertex [below right=1.5cm of a] (e);
                \diagram* {
                    (b) -- [photon] (a),
                    (c) -- [photon] (a),
                    % 标量线箭头流向一致 (例如,从 e 到 a,从 a 到 d)
                    (e) -- [scalar arrow] (a), 
                    (a) -- [scalar arrow] (d),
                };
            \end{feynman}
        \end{tikzpicture}
        = 2ie^2 g^{\mu\nu}
    $$
\end{enumerate}

对于所有外线都有对应的因子:
\begin{itemize}
    \item 入射光子 (${\gamma}$): $\epsilon^\mu(p, \lambda)$
    \item 出射光子 (${\gamma}$): $\epsilon^{\mu*}(p, \lambda)$
    \item 其他复标量粒子: $1$
\end{itemize}

这里有一个suble的点就是有关三点顶角中$p^\mu$还是$-p^\mu$的问题, 我们在这里标动量方向的时候都是保持和粒子流的方向一致的, 这样子导致$p, p'$都是正号. 但是如果我们的$p$在图中标的方向和粒子流相反, 也就是说我们在我们的复标量传播子
$$
    \braket{0|\mathcal \psi_1\psi_2^\dagger|0}=\int\dddd p\frac{i}{p^2-m^2+i\epsilon}\exp{-ip(x_1-x_2)}
$$
中做了个$p\to-p$换元, 得到
$$
    \braket{0|\mathcal \psi_1\psi_2^\dagger|0}=\int\dddd p\frac{i}{p^2-m^2+i\epsilon}\exp{ip(x_1-x_2)}
$$
这会导致$\partial_\mu$的作用多了个$-$号, 于是就从$+ip_\mu$变成了$-ip_\mu$, 从而顶角变成
\begin{equation}
    -ie(p-p')^\mu
\end{equation}

然后我们可以计算几个散射的例子作为尝试.
\begin{example}[pion散射]
    $\pi^+, \pi^-$是一对spin0的复标量粒子, 我们考虑散射
    $$
        \pi^+(p)\pi^-(p')\to\pi^+(k)\pi^-(k').
    $$
    计算到树图阶, 不难发现散射有s-channel以及t-channel, 我们直接计算有
    \begin{align}
        \begin{tikzpicture}[baseline=(current bounding box.center)]
            \begin{feynman}
                \vertex (a);
                \vertex [right=1.5cm of a] (b);
                \vertex [above left=1.5cm of a] (ipion);
                \vertex [below left=1.5cm of a] (iantipion);
                \vertex [above right=1.5cm of b] (opion);
                \vertex [below right=1.5cm of b] (oantipion);
                \diagram* {
                    (a) -- [photon] (b),
                    (ipion) -- [scalar arrow] (a),
                    (a) -- [scalar arrow] (iantipion),
                    (b) -- [scalar arrow] (opion),
                    (oantipion) -- [scalar arrow] (b),
                };
            \end{feynman}
        \end{tikzpicture}&=(-ie)^2(p-p')^\mu(k-k')^\nu\frac{-ig_\mu\nu}{(p+p')^2}=ie^2\frac{u-t}{s}
    \end{align}
    \begin{align}
        \begin{tikzpicture}[baseline=(current bounding box.center)]
            \begin{feynman}
                \vertex (a);
                \vertex [below=1.5cm of a] (b);
                \vertex [above left=1.5cm of a] (ipion);
                \vertex [above right=1.5cm of a] (opion);
                \vertex [below left=1.5cm of b] (iantipion);
                \vertex [below right=1.5cm of b] (oantipion);
                \diagram* {
                    (a) -- [photon] (b),
                    (ipion) -- [scalar arrow] (a),
                    (a) -- [scalar arrow] (opion),
                    (b) -- [scalar arrow] (oantipion),
                    (iantipion) -- [scalar arrow] (b),
                };
            \end{feynman}
        \end{tikzpicture}&=-(-ie)^2(p+k)^\mu(p'+k')^\nu\frac{-ig_\mu\nu}{(p-k')^2}=ie^2\frac{u-s}{t}
    \end{align}
    于是有总振幅
    \begin{equation}
        \mathcal M=e^2\left(\frac{u-t}s+\frac{u-s}t\right)
    \end{equation}
    从而计算得到质心系中的散射截面
    \begin{equation}
        \left(\frac{\d\sigma}{\d\Omega}\right)_{CM}=\frac{e^4}{64\pi^2E_{CM}|\vec p_i|}\left(\frac{u-t}s+\frac{u-s}t\right)^2
    \end{equation}
\end{example}

\newpage

\section{旋量}
\subsection{Lorentz群的性质\label{Lorentz}}
我们首先考虑对矢量的Lorentz变换.
\begin{definition}[Lorentz变换]
    Lorentz变换为一种保内积的变换:
    \begin{equation}
        \bar x^\mu=\Lambda^\mu_{~~\nu}x^\nu
    \end{equation}
    使得
    \begin{equation}
        \bar x^\mu\bar x_\mu=x^\mu x_\mu
    \end{equation}
\end{definition}
\begin{theorem}[Lorentz变换的性质]\label{theorem:lorentz_property}
    \begin{equation}
        \Lambda^\mu_{~~\sigma}g_{\mu\nu}\Lambda^\nu_{~~\rho}=g_{\sigma\rho}
    \end{equation}
\end{theorem}
\begin{proof}
    \begin{equation}
        \bar{x}^2=g_{\mu\nu}\bar{x}^\mu\bar x^\nu=x^\sigma(\Lambda^\mu_{~~\sigma}g_{\mu\nu}\Lambda^\nu_{~~\rho})x^\rho=x^\sigma g_{\sigma\rho} x^\rho
    \end{equation}
\end{proof}
于是我们可以有如下推论:
\begin{theorem}[Lorentz变换的逆矩阵]\label{theorem:Lorentz_inverse}
    \begin{equation}
        (\Lambda^{-1})^{\mu}_{~~\nu}=\Lambda_\nu^{~~\mu}
    \end{equation}
\end{theorem}
\begin{proof}
    由\ref{theorem:lorentz_property}可得
    \begin{equation}
        g^{\rho\beta}\Lambda^\mu_{~~\rho}\Lambda^\nu_{~~\sigma}g_{\mu\nu}=g_{\rho\sigma}g^{\rho\beta}=\delta^\beta_{~~\sigma}
    \end{equation}
    即:
    \begin{equation}
        \Lambda_\nu^{~~\beta}\Lambda^\nu_{~~\sigma}=\delta^\beta_{~~\sigma}
    \end{equation}
    于是可以得证.
\end{proof}
\begin{definition}[$\delta\omega^\mu_{~~\nu}$]\label{def:deltaomega}
    对于无穷小Lorentz变换$\Lambda^\mu_{~~\nu}$, 定义
    \begin{equation}
        \Lambda^\mu_{~~\nu}=\delta^\mu_{~~\nu}+\delta\omega^\mu_{~~\nu}
    \end{equation}
\end{definition}
通过\eqref{theorem:lorentz_property}可以发现$\delta\omega_{\mu\nu}$是反称的. 并且我们还可以进一步写成矩阵形式\:
\begin{equation}
    \delta\omega^\mu_{~~\nu}=\begin{bmatrix}
        0 & v^1 & v^2 & v^3 \\
        v^1 & 0 & \theta^3 & -\theta^2 \\
        v^2 & -\theta^3 & 0 & \theta^1 \\
        v^3 & \theta^2 & -\theta^1 & 0
    \end{bmatrix}
\end{equation}
其中, $v^i$为参考系间的相对速度, $\theta^i$为沿着$i$轴旋转的角度. 或者写为降指标后的结果
\begin{equation}
    \delta\omega_{\mu\nu}=\begin{bmatrix}
        0 & v^1 & v^2 & v^3 \\
        -v^1 & 0 & -\theta^3 & \theta^2 \\
        -v^2 & \theta^3 & 0 & -\theta^1\\
        -v^3 & -\theta^2 & \theta^1 & 0
    \end{bmatrix}
\end{equation}

于是我们还可以进一步写出Lorentz变换的显式形式
\begin{align}\label{lorentz-for-vector}
    \begin{cases}
        \delta x^0=\beta^i x^i\\
        \delta x^i=\beta^i x^0-\epsilon_{ijk}\theta^j x^k
    \end{cases}
\end{align}
\begin{theorem}[$\delta\omega_{\mu\nu}$的性质]
    \begin{equation}
        \delta\omega_{\mu\nu}=\delta\omega_{[\mu\nu]}
    \end{equation}
\end{theorem}

然后我们想要把Lorentz变换的操作抽象化, 一般化, 将其提升到群表示论的高度, 于是我们有一个用抽象的Lorentz变换参数$\omega_{\mu\nu}$表示的对某一个对象进行的Lorentz变换, 这成为Lorentz群的一个表示.
\begin{definition}[无穷小Lorentz群变换的表示$U(\Lambda)$]
    \begin{equation}
        U(\mathbf 1+\delta\omega)=1-\frac i2\delta\omega_{\mu\nu}M^{\mu\nu}
    \end{equation}
    其中,$M^{\mu\nu}=M^{[\mu\nu]}$, 是某一个算符
\end{definition}
以此我们有有限Lorentz群变换的表示
\begin{equation}
    U(\omega)=\exp{-\frac i2\omega_{\mu\nu}M^{\mu\nu}}
\end{equation}

\begin{theorem}[结合律]\label{theorem:U_combine}
    作为群表示, 我们要求$U$满足:
    \begin{equation}
        U(\Lambda\Lambda')=U(\Lambda)U(\Lambda')
    \end{equation}
\end{theorem}
根据定理\ref{theorem:Lorentz_inverse}, 定理\ref{theorem:U_combine}, 定义\ref{def:deltaomega}, 我们要求$U(\Lambda^{-1}\Lambda'\Lambda)=U(\Lambda^{-1})U(\Lambda')U(\Lambda)$, 于是有\ref{theorem:UMU}:
\begin{theorem}\label{theorem:UMU}
    \begin{equation}
        U^{-1}_\Lambda M^{\mu\nu}U_\Lambda=\Lambda^\mu_{~~\rho}\Lambda^\nu_{~~\sigma}M^{\rho\sigma}
    \end{equation}
\end{theorem}
\begin{proof}
    \begin{equation}\label{2eq1}
        U_\Lambda^{-1}U_{\Lambda'}U_\Lambda=1-\frac i2\delta{\omega'}_{\mu\nu}U_\Lambda^{-1}M^{\mu\nu}U_\Lambda
    \end{equation}
    \begin{equation}
        U(\Lambda^{-1}\Lambda'\Lambda)=U(1+\Lambda^{-1}\omega'\Lambda)=1-\frac i2(\Lambda^{-1}\delta\omega'\Lambda)_{\mu\nu}M^{\mu\nu}
    \end{equation}
    计算$(\Lambda^{-1}\delta\omega'\Lambda)^{\mu}_{~~\nu}$
    \begin{equation}
        (\Lambda^{-1}\delta\omega'\Lambda)^{\mu}_{~~\nu}=\Lambda_\sigma^{~~\mu}\delta{\omega'}^\sigma_{~~\rho}\Lambda^\rho_{~~\nu}
    \end{equation}
    于是
    \begin{equation}
        (\Lambda^{-1}\delta\omega'\Lambda)_{\mu\nu}=\Lambda^{\sigma}_{~~\mu}\delta{\omega'}_{\sigma\rho}\Lambda^\rho_{~~\nu}
    \end{equation}
    因此
    \begin{equation}\label{2eq2}
        U(\Lambda^{-1}\Lambda'\Lambda)=U(1+\Lambda^{-1}\omega'\Lambda)=1-\frac i2\Lambda^{\sigma}_{~~\mu}\delta{\omega'}_{\sigma\rho}\Lambda^\rho_{~~\nu}M^{\mu\nu}
    \end{equation}
    将\eqref{2eq1}与\eqref{2eq2}取等我们有
    \begin{equation}
        \delta{\omega'}_{\rho\sigma}U_\Lambda^{-1}M^{\rho\sigma}U_\Lambda=\Lambda^{\sigma}_{~~\mu}\delta{\omega'}_{\sigma\rho}\Lambda^\rho_{~~\nu}M^{\mu\nu}
    \end{equation}
    于是
    \begin{equation}
        U^{-1}_\Lambda M^{\mu\nu}U_\Lambda=\Lambda^\mu_{~~\rho}\Lambda^\nu_{~~\sigma}M^{\rho\sigma}
    \end{equation}
\end{proof}
进一步展开我们可以得到
\begin{theorem}[$M^{\mu\nu}$的对易子]\label{M-commutator}
    \begin{equation}
        [M^{\mu\nu}, M^{\rho\sigma}]=i(-g^{\mu\rho}M^{\nu\sigma}-g^{\sigma\nu}M^{\mu\rho}+g^{\mu\sigma}M^{\nu\rho}+g^{\rho\nu}M^{\mu\sigma})
    \end{equation}
\end{theorem}
\begin{proof}
    展开
    \begin{equation}
        (1+\frac i 2\delta \omega_{\alpha\beta}M^{\alpha\beta})M^{\mu\nu}(1-\frac i 2\delta \omega_{\rho\sigma}M^{\rho\sigma})=(\delta^\mu_{~~\rho}+\delta\omega^\mu_{~~\rho})(\delta^\nu_{~~\sigma}+\delta\omega^\nu_{~~\sigma})M^{\rho\sigma}
    \end{equation}
    化简整理得到
    \begin{equation}
        \frac i2\delta\omega_{\rho\sigma}[M^{\rho\sigma}, M^{\mu\nu}]=\delta\omega_{\rho\sigma}(M^{\mu\sigma}g^{\rho\nu}-M^{\rho\nu}g^{\mu\sigma})
    \end{equation}
    于是
    \begin{equation}\label{eq3}
        [M^{\mu\nu}, M^{\rho\sigma}]=2i(g^{\mu\sigma}M^{\nu\rho}+g^{\rho\nu}M^{\mu\sigma})+A^{\mu\nu\rho\sigma}
    \end{equation}
    其中$A^{\mu\nu\rho\sigma}=A^{\nu\mu\rho\sigma}$, $A^{\mu\nu\rho\sigma}=A^{\mu\nu\sigma\rho}$.\\
    交换$\mu$, $\nu$:
    \begin{equation}\label{eq4}
        [M^{\nu\mu}, M^{\rho\sigma}]=2i(g^{\nu\sigma}M^{\mu\rho}+g^{\rho\mu}M^{\nu\sigma})+A^{\mu\nu\rho\sigma}
    \end{equation}
    注意到$M^{\mu\nu}$反称, \eqref{eq3}+\eqref{eq4}得到:
    \begin{equation}
        A^{\mu\nu\rho\sigma}=-i(g^{\mu\sigma}M^{\nu\rho}+g^{\nu\rho}M^{\mu\sigma}+g^{\mu\sigma}M^{\nu\sigma}+g^{\nu\sigma}M^{\mu\rho})
    \end{equation}
    于是可得
    \begin{important}
        \begin{equation}
            [M^{\mu\nu}, M^{\rho\sigma}]=i(-g^{\mu\rho}M^{\nu\sigma}-g^{\sigma\nu}M^{\mu\rho}+g^{\mu\sigma}M^{\nu\rho}+g^{\rho\nu}M^{\mu\sigma})
        \end{equation}
    \end{important}
\end{proof}
\begin{definition}[Lorentz群生成元]\label{lorentz-generators}
    \begin{equation}
        J^i =\frac12\epsilon_{ijk}M^{jk}\Rightarrow M^{ij}=\epsilon_{ijk}J^k
    \end{equation}
    \begin{equation}
        K^i=M^{i0}
    \end{equation}
\end{definition}
写为矩阵形式就是
\begin{equation}
    M^{\mu\nu}=\begin{bmatrix}
        0 & -K^1 & -K^2 & -K^3\\
        K^1 & 0 & J^3 & -J^2\\
        K^2 & -J^3 & 0 & J^1\\
        K^3 & J^2 & -J^1 & 0
    \end{bmatrix}
\end{equation}

然后我们可以将无穷小Lorentz群表示写为
\begin{equation}
    1+i\theta^iJ^i+i\beta^iK^i
\end{equation}
有限大的写为
\begin{equation}
    \exp{i\theta^iJ^i+i\beta^iK^i}
\end{equation}

然后我们有
\begin{theorem}[Lorentz群生成元的对易关系]\label{lorentz-commutator}
    \begin{equation}
        [J^i, J^j]=i\epsilon_{ijk}J^k
    \end{equation}
    \begin{equation}
        [J^i, K^j]=i\epsilon_{ijk}K^k
    \end{equation}
    \begin{equation}
        [K^i, K^j]=-i\epsilon_{ijk}J^k
    \end{equation}
\end{theorem}
注意, 在这里我们都认为是具体指标的计算, 因此不关心上下标的问题: 矢量就是上标, 体元就是下标, 从而避免(+---)度规三维部分升降指标会多出负号的恼人特性.

\begin{proof}
    \begin{align}
        [J^i, J^j]&=\frac14\epsilon_{iml}\epsilon_{jnp}M^{ml}M^{np}\\
        &=\frac14\epsilon_{iml}\epsilon_{jnp}\cdot 2i(g^{mp}M^{ln}+g^{lm}M^{mp})\\
        &=-\frac i2\epsilon_{iml}\epsilon_{jnp}(\delta_{mp}M^{ln}+\delta_{ln}M^{mp})\\
        &=-i\epsilon_{mli}\epsilon_{mjn}M^{ln}\\
        &=-i(\delta_{lj}\delta_{in}-\delta_{ln}\delta_{ij})M^{ln}\\
        &=iM^{ij}\\
        &=i\epsilon_{ijk}J^k\\
        [J^i, K^j]&=\frac12\epsilon_{imn}\cdot2i(g^{m0}M^{nj}+g^{nj}M^{m0})\\
        &=i\epsilon_{imn}(-\delta_{nj})M^{m0}\\
        &=i\epsilon_{ijk}M^{k0}\\
        &=i\epsilon_{ijk}K^k
    \end{align}
    \begin{align}
        [K^i, K^j]&=[M^{i0}, M^{j0}]\\
        &=-i g^{00}M^{ij}\\
        &=-iM^{ij}\\
        &=-i\epsilon_{ijk}J^k
    \end{align}
\end{proof}
\subsubsection{分解Lorentz群}
正如我们在定义\ref{lorentz-generators}中看到的那样, 洛伦兹群有六个生成元, 从而将Lorentz变换表示为:
\begin{equation}
    \Lambda=\exp{i\theta_i J^i+i\beta_i K^i}
\end{equation}

并且定理\ref{lorentz-commutator}, 有对易关系
\begin{equation}
    [J_i, J_j]=i\epsilon_{ijk}J^k
\end{equation}
\begin{equation}
    [J_i, K_j]=i\epsilon_{ijk}K^k
\end{equation}
\begin{equation}
    [K_i, K_j]=-i\epsilon_{ijk}J^k
\end{equation}

或者写为一个张量形式
\begin{equation}
    M^{\mu\nu}=\begin{bmatrix}
        0 & -K^1 & -K^2 & -K^3\\
        K^1 & 0 & J^3 & -J^2\\
        K^2 & -J^3 & 0 & J^1\\
        K^3 & J^2 & -J^1 & 0
    \end{bmatrix}
\end{equation}

并且根据定理\ref{M-commutator}我们有:
\begin{equation}
    [M^{\mu\nu}, M^{\rho\sigma}]=i(-g^{\mu\rho}M^{\nu\sigma}-g^{\sigma\nu}M^{\mu\rho}+g^{\mu\sigma}M^{\nu\rho}+g^{\rho\nu}M^{\mu\sigma})
\end{equation}

于是
\begin{equation}
    \Lambda=\exp{-\frac i2\omega_{\mu\nu}M^{\mu\nu}}
\end{equation}

我们设
\begin{equation}
    J_i^{\pm}=\frac12(J_i\pm iK_i)
\end{equation}

也就是:
\begin{align}
    \begin{cases}
        & \vec J=\vec J^++\vec J^-\\
        & \vec K=i(\vec J^--\vec J^+)
    \end{cases}
\end{align}

于是有对易关系
\begin{align}
    \begin{cases}
        & [J^+_i, J^+_j]=i\epsilon_{ijk}J^{+k} \\
        & [J^-_i, J^-_j]=i\epsilon_{ijk}J^{-k} \\
        & [J^+_i, J^-_j]=0
    \end{cases}
\end{align}
我们发现,$\vec J^\pm$是解耦的, 而它们分别满足$\mathfrak{su}(2)$的Lie代数关系!

于是我们得到结论
\begin{important}
    \begin{equation}
        \mathfrak{so}(1, 3)=\mathfrak{su}(2)\otimes\mathfrak{su}(2)
    \end{equation}
\end{important}

因此, 我们可以将Lorentz群的不可约表示用两个半整数$(m, n)$表示, 分别代表两个$\mathfrak{su}(2)$部分的角量子数. 

于是我们发现不可约表示$(m, n)$的维度为$(2m+1)(2n+1)$.

\subsubsection{不可约表示}
\begin{example}[$(0, 0)$型]
    其维度为1, 并且其是Lorentz不变的. 因此我们指出$(0, 0)$型就是标量.
\end{example}
\begin{example}[$(\frac12,\frac12)$型]
    其维度为4, 并且我们有生成元
    \begin{equation}
        J^{+i}=J^{-i}=\frac{\sigma^i}2
    \end{equation}

    于是我们有
    \begin{equation}
        \vec J=\frac12\vec\sigma\otimes1+1\otimes\frac12\vec\sigma, \vec K=i\left(1\otimes\frac12\vec\sigma-\frac12\vec\sigma\otimes1\right)
    \end{equation}
    (千万不要直接将两个$\frac12\vec\sigma$相加, 因为它们是分别作用到不同的旋量部分的)

    考虑两旋量$\xi, \eta$, 我们用矩阵$\xi\eta^T\sigma^2$表示这两旋量的张量积.(为什么这里这么奇怪地在最后插入一个$\sigma^2$? 原因在于只有$\sigma^{T2}=-\sigma^2\neq\sigma^2$, 如果没有这个$\sigma^2$就会使得接下来的变换规则非常奇怪)

    而$\vec J, \vec K$对其的作用为
    \begin{align}
        & \vec J(\xi\eta^T)=\frac12(\sigma\xi\eta^T\sigma^2+\xi\eta^T\sigma^T\sigma^2)\\
        & \vec K(\xi\eta^T)=\frac i2(-\sigma\xi\eta^T\sigma^2+\xi\eta^T\sigma^T\sigma^2)
    \end{align}

    需要注意到, 这里出现了$\vec\sigma^T$, 而
    \begin{align}
       &\sigma^{iT}=\sigma^i, i=0,1,3\\
       &\sigma^{2T}=-\sigma^2
    \end{align}

    我们设
    \begin{equation}
        \xi\eta^T\sigma^2=V^\mu\bar\sigma_\mu
    \end{equation}
    (注意这里, 因为单纯$\xi\eta^T\sigma^2$的秩为$1$, 我们要表示任意的旋量其实需要多个$\xi\eta^T\sigma^2$做线性组合. 但是因为线性性使得对于单个$\xi\eta^T\sigma^2$成立的对于它们的线性组合式子也仍然成立, 所以这里出于书写简便性的考虑我们就不妨写一个$\xi\eta^T\sigma^2$来表示$V^\mu\bar\sigma_\mu$)

    计算可以发现
    \begin{align}
        J^2(V^\mu\bar\sigma_\mu)&=\frac12(\sigma^2V^\mu\bar\sigma_\mu+V^\mu\bar\sigma_\mu\sigma^2\sigma^{2T}\sigma^2)\\
        &=\frac12V^\mu(\sigma^2\bar\sigma_\mu-\bar\sigma_\mu\sigma^2)\\
        &=i\bar\sigma_j\epsilon_{j2i}V^i
    \end{align}
    并且对于$k\neq2$, 我们不难计算得到
    \begin{align}
        J^k(V^\mu\bar\sigma_\mu)=i\bar\sigma_j\epsilon_{jki}V^i
    \end{align}
    整理结果有, 对于$k=1,2,3$
    \begin{align}
        J^k(V^\mu\bar\sigma_\mu)=i\bar\sigma_j\epsilon_{jki}V^i
    \end{align}

    同理我们有
    \begin{equation}
        K^i(V^\mu\bar\sigma_\mu)=\frac i2(-\sigma^kV^\mu\bar\sigma_\mu+V^\mu\bar\sigma_\mu\sigma^2\sigma^{kT}\sigma^2)
    \end{equation}
    于是我们可以计算得到
    \begin{equation}
        K^k(V^\mu\bar\sigma_\mu)=-i(V^k\bar\sigma_0+V^0\bar\sigma_k)
    \end{equation}

    所以我们发现, 对于$V^\mu\bar\sigma_\mu$做无穷小Lorentz变换有
    \begin{align}
        \delta(V^\mu\bar\sigma_\mu)&=\Lambda_{\theta^i, \beta^i}(V^\mu\bar\sigma_\mu)-V^\mu\bar\sigma_\mu\\
        &=i\theta^kJ^k(V^\mu\bar\sigma_\mu)+i\beta^kK^k(V^\mu\bar\sigma_mu)\\
        &=-\bar\sigma_j\epsilon_{jki}\theta^kV^i+\beta^k(V^k\bar\sigma_0+V^0\bar\sigma_k)
    \end{align}

    对照\eqref{lorentz-for-vector}这符合Lorentz群作用下矢量的变换, 因此我们指出, $(\frac12, \frac12)$其实代表的就是4矢量.
\end{example}
\begin{example}[$(0,\frac12)$型-右手Weyl旋量]
    其维度为2. 于是我们可以将其记为$\psi_R$, 是一个二维列向量. 根据生成元
    \begin{align}
        \begin{cases}
            &\vec J^-=0\\
            &\vec J^+=\frac{\sigma}2
        \end{cases}
    \end{align}
    我们有得到Lorentz变换的群作用
    \begin{equation}
        \psi_R=\exp{\frac12(i\theta^i\sigma^i+\beta^i\sigma^i)}\psi_R        
    \end{equation}
    对无穷小Lorentz变换有
    \begin{equation}
        \delta\psi_R=\frac12(i\theta^j+\beta^j)\sigma^j\psi_R
    \end{equation}

    我们计算发现
    \begin{equation}
        \delta(\psi_R^\dagger\psi_R)=\beta^i\psi_R^\dagger\sigma^i\psi_R
    \end{equation}
    \begin{equation}
        \delta(\psi_R^\dagger\sigma^i\psi_R)=-\epsilon_{ijk}\theta^j\psi_R^\dagger\sigma^k\psi_R+\beta^i\psi_R^\dagger\psi_R
    \end{equation}
    如果我们将它们组合为$(\psi_R^\dagger\psi_R, \psi_R^\dagger\sigma^i\psi_R)^T$, 可以发现这正是矢量的Lorentz变换形式. 因此我们发现了一个4矢量
    \begin{equation}
        \psi_R^\dagger\sigma^\mu\psi_R
    \end{equation}, 其中
    \begin{equation}
        \sigma^0=\begin{pmatrix}
            1 & 0 \\
            0 & 1
        \end{pmatrix}
    \end{equation}

    从这里我们也可以体会到所谓"旋量是矢量开平方根"的说法的道理: 我们将两个旋量组合在一起, 就能乘出一个矢量.
\end{example}
\begin{example}[$(\frac12,0)$型-左手Weyl旋量]
    其维度为2. 于是我们可以将其记为$\psi_L$, 是一个二维列向量. 生成元为
    \begin{align}
        \begin{cases}
            &\vec J^-=\frac{\sigma}2\\
            &\vec J^+=0\
        \end{cases}
    \end{align}
    我们有得到Lorentz变换的群作用
    \begin{equation}
        \psi_L=\exp{\frac12(i\theta^i\sigma^i-\beta^i\sigma^i)}\psi_R        
    \end{equation}
    对无穷小Lorentz变换有
    \begin{equation}
        \delta\psi_L\frac12(i\theta^j-\beta^j)\sigma^j\psi_R
    \end{equation}

    我们定义$\bar\sigma^\mu=(1, -\sigma^i)^T$, 并计算发现
    \begin{equation}
        \delta(\psi_L^\dagger\psi_L)=-\beta^i\psi_L^\dagger\sigma^i\psi_L=\beta^i\psi_L^\dagger\bar\sigma^i\psi_L
    \end{equation}
    \begin{equation}
        \delta(\psi_L^\dagger\bar\sigma^i\psi_L)=-\sigma_{ijk}\theta^j\psi_L^\dagger\sigma^k\psi_L+\beta^i\psi_L^\dagger\psi_L
    \end{equation}

    于是我们发现
    \begin{equation}
        \psi_L^\dagger\bar\sigma^\mu\psi_L
    \end{equation}
    是一个4矢量
\end{example}

进一步地, 我们还可以发现
\begin{equation}
    \delta(\psi_L^\dagger\psi_R)=\delta(\psi_R^\dagger\psi_L)=0
\end{equation}
于是$\psi_L^\dagger\psi_R, \psi_R^\dagger\psi_L$是Lorentz标量.

再计算
\begin{equation}
    \delta(\psi^\dagger_R\sigma^\mu\partial_\mu\psi_R)
\end{equation}
根据
\begin{equation}
    \delta(\partial_\mu)=\partial_\mu'-\partial_\mu
\end{equation}
即
\begin{align}
    &\delta(\partial_0)=-\beta^i\partial_i\\
    &\delta(\partial_i)=-\beta^i\partial_0+\epsilon_{kji}\theta^j\partial_k
\end{align}
我们有
\begin{equation}
    \delta(\psi^\dagger_R\sigma^\mu\partial_\mu\psi_R)=0
\end{equation}
这个结论对于$\psi^\dagger_L\bar\sigma^\mu\partial_\mu\psi_L$同样成立.

所以我们发现
\begin{equation}
    \psi^\dagger_R\sigma^\mu\partial_\mu\psi_R, \psi^\dagger_L\bar\sigma^\mu\partial_\mu\psi_L
\end{equation}
是Lorentz标量.

\subsection{Dirac旋量}
利用上一节中我们组合出来的标量, 我们可以构造一个Lagrangian
\begin{equation}
    \mathcal L=i\psi^\dagger_R\sigma^\mu\partial_\mu\psi_R+i\psi^\dagger_R\sigma^\mu\partial_\mu\psi_R-m(\psi_R^\dagger\psi_L+\psi_L^\dagger\psi_R)
\end{equation}

我们将$\psi_L, \psi_R$拼到一起
\begin{equation}
    \psi=\begin{pmatrix}
        \psi_L\\
        \psi_R
    \end{pmatrix}
\end{equation}
然后构造
\begin{equation}
    \gamma^\mu=\begin{pmatrix}
        0 & \sigma^\mu\\
        \bar\sigma^\mu & 0
    \end{pmatrix}
\end{equation}
定义
\begin{equation}
    \bar\psi=\psi^\dagger\gamma^0=(\psi_R^\dagger,\psi_L^\dagger)
\end{equation}

就有
\begin{equation}
    \mathcal L=\bar\psi(i\slashed\partial-m)\psi
\end{equation}
其中
\begin{equation}
    \slashed\partial=\gamma^\mu\partial^\mu
\end{equation}

并且从矩阵形式我们注意到
\begin{equation}
    \{\gamma^\mu, \gamma^\nu\}=2g^{\mu\nu}
\end{equation}
以及
\begin{equation}\label{gamma-conj-idx}
    \gamma^{0\dagger}=\gamma^0, \gamma^{i\dagger}=-\gamma^i
\end{equation}
利用
\begin{equation}
    \gamma^0\gamma^0=g^{00}=1, \gamma^i\gamma0=-\gamma^0\gamma^i
\end{equation}
我们可以将式\eqref{gamma-conj-idx}写为更紧凑的形式
\begin{equation}
    \gamma^{\mu\dagger}=\gamma^0\gamma^\mu\gamma^0
\end{equation}

在下一节\ref{clifford}中我们将会看到一般化的对于$\gamma^\mu$性质的讨论.

接着我们继续考虑Dirac旋量, 我们可以从Lagrangian中得到EoM:
\begin{equation}
    (i\slashed\partial-m)\psi=0
\end{equation}

这是一个一阶的PDE, 似乎与预期中的Klein-Gordan方程不符? 我们可以将其左乘$(i\slashed\partial+m)$, 得到
\begin{equation}
    (i\slashed\partial+m)(i\slashed\partial-m)\psi=(-\partial^2-m^2)\psi=0
\end{equation}
于是我们发现, 将其解耦为二阶PDE后, 它仍然是满足Klein-Gordan方程的, 从而具有我们所预期的色散关系
\begin{equation}
    \omega^2=\vec p^2+m^2
\end{equation}

关于EoM的讨论我们见\ref{2ndq-dirac}节, 在那我们将会详细地讨论Dirac方程的解, 并将其二次量子化.

然后我们尝试获得Dirac旋量的Lorentz变换及其生成元. 我们首先直接考虑$\psi$的变换:
\begin{align}
    \delta\psi&=\begin{pmatrix}
        \delta\psi_L\\
        \delta\psi_R
    \end{pmatrix}=\frac i2\theta^i\begin{pmatrix}
        \sigma^i & 0\\
        0 & \sigma^i
    \end{pmatrix}\psi+\frac12v^i\begin{pmatrix}
        \bar\sigma^i & 0\\
        0 & \sigma^i
    \end{pmatrix}\psi
\end{align}
考虑到
\begin{equation}
    [\gamma^j, \gamma^k]=-[\sigma^j, \sigma^k]\begin{pmatrix}
        1 & 0\\
        0 & 1
    \end{pmatrix}=-i\epsilon_{jkl}\sigma^l\begin{pmatrix}
        1 & 0\\
        0 & 1
    \end{pmatrix}
\end{equation}
即
\begin{equation}
    \epsilon_{ijk}[\gamma^j, \gamma^k]=-2i\sigma^i\begin{pmatrix}
        1 & 0\\
        0 & 1
    \end{pmatrix}
\end{equation}
还有
\begin{align}
    \gamma^0\gamma^i=\begin{pmatrix}
        \bar\sigma^i & 0\\
        0 & \sigma^i
    \end{pmatrix}, \gamma^i\gamma^0=\begin{pmatrix}
        -\bar\sigma^i & 0\\
        0 & -\sigma^i
    \end{pmatrix}
\end{align}
即
\begin{equation}
    [\gamma^i, \gamma^0]=-2\begin{pmatrix}
        \bar\sigma^i & 0\\
        0 & \sigma^i
    \end{pmatrix}
\end{equation}
于是
\begin{align}
    \delta\psi&=i\epsilon_{ijk}\theta^i(\frac i4[\gamma^j, \gamma^k])\psi+iv^i(\frac i4[\gamma^i, \gamma^0])\\
    &=-\frac i2\omega_{\mu\nu}(\frac i4[\gamma^\mu, \gamma^\nu])
\end{align}

所以我们发现, 
\begin{definition}[Dirac旋量生成元]
    \begin{equation}
        S^{\mu\nu}=\frac i4[\gamma^\mu, \gamma^\nu]
    \end{equation}
\end{definition}
对Dirac旋量, Lorentz变换为
\begin{equation}
    \psi\to\Lambda_s\psi, \Lambda_s=\exp{-\frac i2\omega_{\mu\nu}S^{\mu\nu}}
\end{equation}
并且有旋转与Boost生成元
\begin{align}
    & J^{i}=\frac12\epsilon_{ijk}S^{jk}=\frac i8\epsilon_{ijk}[\gamma^j, \gamma^k]\\
    & K^i=S^{i0}=\frac i4[\gamma^i, \gamma^0]
\end{align}
\begin{equation}
    \psi\to\Lambda_s\psi, \Lambda_s=\exp{i\vec\theta\cdot\vec J+i\vec v\cdot\vec K}
\end{equation}

并且我们可以验证$S^{\mu\nu}$满足定理\ref{M-commutator}: 首先计算对易子
\begin{equation}\label{S-gamma-commutator}
    [S^{\mu\nu}, \gamma^\rho]=i(\gamma^\mu g^{\nu\rho}-\gamma^\nu g^{\mu\rho})
\end{equation}
所以
\begin{align}
    [S^{\mu\nu}, S_{\rho\sigma}]&=\frac i4\left([S^{\mu\nu}, \gamma^\rho\gamma^\sigma]-[S^{\mu\nu}, \gamma^\sigma\gamma^\rho]\right)\\
    &=\frac i4\left(\gamma^\rho[S^{\mu\nu}, \gamma^\sigma][S^{\mu\nu}, \gamma^\rho]\gamma^\sigma-\gamma^\sigma[S^{\mu\nu}, \gamma^\rho]-[S^{\mu\nu}, \gamma^\sigma]\gamma^\rho\right)
\end{align}
于是可得
\begin{equation}
    [S^{\mu\nu}, S^{\rho\sigma}]=i(-g^{\mu\rho}S^{\nu\sigma}-g^{\sigma\nu}S^{\mu\rho}+g^{\mu\sigma}S^{\nu\rho}+g^{\rho\nu}S^{\mu\sigma})
\end{equation}

\subsection{Clifford代数\label{clifford}}
上一节中, 我们定义出来的$\gamma^\mu$有其独特的代数结构, 本节我们将它抽象出来, 提升到代数的角度研究$\gamma^\mu$.
\begin{definition}[Clifford代数]
    Clifford代数即满足
    \begin{equation}
        \{\gamma^\mu, \gamma^\nu\}=2g^{\mu\nu}
    \end{equation}
    \begin{equation}
        \gamma^{\mu\dagger}=\gamma^0\gamma^\mu\gamma^0
    \end{equation}
    的代数.
\end{definition}
\begin{definition}[Slash]
    对于矢量$A^\mu$, 其slash
    \begin{equation}
        \slashed A\equiv \gamma^\mu A_\mu
    \end{equation}
\end{definition}

对于$\slashed A$我们有如下结论\cite{sredinicki-ugammau}
\begin{theorem}\label{gamma-p-contraction}
    \begin{equation}
        \gamma^\mu\slashed p=p^\mu-2iS^{\mu\nu}p_\nu
    \end{equation}
    \begin{equation}
        \slashed p\gamma^\mu=p^\mu+2iS^{\mu\nu}p_\nu
    \end{equation}
\end{theorem}
\begin{proof}
    \begin{align}
        \gamma^\mu\slashed p&=\frac12\{\gamma^\mu, \gamma^\nu\}p_\nu+\frac12[\gamma^\mu, \gamma^\nu]p_\nu\\
        &=p^\mu-2iS^{\mu\nu}p_\nu
    \end{align}
    \begin{align}
        \slashed p\gamma^\mu&=\frac12\{\gamma^\mu, \gamma^\nu\}p_\nu+\frac12[\gamma^\nu, \gamma^\mu]p_\nu\\
        &=p^\mu+2iS^{\mu\nu}p_\nu
    \end{align}
\end{proof}

\begin{theorem}
    我们有如下常见Clifford代数结论
    \begin{align}
        &\slashed p\slashed p=p^2\\
        &\gamma^\mu\gamma_\mu=4\\
        &\gamma^\mu\slashed p\gamma_\mu=-2\slashed p\\
        &\gamma^\mu\slashed p\slashed q\gamma_\mu=4pq
    \end{align}
\end{theorem}
\begin{proof}
    \begin{align}
        \slashed p\slashed p&=\gamma^\mu\gamma^\nu p_{\mu}p_\nu=\frac12(\gamma^\mu\gamma^\nu+\gamma^\nu\gamma^\mu)p_\mu p_\nu=g^{\mu\nu}p_{\mu}p_{\nu}=p^2\\
        \gamma_\mu\gamma^\mu&=\gamma_\mu\gamma_\nu g^{\mu\nu}=\frac12(\gamma_\mu\gamma_\nu+\gamma_\nu\gamma_\mu)g^{\mu\nu}=g_{\mu\nu}g^{\mu\nu}=4\\
        \gamma_\mu\slashed p\gamma^\mu&=\gamma^\mu\gamma^\alpha p_\alpha\gamma_\mu=(2g^{\mu\alpha}-\gamma^\alpha\gamma^\mu)p_\alpha\gamma_\mu=2\slashed p-4\slashed p=-2\slashed p\\
        \gamma_\mu\slashed p\slashed q\gamma^\mu&=\gamma^\mu\gamma^\alpha p_\alpha q^\beta\gamma_\beta\gamma_\mu=(2g^{\mu\alpha}-\gamma^\alpha\gamma^\mu)(2g_{\beta\mu}-\gamma_\mu\gamma_\beta)p_\alpha p^\beta\notag\\
        &=4pq-2\slashed p\slashed q-2\slashed q\slashed p+4\slashed p\slashed q=4pq
    \end{align}
\end{proof}

\begin{theorem}
    \begin{equation}
        \Lambda_s^{-1}\gamma^\mu\Lambda_s={\Lambda_v}^\mu_{~~\nu}\gamma^\nu
    \end{equation}
    其中$\Lambda_v$即$\omega$对应的对矢量的Lorentz变换矩阵.
\end{theorem}
\begin{proof}
    考虑无穷小Lorentz变换.
    \begin{equation}
        \Lambda_s=(1-\frac12i\omega_{\alpha\beta}S^{\alpha\beta}), \Lambda_s^{-1}=(1+\frac12i\omega_{\alpha\beta}S^{\alpha\beta})
    \end{equation}
    于是
    \begin{align}
        \Lambda_s^{-1}\gamma^\mu\Lambda_s&=(1+\frac i2\omega_{\alpha\beta}S^{\alpha\beta})\gamma^\mu(1-\frac i2\omega_{\alpha\beta}S^{\alpha\beta})\\
        &=\gamma^\mu+\frac i2\omega_{\alpha\beta}[S^{\alpha\beta}, \gamma^\mu]
    \end{align}

    利用式\eqref{S-gamma-commutator}我们有
    \begin{align}
        \Lambda_s^{-1}\gamma^\mu\Lambda_s&=\gamma^\mu-\frac 12\omega_{\alpha\beta}(\gamma^\alpha g^{\beta\mu}-\gamma^\beta g^{\alpha\mu})\\
        &=\gamma^\mu+\frac12\omega_{\beta\alpha}g^{\beta\mu}\gamma^\alpha+\frac12\theta_{\alpha\beta}g^{\alpha\mu}\gamma^\beta\\
        &=\gamma^\mu+\omega^\mu_{~~\nu}\gamma^\nu\\
        &=(\delta^\mu_{~~\nu}+\omega^\mu_{~~\nu})\gamma^\nu\\
        &=(\Lambda_v)^\mu_{~~\nu}\gamma^\nu
    \end{align}
    其中
    \begin{equation}
        (\Lambda_v)^\mu_{~~\nu}=\delta^\mu_{~~\nu}+\omega^\mu_{~~\nu}
    \end{equation}
    即矢量的Lorentz变换.\\
\end{proof}
\begin{theorem}
    \begin{equation}
        \gamma^0\Lambda_s^\dagger\gamma^0=\Lambda_s^{-1}
    \end{equation}
\end{theorem}
\begin{proof}
    我们求生成元的共轭
    \begin{align}
        S^{\mu\nu\dagger}&=-\frac i4(\gamma^\mu\gamma^\nu-\gamma^\nu\gamma^\mu)^\dagger\\
        &=-\frac i4(\gamma^{\nu\dagger}\gamma^{\mu\dagger}-\gamma^{\mu\dagger}\gamma^{\nu\dagger})\\
        &=\frac i4(\gamma^0\gamma^\mu\gamma^0\gamma^0\gamma^\nu\gamma^0-\gamma^0\gamma^\nu\gamma^0\gamma^0\gamma^\mu\gamma^0)\\
        &=\gamma^0S^{\mu\nu}\gamma^0
    \end{align}

    然后由于$\gamma^0\gamma^0=1$, 所以$\gamma^0$可以从$\Lambda_s^\dagger$的两边拎入指数中:
    \begin{equation}
        \gamma^0\Lambda_s^\dagger\gamma^0=\gamma^0\exp{\frac i2\omega_{\mu\nu}S^{\mu\nu\dagger}}\gamma^0=\exp{\frac i2\omega_{\mu\nu}\gamma^0S^{\mu\nu\dagger}\gamma^0}=\exp{\frac i2\omega_{\mu\nu}S^{\mu\nu}}=\Lambda_s^{-1}
    \end{equation}
\end{proof}

\begin{theorem}
    $\bar\psi\psi$是Lorentz标量, $\bar\psi\gamma^\mu\psi$是Loretnz矢量, $\bar\psi\gamma^\mu\gamma^\nu\psi$是Lorentz张量.
\end{theorem}
\begin{proof}
    \begin{align}
        \bar\psi\psi=\psi^\dagger\gamma^0\psi\to \psi^\dagger\Lambda_s^\dagger\gamma^0\Lambda_s\psi=\psi^\dagger\gamma^0\Lambda_s^{-1}\Lambda_s\psi=\psi^\dagger\gamma^0\psi=\bar\psi\psi
    \end{align}
    \begin{align}
        \bar\psi\gamma^\mu\psi=\psi^\dagger\gamma^0\gamma^\mu\psi&\to\psi^\dagger\Lambda_s^\dagger\gamma^0\gamma^\mu\Lambda_s\psi=\psi^\dagger\Lambda_s^{-1}\gamma^\mu\Lambda_s\psi\notag\\
        &=\psi^\dagger\gamma^0(\Lambda_v)^\mu_{~~\nu}\psi=(\Lambda_v)^\mu_{~~\nu}\gamma^\nu\bar\psi\gamma^\mu\psi
    \end{align}
    \begin{align}
        \bar\psi\gamma^\mu\gamma^\nu\psi=\psi^\dagger\gamma^0\gamma^\mu\gamma^\nu\psi&\to\psi^\dagger\Lambda_s^\dagger\gamma^0\gamma^\mu\gamma^\nu\Lambda_s\psi=\psi^\dagger\Lambda_s^{-1}\gamma^\mu\Lambda_s\Lambda_s^{-1}\gamma^\nu\Lambda_s\psi\notag\\
        &=\psi^\dagger\gamma^0(\Lambda_v)^\mu_{~~\alpha}\gamma^\alpha(\Lambda_v)^\nu_{~~\beta}\gamma^\beta\psi=(\Lambda_v)^\mu_{~~\alpha}(\Lambda_v)^\nu_{~~\beta}\bar\psi\gamma^\alpha\gamma^\beta\psi
    \end{align}
\end{proof}

关于$\gamma$矩阵乘积的迹, 我们还有如下定理\cite{griffthsClifford}
\begin{theorem}[$\gamma$乘积迹定理]
    对于奇数个$\gamma$乘积
    \begin{equation}
        \rm{Tr}(\gamma^{\mu_1}\gamma^{\mu_2}\cdots\gamma^{\mu_{2n-1}})=0
    \end{equation}
    对于偶数个$\gamma$乘积, 我们有
    \begin{align}
        &\rm{Tr}(\gamma^\mu\gamma^\nu)=4g^{\mu\nu}\\
        &\rm{Tr}(\gamma^\mu\gamma^\nu\gamma^\lambda\gamma^\sigma)=4(g^{\mu\nu}g^{\lambda\sigma}+g^{\mu\sigma}g^{\lambda\nu}-g^{\mu\lambda}g^{\nu\sigma})\\
        &\cdots\notag
    \end{align}
\end{theorem}

特别地, 如果$\gamma$乘积后最外侧的左右两个$\gamma$指标缩并, 即如$\gamma^\mu\gamma^\nu\gamma_\mu$, 我们有
\begin{theorem}[$\gamma$缩并定理]
    \begin{align}
        &\gamma_\mu\gamma^\mu=4\\
        &\gamma_\mu\gamma^\nu\gamma^\mu=-2\gamma^\nu\\
        &\gamma_\mu\gamma^\nu\gamma^\nu\gamma^\mu=4g^{\nu\lambda}\\
        &\gamma_\mu\gamma^{\nu}\gamma^\lambda\gamma^\sigma\gamma^\mu=-2\gamma^\sigma\gamma^\lambda\gamma^\nu\\
        &\cdots\notag
    \end{align}
\end{theorem}

\subsection{Dirac旋量的二次量子化\label{2ndq-dirac}}
\subsubsection{旋量运动方程}
首先是第一步, 解EoM. 设在Weyl基底下一个一般的解为$(\psi_L \psi_R)^T$, 于是EoM可以写为
\begin{equation}
    \begin{pmatrix}
        -m & i\sigma^\mu\partial_\mu\\
        i\bar\sigma^\mu\partial_\mu & -m
    \end{pmatrix}\begin{pmatrix}
        \psi_L\\
        \psi_R
    \end{pmatrix}=0
\end{equation}

在动量空间中有
\begin{align}
    &\sigma^\mu p_\mu\psi_R=(E-\sigma\cdot\vec p)\psi_R=m\psi_L\\
    &\bar\sigma^\mu p_\mu\psi_L=(E+\sigma\cdot\vec p)\psi_L=m\psi_R
\end{align}

对于零质量Fermion, 这个方程是解耦的:
\begin{align}
    &\sigma^\mu p_\mu\psi_R=(E-\vec\sigma\cdot\vec p)\psi_R=0\\
    &\bar\sigma^\mu p_\mu\psi_L=(E+\vec\sigma\cdot\vec p)\psi_L=0
\end{align}

\begin{definition}[螺旋度]
    \begin{equation}
        H=\frac{\vec\sigma\cdot\vec p}{|p|}
    \end{equation}
\end{definition}

可以发现
\begin{align}
    & H\psi_R=\psi_R\\
    & H\psi_L=-\psi_L\\
\end{align}

可见, $\psi_R, \psi_L$分别是螺旋度的本征矢. 这里螺旋度的物理意义就是, 自旋指向和运动方向的夹角. 在这里我们发现, Weyl旋量的左右手其实分别就是自旋方向和运动方向分别是左手螺旋和右手螺旋的关系.

我们一般认为中微子的$m\approx0$, 所以中微子是具有固定的手性. 由于我们世界Parity的破缺, 导致自然界中其实基本上只存在左手中微子.

\subsubsection{旋量的极化}
一般性的讨论结束, 我们开始从EoM中寻求可以拿来正则量子化的解. 设正能解$\psi=u^s\exp{-ipx}$, 负能解$\psi=v^s\exp{ipx}$

对于$u^s$
\begin{equation}
    (\slashed p-m)u^s=0
\end{equation}
在Weyl基底中即
\begin{equation}
    \begin{pmatrix}
        -m & p^\mu\sigma_\mu\\
        p^\mu\bar\sigma_\mu & -m
    \end{pmatrix}u^s=0
\end{equation}

根据
\begin{align}
    (\vec p\cdot\vec\sigma)^2&=p^ip^j\sigma^i\sigma^j=p^ip^j(\delta_{ij}+i\epsilon_{ijk}\sigma^k)=p^ip^i=\vec p^2
\end{align}
然后
\begin{align}
    (p\cdot\sigma)(p\cdot\bar\sigma)&=(p^0\sigma^0-\vec p\cdot\vec\sigma)(p^0\sigma^0+\vec p\cdot\vec\sigma)\\
    &=(p^0)^2-(\vec p\cdot\sigma)^2=(p^0)^2-(\vec p)^2=p^2=m^2
\end{align}
即(都是算数平方根, 取正根, 这里没有负根的情况)
\begin{equation}
    \sqrt{(p\cdot\sigma)(p\cdot\bar\sigma)}=\sqrt{m^2}=m
\end{equation}
我们可以计算验证
\begin{align}
    \begin{pmatrix}
        -m & p^\mu\sigma_\mu\\
        p^\mu\bar\sigma_\mu & -m
    \end{pmatrix}\begin{pmatrix}
        \sqrt{p\sigma}\zeta_s\\
        \sqrt{p\bar\sigma}\zeta_s
    \end{pmatrix}&=\begin{pmatrix}
        -m\sqrt{p\sigma}\zeta_s+p\sigma\sqrt{p\bar\sigma}\zeta_s\\
        p\bar\sigma\sqrt{p\sigma}\zeta_s-m\sqrt{p\bar\sigma}\zeta_s
    \end{pmatrix}\\
    &=\begin{pmatrix}
        \sqrt{p\sigma}\left(-m+\sqrt{(p\bar\sigma)(p\sigma)}\right)\zeta_s\\
        \sqrt{p\bar\sigma}\left(\sqrt{(p\sigma)(p\bar\sigma)}-m\right)\zeta_s
    \end{pmatrix}=0
\end{align}

所以
\begin{equation}
    u^s=\begin{pmatrix}
        \sqrt{p\sigma}\zeta_s\\
        \sqrt{p\bar\sigma}\zeta_s
    \end{pmatrix}
\end{equation}
是EoM的解.

对于$v^s$同理, 它满足
\begin{equation}
    (\slashed p+m)v^s=0
\end{equation}

可以验证
\begin{equation}
    v^s=\begin{pmatrix}
        \sqrt{p\sigma}\eta_s\\
        -\sqrt{p\bar\sigma}\eta_s
    \end{pmatrix}
\end{equation}
是满足EoM的解.

然后我们取正交基底张成$u^s, v^s$的解空间, 即我们取$\zeta_1, \zeta_2$以及$\eta_1, \eta_2$满足
\begin{equation}
    \zeta_r^\dagger\zeta_s=\delta_{rs}, \eta_r^\dagger\eta_s=\delta_{rs}
\end{equation}
然后取$\zeta_s, \eta_s$分别代入$u^s, v^s$得到$u^s, v^s$的解空间的基底.

我们可以验证正交性
\begin{equation}
    \bar u^ru^s=\begin{pmatrix}
        \sqrt{p\bar\sigma}\zeta_r^\dagger & \sqrt{p\sigma}\zeta_r^\dagger
    \end{pmatrix}\begin{pmatrix}
        \sqrt{p\sigma}\zeta_s\\
        \sqrt{p\bar\sigma}\zeta_s
    \end{pmatrix}=m\delta^{rs}+m\delta^{rs}=2m\delta^{rs}
\end{equation}
\begin{equation}
    \bar v^rv^s=\begin{pmatrix}
        -\sqrt{p\bar\sigma}\eta_r^\dagger & \sqrt{p\sigma}\eta_r^\dagger
    \end{pmatrix}\begin{pmatrix}
        \sqrt{p\sigma}\eta_s\\
        -\sqrt{p\bar\sigma}\eta_s
    \end{pmatrix}=-m\delta^{rs}-m\delta^{rs}=-2m\delta^{rs}
\end{equation}
\begin{equation}
    \bar u^rv^s=\begin{pmatrix}
        \sqrt{p\bar\sigma}\zeta_r^\dagger & \sqrt{p\sigma}\zeta_r^\dagger
    \end{pmatrix}\begin{pmatrix}
        \sqrt{p\sigma}\eta_s\\
        -\sqrt{p\bar\sigma}\eta_s
    \end{pmatrix}=0
\end{equation}

于是
\begin{theorem}
    \begin{equation}
        \bar u^ru^s=-\bar v^rv^s=2m\delta_{rs}, \bar u^rv^s=\bar v^ru^s=0
    \end{equation}
\end{theorem}

\begin{theorem}
    \begin{equation}
        \sum_s u^s\bar u^s=\slashed p+m
    \end{equation}
    \begin{equation}
        \sum_s v^s\bar v^s=\slashed p-m
    \end{equation}
\end{theorem}
\begin{proof}
    根据$\zeta_1, \zeta_2$是$\mathbb{C}^2$上的完备正交基底, 所以
    \begin{equation}
        \sum_s \zeta_s\zeta_s^\dagger=1
    \end{equation}

    于是
    \begin{align}
        \sum_s u^s\bar u^s&=\sum_s\begin{pmatrix}
            \sqrt{p\sigma}\zeta_s\\
            \sqrt{p\bar\sigma}\zeta_s
        \end{pmatrix}\begin{pmatrix}
            \sqrt{p\bar\sigma}\zeta_s^\dagger & \sqrt{p\sigma}\zeta_s^\dagger
        \end{pmatrix}\notag\\
        &=\sum_s\begin{pmatrix}
            \sqrt{(p\sigma)(p\bar\sigma)}\zeta_s\zeta_s^\dagger & p\sigma\zeta_s\zeta_s^\dagger\\
            p\bar\sigma\zeta_s\zeta_s^\dagger & \sqrt{(p\sigma)(p\bar\sigma)}\zeta_s\zeta_s^\dagger
        \end{pmatrix}\notag\\
        &=\begin{pmatrix}
            m & p\sigma\\
            p\bar\sigma & m
        \end{pmatrix}=p\gamma+m=\slashed p+m
    \end{align}
    \begin{align}
        \sum_s v^s\bar v^s&=\sum_s\begin{pmatrix}
            \sqrt{p\sigma}\eta_s\\
            -\sqrt{p\bar\sigma}\eta_s
        \end{pmatrix}\begin{pmatrix}
            -\sqrt{p\bar\sigma}\eta_s^\dagger & \sqrt{p\sigma}\eta_s^\dagger
        \end{pmatrix}\notag\\
        &=\sum_s\begin{pmatrix}
            -\sqrt{(p\sigma)(p\bar\sigma)}\eta_s\eta_s^\dagger & p\sigma\eta_s\eta_s^\dagger\\
            p\bar\sigma\eta_s\eta_s^\dagger & -\sqrt{(p\sigma)(p\bar\sigma)}\eta_s\eta_s^\dagger
        \end{pmatrix}\notag\\
        &=\begin{pmatrix}
            -m & p\sigma\\
            p\bar\sigma & -m
        \end{pmatrix}=p\gamma-m=\slashed p-m
    \end{align}
\end{proof}

\begin{theorem}\label{ugammau-contraction}
    对于$u_s, v_s$我们有\cite{sredinicki-ugammau}:
    \begin{equation}
        2m\bar u_{s'}(\vec p')\gamma^\mu u_s(\vec p)=\bar u_{s'}(\vec p')\left[(p'+p)^\mu-2iS^{\mu\nu}(p'-p)_\nu\right]u_s(\vec p)
    \end{equation}
    \begin{equation}
        -2m\bar v_{s'}(\vec p')\gamma^\mu v_s(\vec p)=\bar v_{s'}(\vec p')\left[(p'+p)^\mu-2iS^{\mu\nu}(p'-p)_\nu\right]v_s(\vec p)
    \end{equation}
    \begin{equation}
        2m\bar u_{s'}(\vec p')\gamma^\mu v_s(\vec p)=\bar u_{s'}(\vec p')\left[(p'-p)^\mu-2iS^{\mu\nu}(p'+p)_\nu\right]v_s(\vec p)
    \end{equation}
    \begin{equation}
        -2m\bar v_{s'}(\vec p')\gamma^\mu u_s(\vec p)=\bar v_{s'}(\vec p')\left[(p'-p)^\mu-2iS^{\mu\nu}(p'+p)_\nu\right]u_s(\vec p)
    \end{equation}
\end{theorem}
\begin{proof}
    \begin{align}
        2m\bar u_{s'}(\vec p')\gamma^\mu u_s(\vec p)&=\bar u_{s'}(\vec p')\slashed p'\gamma^\mu u_s(\vec p)+\bar u_{s'}(\vec p')\gamma^\mu\slashed pu_s(\vec p)\\
        &=\bar u_{s'}(\vec p')\left(\slashed p'\gamma^\mu+\gamma^\mu\slashed p\right)u_s(\vec p)\\
        &=\bar u_{s'}(\vec p')\left[(p'+p)^\mu-2iS^{\mu\nu}(p'-p)_\nu\right]u_s(\vec p)
    \end{align}
    其中, 第三个等号利用定理\ref{gamma-p-contraction}. 而对于后三个等式的证明同理, 在此不赘述.
\end{proof}

对定理\ref{ugammau-contraction}取$p'=p$, 就有
\begin{equation}
    \bar u^s(\vec p)\gamma^\mu u_{s'}(\vec p)=2p^\mu\delta_{ss'}\label{gamma-p-p-contraction}
\end{equation}
\begin{equation}
    \bar v^s(\vec p)\gamma^\mu v_{s'}(\vec p)=2p^\mu\delta_{ss'}
\end{equation}
不过需要注意$\bar u^s(\vec p)\gamma^\mu v_{s'}(\vec p)\neq0$:
\begin{equation}
    2m\bar u^s(\vec p)\gamma^\mu v_{s'}(\vec p)=\bar u^s\slashed p\gamma^\mu v_{s'}-\bar u^s\gamma^\mu\slashed pv_{s'}=\bar u^sp_\nu[\gamma^\nu, \gamma^\mu]v_{s'}\neq0
\end{equation}
. 但是对于$\vec p'=-\vec p$, 我们根据此定理有推论
\begin{theorem}\label{ubar-gamma-v}
    \begin{equation}
        \bar u_{s'}(\vec p)\gamma^0 v_s(-\vec p)=\bar v_{s'}(\vec p)\gamma^0 u_s(-\vec p)=0
    \end{equation}
\end{theorem}

\subsubsection{反对易的二次量子化与关联函数}
关于$u, v$性质的讨论告一段落. 接下来由于涉及到对易子的问题, 为了能够区分对易子的乘法顺序以及做乘法的方式(内积还是外积), 我们对$\gamma, \psi$引入指标:
\begin{align}
    &\psi\to\psi_A\\
    &\bar\psi\to\bar\psi^A\\
    &\gamma^\mu\to\gamma^{\mu~~B}_{~~A}\\
    &\slashed p\to\slashed p_A^{~~B}
\end{align}

那么
\begin{align}
    &\bar\psi\psi\to\bar\psi^A\psi_A=\psi_A\bar\psi^A\\
    &\psi\bar\psi\to\psi_A\bar\psi^B=\bar\psi^B\psi_A
\end{align}

首先对$\mathcal L$做Legdren变换, 注意到$\mathcal L$中只有$\partial_0\psi$而没有$\partial_0\psi^\dagger$, 因此我们只需要定义
\begin{equation}
    \pi=\pa{\mathcal L}{\partial_0\psi}=i\bar\psi\gamma^0=i\psi^\dagger
\end{equation}
而不需要定义共轭动量(注意$\pi^{(\dagger)}$是指$\psi^\dagger$的正则动量, 而不是正则动量$\pi$的共轭, 即$\pi^{(\dagger)}\neq\pi^\dagger$)
\begin{equation}
    \pi^{(\dagger)}=\pa{\mathcal L}{\partial_0\psi^\dagger}=0
\end{equation}
, 从而
\begin{align}
    \mathcal H&=\pi\partial_0\psi-\mathcal L=\bar\psi(m-i\gamma^i\partial_i)\psi.
\end{align}

引入$\a p, \b p$, 我们可以将Dirac场量子化
\begin{align}
    &\psi_A=\int\ldsq{p}\sum_s(u^s_A\a{p}^s\exp{-ipx}+v^{s}_A\b{p}^{s\dagger}\exp{ipx})\\
    &\bar\psi^A=\int\ldsq{p}\sum_s(\bar u^{sA}\a{p}^{s\dagger}\exp{ipx}+\bar v^{sA}\b{p}^{s}\exp{-ipx}).
\end{align}

我们可以计算得到
\begin{align}
    m\int\d^3x\bar\psi_x\psi_x&=\int\d^3x\int\ldsq p\sum_s\left(\bar u^s\a p^{s\dagger}\exp{ipx}+\bar v^s\b p^s\exp{-ipx}\right)\notag\\
    &\;\;\int\ldsq q\sum_{s'}\left(u^{s'}\a q^{s'}\exp{-iqx}+v^{s'}\b q^{s'\dagger}\exp{iqx}\right)\\
    &=\int\ddd p\frac{m^2}{\om p}\sum_s\left(\a p^\dagger\a p-\b p\b p^\dagger\right)
\end{align}
以及
\begin{align}
    \int\d^3x\bar\psi(-i\gamma^i\partial_i)\psi&=\int\d^3x\ldsq q\sum_s\left(\bar u^s\a q^{s\dagger}\exp{iqx}+\bar v^s\b q^s\exp{-iqx}\right)\notag\\
    &\;\;\int\ldsq p[\gamma^ip_i]\sum_{s'}\left(-u^{s'}\a p\exp{-ipx}+v^{s'}\b p^{s'\dagger}\exp{ipx}\right)\\
    &=\int\ld p\sum_{ss'}\left(-\bar u^s\gamma^ip_i u^{s'}\a p^{s\dagger}\a p^{s'}+\bar v^s\gamma^ip_iv^{s'}\b p^s\b p^{s'\dagger}\right)
\end{align}
利用式\eqref{gamma-p-p-contraction}, 我们有
\begin{align}
    \bar u^s\gamma^ip_i u^{s'}&=-2\vec p^2\delta_{ss'}\\
    \bar v^s\gamma^ip_i v^{s'}&=-2\vec p^2\delta_{ss'}
\end{align}
从而有
\begin{align}
    \int\d^3x\bar\psi(-i\gamma^i\partial_i)\psi&=\int\ddd p\frac{\vec p^2}{\om p}\sum_s\left(\a p^\dagger\a p-\b p\b p^\dagger\right)
\end{align}
利用这两式, 我们可以得到Hamiltonian
\begin{equation}
    H=\int\d^3\mathcal H=\int\ddd p\om p\sum_s\left(\a p^\dagger\a p-\b p\b p^\dagger\right)
\end{equation}

类似地, 我们利用正则对易关系, $\b p, \b p^\dagger$给对易过来, 似乎就可以得到最终结果了...?
\begin{equation}
    H=\int\d^3\mathcal H=\int\ddd p\om p\sum_s\left(\a p^\dagger\a p-\b p^\dagger\b p\right)
\end{equation}
但是这个式子有个巨大的问题: 非正定! 也就是说, 如果$\b p^\dagger$激发产生一个反粒子, 那么它的能量是负的! 而大自然会倾向于低能量的状态, 也就是这会导致Dirac旋量场会自发放出无穷大的能量! 这个在物理上显然是荒谬的. 而我们可以确信我们上述的计算过程是没有问题的, 那么只有两个地方的假设可能是错的: 1. Dirac场能用极化旋量和$\a p, \b p$表示. 2. 正则对易关系.

而第一个假设如果放弃, 我们会失去一切. 因此我们的首选做法是放弃第二个假设. 注意到, 如果我们引入反对易关系:
\begin{align}
    & \{\a p, \a q^\dagger\}=\dpi3\delta^3(\vec p-\vec q), \{\a b, \b q^\dagger\}=\dpi3\delta^3(\vec p-\vec q)\\
    & \{\a p, \a q\}=\{\a p^\dagger, \a q^\dagger\}=\{\b p, \b q\}=\{\b p^\dagger, \b q^\dagger\}=0
\end{align}

则Hamiltonian可以对易为
\begin{equation}
    H=\int\d^3\mathcal H=\int\ddd p\om p\sum_s\left(\a p^\dagger\a p+\b p^\dagger\b p+\mathcal V\right)
\end{equation}
这样就可以得到正确的结果.

反对易关系的引入, 还导致了在统计上与对易关系的玻色子的重大区别: Pauli不相容原理, 即
\begin{equation}
    \a p^\dagger\a p^\dagger\ket0=0.
\end{equation}
我们不能让两个电子处于完全一样的态, 这样的粒子称为费米子, 遵从Fermi-Dirac分布, 而满足对易关系的粒子称为玻色子, 遵从Bose-Einstein分布. 接下来我们还将从关联函数协变性的角度更深刻地看到费米子反对易性的必要性.

然后我们尝试计算两点关联函数, 首先有
\begin{align}
    \braket{0|\psi_A(y)\bar\psi^B(x)|0}&=\int\ddd{p}\frac{\slashed p_A^{~~B}+m}{2\om p}\exp{ip(x-y)}
\end{align}
\begin{align}
    \braket{0|\bar\psi^B(x)\psi_A(y)|0}&=\int\ddd{p}\frac{\slashed p_A^{~~B}-m}{2\om p}\exp{-ip(x-y)}
\end{align}

考虑Fermion的反交换性, 我们很自然地要求
\begin{equation}
    \mathcal T\left\{\psi_x\bar\psi_y\right\}=-\mathcal T\left\{\bar\psi_y\psi_x\right\}
\end{equation}
这需要我们定义对Dirac场的编时算符为
\begin{equation}
    \mathcal T\left\{\psi_A(y)\bar\psi^B(x)\right\}\equiv\Theta(y^0-x^0)\psi_A(y)\bar\psi^B(x)-\Theta(x^0-y^0)\bar\psi^B(x)\psi_A(y)
\end{equation}

这样我们经过类似\ref{2pt-real-scalar}节后半段的方法计算可以得到
\begin{align}
    \Theta(y^0-x^0)\psi_A(y)\bar\psi^B(x)&=\int\frac{i\d\omega}{2\pi}\frac{\exp{i\omega(t_x-t_y)}}{\omega+i\epsilon}\int\ld p[\slashed p+m]\exp{ip(x-y)}\\
    &=\int i\dddd p\frac{\slashed p-(p^0-\om p)\gamma^0+m}{2\om p(p^0-\om p+i\epsilon)}\exp{ip(x-y)}\label{dirac-spinor-2pt1}\\
    \mathcal T\left\{\bar\psi_y\psi_x\right\}&=\int i\dddd p\frac{-\slashed p+(p^0+\om p)\gamma^0-m}{2\om p(-p^0-\om p+i\epsilon)}\exp{ip(x-y)}\label{dirac-spinor-2pt2}
\end{align}

将\eqref{dirac-spinor-2pt1}, \eqref{dirac-spinor-2pt2}两式相减即可得Feynman传播子
\begin{equation}
    \braket{\psi_A(y)\bar\psi^B(x)}=\int\dddd p\frac{i(\slashed p+m)}{p^2-m^2+i\epsilon}\exp{ip(x-y)}
\end{equation}
不难看出这是协变的.

\kaishu
而如果我们仍然强加对易关系到编时算符上, 即定义
\begin{equation}
    \mathcal T\left\{\psi_A(y)\bar\psi^B(x)\right\}\equiv\Theta(y^0-x^0)\psi_A(y)\bar\psi^B(x)+\Theta(x^0-y^0)\bar\psi^B(x)\psi_A(y)
\end{equation}
则需要将\eqref{dirac-spinor-2pt1}, \eqref{dirac-spinor-2pt2}两式相加, 得到
\begin{equation}
    \braket{\psi_A(y)\bar\psi^B(x)}=\int i\dddd p\frac{\omega}{\om p}\frac{\slashed p-\frac{p^2-m^2}{\omega}\gamma^0+m}{p^2-m^2+i\epsilon}
\end{equation}
这又丑又完全不协变, 是一个灾难性的结果.
\songti

\subsection{Majorana旋量}

\newpage
\section{离散变换}
在本节我们尝试讨论对场的三种幺正离散变换, 即$C,P,T$变换. 所谓的$C$, 其实就是电荷共轭(Charge Conjugate), 就是指将粒子变为反粒子, 反转其的所有量子荷(比如电荷, 但还包括轻子数、重子数等所有性质). 而$P$就是大名鼎鼎的宇称(Parity), 就是指$\vec x\to-\vec x$的镜像变换. $T$则是时间反演(Time reversal), 也就是将对象的运动过程反演: $t\to-t$. 

\subsection{C变换}
$C$变换就是取反粒子, 若我们有一个粒子态$\ket p$, 我们将$C$作用在它上面, 根据要求我们有:
\begin{equation}
    C\ket p=\ket{\bar p}
\end{equation}
其中$\ket{\bar p}$即$\ket p$的反粒子态.

而这也就是说
\begin{align}
    & C\b{p}^\dagger\ket0=\a{p}^\dagger\ket 0\\
    & C\a{p}^\dagger\ket0=\b{p}^\dagger\ket 0
\end{align}

在$\b{p}^\dagger, \a{p}^\dagger$与$\ket 0$间插入$C^\dagger c$, 我们有
\begin{align}
    & C\b{p}^\dagger C^\dagger C\ket0=\a{p}^\dagger\ket 0\\
    & C\a{p}^\dagger C^\dagger C\ket0=\b{p}^\dagger\ket 0
\end{align}

这暗示我们
\begin{align}
    & C\b{p}^\dagger C^\dagger=\a{p}^\dagger\\
    & C\a{p}^\dagger C^\dagger=\b{p}^\dagger
\end{align}

然后我们以实标量场、复标量场、Dirac旋量场为例讨论$C$的具体作用.
\begin{example}[实标量场的电荷共轭]
    由于实标量场只有一组产生湮灭算符, 它自己就是自己的反粒子, $C$变换对其不起作用, 于是我们有
    \begin{eqnarray}
        C=1
    \end{eqnarray}
\end{example}
\begin{example}[复标量场的电荷共轭]
    直接套用结论我们有
    \begin{align}
        &C\psi(x)C^\dagger=\int\ldsq p\left(\a p\exp{-ipx}+\b p^\dagger\exp{ipx}\right)=\psi^\dagger(x)\\
        &C\pi(x)C^\dagger=\int\ddd pi\sqrt{\frac{\om p}2}\left(\b p^\dagger\exp{ipx}-\a p\exp{-ipx}\right)=\pi^\dagger(x)
    \end{align}
\end{example}
\begin{example}[Dirac旋量的电荷共轭]
    直接利用结论我们有
    \begin{align}
        &C\psi_AC^\dagger=\int\ldsq{p}\sum_s(u^s_A\b{p}^s\exp{-ipx}+v^{s}_A\a{p}^{s\dagger}\exp{ipx})
    \end{align}
    将其与
    \begin{equation}
        \psi_A=\int\ldsq{p}\sum_s(u^s_A\a{p}^s\exp{-ipx}+v^{s}_A\b{p}^{s\dagger}\exp{ipx})\\
    \end{equation}
    对比, 我们发现, 这暗示着
    \begin{equation}
        \psi_C=C\psi C^\dagger=\boldsymbol{\mathrm C}\psi^*=\int\ldsq{p}\sum_s(\boldsymbol{\mathrm C}v^{s*}\b{p}^{s}\exp{-ipx}+\boldsymbol{\mathrm C}u^{s*}\a{p}^{s\dagger}\exp{ipx})
    \end{equation}
    其中$\boldsymbol{\mathrm C}$为一个矩阵. 这也就是说
    \begin{align}
        \boldsymbol{\mathrm C}v^{s*}=u^s\\
        \boldsymbol{\mathrm C}u^{s*}=v^s\label{uv-conguate-star}.
    \end{align}
    而这意味着
    \begin{align}
        (\slashed p-m)\boldsymbol{\mathrm C}v^{s*}=0\\
        (\slashed p+m)\boldsymbol{\mathrm C}u^{s*}=0.
    \end{align}
    又因为
    \begin{align}
        m\boldsymbol{\mathrm C}u^{s*}&=\boldsymbol{\mathrm C}(mu)^{s*}\\
        &=p_\mu\boldsymbol{\mathrm C}\gamma^{\mu*} u^{s*}
    \end{align}
    同时根据式\eqref{uv-conguate-star}, 我们有
    \begin{align}
        m\boldsymbol{\mathrm C}u^{s*}=mv^s=-p_\mu\gamma^\mu v^s=-p_\mu\gamma^\mu\boldsymbol{\mathrm C}u^{s*}
    \end{align}
    所以有
    \begin{equation}
        \boldsymbol{\mathrm C}\gamma^\mu\boldsymbol{\mathrm C}^\dagger=-\gamma^{\mu*}.
    \end{equation}

    又注意到, 
    \begin{align}
        \gamma^{\mu*}=\gamma^\mu, \mu=0,1,3\\
        \gamma^{2*}=-\gamma^2.
    \end{align}
    所以我们不难猜想, $\boldsymbol{\mathrm C}$应当与$\gamma^2$有关. 不难验证,
    \begin{equation}
        \boldsymbol{\mathrm C}=i\gamma^2
    \end{equation}
    是满足这一要求的电荷共轭矩阵.

    % 我们想要将其写为与$\psi^*$(注意, 这里如果是$\psi^\dagger$那么就变成行向量了, 结构不同, 而$\psi^*$则只是取共轭, 也就是将$a, b$取$\dagger$, 给$u, v$取$*$却不取$\dagger$)有关的东西, 那么我们就需要获得$u$与$v$之间的关系.
\end{example}

\subsection{P变换}


\subsection{T变换}

\newpage
\section{旋量QED}
然后我们试图将Dirac旋量与电磁场耦合, 考虑Dirac旋量与电磁场的相互作用, 即标准的量子电动力学(QED).
\subsection{最小耦合}
同样考虑局部$U(1)$规范变换, 即
\begin{equation}
    \psi\to\exp{-i\alpha}\psi
\end{equation}
设
\begin{equation}
    \partial_\mu\to D_\mu=\partial_\mu+ieA_\mu
\end{equation}
让局域$U(1)$规范变换对$A_\mu$满足
\begin{equation}
    A_\mu\to A_\mu+\frac1e\partial_\mu\alpha
\end{equation}
则
\begin{equation}
    D_\mu(\exp{-i\alpha}\psi)=\exp{-i\alpha}D_\mu\psi
\end{equation}

于是根据最小耦合原理我们可以得到Lagrangian
\begin{equation}
    \mathcal L=\bar\psi(i\slashed D-m)\psi-\frac14F^2
\end{equation}
即
\begin{equation}
    \mathcal L=\bar\psi(i\partial_\mu-m)\psi-eA_\mu\bar\psi\gamma^\mu\psi
\end{equation}

于是我们发现电流项
\begin{equation}
    J^\mu=e\bar\psi\gamma^\mu\psi
\end{equation}

还有EoM
\begin{align}
    \begin{cases}
        &(i\slashed\partial-m)\psi=e\slashed A\psi\\
        &\partial_\mu F^{\mu\nu}=e\bar\psi\gamma^\nu\psi
    \end{cases}
\end{align}

\subsection{旋量LSZ公式}
我们首先有引理
\begin{lemma}
    \begin{equation}
        \int\d^4xi\exp{ipx}\bar u_s(m-i\slashed\partial)\psi=\sqrt{2\om p}(\a p^s(+\infty)-\a p^s(-\infty))
    \end{equation}
    \begin{equation}
        \int\d^4xi\exp{-ipx}\bar v_s(-m-i\slashed\partial)\psi=\sqrt{2\om p}(\b p^{\dagger s}(+\infty)-\b p^{\dagger s}(-\infty))
    \end{equation}
    \begin{equation}
        \int\d^4xi\exp{-ipx}\Tr\left[(-m-i\slashed\partial)u_s\bar\psi\right]=\sqrt{2\om p}(\a p^{\dagger s}(+\infty)-\a p^{\dagger s}(-\infty))
    \end{equation}
    \begin{equation}
        \int\d^4xi\exp{ipx}\Tr\left[(m-i\slashed\partial)v_s\bar\psi\right]=\sqrt{2\om p}(\b p^{\dagger s}(+\infty)-\b p^{\dagger s}(-\infty))
    \end{equation}
\end{lemma}
\begin{proof}
    \begin{align}
        \int\d^4xi\exp{ipx}\bar u_s(m-i\slashed\partial)\psi&=\int\d^4xi\exp{ipx}\bar u_s(m--i\gamma^i\partial_i-i\gamma^0\partial_0)\psi\\
        &=\int\d^4xi\exp{ipx}\bar u_s(m-\gamma^ip_i-i\gamma^0\partial_0)\psi=
    \end{align}
    根据
    \begin{equation}
        \partial_0(\exp{ipx}\bar u_s\gamma^0\psi)=ip_0\exp{ipx}\bar u_s\gamma^0\psi+\exp{ipx}\bar u_s\gamma^0\partial_0\psi
    \end{equation}
    我们有
    \begin{align}
        &\;\;\int\d^4xi\exp{ipx}\bar u_s(m-i\slashed\partial)\psi\\
        &=\int\d^4xi\exp{ipx}\bar u^s(m-\slashed p)\psi\notag\\
        &\;+\int\d^3x\exp{ipx}\bar u_s\gamma^0\int\ldsq q{}\sum_{s'}\left(u_{s'}\a q^s\exp{-iqx}+v_s\b q^{\dagger s}\exp{iqx}\right)\Big|_{-\infty}^{+\infty}\\
        &=\bar u_s\gamma^0\int\ldsq q{\dpi3}\sum_{s'}\left(u_{s'}\a q^s\delta^3(\vec p-\vec q)+v_s\b q^{\dagger s}\delta^3(\vec p+\vec q)\right)\Big|_{-\infty}^{+\infty}
    \end{align}
    而根据定理\ref{ugammau-contraction}我们知道
    \begin{equation}
        \bar u_s(\vec p)\gamma^0u_{s'}(\vec p)=2p^0\delta_{ss'}, \bar u_s(\vec p)\gamma^0v_{s'}(\vec p)=0
    \end{equation}
    从而有
    \begin{equation}
        \int\d^4xi\exp{ipx}\bar u_s(m-i\slashed\partial)\psi=\sqrt{2\om p}(\a p^s(+\infty)-\a p^s(-\infty))
    \end{equation}

    剩下三个式子证明同理.
\end{proof}

于是我们可以有最终结论
\begin{align}
    \braket{f, +\infty|i, -\infty}&=\int\d^4xi\exp{ip_1x_1}\bar u_s(m-i\slashed\partial_1)\textcolor{blue}{\int\d^4x_2i\exp{-ipx}\bar v_s(-m-i\slashed\partial_2)}\notag\\
    &\;\partial_{3\mu}\partial_{4\nu}\braket{\psi_1\psi_2...\bar\psi_3\bar\psi_4}\notag\\
    &\;\int\d^4xi\exp{-ip_3x_3}(-m-i\gamma^\mu)u_s\textcolor{blue}{\int\d^4xi\exp{ip_4x_4}(m-i\gamma^\nu)v_s}
\end{align}
其中, 第一行第一个为出射正粒子, 第一行第二个为入射反粒子, 第三行第一个为入射正粒子, 第三行第四个为出射反粒子.

\subsection{旋量QED Feynman规则}
我们可以不难从Lagrangian中读出Feynman规则
\begin{enumerate}
    \item Dirac旋量传播子
    $$
        \begin{tikzpicture}[baseline=(current bounding box.center)]
            \begin{feynman}
                \vertex (a) {\(p\)};
                \vertex [right=2cm of a] (b);
                \diagram* {
                    (a) -- [fermion, momentum'=\(p\)] (b),
                };
            \end{feynman}
        \end{tikzpicture}
        = \frac{i(\slashed{p} + m)}{p^2 - m^2 + i\epsilon}
    $$
    \item 光子传播子
    $$
        \begin{tikzpicture}[baseline=(current bounding box.center)]
            \begin{feynman}
                \vertex (a) {\(\mu\)};
                \vertex [right=2cm of a] (b) {\(\nu\)};
                \diagram* {
                    (a) -- [photon, momentum'=\(p\)] (b),
                };
            \end{feynman}
        \end{tikzpicture}
        = \frac{-i g_{\mu\nu}}{p^2 + i\epsilon} \quad (\text{Feynman 规范})
    $$
    \item 旋量-光子三点顶角
    $$
    \begin{tikzpicture}[baseline=(current bounding box.center)]
        \begin{feynman}
            \vertex (a);
            \vertex [above right=1.5cm of a] (b);
            \vertex [below right=1.5cm of a] (c);
            \vertex [left=1.5cm of a] (d);
            \diagram* {
                (b) -- [fermion] (a) -- [fermion] (c),
                (a) -- [photon] (d),
            };
        \end{feynman}
    \end{tikzpicture}
    = -ie\gamma^\mu
    $$
\end{enumerate}

对于每一个过程的外线,都有对应的因子:
\begin{itemize}
    \item 入射电子(${e^-}$): $u(p, s)$
    \item 出射电子 (${e^-}$): $\bar{u}(p, s)$
    \item 入射正电子 (${e^+}$): $\bar{v}(p, s)$
    \item 出射正电子 (${e^+}$): $v(p, s)$
    \item 入射光子 (${\gamma}$): $\epsilon^\mu(p, \lambda)$
    \item 出射光子 (${\gamma}$): $\epsilon^{\mu*}(p, \lambda)$
\end{itemize}

需要注意, 由于Fermion场的反对易性, 对于一些构型我们会有额外的$-1$因子, 比如Fermion环. 比较保险的方法是将图还原到Dyson级数的缩并中具体地检查正负号.
