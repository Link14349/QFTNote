% !TEX program = xelatex
% !TEX root = 笔记.tex
\documentclass[cn,hazy,blue,14pt,normal]{elegantnote}
\title{量子场论笔记}
\author{郑元昊}
\date{\zhtoday}
\institute{复旦大学物理系}

\usepackage{array}
\usepackage{amssymb,amsfonts}
\usepackage{amsmath}
\usepackage{mathrsfs}
\usepackage{braket}
\usepackage{tcolorbox}
\usepackage{xparse}
\usepackage{tikz}
\usepackage{subcaption}
\usepackage{makecell}
\usepackage{tikz-feynman}
\usepackage{slashed}
\usepackage{cancel}
\usepackage{geometry}
\usepackage{feynmp-auto}
\makeatletter
\def\input@path{{./texmf/}}
\makeatother
\usepackage{tikz-feynman}
\tikzfeynmanset{compat=1.1.0} % 推荐设置兼容性版本

% 定义通用的样式设置,减少重复代码
\tikzfeynmanset{
    my_diagram/.style={
        small, 
        inline=(a.center), 
        % baseline=-0.5ex, % 垂直居中对齐
        horizontal=a to b,
        dot, % <-- 添加这一行:将所有顶点渲染为实心黑点
        % /tikz/dot/minimum size=4pt, % 可选:调整点的大小 (默认值通常是 4pt 左右)
    }
}
\tikzset{
    scalar arrow/.style={
        dashed, % 虚线
        postaction={decorate}, % 允许装饰
        decoration={
            markings,
            % 使用实心的 'stealth' 箭头,并设置其长度和宽度
            mark=at position 0.5 with {\arrow[scale=1.5]{Stealth}} 
            % 或者使用更通用的 latex 箭头并加大 scale:
            % mark=at position 0.5 with {\arrow[scale=1.5]{latex}} 
        }
    }
}
\tikzset{
    % 定义动量箭头样式
    my momentum arrow/.style={
        % 使用 postaction 来在线条绘制完成后再执行一个动作
        postaction={
            decorate,
            decoration={
                markings,
                mark=at position 0.5 with {\arrow{>}}, % 在中点放置一个箭头
            },
        },
        draw=none, % 不绘制线本身,只绘制箭头
    },
    % 动量标签位置调整
    momentum label/.style={
        sloped, % 让标签与路径平行
        above,  % 标签在线的上方
        yshift=2mm, % 垂直微调
    }
}

\usetikzlibrary{calc}
\usetikzlibrary{arrows.meta}
\usetikzlibrary{bending}

% 定义重要结论环境
\newtcolorbox{important}{
    colback=white,    % 背景色
    colframe=black,   % 边框色
    sharp corners,    % 直角
    boxrule=1.5pt,    % 边框粗细
    coltitle=black
}

\numberwithin{equation}{subsection}
\numberwithin{figure}{section}
\numberwithin{table}{section}
\newcommand{\de}[2]{\frac{\mathrm d#1}{\mathrm d#2}}
\newcommand{\pa}[2]{\frac{\partial #1}{\partial #2}}
\renewcommand{\vec}[1]{\mathbf{#1}}
\renewcommand{\rm}[1]{\mathrm{#1}}
\renewcommand{\bf}[1]{\mathbf{#1}}
\renewcommand{\exp}[1]{\mathrm e^{#1}}
\newcommand{\ddd}[1]{\frac{\mathrm d^3#1}{(2\pi)^3}}
\newcommand{\dddd}[1]{\frac{\mathrm d^4#1}{(2\pi)^4}}
\renewcommand{\a}[1]{a_{\vec #1}}
\renewcommand{\b}[1]{b_{\vec #1}}
\newcommand{\om}[1]{\omega_{\vec #1}}
\newcommand{\ad}[1]{a^\dagger_{\vec #1}}
\renewcommand{\d}{\mathrm d}
\newcommand{\dx}{\mathrm dx}
\newcommand{\dy}{\mathrm dy}
\newcommand{\dz}{\mathrm dz}
\newcommand{\dw}{\mathrm dw}
\newcommand{\dt}{\mathrm dt}
% \renewcommand{\dp}{\mathrm dp}
\newcommand{\dpi}[1]{(2\pi)^{#1}}
\renewcommand{\vec}[1]{\mathbf{#1}}
\newcommand{\Tr}{\mathrm{Tr}}
\newcommand{\uvec}[1]{\mathbf{\hat{#1}}}
\newcommand{\e}{\mathrm e}
\newcommand{\tensor}[1]{\overleftrightarrow{\vec{#1}}}
\NewDocumentCommand{\ld}{m O{1}}{\frac{\mathrm d^3#1}{(2\pi)^3}\frac{#2}{2\omega_{\vec #1}}}
\NewDocumentCommand{\ldsq}{m O{1}}{\frac{\mathrm d^3#1}{(2\pi)^3}\frac{#2}{\sqrt{2\omega_{\vec #1}}}}
\NewDocumentCommand{\lips}{m O{1}}{\frac{\mathrm d^3#1}{(2\pi)^3}\frac{#2}{2\omega_{\vec #1}}}

\begin{document}
\maketitle
\newpage

\setcounter{tocdepth}{3} % 0: chapter/part, 1: section, 2: subsection, 3: subsubsection
\tableofcontents

\newpage
\section{绪论}
\subsection{Why QFT?}
QFT is a framework for incorporating an infinite number of quantum degrees of
freedom, arranged in spacetime, with the interactions constrained by locality, symmetry,
unitarity, and causality. And it is the language of the laws of nature.\cite{qft:lecture1}
\begin{itemize}
    \item Locality: interactions occur at single points in spacetime
    \item Symmetry: the laws of physics are invariant under certain transformations
    \item Unitarity: the total probability of all possible outcomes of a quantum event is 1
    \item Causality: cause precedes effect, and information cannot travel faster than light
\end{itemize}
\subsection{符号约定}
\begin{itemize}
    \item 采用自然Heaviside-Lorentz单位制: $\hbar = c = \mu_0 = \epsilon_0 = 1$
    \item 度规: $g_{\mu\nu} = \mathrm{diag}(1,-1,-1,-1)$
    \item 爱因斯坦求和约定: 重复指标隐含求和, 并且希腊字母如$\mu,\nu,\lambda\cdots$取0,1,2,3, 拉丁字母如$i,j,k\cdots$取1,2,3
    \item 在不影响歧义的情况下, 对于标量场$\phi$, 其偏导$\mu$分量记为$\phi_\mu=\partial_\mu\phi$
    \item 在不影响歧义的情况下, $p_\mu x^\mu$类的缩并可简记为$px$
    \item $\phi_\mathbf x\equiv\phi(\mathbf x)$
    \item $\phi_i\equiv\phi(x_i)$
    \item $\delta_{xy}\equiv\delta^4(x-y)$
    \item $\delta_{ij}\equiv\delta^4(x_i-x_j)$
    \item $D_{xy}\equiv D_F(x-y)$
    \item $D_{ij}\equiv D_F(x_i-x_j)$
    \item $\braket{\phi_1\phi_2\cdots\phi_n}\equiv\braket{\Omega|\mathcal T\phi_1\phi_2\cdots\phi_n|\Omega}$
\end{itemize}
\begin{equation}
    \Theta(x)=\begin{cases}
        1, & x>0\\
        0, & x<0
    \end{cases}
\end{equation}
单位元电荷
\begin{equation}
    e=-|e|
\end{equation}
精细结构常数
\begin{equation}
    \alpha=\frac{e^2}{4\pi}\approx \frac{1}{137}
\end{equation}
作用量
\begin{equation}
    S = \int \mathcal{L} \mathrm d^4x=\int (-m\mathrm d\tau-qA_\mu\mathrm dx^\mu)-\int \mathrm d^4x \frac14 F_{\mu\nu}F^{\mu\nu}
\end{equation}
其中,$F_{\mu\nu}$为电磁场张量
\begin{equation}
    F_{\mu\nu} = \mathrm dA_{\mu\nu}
\end{equation}
$\mathrm d$为外微分算符,$A_\mu$为电磁四势.\\
展开为分量形式有: 
\begin{subequations}
    \begin{align}
        F_{\mu\nu} &= \begin{bmatrix}
            0 & E_x & E_y & E_z \\
            -E_x & 0 & -B_z & B_y \\
            -E_y & B_z & 0 & -B_x \\
            -E_z & -B_y & B_x & 0
        \end{bmatrix}\\
        F^{\mu\nu} &= \begin{bmatrix}
            0 & -E_x & -E_y & -E_z \\
            E_x & 0 & -B_z & B_y \\
            E_y & B_z & 0 & -B_x \\
            E_z & -B_y & B_x & 0
        \end{bmatrix}
    \end{align}
\end{subequations}
以及有粒子动力学方程:
\begin{equation}
    m\frac{\mathrm d^2x^\mu}{\mathrm d\tau^2}=qF^\mu_{~~\nu}\frac{\mathrm dx^\nu}{\mathrm d\tau}
\end{equation}
还有Maxwell方程组的协变形式:
\begin{equation}
    \begin{cases}
        \partial_\mu F^{\mu\nu} = J^\nu\\
        \partial_{[\mu} F_{\nu\lambda]} = 0
    \end{cases}
\end{equation}
化为矢量方程组:
\begin{equation}
    \begin{cases}
        \nabla \cdot \mathbf{E} &= \rho \\
        \nabla \cdot \mathbf{B} &= 0 \\
        \nabla \times \mathbf{E} &= -\frac{\partial \mathbf{B}}{\partial t} \\
        \nabla \times \mathbf{B} &= \mathbf{J} + \frac{\partial \mathbf{E}}{\partial t}
    \end{cases}
\end{equation}

\newpage
\section{数学基础}
\subsection{泛函}
\begin{definition}[泛函]
    泛函$f: \mathscr{F} \to \mathbb{R}$是定义在某个函数空间上的映射, 全体泛函组成的集合记为$\mathcal{F}$
\end{definition}
\begin{example}
    设$\mathcal{F}$是定义在区间$[a,b]$上的所有实值连续函数的空间,则
    \begin{equation}
        f[\phi] = \int_a^b \phi(x) \, \mathrm{d}x
    \end{equation}
    是$\mathcal{F}$上的一个泛函
\end{example}

\subsection{$\delta$函数}
\begin{definition}[$\delta$函数]
    $\delta(x)$是满足
    \begin{equation}
        \int f(x) \delta(x) \mathrm dx = f(0)
    \end{equation}
    的函数
\end{definition}
\begin{theorem}
    \begin{equation}
        \int \mathrm dx \mathrm e^{ikx}=2\pi \delta(x)\label{matheq1}
    \end{equation}
\end{theorem}
\begin{theorem}
    \begin{equation}
        \int_a^b f(x) \delta(g(x)) \mathrm dx = \sum_{i\in \{i~|~g(x_i)=0, x_i\in[a, b]\}}\frac{f(x_i)}{|g'(x_i)|}= \sum_{i\in \{i~|~g(x_i)=0\}}\frac{f(x_i)}{|g'(x_i)|}\Theta(b-x)\Theta(x-a)
    \end{equation}
\end{theorem}

\subsection{变分}
\begin{definition}[变分]
    设泛函$f: \mathcal{F} \to \mathbb{R}$\\
    其变分
    \begin{equation}
        \frac{\delta f}{\delta \phi(x_0)} = \lim_{\epsilon \to 0} \frac{f[\phi(x) + \epsilon \delta(x-x_0)] - f[\phi]}{\epsilon}
    \end{equation}
\end{definition}
\begin{example}
    设泛函
    \begin{equation}
        f[\phi] = \phi
    \end{equation}
    则
    \begin{equation}
        \frac{\delta f}{\delta \phi(x_0)} = \delta(x-x_0)
    \end{equation}
\end{example}
\begin{example}
    设泛函
    \begin{equation}
        f[\phi] = g(\phi), \text{其中} g: \mathbb{R} \to \mathbb{R}
    \end{equation}
    则
    \begin{equation}
        \frac{\delta f}{\delta \phi(x_0)} = \frac{\partial g}{\partial \phi} \delta(x-x_0)
    \end{equation}
\end{example}
\begin{example}
    设泛函
    \begin{equation}
        f[\phi] = \int g(\phi) \mathrm dx, \text{其中} g: \mathbb{R} \to \mathbb{R}
    \end{equation}
    则
    \begin{equation}
        \frac{\delta f}{\delta \phi(x_0)} = \left.\frac{\partial g}{\partial \phi}\right|_{\phi(x_0)} \delta(x-x_0)
    \end{equation}
\end{example}
\begin{example}
    设泛函
    \begin{equation}
        f[\phi] = \int g(\phi, \nabla\phi) \mathrm d^3x, \text{其中} g: \mathbb{R} \times \mathbb{R}^3 \to \mathbb{R}
    \end{equation}
    则
    \begin{equation}
        \frac{\delta f}{\delta \phi(x_0)} = \left.\left(\frac{\partial g}{\partial \phi} - \partial_i\frac{\partial g}{\partial \phi_i}\right)\right|_{\nabla\phi(x_0)}
    \end{equation}
\end{example}
参考: \cite{functionals}

% \subsection{群\&群表示论}


\subsection{Lorentz群}
\begin{definition}[Lorentz变换]
    Lorentz变换为一种保内积的变换:
    \begin{equation}
        \bar x^\mu=\Lambda^\mu_{~~\nu}x^\nu
    \end{equation}
    使得
    \begin{equation}
        \bar x^\mu\bar x_\mu=x^\mu x_\mu
    \end{equation}
\end{definition}
\begin{theorem}[Lorentz变换的性质]\label{theorem:lorentz_property}
    \begin{equation}
        \Lambda^\mu_{~~\sigma}g_{\mu\nu}\Lambda^\nu_{~~\rho}=g_{\sigma\rho}
    \end{equation}
\end{theorem}
\begin{proof}
    \begin{equation}
        \bar{x}^2=g_{\mu\nu}\bar{x}^\mu\bar x^\nu=x^\sigma(\Lambda^\mu_{~~\sigma}g_{\mu\nu}\Lambda^\nu_{~~\rho})x^\rho=x^\sigma g_{\sigma\rho} x^\rho
    \end{equation}
\end{proof}
于是我们可以有如下推论:
\begin{theorem}[Lorentz变换的逆矩阵]\label{theorem:Lorentz_inverse}
    \begin{equation}
        (\Lambda^{-1})^{\mu}_{~~\nu}=\Lambda_\nu^{~~\mu}
    \end{equation}
\end{theorem}
\begin{proof}
    由\ref{theorem:lorentz_property}可得
    \begin{equation}
        g^{\rho\beta}\Lambda^\mu_{~~\rho}\Lambda^\nu_{~~\sigma}g_{\mu\nu}=g_{\rho\sigma}g^{\rho\beta}=\delta^\beta_{~~\sigma}
    \end{equation}
    即:
    \begin{equation}
        \Lambda_\nu^{~~\beta}\Lambda^\nu_{~~\sigma}=\delta^\beta_{~~\sigma}
    \end{equation}
    于是可以得证.
\end{proof}
\begin{definition}[$\delta\omega^\mu_{~~\nu}$]\label{def:deltaomega}
    对于无穷小Lorentz变换$\Lambda^\mu_{~~\nu}$, 定义
    \begin{equation}
        \Lambda^\mu_{~~\nu}=\delta^\mu_{~~\nu}+\delta\omega^\mu_{~~\nu}
    \end{equation}
\end{definition}
通过\eqref{theorem:lorentz_property}可以发现$\delta\omega_{\mu\nu}$是反称的.
\begin{theorem}[$\delta\omega_{\mu\nu}$的性质]
    \begin{equation}
        \delta\omega_{\mu\nu}=\delta\omega_{[\mu\nu]}
    \end{equation}
\end{theorem}
\begin{definition}[对$\phi$的Unitary变换$U(\Lambda)$]
    \begin{equation}
        U(\mathbf 1+\delta\omega)=1+\frac i2\delta\omega_{\mu\nu}M^{\mu\nu}
    \end{equation}
    其中,$M^{\mu\nu}=M^{[\mu\nu]}$
\end{definition}
\begin{theorem}[结合律]\label{theorem:U_combine}
    我们要求$U$满足:
    \begin{equation}
        U(\Lambda\Lambda')=U(\Lambda)U(\Lambda')
    \end{equation}
\end{theorem}
根据定理\ref{theorem:Lorentz_inverse}, 定理\ref{theorem:U_combine}, 定义\ref{def:deltaomega}, 我们要求$U(\Lambda^{-1}\Lambda'\Lambda)=U(\Lambda^{-1})U(\Lambda')U(\Lambda)$, 于是有\ref{theorem:UMU}:
\begin{theorem}\label{theorem:UMU}
    \begin{equation}
        U^{-1}_\Lambda M^{\mu\nu}U_\Lambda=\Lambda^\mu_{~~\rho}\Lambda^\nu_{~~\sigma}M^{\rho\sigma}
    \end{equation}
\end{theorem}
\begin{proof}
    \begin{equation}\label{2eq1}
        U_\Lambda^{-1}U_{\Lambda'}U_\Lambda=1+\frac i2\delta{\omega'}_{\mu\nu}U_\Lambda^{-1}M^{\mu\nu}U_\Lambda
    \end{equation}
    \begin{equation}
        U(\Lambda^{-1}\Lambda'\Lambda)=U(1+\Lambda^{-1}\omega'\Lambda)=1+\frac i2(\Lambda^{-1}\delta\omega'\Lambda)_{\mu\nu}M^{\mu\nu}
    \end{equation}
    计算$(\Lambda^{-1}\delta\omega'\Lambda)^{\mu}_{~~\nu}$
    \begin{equation}
        (\Lambda^{-1}\delta\omega'\Lambda)^{\mu}_{~~\nu}=\Lambda_\sigma^{~~\mu}\delta{\omega'}^\sigma_{~~\rho}\Lambda^\rho_{~~\nu}
    \end{equation}
    于是
    \begin{equation}
        (\Lambda^{-1}\delta\omega'\Lambda)_{\mu\nu}=\Lambda^{\sigma}_{~~\mu}\delta{\omega'}_{\sigma\rho}\Lambda^\rho_{~~\nu}
    \end{equation}
    因此
    \begin{equation}\label{2eq2}
        U(\Lambda^{-1}\Lambda'\Lambda)=U(1+\Lambda^{-1}\omega'\Lambda)=1+\frac i2\Lambda^{\sigma}_{~~\mu}\delta{\omega'}_{\sigma\rho}\Lambda^\rho_{~~\nu}M^{\mu\nu}
    \end{equation}
    将\eqref{2eq1}与\eqref{2eq2}取等我们有
    \begin{equation}
        \delta{\omega'}_{\rho\sigma}U_\Lambda^{-1}M^{\rho\sigma}U_\Lambda=\Lambda^{\sigma}_{~~\mu}\delta{\omega'}_{\sigma\rho}\Lambda^\rho_{~~\nu}M^{\mu\nu}
    \end{equation}
    于是
    \begin{equation}
        U^{-1}_\Lambda M^{\mu\nu}U_\Lambda=\Lambda^\mu_{~~\rho}\Lambda^\nu_{~~\sigma}M^{\rho\sigma}
    \end{equation}
\end{proof}
进一步展开我们可以得到
\begin{theorem}[$M^{\mu\nu}$的对易子]
    \begin{equation}
        [M^{\mu\nu}, M^{\rho\sigma}]=i(g^{\mu\rho}M^{\nu\sigma}+g^{\sigma\nu}M^{\mu\rho}-g^{\mu\sigma}M^{\nu\rho}-g^{\rho\nu}M^{\mu\sigma})
    \end{equation}
\end{theorem}
\begin{proof}
    展开
    \begin{equation}
        (1-\frac i 2\delta \omega_{\alpha\beta}M^{\alpha\beta})M^{\mu\nu}(1+\frac i 2\delta \omega_{\rho\sigma}M^{\rho\sigma})=(\delta^\mu_{~~\rho}+\delta\omega^\mu_{~~\rho})(\delta^\nu_{~~\sigma}+\delta\omega^\nu_{~~\sigma})M^{\rho\sigma}
    \end{equation}
    化简整理得到
    \begin{equation}
        -\frac i2\delta\omega_{\rho\sigma}[M^{\rho\sigma}, M^{\mu\nu}]=\delta\omega_{\rho\sigma}(M^{\mu\sigma}g^{\rho\nu}-M^{\rho\nu}g^{\mu\sigma})
    \end{equation}
    于是
    \begin{equation}\label{eq3}
        [M^{\mu\nu}, M^{\rho\sigma}]=2i(-g^{\mu\sigma}M^{\nu\rho}-g^{\rho\nu}M^{\mu\sigma})+A^{\mu\nu\rho\sigma}
    \end{equation}
    其中$A^{\mu\nu\rho\sigma}=A^{\nu\mu\rho\sigma}$, $A^{\mu\nu\rho\sigma}=A^{\mu\nu\sigma\rho}$.\\
    交换$\mu$, $\nu$:
    \begin{equation}\label{eq4}
        [M^{\nu\mu}, M^{\rho\sigma}]=2i(-g^{\nu\sigma}M^{\mu\rho}-g^{\rho\mu}M^{\nu\sigma})+A^{\mu\nu\rho\sigma}
    \end{equation}
    注意到$M^{\mu\nu}$反称, \eqref{eq3}+\eqref{eq4}得到:
    \begin{equation}
        A^{\mu\nu\rho\sigma}=i(g^{\mu\sigma}M^{\nu\rho}+g^{\nu\rho}M^{\mu\sigma}+g^{\mu\sigma}M^{\nu\sigma}+g^{\nu\sigma}M^{\mu\rho})
    \end{equation}
    于是可得
    \begin{important}
        \begin{equation}
            [M^{\mu\nu}, M^{\rho\sigma}]=i(g^{\mu\rho}M^{\nu\sigma}+g^{\sigma\nu}M^{\mu\rho}-g^{\mu\sigma}M^{\nu\rho}-g^{\rho\nu}M^{\mu\sigma})
        \end{equation}
    \end{important}
\end{proof}
\begin{definition}[(伪)旋转生成元]
    \begin{equation}
        J_i = \frac12\epsilon_{ijk}M^{jk}
    \end{equation}
    \begin{equation}
        K_i = M_{i0}
    \end{equation}
\end{definition}
\begin{theorem}[(伪)旋转生成元的对易关系]
    \begin{equation}
        [J_i, J_j]=i\epsilon_{ijk}J^k
    \end{equation}
    \begin{equation}
        [J_i, K_j]=i\epsilon_{ijk}K^k
    \end{equation}
    \begin{equation}
        [K_i, K_j]=-i\epsilon_{ijk}J^k
    \end{equation}
\end{theorem}
\begin{proof}
    注意到$g=\mathrm{diag}(1,-1,-1,-1)$
    \begin{equation}
        \begin{split}
            [J_i, J_j]&=-\frac i2\epsilon_{ikl}\epsilon_{jmn}(g^{nk}M^{lm}+g^{ml}M^{kn})\\
            &=-\frac i2\epsilon_{ikl}\epsilon_{jmn}(\textcolor{red}{-\delta^{nk}}M^{lm}\textcolor{red}{-\delta^{ml}}M^{kn})\\
            &=\frac i2(\epsilon_{kli}\epsilon_{kjm}M^{lm}+\epsilon_{lik}\epsilon_{lnj}M^{kn})\\
            &=\frac i2\left((\delta_{lj}\delta_{im}-\delta_{lm}\delta_{ij}M^{lm}+(\delta_{in}))\right)\\
            &=-iM_{ij}
        \end{split}
    \end{equation}
    又因为
    \begin{equation}
        \epsilon_{kij}\epsilon^{kmn}=-(\delta_i^m\delta_j^n-\delta_i^n\delta_j^m)
    \end{equation}
    \textcolor{red}{(不要忘记我们升指标的时候我们度规三次, 而(+ - - -)度规在三维空间的诱导度规为$\mathrm{diag}(-1,-1,-1)$, 因此总体上我们乘了三次-1,产生一个额外的负号)}\\
    因此
    \begin{equation}\label{eq5}
        \epsilon_{ijk}J^k=\frac12\epsilon_{kij}\epsilon^{kmn}M_{mn}=-M_{ij}
    \end{equation}
    然后即得
    \begin{equation}
        [J_i, J_j]=i\epsilon_{ijk}J^k
    \end{equation}
    接着尝试证明$[J_i,K_j]$.
    \begin{equation}
        [J_i,K_j]=[\frac12\epsilon_{imn}M^{mn}, M_{j0}]=\frac12\epsilon_{imn}[M^{mn}, M_{j0}]
    \end{equation}
    计算$[M^{\mu\nu}, M_{\rho\sigma}]$
    \begin{equation}
        [M^{\mu\nu}, M_{\rho\sigma}]=-2i(\delta_\sigma^{~~\mu}M^\nu_{~~\rho}+\delta_\rho^{~~\nu}M^\mu_{~~\sigma})
    \end{equation}
    于是
    \begin{equation}
        [J_i, K_j]=\frac12\epsilon_{ikl}(-2i)\left(\delta_0^{~~k}M^l_{~~j}+\delta_j^{~~l}M^k_{~~0}\right)=i\epsilon_{ijk}M^k_{~~0}
    \end{equation}
    又因为$g_{\mu\nu}$升$0$指标不会改变符号, 因此我们可以直接升指标然后得到
    \begin{equation}
        [J_i, K_j]=i\epsilon_{ijk}M^{k0}=i\epsilon_{ijk}K^k
    \end{equation}
    最后,考虑证明:
    \begin{equation}
        [K_i, K_j]=[M_{i0}, M_{j0}]
    \end{equation}
    因为
    \begin{equation}
        [M_{\mu\nu}, M_{\rho\sigma}]=i(g_{\mu\rho}M_{\nu\sigma}+g_{\sigma\nu}M_{\mu\rho}-g_{\mu\sigma}M_{\nu\rho}-g_{\rho\nu}M_{\mu\sigma})
    \end{equation}
    于是
    \begin{equation}
        [K_i, K_j]=i(g_{ij}M_{00}+g_{00}M_{ij}-g_{i0}M_{0j}-g_{j0}M_{i0})=iM_{ij}
    \end{equation}
    利用\eqref{eq5}我们得到
    \begin{equation}
        [K_i, K_j]=-i\epsilon_{ijk}J^k
    \end{equation}
\end{proof}

\newpage
\section{狭义相对论动力学技巧}
一些常用技巧\cite{griffthsPPSRTrick}:
\begin{enumerate}
    \item $\vec v=\frac{\vec p}{E}$
    \item 使用四矢量以及不变量点积
    \item 使用质心系简化计算
\end{enumerate}
\begin{definition}[Mandelstam变量]
    对于以$m_1, p_1$, $m_2, p_2$的粒子入射, 以$m_3, p_3$ $m_4, p_4$的粒子出射的动力学系统
    \begin{equation}
        s=(p_1+p_2)^2=(p_3+p_4)^2, \quad t=(p_1-p_3)^2=(p_2-p_4)^2, \quad u=(p_1-p_4)^2=(p_2-p_3)^2
    \end{equation}
\end{definition}
\begin{theorem}[s+t+u守恒]
    \begin{equation}
        s+t+u=m_1^2+m_2^2+m_3^2+m_4^2
    \end{equation}
\end{theorem}
\begin{proof}
    我们取质心系, 可以写出$p_1, p_2, p_3, p_4$
    \begin{align}
        p_1=(E_1, \vec p), \quad p_2=(E_2, -\vec p)\\
        p_3=(E_3, \vec k), \quad p_4=(E_4, -\vec k)
    \end{align}

    于是
    \begin{equation}
        s=(E_1+E_2)^2, \quad t=(E_3-E_1)^2-(\vec k-\vec p)^2, \quad u=(E_4-E_1)^2-(\vec k+\vec p)^2
    \end{equation}

    计算可得:
    \begin{align}
        s+t+u&=E_1^2+E_2^2+E_3^2+E_4^2-2\vec p^2-2\vec k^2+2E_1^2+2E_1E_2-2E_1E_3-2E_1E_4\\
        &=m_1^2+m_2^2+m_3^2+m_4^2+2E_1(E_1+E_2-E_3-E_4)\\
        &=m_1^2+m_2^2+m_3^2+m_4^2
    \end{align}
\end{proof}

\begin{theorem}[质心系中$m_1$的能量]
    \begin{equation}
        E^{\mathbf{CM}}_1=\frac{s+m_1^2-m_2^2}{2\sqrt s}
    \end{equation}
\end{theorem}
\begin{theorem}[实验室系($m_2$静止)中$m_1$的能量]
    \begin{equation}
        E^{\mathbf{lab}}_1=\frac{s-m_1^2-m_2^2}{2m_2}
    \end{equation}
\end{theorem}
\begin{theorem}[质心系总能量]
    \begin{equation}
        E^{\mathbf{CM}}_{\mathbf{TOT}}=\sqrt s
    \end{equation}
\end{theorem}


\newpage
\section{经典场论}
\subsection{拉氏量、作用量与Euler-Lagrange方程}
\begin{definition}[拉氏量\&拉氏量密度]
    拉氏量
    \begin{equation}
        L(t)=\int \mathrm d^3x \, \mathcal{L}(\phi, \partial_\mu \phi)
    \end{equation}
    其中$\mathcal L$即拉氏量密度
\end{definition}
\begin{definition}[作用量]
    作用量$S$是拉氏量密度在时空上的积分
    \begin{equation}
        S = \int L \, \mathrm{d}t=\int\mathcal L\,\mathrm d^4x
    \end{equation}
\end{definition}
\theorem[标量的Euler-Lagrange方程]
\begin{equation}
    \frac{\partial\mathcal L}{\partial\phi}-\partial_\mu\frac{\partial\mathcal L}{\partial\phi_\mu}=0
\end{equation}
\begin{proof}
    设$\phi\to\phi+\delta\phi$,则
    \begin{equation}
        \delta S = \int \mathrm d^4x \left(\frac{\partial\mathcal L}{\partial\phi}\delta\phi + \frac{\partial\mathcal L}{\partial\phi_\mu}\delta\phi_\mu\right)
    \end{equation}
    对第二项分部积分,忽略边界项,有
    \begin{equation}
        \delta S = \int \mathrm d^4x \left(\frac{\partial\mathcal L}{\partial\phi} - \partial_\mu\frac{\partial\mathcal L}{\partial\phi_\mu}\right)\delta\phi
    \end{equation}
    由$\delta S=0$可得Euler-Lagrange方程
\end{proof}
\begin{example}
    自由粒子的拉氏量密度
    \begin{equation}
        \mathcal L = \frac12\partial_\mu\phi\partial^\mu\phi - \frac12 m^2\phi^2
    \end{equation}
    代入Euler-Lagrange方程, 有
    \begin{equation}
        (\partial_\mu\partial^\mu + m^2)\phi = 0
    \end{equation}
    即Klein-Gordon方程
\end{example}
\begin{theorem}[矢量场的Euler-Lagrange方程]
    设矢量场为$A_\mu$, 
    定义$F_{\mu\nu}=\mathrm dA_{\mu\nu}$, $\Pi^{\mu\nu}=\frac{\partial\mathcal L}{\partial F_{\mu\nu}}, B^\nu=\frac{\partial \mathcal L}{\partial A_\nu}$, 那么
    \begin{equation}
        B^\mu-2\partial_\nu\Pi^{[\nu\mu]}=0
    \end{equation}
\end{theorem}
\begin{proof}
    \begin{equation}
        \delta\mathcal L=\Pi^{\mu\nu}\delta \mathrm dA_{\mu\nu}+\frac{\partial\mathcal L}{\partial A_\mu}\delta A_\mu
    \end{equation}
    交换$\delta$与外微分算子$\mathrm d$, 并利用乘法法则有:
    \begin{subequations}
        \begin{align}
            \delta\mathcal L&=2\partial_{[\mu}(\Pi^{\mu\nu}\delta A_{\nu]})-2(\partial_{[\mu}\Pi^{\mu\nu})\delta A_\nu+\frac{\partial\mathcal L}{\partial A_\mu}\delta A_\mu\\
            &=2\partial_{[\mu}(\Pi^{\mu\nu}\delta A_{\nu]})-2(\partial_\nu\Pi^{[\nu\mu]})\delta A_\mu+\frac{\partial\mathcal L}{\partial A_\mu}\delta A_\mu
        \end{align}
    \end{subequations}
    忽略边界项, 由$\delta S=0$可得Euler-Lagrange方程.
\end{proof}
\begin{example}
    电磁场的拉氏量密度
    \begin{equation}
        \mathcal L = -\frac14 F_{\mu\nu}F^{\mu\nu}-J^\mu A_\mu
    \end{equation}
    代入Euler-Lagrange方程, 有
    \begin{equation}
        \partial_\mu F^{\mu\nu} = J^\nu
    \end{equation}
    即Maxwell方程.
\end{example}
\subsection{对称性与守恒量}
\songti 在这里我们考虑连续对称性.
\begin{definition}[连续对称性]
    连续对称性是指在某个参数$\alpha$下,场的变换
    \begin{equation}
        \phi(x) \to \phi'(x) = \phi(x) + \alpha \Delta\phi(x)
    \end{equation}
    其中, $\Delta$为某一算符.\\
    若系统具有该变换的对称性, 则运动方程应该不变.\\
    即:
    \begin{equation}
        \mathcal L \to \mathcal L+\alpha\partial_{\mu} J^\mu
    \end{equation}
\end{definition}
\theorem[Noether定理]
对于存在某对称性的系统, 存在守恒流
\begin{equation}
    j^\mu=\Pi^\mu\Delta\phi - J^\mu
\end{equation}
\begin{proof}
    \begin{subequations}
        \begin{align}
            \delta\mathcal L 
            &= \alpha\frac{\partial\mathcal L}{\partial\phi}\Delta\phi + \alpha\frac{\partial\mathcal L}{\partial\phi_\mu}\partial_\mu\Delta\phi\\
            &= \alpha\left(\frac{\partial\mathcal L}{\partial\phi} - \alpha\partial_\mu\frac{\partial\mathcal L}{\partial\phi_\mu}\right)\Delta\phi + \partial_\mu\left(\frac{\partial\mathcal L}{\partial\phi_\mu}\Delta\phi\right)\\
            &= \alpha\partial_\mu\left(\Pi^\mu\Delta\phi\right)
        \end{align}
    \end{subequations}
    又因为系统存在对称性:
    \begin{equation}
        \delta\mathcal L = \alpha\partial_\mu J^\mu
    \end{equation}
    因此:
    \begin{equation}
        \partial_\mu(\Pi^\mu\Delta\phi-J^\mu)=\partial_\mu j^\mu = 0
    \end{equation}
\end{proof}
\begin{example}[复标量场的$U(1)$对称性]\label{csU1}
    考虑复标量场
    \begin{equation}
        \mathcal L = \partial^\mu\psi^*\partial_\mu\psi - m^2\psi^*\psi
    \end{equation}
    其对称性为
    \begin{equation}
        \psi \to \psi' = e^{i\alpha}\psi, \quad \psi^* \to \psi'^* = e^{-i\alpha}\psi^*
    \end{equation}
    不难发现, $\mathcal L$与$\psi$的相位无关,因此:
    \begin{equation}
        \mathcal L \to \mathcal L + 0\Rightarrow J^\mu = 0
    \end{equation}
    于是, 由Noether定理可得守恒流
    \begin{equation}
        j^\mu = i(\psi\partial^\mu\psi^* - \psi^*\partial^\mu\psi)
    \end{equation}
    我们可以代入验证其守恒:
    \begin{equation}
        \partial_\mu j^\mu = i(\psi\partial^2\psi^*-\psi^*\partial^2\psi)
    \end{equation}
    代入运动方程$(\partial^2+m^2)\psi=0$
    \begin{equation}
        \partial_\mu j^\mu = i(\psi(-m^2\psi^*) - \psi^*(-m^2\psi)) = 0
    \end{equation}
\end{example}
\begin{example}[标量场的时空平移对称性]\label{ex:ct_scalar_translation}
    时空中有Killing矢量场$\xi^\mu$, 其满足Killing方程
    \begin{equation}
        \nabla_{(\mu}\xi_{\nu)} = 0
    \end{equation}
    缩并有:
    \begin{equation}
        \nabla_\mu\xi^\mu = 0
    \end{equation}
    在这里, $\Delta$算符即Lie导数$\mathscr L_\xi$.\\
    对$\mathcal L$沿着$-\xi^\mu$的方向进行平移, 有
    \begin{equation}
        \mathcal L\to \mathcal L + \alpha \mathscr L_\xi\mathcal L=\mathcal L + \alpha \xi^\mu\nabla_\mu\mathcal L=\mathcal L + \alpha\nabla_\mu(\xi^\mu\mathcal L)
    \end{equation}
    因此, $J^\mu = \xi^\mu\mathcal L$. 由Noether定理可得守恒流
    \begin{equation}
    \begin{split}
        j^\mu &= \Pi^\mu\Delta\phi - J^\mu = \Pi^\mu\xi^\nu\nabla_\nu\phi-\xi^\mu\mathcal L\\
        &= \xi^\nu(\nabla^\mu\phi\nabla_\nu\phi-\mathcal L\delta^\mu_{~~\nu})
    \end{split}
    \end{equation}
    于是有能动张量
    \begin{equation}
        T_{\mu\nu}=\nabla_\mu\phi\nabla_\nu\phi - \mathcal L g_{\mu\nu}
    \end{equation}
    其满足:
    \begin{equation}
        \nabla_\mu T^{\mu\nu} = 0
    \end{equation}
    由此可见, 能动量守恒完全是时空平移不变性的体现.
\end{example}
\begin{example}[矢量场的时空平移对称性]
    类似\ref{ex:ct_scalar_translation}, 时空中有Killing矢量场$\xi^\mu$.
    我们首先设两个辅助场:
    \begin{align}
        &\Pi^{\mu\nu}=2\pa{\mathcal L}{F_{\mu\nu}}\\
        &B^\mu=\pa{\mathcal L}{A_\mu}
    \end{align}
    沿着$-\xi^\mu$变换, 同样有:
    \begin{equation}
        \mathcal L\to \mathcal L + \alpha \xi^\mu\nabla_\mu\mathcal L=\mathcal L + \alpha\nabla_\mu(\xi^\mu\mathcal L)
    \end{equation}
    \begin{equation}
        A_\mu \to A_\mu + \alpha\mathscr{L}_\xi A_\mu
    \end{equation}
    \begin{equation}
        F_{\mu\nu} \to F_{\mu\nu} + \alpha\mathscr{L}_\xi F_{\mu\nu}
    \end{equation}
    需要注意的是, $\mathscr L_\xi$与外微分算子$\mathrm d$不对易, 因此
    \begin{equation}
        \mathscr L_\xi F_{\mu\nu} = \mathscr L_\xi \mathrm d A_{\mu\nu} \neq \mathrm d(\mathscr L_\xi A)_{\mu\nu}
    \end{equation}
    而
    \begin{equation}
        \begin{split}
            \mathscr L_\xi F_{\mu\nu} &= \xi^\lambda\nabla_\lambda F_{\mu\nu} + F_{\lambda\nu}\nabla_\mu\xi^\lambda + F_{\mu\lambda}\nabla_\nu\xi^\lambda\\
            &= \xi^\lambda\nabla_\lambda F_{\mu\nu}+\nabla_{\mu}(F_{\lambda\nu\xi^\lambda})-\xi^\lambda\nabla_\mu F_{\lambda\nu}\\
            &\quad+\nabla_\nu(F_{\mu\lambda}\xi^\lambda)-\xi^\lambda\nabla_\nu F_{\mu\lambda}\\
            &= \xi^\lambda\nabla_\lambda F_{\mu\nu}+\xi^\lambda\nabla_\mu F_{\nu\lambda}+\xi^\lambda\nabla_\nu F_{\lambda\mu}+\nabla_{\mu}(F_{\lambda\nu}\xi^\lambda)+\nabla_\nu(F_{\mu\lambda}\xi^\lambda)
        \end{split}
    \end{equation}
    由外微分算子性质有$\nabla_{[\mu}F_{\nu\lambda]}=0$, 可知前三项为零.\\
    因此
    \begin{equation}
        \mathscr L_\xi F_{\mu\nu}=\nabla_{\mu}(F_{\lambda\nu}\xi^\lambda)+\nabla_\nu(F_{\mu\lambda}\xi^\lambda)
    \end{equation}
    于是
    \begin{equation}
        \begin{split}
            \mathscr L_\xi\mathcal L &= \nabla_\mu(2\Pi^{\mu\nu F_{\lambda\nu}\xi^\lambda})-F_{\mu\nu}\xi^\lambda B^\nu+B^\nu(\xi^\lambda\nabla_\lambda A_\nu+A_\lambda\nabla_\nu\xi^\lambda)\\
            &= \nabla_\mu(2\Pi^{\mu\nu F_{\lambda\nu}\xi^\lambda})-F_{\mu\nu}\xi^\lambda B^\nu+B^\nu\xi^\lambda\nabla_\lambda A_\nu-B^\nu\xi^\lambda\nabla_\nu A_\lambda\\
            &\quad+B^\nu\nabla_\nu(A_\lambda\xi^\lambda)
        \end{split}
    \end{equation}
    这里我们取规范
    \begin{equation}
        \nabla_\nu B^\nu=0
    \end{equation}
    故:
    \begin{equation}
        \mathscr L_\xi\mathcal L=\nabla_\mu(2\Pi^{\mu\nu}F_{\lambda\nu}\xi^\lambda+B^\mu A_\nu\xi^\nu)
    \end{equation}
    类似地, 可以得到能动张量:
    \begin{equation}
        T_{\mu\nu} = -2\Pi_{\mu}^{~~\lambda}F_{\lambda\nu}+B_\mu A_\nu-g_{\mu\nu}\mathcal L
    \end{equation}
    参考:本人25年首考后写的笔记\cite{LinkZhihu}(很惭愧, 现在尝试重新推导的时候卡了好久...无奈看了当时的笔记才推出来, 真是奇怪, 明明当时也是我自己推出来的, 怎么现在一点都推不出来了呢= =).
\end{example}
\subsection{哈密顿量}
\begin{definition}[共轭动量]
    共轭动量
    \begin{equation}
        \Pi = \frac{\partial\mathcal L}{\partial\dot\phi}
    \end{equation}
\end{definition}
做Legendre变换, 有Hamiltonian:
\begin{equation}
    H(\phi, \Pi, \nabla\phi) = \int \mathrm d^3x \, \Pi\dot\phi-L=\int \mathrm d^3x \, (\Pi\dot\phi-\mathcal L)
\end{equation}
定义Hamiltonian密度:
\begin{equation}
    \mathcal H(\phi, \Pi, \nabla\phi) = \Pi\dot\phi - \mathcal L
\end{equation}
定义Poisson括号$\{, \}: \mathcal F\times\mathcal F\to\mathbb{R}$:
\begin{equation}
    \{F, G\}=\displaystyle\int \mathrm d^3x \left(\frac{\delta F}{\delta\phi(x)}\frac{\delta G}{\delta\Pi(y)}-\frac{\delta G}{\delta\phi(y)}\frac{\delta F}{\delta\Pi(x)}\right)
\end{equation}
\begin{example}
    \begin{equation}
        \{\phi(\mathbf x), \Pi(\mathbf y)\} = \delta^3(\mathbf x - \mathbf y)
    \end{equation}
\end{example}
\begin{proof}
    设$F=\phi(\mathbf x), G=\Pi(\mathbf y)$, 则
    \begin{equation}
        \frac{\delta F}{\delta\phi(x)} = \delta^3(\mathbf x - \mathbf x'), \quad \frac{\delta F}{\delta\Pi(x)} = 0
    \end{equation}
    \begin{equation}
        \frac{\delta G}{\delta\phi(x)} = 0, \quad \frac{\delta G}{\delta\Pi(x)} = \delta^3(\mathbf y - \mathbf x')
    \end{equation}
    代入Poisson括号定义, 有
    \begin{equation}
        \{\phi(\mathbf x), \Pi(\mathbf y)\} = \int \mathrm d^3x' (\delta^3(\mathbf x - \mathbf x')\delta^3(\mathbf y - \mathbf x') - 0) = \delta^3(\mathbf x - \mathbf y)
    \end{equation}
\end{proof}
\theorem[哈密顿正则方程]
\begin{equation}
    \begin{cases}
        \dot\phi(\mathbf x) = \{\phi_{\mathbf x}, H\}=\frac{\delta H}{\delta\Pi}=\frac{\partial\mathcal H}{\partial\Pi}\\
        \dot\Pi(\mathbf x) = \{\Pi_{\mathbf x}, H\}=-\frac{\delta H}{\delta\phi}=-\frac{\partial\mathcal H}{\partial\phi}+\partial_i\frac{\partial\mathcal H}{\partial(\phi_i)}
    \end{cases}
\end{equation}
\begin{proof}
    \begin{equation}
        \begin{split}
            \delta H &= \int \mathrm d^3x \left(\delta\Pi_\mathbf x\dot\phi_\mathbf x+\Pi_\mathbf x\delta\dot\phi_\mathbf x-\frac{\partial\mathcal L}{\partial\phi}\delta\phi-\Pi_\mathbf x\delta\dot\phi-\frac{\partial\mathcal L}{\partial\phi_i}\delta\phi_i\right)\\
            &= \int \mathrm d^3x \left(\delta\Pi_\mathbf x\dot\phi_\mathbf x - \frac{\partial\mathcal L}{\partial\phi}\delta\phi - \frac{\partial\mathcal L}{\partial\phi_i}\delta\phi_i\right)\\
        \end{split}
    \end{equation}
    分部积分并消去边缘项有:
    \begin{equation}
        \delta H= \int \mathrm d^3x \left(\delta\Pi_\mathbf x\dot\phi_\mathbf x - \frac{\partial\mathcal L}{\partial\phi}\delta\phi + \partial_i\frac{\partial\mathcal L}{\partial\phi_i}\delta\phi\right)\\
    \end{equation}
    于是可得:
    \begin{equation}
        \begin{cases}
            \dot\phi(\mathbf x) = \frac{\partial\mathcal H}{\partial\Pi}\\
            \dot\Pi(\mathbf x) = -\frac{\partial\mathcal H}{\partial\phi}+\partial_i\frac{\partial\mathcal H}{\partial(\phi_i)}
        \end{cases}
    \end{equation}
\end{proof}
讨论: 在转换到Hamilton力学的过程中, Legendre变换给予了$\dot\phi$特殊的地位, 使得我们选定的参考系的时间轴$t$具有了特殊的地位, 因此破坏了洛伦兹协变性.

\newpage
\section{二次量子化}
\songti 二次量子化是将场(如电磁场、电子场)本身进行量子化的框架,它将描述单粒子概率幅的经典场提升为场算符,其激发则对应粒子的产生与湮灭,从而自然地描述了粒子数可变的多粒子系统. 相比之下, 一次量子化中粒子是给定的,其运动(波函数)是量子的.
\subsection{自由实标量场的量子化}
% \begin{enumerate}
%     \item 将场的动力学方程转换为算符方程
%     \item 找到动力学方程的一般解
%     \item 将一般解的积分常数升级为常算符
%     \item 施加量子化条件
%     \item 用常算符构造Hilbert空间
% \end{enumerate}
\begin{enumerate}
    \item 写下Lagrangian
    \item 得到Hamiltonian
    \item 做正则变换解耦
    \item 施加量子化条件
    \item 得到产生湮灭算符
    \item 得到场方程
\end{enumerate}
\kaishu 注: 这里流程没按周洋讲的来, 因为我略微感觉他那样子做有一点点奇怪, 有些地方存在神秘的天降系数, 按照这样从Lagrangian出发的流程会更清晰一些.\songti
写下Lagrangian:
\begin{equation}
    \mathcal L=\frac12\partial_\mu\phi\partial^\mu\phi - \frac12 m^2\phi^2
\end{equation}
我们有EoM:
\begin{equation}
    (\partial_\mu\partial^\mu + m^2)\phi = 0
\end{equation}
或者写成
\begin{equation}
    (\Box + m^2)\phi = 0
\end{equation}
并有色散关系:
\begin{equation}
    \omega^2=\vec p^2+m^2
\end{equation}
满足这个关系的称为on-shell.

得到Hamiltonian:
\begin{equation}
    \mathcal H=\frac12(\pi^2+m^2\phi^2+(\nabla\phi)^2)
\end{equation}

做正则变换, 有母函数
\begin{equation}
    U=-\int\Pi(\vec k)\pi(\vec x)\mathrm e^{i\vec k\cdot\vec x}\mathrm d^3x\mathrm d^3k
\end{equation}

因此
\begin{align}
    &\phi(\vec x)=\int\Phi_{\vec k}\mathrm e^{i\vec k\cdot\vec x}\mathrm d^3k\\
    &\pi(\vec x)=\frac1{(2\pi)^3}\int\Pi_{\vec k}\mathrm e^{-i\vec k\cdot\vec x}\mathrm d^3k
\end{align}
\begin{align}
    &\Phi_{\vec k}=\frac1{(2\pi)^3}\int\phi(\vec x)\mathrm e^{-i\vec k\cdot\vec x}\mathrm d^3x\\
    &\Pi_{\vec k}=\int\pi(\vec x)\mathrm e^{i\vec k\cdot\vec x}\mathrm d^3x
\end{align}
并且有:
\begin{equation}
    \Phi_{\vec k}^\dagger=\Phi_{-\vec k}, \quad \Pi_{\vec k}^\dagger=\Pi_{-\vec k}
\end{equation}
可以得到解耦后的Hamiltonian:
\begin{equation}
    H=\frac12\int \mathrm d^3k (\frac1{(2\pi)^3}\Pi_{\vec k}\Pi_{-\vec k}+(2\pi)^3\omega^2\Phi_{\vec k}\Phi_{-\vec k})
\end{equation}
其中$\omega^2=m^2+k^2$

添加量子化条件:
\begin{equation}
    [\phi(\vec x, t),\pi(\vec y, t)]=i\delta^3(x-y)
\end{equation}
于是有:
\begin{equation}
    [\Phi_{\vec k}, \Pi_{\vec p}]=i\delta^3(k-p)
\end{equation}
令:
\begin{align}
    &a_{\vec k}=(2\pi)^3\sqrt{\frac{\omega}{2}}\left(\Phi_{\vec k}+\frac i{\omega}\frac1{(2\pi)^3}\Pi_{\vec k}^\dagger\right)=\frac1{\sqrt{2\om k}}\int\d^3x\exp{ikx}(\omega\phi+i\pi)\\
    &a^\dagger_{\vec k}=(2\pi)^3\sqrt{\frac{\omega}{2}}\left(\Phi_{\vec k}^\dagger-\frac i{\omega}\frac1{(2\pi)^3}\Pi_{\vec k}\right)=\frac1{\sqrt{2\om k}}\int\d^3x\exp{-ikx}(\omega\phi-i\pi)
\end{align}

有对易子:
\begin{equation}
    [a_{\vec k}, a_{\vec p}^\dagger]=(2\pi)^3\delta^3(\vec k-\vec p)
\end{equation}
更准确地说, 对于$[a_{\vec k}, a_{\vec k}^\dagger]$:
\begin{equation}
    [a_{\vec k}, a_{\vec k}^\dagger]=\mathcal V
\end{equation}
$\mathcal V$为系统的总体积.

于是Hamiltonian可以被对角化:
\begin{equation}
    H=\int\frac{\mathrm d^3p}{(2\pi)^3}\omega_{\vec p}(a^\dagger_{\vec p} a_{\vec p}+\frac12\mathcal V)
\end{equation}

Fourier逆变换可以得到:
\begin{align}
    &\phi(\vec x)=\int\frac{\mathrm d^3p}{(2\pi)^3}\frac1{\sqrt{2\omega_{\vec p}}}(a^\dagger_{\vec p}\mathrm e^{-i\vec p\cdot\vec x}+a_{\vec p}\mathrm e^{i\vec p\cdot\vec x})\\
    &\pi(\vec x)=\int\frac{\mathrm d^3p}{(2\pi)^3} i\sqrt{\frac{\omega_{\vec p}}{2}}(a^\dagger_{\vec p} e^{-i\vec p\cdot\vec x}-a_{\vec p} e^{i\vec p\cdot\vec x})
\end{align}

最后利用时间演化算符得到场方程:
\begin{important}
    \begin{equation}\label{ch4freephi}
        \phi(x)=\mathrm e^{iHt}\phi(\vec x)\mathrm e^{-iHt}=\int\frac{\mathrm d^3p}{(2\pi)^3}\frac1{\sqrt{2\omega_{\vec p}}}(a_{\vec p}^\dagger\mathrm e^{ipx}+a_{\vec p}\mathrm e^{-ipx})
    \end{equation}
    \begin{equation}\label{ch4freepi}
        \pi(x)=\mathrm e^{iHt}\pi(\vec x)\mathrm e^{-iHt}=\int\frac{\mathrm d^3p}{(2\pi)^3} i\sqrt{\frac{\omega_{\vec p}}{2}}(a^\dagger_{\vec p} \e^{ipx}-a_{\vec p} \e^{-ipx})
    \end{equation}
\end{important}

我们可以计算所谓的真空零点能密度:
\begin{equation}
    \frac{\bra0 H\ket0}{\mathcal V}=\int\frac{\mathrm d^3p}{(2\pi)^3}\frac{\omega_{\vec p}}{2}=\int\frac{\mathrm d^3p}{(2\pi)^3}\frac{\sqrt{p^2+m^2}}{2}=+\infty
\end{equation}

我们还可以定义动量算子:
\begin{equation}
    P^i=\int\mathrm\partial^0\phi\partial^i\pi\mathrm d^3x=-\int\phi_t\phi_i\mathrm d^3x
\end{equation}
即:
\begin{align}
    \vec P&=-\int\phi_t\nabla\phi\mathrm d^3x\\
    &=-\int\mathrm d^3x\int\ldsq{p}[i\omega_{\vec p}](a^\dagger_{\vec p} e^{ipx}-a_{\vec p} e^{-ipx})\\
    &\quad\int\ldsq{q}[i\vec q](-a^\dagger_{\vec q}\exp{iqx}+a_{\vec q}\exp{-iqx})\\
    &=\int\ld{p}[\vec p\omega_{\vec p}](a^\dagger_{\vec p} a_{\vec p}+a_{\vec p}a^\dagger_{\vec p})\\
    &=\int\ddd p\vec p a^\dagger_{\vec p} a_{\vec p}
\end{align}
% 动力学方程:
% \begin{equation}
%     (\partial^2+m^2)\phi=0
% \end{equation}
% 利用Fourier变换, 得:
% \begin{equation}
%     \omega^2=p^2+m^2
% \end{equation}
% 于是, 有
% \begin{align}
%     &\phi(x)=\int \frac{\mathrm d^3p}{(2\pi)^3} \frac{1}{\sqrt{2\omega_{\vec p}}}(a_{\vec p} e^{-ipx}+a_{\vec p}^\dagger e^{ipx})\\
%     &\pi(x)=\dot\phi(x)=\int \frac{\mathrm d^3p}{(2\pi)^3} i\sqrt{\frac{\omega}{2}}(-a_{\vec p} e^{-ipx}+a_{\vec p}^\dagger e^{ipx})
% \end{align}
% 做Fourier逆变换, 有:
% \begin{align}
%     &\Phi(\vec p)=\int \mathrm d^3x \, \phi(x)e^{-ipx}=
% \end{align}
% 施加量子化条件:
% \begin{equation}
%     [\phi(\vec x, t),\pi(\vec y, t)]=i\delta^3(x-y)
% \end{equation}

这里需要补充一点:
\begin{theorem}[Lorentz不变的体元]
    \begin{equation}
        \int\frac{\mathrm d^3p}{(2\pi)^3}\frac1{2E}
    \end{equation}
    是一个Lorentz不变量
\end{theorem}
\begin{proof}
    考虑
    \begin{equation}
        \int\frac{\mathrm d^4p}{(2\pi)^4}\Theta(p^0) 2\pi\delta(p^2-m^2)
    \end{equation}
    可以发现其等于
    \begin{equation}
        \int\frac{\mathrm d^3p}{(2\pi)^3}\frac1{2E}
    \end{equation}
\end{proof}
然后我们就可以定义真空态以及Fock空间:
\begin{definition}[真空态]
    真空态$\ket0$满足
    \begin{equation}
        a_{\vec p}\ket0=0, \quad \forall \vec p
    \end{equation}
\end{definition}
\begin{definition}[Fock空间]
    Fock空间为
    \begin{equation}
        \mathcal H=\bigoplus_{n=1}\mathcal H_n
    \end{equation}
    其中$\mathcal H_n$为$n$粒子空间, 即:
    \begin{equation}
        \mathcal H_n=\mathrm{span}\{a_{\vec p_1}^\dagger a_{\vec p_2}^\dagger \cdots a_{\vec p_n}^\dagger \ket0 \,|\, \forall \vec p_i\}
    \end{equation}
\end{definition}
接下来我们检查自由标量场的二次量子化结果与我们的经典一次量子化结果相一致.

\begin{definition}[动量本征态]
    \begin{equation}
        \ket{\vec p}=\sqrt{2\omega_{\vec p}}a^\dagger_{\vec p}\ket0
    \end{equation}
    \begin{equation}
        \ket{\vec p\vec q}=\sqrt{4\omega_{\vec p}\omega_{\vec q}}a^\dagger_{\vec p}a^\dagger_{\vec q}\ket0
    \end{equation}
\end{definition}
我们不难验证:
\begin{equation}
    \braket{\vec p|\vec q}=2\omega_{\vec p}(2\pi)^3\delta^3(\vec p-\vec q)
\end{equation}
\begin{equation}
    \braket{\vec p'\vec q'|\vec p\vec q}=4\omega_{\vec p}\omega_{\vec q}(2\pi)^6\left(\delta^3(\vec p-\vec p')\delta^3(\vec q-\vec q')+\delta^3(\vec p'-\vec q)\delta^3(\vec q'-\vec p)\right)
\end{equation}
\begin{equation}
    \vec P\ket{\vec p}=\vec p\ket{\vec p}
\end{equation}

\begin{definition}[位置本征态]
    定义位置产生算符:
    \begin{align}
        &\psi^\dagger(x)=\int\ldsq pa^\dagger_{\vec p}\mathrm e^{ipx}\\
        &\psi(x)=\int\ldsq pa_{\vec p}\mathrm e^{-ipx}
    \end{align}
    以及位置本征态
    \begin{equation}
        \ket{\vec x}=\psi^\dagger(x)\ket0
    \end{equation}
\end{definition}
我们可以发现:
\begin{equation}
    \ket{\vec x}=\psi^\dagger(x)\ket0=\int\ld{p}\exp{ipx}\ket{\vec p}
\end{equation}
\begin{equation}
    \braket{\vec p|\vec x}=\exp{ipx}
\end{equation}

可以发现与我们在QM里学的一致.

\subsection{自由复标量场量子化}
\begin{equation}
    \mathcal L=\partial_\mu\psi^*\partial^\mu\psi-m^2\psi^*\psi
\end{equation}
其中$\psi=\phi_1+i\phi_2$

\begin{align}
    \pi^\mu=\partial^\mu\psi^*
\end{align}

正则变换:
\begin{align}
    \begin{cases}
        &\Psi_{\vec k}=\frac1{\dpi3}\int\d^3x\psi_{\vec x}\exp{-i\vec k\cdot\vec x}\\
        &\Pi_{\vec k}=\int\d^3x\pi_{\vec x}\exp{i\vec k\cdot\vec x}
    \end{cases}
\end{align}

于是得到:
\begin{equation}
    H=\int\d^3k\left(\frac1{\dpi3}\Pi^\dagger_{-\vec k}\Pi_{\vec k}+\dpi3\om p^2\Psi_{-\vec k}^\dagger\Psi_\vec k\right)
\end{equation}

添加正则量子化条件:
\begin{equation}
    [\psi_{\vec x}, \pi_{\vec y}]=i\delta^3(\vec x-\vec y), \quad [\psi_\vec x, \psi_\vec y]=[\pi_\vec x, \pi_\vec y]=0
\end{equation}

于是我们得到:
\begin{align}
    [\Psi_\vec p, \Pi_\vec q]=[\Psi^\dagger_\vec p, \Pi^\dagger_\vec q]=i\delta^3(\vec p-\vec q), \quad [\Psi_\vec p, \Psi_\vec q]=[\Pi_\vec p, \Pi_\vec q]=0
\end{align}

设
\begin{align}
    &\a p=\dpi3\sqrt{\frac{\om p}2}(\Psi_\vec p+\frac i{\dpi3\om p}\Pi^\dagger_{-\vec p})\\
    &\b p=\dpi3\sqrt{\frac{\om p}2}(\Psi_{-\vec p}^\dagger+\frac i{\dpi3\om p}\Pi_{\vec p})
\end{align}

我们有:
\begin{align}
    [\a p, \a q^\dagger]&=i\dpi3\delta^3(\vec p-\vec q), \quad [\a p, \a q]=[\a p^\dagger, \a q^\dagger]=0\\
    [\b p, \b q^\dagger]&=i\dpi3\delta^3(\vec p-\vec q), \quad [\b p, \b q]=[\b p^\dagger, \b q^\dagger]=0
\end{align}

然后可以对角化Hamiltonian
\begin{equation}
    H=\int\ddd p\om p\left(\a p^\dagger\a p+\b p^\dagger\b p+\mathcal V\right)
\end{equation}


然后有:
\begin{important}
    \begin{align}
        &\psi(x)=\int\ldsq p\left(\a p\exp{-ipx}+\b p^\dagger\exp{ipx}\right)\\
        &\pi(x)=\int\ddd pi\sqrt{\frac{\om p}2}\left(\a p^\dagger\exp{ipx}-\b p\exp{-ipx}\right)
    \end{align}
\end{important}

例\ref{csU1}指出复标量场的$U(1)$对称性导致有守恒荷$Q$
\begin{equation}
    Q\equiv J_N^0=i\left[(\partial^0\psi^\dagger)\psi-(\partial^0\psi)\psi^\dagger\right]
\end{equation}

经过计算我们可以得到:
\begin{equation}
    Q_N=\int\ddd p\left(\b p^\dagger\b p-\a p^\dagger\a p\right)
\end{equation}

\kaishu 讨论: 复标量场是双自由度的, 因此它有两个产生湮灭算子$a, b$. $a, b$分别代表正粒子和反粒子的运动模式. 而从其的$U(1)$对称性导出的守恒量中可以得出结论, 正粒子与反粒子数目之差为常数.\songti

\subsection{两点关联函数}
\begin{definition}[关联函数$D(x-y)$\cite{peskinCausality}]
    \begin{equation}
        D(x-y)\equiv\braket{0|\phi(x)\phi(y)|0}
    \end{equation}
\end{definition}
计算可得:
\begin{theorem}
    \begin{align}
        D(x-y)&=\int\ldsq{p}\int\ldsq{q}\exp{iqy-ipx}\braket{0|\a p\ad q|0}\\
        &=\int\ld{p}\exp{-ip(x-y)}
    \end{align}
\end{theorem}
可以发现, 这是一个Lorentz不变量(更准确来说, 是$\mathcal P\times\rm{SO}(1,3)$的不变量, 由于存在$\Theta$它不能在$\mathcal T$下不变).

\kaishu 讨论: 两点关联函数的意义是什么? 不难证明这就是上一节中的$\braket{x|y}$, 也就是说, 它表示在时空点$y$处(假设y更早发生)激发一个粒子, 在时空点$x$处测量到它的概率密度. \songti

我们分类讨论类时、类空间隔的关联函数$D(x-y)$:\\
类时: 我们不妨假设$x^0>y^0$, 那么我们可以做Lorentz变换, 使得$\vec x'=\vec y'=0$. 设$t={x'}^0-{y'}^0$, 则:
\begin{equation}
    D(x-y)=\int\ld p\exp{-i\omega_{\vec p}t'}=\int_m^{+\infty}\sqrt{\omega^2-m^2}\exp{-i\omega t'}\frac{\d\omega}{\dpi2}
\end{equation}
对于$t'\rightarrow+\infty$
\begin{equation}
    D(x-y)\sim\exp{-imt'}
\end{equation}
类空: 我们可以做Lorentz变换, 使得$x^0=y^0=0$, 并设$\vec r=\vec x-\vec y$, 则:
\begin{align}
    D(x-y)&=\int\ld p\exp{i\vec p\cdot(\vec x-\vec y)}\\
    &=\int_0^{+\infty}\frac{2\pi p^2\d p}{\dpi3}\frac1{2\omega_{\vec p}}\int_0^\pi\exp{ipr\cos\theta}\sin\theta\d\theta\\
    &=\int_0^{+\infty}\frac{-i}{2\dpi2}\frac{\exp{ipr}-\exp{-ipr}}r\frac{p\d p}{\sqrt{p^2+m^2}}\\
    &=-\frac i{2\dpi2r}\int_{-\infty}^{+\infty}\frac{p}{\sqrt{p^2+m^2}}\exp{ipr}\d p
\end{align}
注意到, $p=\pm im$是被积分函数的两个支点, 于是将积分路径改为图\ref{fig:q2Dspace}中沿着上半部分支割线的路径. 然后可得:
\begin{equation}
    D(x-y)=\frac1{\dpi2 r}\int_m^{+\infty}\frac{\omega\exp{-\omega r}}{\sqrt{\omega^2-m^2}}\d\omega
\end{equation}
对于$r\rightarrow+\infty$
\begin{equation}
    D(x-y)\sim\exp{-mr}
\end{equation}
我们发现, 即使是类空间隔, $D(x-y)$仍不为0, 这说明$\phi(x), \phi(y)$在空间中存在重叠(overlap). 
\begin{figure}
    \centering
    \begin{tikzpicture}[x=0.75pt,y=0.75pt,yscale=-.7,xscale=.7]
        \draw  (186,155.64) -- (402,155.64)(296.16,39) -- (296.16,255) (395,150.64) -- (402,155.64) -- (395,160.64) (291.16,46) -- (296.16,39) -- (301.16,46)  ;
        %Shape: Circle [id:dp009460912696354407] 
        \draw  [fill={rgb, 255:red, 0; green, 0; blue, 0 }  ,fill opacity=1 ] (294.83,126) .. controls (294.83,125.17) and (295.5,124.5) .. (296.33,124.5) .. controls (297.16,124.5) and (297.83,125.17) .. (297.83,126) .. controls (297.83,126.83) and (297.16,127.5) .. (296.33,127.5) .. controls (295.5,127.5) and (294.83,126.83) .. (294.83,126) -- cycle ;
        %Shape: Circle [id:dp9972292459262887] 
        \draw  [fill={rgb, 255:red, 0; green, 0; blue, 0 }  ,fill opacity=1 ] (294.83,185) .. controls (294.83,184.17) and (295.5,183.5) .. (296.33,183.5) .. controls (297.16,183.5) and (297.83,184.17) .. (297.83,185) .. controls (297.83,185.83) and (297.16,186.5) .. (296.33,186.5) .. controls (295.5,186.5) and (294.83,185.83) .. (294.83,185) -- cycle ;
        %Straight Lines [id:da906582585507226] 
        \draw    (296,40.17) .. controls (297.67,41.83) and (297.68,43.5) .. (296.02,45.17) .. controls (294.36,46.84) and (294.37,48.51) .. (296.04,50.17) .. controls (297.71,51.83) and (297.72,53.5) .. (296.06,55.17) .. controls (294.4,56.84) and (294.41,58.51) .. (296.08,60.17) .. controls (297.75,61.83) and (297.76,63.5) .. (296.1,65.17) .. controls (294.44,66.84) and (294.45,68.51) .. (296.12,70.17) .. controls (297.79,71.83) and (297.8,73.5) .. (296.14,75.17) .. controls (294.48,76.84) and (294.49,78.51) .. (296.16,80.17) .. controls (297.83,81.84) and (297.83,83.5) .. (296.17,85.17) .. controls (294.51,86.84) and (294.52,88.51) .. (296.19,90.17) .. controls (297.86,91.83) and (297.87,93.5) .. (296.21,95.17) .. controls (294.55,96.84) and (294.56,98.51) .. (296.23,100.17) .. controls (297.9,101.83) and (297.91,103.5) .. (296.25,105.17) .. controls (294.59,106.84) and (294.6,108.51) .. (296.27,110.17) .. controls (297.94,111.83) and (297.95,113.5) .. (296.29,115.17) .. controls (294.63,116.84) and (294.64,118.51) .. (296.31,120.17) .. controls (297.98,121.83) and (297.99,123.5) .. (296.33,125.17) -- (296.33,126) -- (296.33,126) ;
        %Straight Lines [id:da6403683789941244] 
        \draw    (296.33,183.5) .. controls (298,185.17) and (298,186.83) .. (296.33,188.5) .. controls (294.66,190.17) and (294.66,191.83) .. (296.33,193.5) .. controls (298,195.17) and (298,196.83) .. (296.33,198.5) .. controls (294.66,200.17) and (294.66,201.83) .. (296.33,203.5) .. controls (298,205.17) and (298,206.83) .. (296.33,208.5) .. controls (294.66,210.17) and (294.66,211.83) .. (296.33,213.5) .. controls (298,215.17) and (298,216.83) .. (296.33,218.5) .. controls (294.66,220.17) and (294.66,221.83) .. (296.33,223.5) .. controls (298,225.17) and (298,226.83) .. (296.33,228.5) .. controls (294.66,230.17) and (294.66,231.83) .. (296.33,233.5) .. controls (298,235.17) and (298,236.83) .. (296.33,238.5) .. controls (294.66,240.17) and (294.66,241.83) .. (296.33,243.5) .. controls (298,245.17) and (298,246.83) .. (296.33,248.5) .. controls (294.66,250.17) and (294.66,251.83) .. (296.33,253.5) -- (296.33,255) -- (296.33,255) ;
        %Straight Lines [id:da20632335279729674] 
        \draw    (288.17,41) -- (287.83,131) ;
        %Straight Lines [id:da03237870466308923] 
        \draw    (304,40.17) -- (303.67,130.17) ;
        %Shape: Arc [id:dp3060537526358036] 
        \draw  [draw opacity=0] (303.67,129.25) .. controls (303.67,129.25) and (303.67,129.25) .. (303.67,129.25) .. controls (303.67,129.25) and (303.67,129.25) .. (303.67,129.25) .. controls (303.67,133.62) and (300.12,137.17) .. (295.75,137.17) .. controls (291.41,137.17) and (287.88,133.67) .. (287.83,129.34) -- (295.75,129.25) -- cycle ; \draw   (303.67,129.25) .. controls (303.67,129.25) and (303.67,129.25) .. (303.67,129.25) .. controls (303.67,129.25) and (303.67,129.25) .. (303.67,129.25) .. controls (303.67,133.62) and (300.12,137.17) .. (295.75,137.17) .. controls (291.41,137.17) and (287.88,133.67) .. (287.83,129.34) ;  
        %Straight Lines [id:da32051585297040974] 
        \draw    (288.33,78.83) -- (288.14,83.01) ;
        \draw [shift={(288,86)}, rotate = 272.66] [fill={rgb, 255:red, 0; green, 0; blue, 0 }  ][line width=0.08]  [draw opacity=0] (8.93,-4.29) -- (0,0) -- (8.93,4.29) -- cycle    ;
        %Straight Lines [id:da6503474180961808] 
        \draw    (304,88.5) -- (303.78,82.83) ;
        \draw [shift={(303.67,79.83)}, rotate = 87.8] [fill={rgb, 255:red, 0; green, 0; blue, 0 }  ][line width=0.08]  [draw opacity=0] (8.93,-4.29) -- (0,0) -- (8.93,4.29) -- cycle    ;    
        \draw (306.33,123.9) node [anchor=north west][inner sep=0.75pt]  [font=\small]  {$C$};
    \end{tikzpicture}
    \caption{$f(p)=\frac{p}{\sqrt{p^2+m^2}}\exp{ipr}$}\label{fig:q2Dspace}
\end{figure}

\kaishu 讨论: 类空间隔的关联函数非零能说明这违反了因果律吗?并不能, 因为所谓的因果律需要是指类空间隔的测量之间互相不影响, 也就是说交换这两个算子作用在态上的顺序不会影响结果. 这暗示我们或许应当计算两个算符的对易子来检验我们的理论是否违背了因果律, 而最直接的检验就是计算$[\phi(x), \phi(y)]$. \songti

因此我们考虑$\phi(x), \phi(y)$的对易子:
\begin{equation}
    [\phi(x), \phi(y)]=\braket{0|[\phi(x), \phi(y)]|0}=D(x-y)-D(y-x)
\end{equation}

再次尝试对类时类空间隔分类讨论:\\
类空: 首先将换参考系, 使得$x$, $y$在同一三维空间中, 然后利用Parity算符$\mathcal P$, 显然可以使得$D(x-y)\rightarrow D(y-x)$, 而$\mathcal P$操作不会改变结果, 于是我们有:
\begin{equation}
    [\phi(x), \phi(y)]=0
\end{equation}
类时: 由于$[\phi(x), \phi(y)]$不在$\mathcal T$中保持不变, 因此我们无法如类空般做到交换$x$, $y$, 因此我们不能得到$[\phi(x), \phi(y)]=0$.

可以发现, 对易子对类空间隔一定为0而对类时间隔则不一定, 这正是我们想要的因果性!

然后我们尝试进一步计算对易子
\begin{align}
    [\phi(x), \phi(y)]&=D(x-y)-D(y-x)\\
    &=\int\ld p(\exp{-ip(x-y)}-\exp{ip(x-y)})\\
    &=\int\ddd p(\frac1{2\omega_p}\exp{-ip(x-y)}+\frac1{-2\omega_p}\exp{ip(x-y)})
\end{align}
注意到对第二项做$\vec p\rightarrow-\vec p$换元结果不变, 于是:
\begin{align}
    [\phi(x), \phi(y)]&=\int\ddd p\exp{i\vec p\cdot(\vec x-\vec y)}\left(\frac{\exp{-i\omega_p(x^0-y^0)}}{2\omega_p}+\frac{\exp{i\omega_p(x^0-y^0)}}{-2\omega_p}\right)
\end{align}

观察这个形式, 令我们想到留数定理, 这两项就是留数的相加, 我们可以将其化为:
\begin{align}
    \frac{\exp{-i\omega_p(x^0-y^0)}}{2\omega_p}+\frac{\exp{i\omega_p(x^0-y^0)}}{-2\omega_p}&=\frac{1}{-2\pi i}\left(-2\pi i\rm{Res}(...)-2\pi i\rm{Res}(...)\right)\label{q2eq1}\\
    &=\frac i{2\pi}\int_C\d\omega\frac{\exp{-i\omega(x^0-y^0)}}{(\omega+\omega_{\vec p})(\omega-\omega_{\vec p})}\\
    &=\frac i{2\pi}\int_C\d\omega\frac{\exp{-i\omega(x^0-y^0)}}{\omega^2-\vec p^2-m^2}\\
    &=\int_C\frac{\d\omega}{2\pi}\frac{i}{p^2-m^2}\exp{-i\omega(x^0-y^0)}\label{q2eq2}
\end{align}
\begin{figure}
    \begin{subfigure}[b]{0.45\textwidth}
        \centering
        % \tikzset{every picture/.style={line width=0.75pt}} %set default line width to 0.75pt        
        \begin{tikzpicture}[x=0.75pt,y=0.75pt,yscale=-.7,xscale=.7]
            \draw  (108,169.55) -- (470,169.55)(285.38,21.5) -- (285.38,303.5) (463,164.55) -- (470,169.55) -- (463,174.55) (280.38,28.5) -- (285.38,21.5) -- (290.38,28.5)  ;
            \draw  [fill={rgb, 255:red, 0; green, 0; blue, 0 }  ,fill opacity=1 ] (246,170) .. controls (246,168.9) and (246.9,168) .. (248,168) .. controls (249.1,168) and (250,168.9) .. (250,170) .. controls (250,171.1) and (249.1,172) .. (248,172) .. controls (246.9,172) and (246,171.1) .. (246,170) -- cycle ;
            \draw  [fill={rgb, 255:red, 0; green, 0; blue, 0 }  ,fill opacity=1 ] (322,170) .. controls (322,168.9) and (322.9,168) .. (324,168) .. controls (325.1,168) and (326,168.9) .. (326,170) .. controls (326,171.1) and (325.1,172) .. (324,172) .. controls (322.9,172) and (322,171.1) .. (322,170) -- cycle ;
            \draw    (137,169.75) -- (203.5,169.27) ;
            \draw [shift={(206.5,169.25)}, rotate = 179.59] [fill={rgb, 255:red, 0; green, 0; blue, 0 }  ][line width=0.08]  [draw opacity=0] (8.93,-4.29) -- (0,0) -- (8.93,4.29) -- cycle    ;
            \draw    (374.5,170.25) -- (380.5,169.92) ;
            \draw [shift={(383.5,169.75)}, rotate = 176.82] [fill={rgb, 255:red, 0; green, 0; blue, 0 }  ][line width=0.08]  [draw opacity=0] (8.93,-4.29) -- (0,0) -- (8.93,4.29) -- cycle    ;
            \draw  [draw opacity=0] (152.02,170.24) .. controls (152.21,96.38) and (211.78,36.2) .. (285.85,35.47) .. controls (360.49,34.74) and (421.59,94.66) .. (422.32,169.3) .. controls (422.32,169.61) and (422.32,169.93) .. (422.33,170.24) -- (287.17,170.62) -- cycle ; \draw   (152.02,170.24) .. controls (152.21,96.38) and (211.78,36.2) .. (285.85,35.47) .. controls (360.49,34.74) and (421.59,94.66) .. (422.32,169.3) .. controls (422.32,169.61) and (422.32,169.93) .. (422.33,170.24) ;  
            \draw    (252,40.25) -- (246.75,42.55) ;
            \draw [shift={(244,43.75)}, rotate = 336.37] [fill={rgb, 255:red, 0; green, 0; blue, 0 }  ][line width=0.08]  [draw opacity=0] (8.93,-4.29) -- (0,0) -- (8.93,4.29) -- cycle    ;
            \draw  [draw opacity=0] (241.23,170) .. controls (241.23,166.32) and (244.19,163.3) .. (247.89,163.24) .. controls (251.62,163.17) and (254.7,166.15) .. (254.76,169.89) .. controls (254.77,169.94) and (254.77,170) .. (254.77,170.06) -- (248,170) -- cycle ; \draw   (241.23,170) .. controls (241.23,166.32) and (244.19,163.3) .. (247.89,163.24) .. controls (251.62,163.17) and (254.7,166.15) .. (254.76,169.89) .. controls (254.77,169.94) and (254.77,170) .. (254.77,170.06) ;  
            \draw  [draw opacity=0] (317.23,170) .. controls (317.23,166.32) and (320.19,163.3) .. (323.89,163.24) .. controls (327.62,163.17) and (330.7,166.15) .. (330.76,169.89) .. controls (330.77,169.94) and (330.77,170) .. (330.77,170.06) -- (324,170) -- cycle ; \draw   (317.23,170) .. controls (317.23,166.32) and (320.19,163.3) .. (323.89,163.24) .. controls (327.62,163.17) and (330.7,166.15) .. (330.76,169.89) .. controls (330.77,169.94) and (330.77,170) .. (330.77,170.06) ;  
            \draw (239,172.9) node [anchor=north west][inner sep=0.75pt]  [font=\footnotesize]  {$-\omega _{\vec{p}}$};
            \draw (317.76,173.4) node [anchor=north west][inner sep=0.75pt]  [font=\footnotesize]  {$\omega _{\vec{p}}$};
        \end{tikzpicture}
        \caption{$x^0<y^0$的围道}
        \label{fig:q1f1a}
    \end{subfigure}
    \hfill
    \begin{subfigure}[b]{0.45\textwidth}
        \centering
        \begin{tikzpicture}[x=0.75pt,y=0.75pt,yscale=-.7,xscale=.7]
            \draw  (108,169.55) -- (470,169.55)(285.38,21.5) -- (285.38,303.5) (463,164.55) -- (470,169.55) -- (463,174.55) (280.38,28.5) -- (285.38,21.5) -- (290.38,28.5)  ;
            \draw  [fill={rgb, 255:red, 0; green, 0; blue, 0 }  ,fill opacity=1 ] (246,170) .. controls (246,168.9) and (246.9,168) .. (248,168) .. controls (249.1,168) and (250,168.9) .. (250,170) .. controls (250,171.1) and (249.1,172) .. (248,172) .. controls (246.9,172) and (246,171.1) .. (246,170) -- cycle ;
            \draw  [fill={rgb, 255:red, 0; green, 0; blue, 0 }  ,fill opacity=1 ] (322,170) .. controls (322,168.9) and (322.9,168) .. (324,168) .. controls (325.1,168) and (326,168.9) .. (326,170) .. controls (326,171.1) and (325.1,172) .. (324,172) .. controls (322.9,172) and (322,171.1) .. (322,170) -- cycle ;
            \draw    (137,169.75) -- (203.5,169.27) ;
            \draw [shift={(206.5,169.25)}, rotate = 179.59] [fill={rgb, 255:red, 0; green, 0; blue, 0 }  ][line width=0.08]  [draw opacity=0] (8.93,-4.29) -- (0,0) -- (8.93,4.29) -- cycle    ;
            \draw    (374.5,170.25) -- (380.5,169.92) ;
            \draw [shift={(383.5,169.75)}, rotate = 176.82] [fill={rgb, 255:red, 0; green, 0; blue, 0 }  ][line width=0.08]  [draw opacity=0] (8.93,-4.29) -- (0,0) -- (8.93,4.29) -- cycle    ;
            \draw  [draw opacity=0] (422.32,169.25) .. controls (422.32,169.5) and (422.32,169.75) .. (422.32,170.01) .. controls (422.66,244.65) and (362.43,305.43) .. (287.79,305.77) .. controls (213.15,306.11) and (152.36,245.87) .. (152.02,171.23) .. controls (152.02,170.92) and (152.02,170.6) .. (152.02,170.29) -- (287.17,170.62) -- cycle ; \draw   (422.32,169.25) .. controls (422.32,169.5) and (422.32,169.75) .. (422.32,170.01) .. controls (422.66,244.65) and (362.43,305.43) .. (287.79,305.77) .. controls (213.15,306.11) and (152.36,245.87) .. (152.02,171.23) .. controls (152.02,170.92) and (152.02,170.6) .. (152.02,170.29) ;  
            \draw    (379,270.25) -- (375.3,273.33) ;
            \draw [shift={(373,275.25)}, rotate = 320.19] [fill={rgb, 255:red, 0; green, 0; blue, 0 }  ][line width=0.08]  [draw opacity=0] (8.93,-4.29) -- (0,0) -- (8.93,4.29) -- cycle    ;
            \draw  [draw opacity=0] (241.23,170) .. controls (241.23,166.32) and (244.19,163.3) .. (247.89,163.24) .. controls (251.62,163.17) and (254.7,166.15) .. (254.76,169.89) .. controls (254.77,169.94) and (254.77,170) .. (254.77,170.06) -- (248,170) -- cycle ; \draw   (241.23,170) .. controls (241.23,166.32) and (244.19,163.3) .. (247.89,163.24) .. controls (251.62,163.17) and (254.7,166.15) .. (254.76,169.89) .. controls (254.77,169.94) and (254.77,170) .. (254.77,170.06) ;  
            \draw  [draw opacity=0] (317.23,170) .. controls (317.23,166.32) and (320.19,163.3) .. (323.89,163.24) .. controls (327.62,163.17) and (330.7,166.15) .. (330.76,169.89) .. controls (330.77,169.94) and (330.77,170) .. (330.77,170.06) -- (324,170) -- cycle ; \draw   (317.23,170) .. controls (317.23,166.32) and (320.19,163.3) .. (323.89,163.24) .. controls (327.62,163.17) and (330.7,166.15) .. (330.76,169.89) .. controls (330.77,169.94) and (330.77,170) .. (330.77,170.06) ;  
            \draw (239,172.9) node [anchor=north west][inner sep=0.75pt]  [font=\footnotesize]  {$-\omega _{\vec{p}}$};
            \draw (317.76,173.4) node [anchor=north west][inner sep=0.75pt]  [font=\footnotesize]  {$\omega _{\vec{p}}$};
        \end{tikzpicture}
        \caption{$x^0>y^0$的围道}
        \label{fig:q1f1a}
    \end{subfigure}
    \caption{$D_R(x-y)$围道示意图}
    \label{fig:q2f1}
\end{figure}

为了让大圆弧不会对积分结果产生贡献, 对于$x^0<y^0$我们取围道如图\ref{fig:q1f1a}; 对于$x^0>y^0$我们取围道: \ref{fig:q1f1a}

于是我们可以定义新的关联函数
\begin{theorem}[推迟关联函数$D_R(x-y)$]
    取围道$C$如图\ref{fig:q2f1}
    \begin{equation}
        D_R(x-y)\equiv\int\dddd p \frac i{p^2-m^2}\exp{-ip(x-y)}\label{q2eq3}
    \end{equation}
    或者等价于取一个无穷小正数$\epsilon$:
    \begin{equation}
        D_R(x-y)\equiv\lim_{\epsilon\rightarrow0^+}\int\dddd p \frac i{(p+i\epsilon)^2-m^2}\exp{-ip(x-y)}
    \end{equation}
\end{theorem}
不难发现, 对于$x^0<y^0$, 围道内不含有任何pole, 故$D_R(x-y)=0$, 而对于$x^0>y^0$, 围道内含有pole, 故$D_R(x-y)=D(x-y)-D(y-x)$

于是我们发现
\begin{equation}
    D_R(x-y)=\Theta(x^0-y^0)[\phi(x), \phi(y)]
\end{equation}
这令人想起电动力学中我们学过的推迟势, 故而得名.

类似地, 我们可以定义提前关联函数
\begin{theorem}[提前关联函数$D_A(x-y)$]
    取围道$C$如图\ref{fig:q2f2}
    \begin{equation}
        D_A(x-y)\equiv\int\dddd p \frac i{p^2-m^2}\exp{-ip(x-y)}\label{q2eq4}
    \end{equation}
    或者等价于取一个无穷小正数$\epsilon$:
    \begin{equation}
        D_A(x-y)\equiv\lim_{\epsilon\rightarrow0^+}\int\dddd p \frac i{(p-i\epsilon)^2-m^2}\exp{-ip(x-y)}
    \end{equation}
\end{theorem}
我们可以发现:
\begin{equation}
    D_A(x-y)=-\Theta(y^0-x^0)[\phi(x), \phi(y)]
\end{equation}

这正与电动力学中所谓的提前势相对应.
\begin{figure}
    \begin{subfigure}[b]{0.45\textwidth}
        \centering
        \begin{tikzpicture}[x=0.75pt,y=0.75pt,yscale=-.7,xscale=.7]
            \draw  (108,169.55) -- (470,169.55)(285.38,21.5) -- (285.38,303.5) (463,164.55) -- (470,169.55) -- (463,174.55) (280.38,28.5) -- (285.38,21.5) -- (290.38,28.5)  ;
            %Flowchart: Connector [id:dp1477182047051221] 
            \draw  [fill={rgb, 255:red, 0; green, 0; blue, 0 }  ,fill opacity=1 ] (246,170) .. controls (246,168.9) and (246.9,168) .. (248,168) .. controls (249.1,168) and (250,168.9) .. (250,170) .. controls (250,171.1) and (249.1,172) .. (248,172) .. controls (246.9,172) and (246,171.1) .. (246,170) -- cycle ;
            %Flowchart: Connector [id:dp1451620126132538] 
            \draw  [fill={rgb, 255:red, 0; green, 0; blue, 0 }  ,fill opacity=1 ] (322,170) .. controls (322,168.9) and (322.9,168) .. (324,168) .. controls (325.1,168) and (326,168.9) .. (326,170) .. controls (326,171.1) and (325.1,172) .. (324,172) .. controls (322.9,172) and (322,171.1) .. (322,170) -- cycle ;
            %Straight Lines [id:da554770276238623] 
            \draw    (137,169.75) -- (203.5,169.27) ;
            \draw [shift={(206.5,169.25)}, rotate = 179.59] [fill={rgb, 255:red, 0; green, 0; blue, 0 }  ][line width=0.08]  [draw opacity=0] (8.93,-4.29) -- (0,0) -- (8.93,4.29) -- cycle    ;
            %Straight Lines [id:da2352474286511015] 
            \draw    (374.5,170.25) -- (380.5,169.92) ;
            \draw [shift={(383.5,169.75)}, rotate = 176.82] [fill={rgb, 255:red, 0; green, 0; blue, 0 }  ][line width=0.08]  [draw opacity=0] (8.93,-4.29) -- (0,0) -- (8.93,4.29) -- cycle    ;
            %Shape: Arc [id:dp2545574686176101] 
            \draw  [draw opacity=0] (152.03,171.94) .. controls (152.03,171.69) and (152.02,171.44) .. (152.02,171.19) .. controls (151.71,96.54) and (211.97,35.78) .. (286.61,35.47) .. controls (361.25,35.16) and (422.01,95.41) .. (422.32,170.05) .. controls (422.33,170.37) and (422.33,170.68) .. (422.33,171) -- (287.17,170.62) -- cycle ; \draw   (152.03,171.94) .. controls (152.03,171.69) and (152.02,171.44) .. (152.02,171.19) .. controls (151.71,96.54) and (211.97,35.78) .. (286.61,35.47) .. controls (361.25,35.16) and (422.01,95.41) .. (422.32,170.05) .. controls (422.33,170.37) and (422.33,170.68) .. (422.33,171) ;  
            %Straight Lines [id:da07283350213642492] 
            \draw    (217,55.25) -- (212.52,58.13) ;
            \draw [shift={(210,59.75)}, rotate = 327.26] [fill={rgb, 255:red, 0; green, 0; blue, 0 }  ][line width=0.08]  [draw opacity=0] (8.93,-4.29) -- (0,0) -- (8.93,4.29) -- cycle    ;
            %Shape: Arc [id:dp1734533961545065] 
            \draw  [draw opacity=0] (241.24,170.23) .. controls (241.36,174.31) and (244.42,177.55) .. (248.13,177.49) .. controls (251.86,177.42) and (254.83,174.02) .. (254.76,169.89) .. controls (254.76,169.83) and (254.76,169.77) .. (254.76,169.71) -- (248,170) -- cycle ; \draw   (241.24,170.23) .. controls (241.36,174.31) and (244.42,177.55) .. (248.13,177.49) .. controls (251.86,177.42) and (254.83,174.02) .. (254.76,169.89) .. controls (254.76,169.83) and (254.76,169.77) .. (254.76,169.71) ;  
            %Shape: Arc [id:dp1937528136867961] 
            \draw  [draw opacity=0] (317.27,169.22) .. controls (317.24,169.52) and (317.23,169.81) .. (317.24,170.11) .. controls (317.31,174.25) and (320.39,177.55) .. (324.13,177.49) .. controls (327.86,177.43) and (330.84,174.02) .. (330.76,169.89) .. controls (330.76,169.83) and (330.76,169.77) .. (330.76,169.71) -- (324,170) -- cycle ; \draw   (317.27,169.22) .. controls (317.24,169.52) and (317.23,169.81) .. (317.24,170.11) .. controls (317.31,174.25) and (320.39,177.55) .. (324.13,177.49) .. controls (327.86,177.43) and (330.84,174.02) .. (330.76,169.89) .. controls (330.76,169.83) and (330.76,169.77) .. (330.76,169.71) ;  

            % Text Node
            \draw (236,178.9) node [anchor=north west][inner sep=0.75pt]  [font=\footnotesize]  {$-\omega _{\vec{p}}$};
            % Text Node
            \draw (316.73,180.9) node [anchor=north west][inner sep=0.75pt]  [font=\footnotesize]  {$\omega _{\vec{p}}$};
        \end{tikzpicture}
        \caption{$x^0<y^0$的围道}
        \label{fig:q1f2a}
    \end{subfigure}
    \hfill
    \begin{subfigure}[b]{0.45\textwidth}
        \centering
        \begin{tikzpicture}[x=0.75pt,y=0.75pt,yscale=-.7,xscale=.7]
            \draw  (108,169.55) -- (470,169.55)(285.38,21.5) -- (285.38,303.5) (463,164.55) -- (470,169.55) -- (463,174.55) (280.38,28.5) -- (285.38,21.5) -- (290.38,28.5)  ;
            %Flowchart: Connector [id:dp1477182047051221] 
            \draw  [fill={rgb, 255:red, 0; green, 0; blue, 0 }  ,fill opacity=1 ] (246,170) .. controls (246,168.9) and (246.9,168) .. (248,168) .. controls (249.1,168) and (250,168.9) .. (250,170) .. controls (250,171.1) and (249.1,172) .. (248,172) .. controls (246.9,172) and (246,171.1) .. (246,170) -- cycle ;
            %Flowchart: Connector [id:dp1451620126132538] 
            \draw  [fill={rgb, 255:red, 0; green, 0; blue, 0 }  ,fill opacity=1 ] (322,170) .. controls (322,168.9) and (322.9,168) .. (324,168) .. controls (325.1,168) and (326,168.9) .. (326,170) .. controls (326,171.1) and (325.1,172) .. (324,172) .. controls (322.9,172) and (322,171.1) .. (322,170) -- cycle ;
            %Straight Lines [id:da554770276238623] 
            \draw    (137,169.75) -- (203.5,169.27) ;
            \draw [shift={(206.5,169.25)}, rotate = 179.59] [fill={rgb, 255:red, 0; green, 0; blue, 0 }  ][line width=0.08]  [draw opacity=0] (8.93,-4.29) -- (0,0) -- (8.93,4.29) -- cycle    ;
            %Straight Lines [id:da2352474286511015] 
            \draw    (374.5,170.25) -- (380.5,169.92) ;
            \draw [shift={(383.5,169.75)}, rotate = 176.82] [fill={rgb, 255:red, 0; green, 0; blue, 0 }  ][line width=0.08]  [draw opacity=0] (8.93,-4.29) -- (0,0) -- (8.93,4.29) -- cycle    ;
            %Shape: Arc [id:dp2545574686176101] 
            \draw  [draw opacity=0] (422.32,169.25) .. controls (422.32,169.5) and (422.32,169.75) .. (422.32,170.01) .. controls (422.66,244.65) and (362.43,305.43) .. (287.79,305.77) .. controls (213.15,306.11) and (152.36,245.87) .. (152.02,171.23) .. controls (152.02,170.92) and (152.02,170.6) .. (152.02,170.29) -- (287.17,170.62) -- cycle ; \draw   (422.32,169.25) .. controls (422.32,169.5) and (422.32,169.75) .. (422.32,170.01) .. controls (422.66,244.65) and (362.43,305.43) .. (287.79,305.77) .. controls (213.15,306.11) and (152.36,245.87) .. (152.02,171.23) .. controls (152.02,170.92) and (152.02,170.6) .. (152.02,170.29) ;  
            %Straight Lines [id:da07283350213642492] 
            \draw    (379,270.25) -- (375.3,273.33) ;
            \draw [shift={(373,275.25)}, rotate = 320.19] [fill={rgb, 255:red, 0; green, 0; blue, 0 }  ][line width=0.08]  [draw opacity=0] (8.93,-4.29) -- (0,0) -- (8.93,4.29) -- cycle    ;
            %Shape: Arc [id:dp1734533961545065] 
            \draw  [draw opacity=0] (241.24,170.23) .. controls (241.36,174.31) and (244.42,177.55) .. (248.13,177.49) .. controls (251.86,177.42) and (254.83,174.02) .. (254.76,169.89) .. controls (254.76,169.83) and (254.76,169.77) .. (254.76,169.71) -- (248,170) -- cycle ; \draw   (241.24,170.23) .. controls (241.36,174.31) and (244.42,177.55) .. (248.13,177.49) .. controls (251.86,177.42) and (254.83,174.02) .. (254.76,169.89) .. controls (254.76,169.83) and (254.76,169.77) .. (254.76,169.71) ;  
            %Shape: Arc [id:dp1937528136867961] 
            \draw  [draw opacity=0] (317.27,169.22) .. controls (317.24,169.52) and (317.23,169.81) .. (317.24,170.11) .. controls (317.31,174.25) and (320.39,177.55) .. (324.13,177.49) .. controls (327.86,177.43) and (330.84,174.02) .. (330.76,169.89) .. controls (330.76,169.83) and (330.76,169.77) .. (330.76,169.71) -- (324,170) -- cycle ; \draw   (317.27,169.22) .. controls (317.24,169.52) and (317.23,169.81) .. (317.24,170.11) .. controls (317.31,174.25) and (320.39,177.55) .. (324.13,177.49) .. controls (327.86,177.43) and (330.84,174.02) .. (330.76,169.89) .. controls (330.76,169.83) and (330.76,169.77) .. (330.76,169.71) ;  
            % Text Node
            \draw (235.5,151.4) node [anchor=north west][inner sep=0.75pt]  [font=\footnotesize]  {$-\omega _{\vec{p}}$};
            % Text Node
            \draw (316.73,152.9) node [anchor=north west][inner sep=0.75pt]  [font=\footnotesize]  {$\omega _{\vec{p}}$};    
        \end{tikzpicture}
        \caption{$x^0>y^0$的围道}
        \label{fig:q1f2b}
    \end{subfigure}
    \caption{$D_A(x-y)$围道示意图}
    \label{fig:q2f2}
\end{figure}
\begin{figure}
    \begin{subfigure}[b]{0.45\textwidth}
        \centering
        \begin{tikzpicture}[x=0.75pt,y=0.75pt,yscale=-.7,xscale=.7]
            \draw  (108,169.55) -- (470,169.55)(285.38,21.5) -- (285.38,303.5) (463,164.55) -- (470,169.55) -- (463,174.55) (280.38,28.5) -- (285.38,21.5) -- (290.38,28.5)  ;
            %Flowchart: Connector [id:dp1477182047051221] 
            \draw  [fill={rgb, 255:red, 0; green, 0; blue, 0 }  ,fill opacity=1 ] (246,170) .. controls (246,168.9) and (246.9,168) .. (248,168) .. controls (249.1,168) and (250,168.9) .. (250,170) .. controls (250,171.1) and (249.1,172) .. (248,172) .. controls (246.9,172) and (246,171.1) .. (246,170) -- cycle ;
            %Flowchart: Connector [id:dp1451620126132538] 
            \draw  [fill={rgb, 255:red, 0; green, 0; blue, 0 }  ,fill opacity=1 ] (322,170) .. controls (322,168.9) and (322.9,168) .. (324,168) .. controls (325.1,168) and (326,168.9) .. (326,170) .. controls (326,171.1) and (325.1,172) .. (324,172) .. controls (322.9,172) and (322,171.1) .. (322,170) -- cycle ;
            %Straight Lines [id:da554770276238623] 
            \draw    (137,169.75) -- (203.5,169.27) ;
            \draw [shift={(206.5,169.25)}, rotate = 179.59] [fill={rgb, 255:red, 0; green, 0; blue, 0 }  ][line width=0.08]  [draw opacity=0] (8.93,-4.29) -- (0,0) -- (8.93,4.29) -- cycle    ;
            %Straight Lines [id:da2352474286511015] 
            \draw    (374.5,170.25) -- (380.5,169.92) ;
            \draw [shift={(383.5,169.75)}, rotate = 176.82] [fill={rgb, 255:red, 0; green, 0; blue, 0 }  ][line width=0.08]  [draw opacity=0] (8.93,-4.29) -- (0,0) -- (8.93,4.29) -- cycle    ;
            %Shape: Arc [id:dp2545574686176101] 
            \draw  [draw opacity=0] (152.03,169.25) .. controls (152.75,95.5) and (212.61,35.78) .. (286.61,35.47) .. controls (360.98,35.16) and (421.58,94.98) .. (422.32,169.25) -- (287.17,170.62) -- cycle ; \draw   (152.03,169.25) .. controls (152.75,95.5) and (212.61,35.78) .. (286.61,35.47) .. controls (360.98,35.16) and (421.58,94.98) .. (422.32,169.25) ;  
            %Straight Lines [id:da07283350213642492] 
            \draw    (217,55.25) -- (212.52,58.13) ;
            \draw [shift={(210,59.75)}, rotate = 327.26] [fill={rgb, 255:red, 0; green, 0; blue, 0 }  ][line width=0.08]  [draw opacity=0] (8.93,-4.29) -- (0,0) -- (8.93,4.29) -- cycle    ;
            %Shape: Arc [id:dp1734533961545065] 
            \draw  [draw opacity=0] (241.24,170.23) .. controls (241.36,174.31) and (244.42,177.55) .. (248.13,177.49) .. controls (251.86,177.42) and (254.83,174.02) .. (254.76,169.89) .. controls (254.76,169.83) and (254.76,169.77) .. (254.76,169.71) -- (248,170) -- cycle ; \draw   (241.24,170.23) .. controls (241.36,174.31) and (244.42,177.55) .. (248.13,177.49) .. controls (251.86,177.42) and (254.83,174.02) .. (254.76,169.89) .. controls (254.76,169.83) and (254.76,169.77) .. (254.76,169.71) ;  
            %Shape: Arc [id:dp1937528136867961] 
            \draw  [draw opacity=0] (330.76,169.72) .. controls (330.6,165.64) and (327.52,162.42) .. (323.82,162.51) .. controls (320.22,162.6) and (317.35,165.79) .. (317.24,169.72) -- (324,170) -- cycle ; \draw   (330.76,169.72) .. controls (330.6,165.64) and (327.52,162.42) .. (323.82,162.51) .. controls (320.22,162.6) and (317.35,165.79) .. (317.24,169.72) ;  

            % Text Node
            \draw (236,178.9) node [anchor=north west][inner sep=0.75pt]  [font=\footnotesize]  {$-\omega _{\vec{p}}$};
            % Text Node
            \draw (317.73,179.4) node [anchor=north west][inner sep=0.75pt]  [font=\footnotesize]  {$\omega _{\vec{p}}$};
        \end{tikzpicture}
        \caption{$x^0<y^0$的围道}
        \label{fig:q1f3a}
    \end{subfigure}
    \hfill
    \begin{subfigure}[b]{0.45\textwidth}
        \centering
        \begin{tikzpicture}[x=0.75pt,y=0.75pt,yscale=-.7,xscale=.7]
            %Shape: Axis 2D [id:dp2814236532414889] 
            \draw  (108,169.55) -- (470,169.55)(285.38,21.5) -- (285.38,303.5) (463,164.55) -- (470,169.55) -- (463,174.55) (280.38,28.5) -- (285.38,21.5) -- (290.38,28.5)  ;
            %Flowchart: Connector [id:dp1477182047051221] 
            \draw  [fill={rgb, 255:red, 0; green, 0; blue, 0 }  ,fill opacity=1 ] (246,170) .. controls (246,168.9) and (246.9,168) .. (248,168) .. controls (249.1,168) and (250,168.9) .. (250,170) .. controls (250,171.1) and (249.1,172) .. (248,172) .. controls (246.9,172) and (246,171.1) .. (246,170) -- cycle ;
            %Flowchart: Connector [id:dp1451620126132538] 
            \draw  [fill={rgb, 255:red, 0; green, 0; blue, 0 }  ,fill opacity=1 ] (322,170) .. controls (322,168.9) and (322.9,168) .. (324,168) .. controls (325.1,168) and (326,168.9) .. (326,170) .. controls (326,171.1) and (325.1,172) .. (324,172) .. controls (322.9,172) and (322,171.1) .. (322,170) -- cycle ;
            %Straight Lines [id:da554770276238623] 
            \draw    (137,169.75) -- (203.5,169.27) ;
            \draw [shift={(206.5,169.25)}, rotate = 179.59] [fill={rgb, 255:red, 0; green, 0; blue, 0 }  ][line width=0.08]  [draw opacity=0] (8.93,-4.29) -- (0,0) -- (8.93,4.29) -- cycle    ;
            %Straight Lines [id:da2352474286511015] 
            \draw    (374.5,170.25) -- (380.5,169.92) ;
            \draw [shift={(383.5,169.75)}, rotate = 176.82] [fill={rgb, 255:red, 0; green, 0; blue, 0 }  ][line width=0.08]  [draw opacity=0] (8.93,-4.29) -- (0,0) -- (8.93,4.29) -- cycle    ;
            %Shape: Arc [id:dp2545574686176101] 
            \draw  [draw opacity=0] (422.32,172.01) .. controls (421.58,245.76) and (361.71,305.47) .. (287.72,305.77) .. controls (213.34,306.07) and (152.76,246.23) .. (152.03,171.96) -- (287.17,170.62) -- cycle ; \draw   (422.32,172.01) .. controls (421.58,245.76) and (361.71,305.47) .. (287.72,305.77) .. controls (213.34,306.07) and (152.76,246.23) .. (152.03,171.96) ;  
            %Straight Lines [id:da07283350213642492] 
            \draw    (361.5,283.75) -- (357.02,286.63) ;
            \draw [shift={(354.5,288.25)}, rotate = 327.26] [fill={rgb, 255:red, 0; green, 0; blue, 0 }  ][line width=0.08]  [draw opacity=0] (8.93,-4.29) -- (0,0) -- (8.93,4.29) -- cycle    ;
            %Shape: Arc [id:dp1734533961545065] 
            \draw  [draw opacity=0] (241.24,170.23) .. controls (241.36,174.31) and (244.42,177.55) .. (248.13,177.49) .. controls (251.86,177.42) and (254.83,174.02) .. (254.76,169.89) .. controls (254.76,169.83) and (254.76,169.77) .. (254.76,169.71) -- (248,170) -- cycle ; \draw   (241.24,170.23) .. controls (241.36,174.31) and (244.42,177.55) .. (248.13,177.49) .. controls (251.86,177.42) and (254.83,174.02) .. (254.76,169.89) .. controls (254.76,169.83) and (254.76,169.77) .. (254.76,169.71) ;  
            %Shape: Arc [id:dp1937528136867961] 
            \draw  [draw opacity=0] (330.76,169.72) .. controls (330.6,165.64) and (327.52,162.42) .. (323.82,162.51) .. controls (320.22,162.6) and (317.35,165.79) .. (317.24,169.72) -- (324,170) -- cycle ; \draw   (330.76,169.72) .. controls (330.6,165.64) and (327.52,162.42) .. (323.82,162.51) .. controls (320.22,162.6) and (317.35,165.79) .. (317.24,169.72) ;  

            % Text Node
            \draw (236,178.9) node [anchor=north west][inner sep=0.75pt]  [font=\footnotesize]  {$-\omega _{\vec{p}}$};
            % Text Node
            \draw (317.73,179.4) node [anchor=north west][inner sep=0.75pt]  [font=\footnotesize]  {$\omega _{\vec{p}}$};
        \end{tikzpicture}
        \caption{$x^0>y^0$的围道}
        \label{fig:q1f3b}
    \end{subfigure}
    \caption{$D_F(x-y)$围道示意图}
    \label{fig:q2f3}
\end{figure}
\begin{figure}
    \begin{subfigure}[b]{0.45\textwidth}
        \centering
        \begin{tikzpicture}[x=0.75pt,y=0.75pt,yscale=-.7,xscale=.7]
            \draw  (108,169.55) -- (470,169.55)(285.38,21.5) -- (285.38,303.5) (463,164.55) -- (470,169.55) -- (463,174.55) (280.38,28.5) -- (285.38,21.5) -- (290.38,28.5)  ;
            %Flowchart: Connector [id:dp1477182047051221] 
            \draw  [fill={rgb, 255:red, 0; green, 0; blue, 0 }  ,fill opacity=1 ] (246,170) .. controls (246,168.9) and (246.9,168) .. (248,168) .. controls (249.1,168) and (250,168.9) .. (250,170) .. controls (250,171.1) and (249.1,172) .. (248,172) .. controls (246.9,172) and (246,171.1) .. (246,170) -- cycle ;
            %Flowchart: Connector [id:dp1451620126132538] 
            \draw  [fill={rgb, 255:red, 0; green, 0; blue, 0 }  ,fill opacity=1 ] (322,170) .. controls (322,168.9) and (322.9,168) .. (324,168) .. controls (325.1,168) and (326,168.9) .. (326,170) .. controls (326,171.1) and (325.1,172) .. (324,172) .. controls (322.9,172) and (322,171.1) .. (322,170) -- cycle ;
            %Straight Lines [id:da554770276238623] 
            \draw    (137,169.75) -- (203.5,169.27) ;
            \draw [shift={(206.5,169.25)}, rotate = 179.59] [fill={rgb, 255:red, 0; green, 0; blue, 0 }  ][line width=0.08]  [draw opacity=0] (8.93,-4.29) -- (0,0) -- (8.93,4.29) -- cycle    ;
            %Straight Lines [id:da2352474286511015] 
            \draw    (374.5,170.25) -- (380.5,169.92) ;
            \draw [shift={(383.5,169.75)}, rotate = 176.82] [fill={rgb, 255:red, 0; green, 0; blue, 0 }  ][line width=0.08]  [draw opacity=0] (8.93,-4.29) -- (0,0) -- (8.93,4.29) -- cycle    ;
            %Shape: Arc [id:dp2545574686176101] 
            \draw  [draw opacity=0] (152.03,169.39) .. controls (152.68,95.64) and (212.47,35.86) .. (286.47,35.47) .. controls (360.84,35.08) and (421.5,94.84) .. (422.32,169.11) -- (287.17,170.62) -- cycle ; \draw   (152.03,169.39) .. controls (152.68,95.64) and (212.47,35.86) .. (286.47,35.47) .. controls (360.84,35.08) and (421.5,94.84) .. (422.32,169.11) ;  
            %Straight Lines [id:da07283350213642492] 
            \draw    (216,56.25) -- (211.52,59.13) ;
            \draw [shift={(209,60.75)}, rotate = 327.26] [fill={rgb, 255:red, 0; green, 0; blue, 0 }  ][line width=0.08]  [draw opacity=0] (8.93,-4.29) -- (0,0) -- (8.93,4.29) -- cycle    ;
            %Shape: Arc [id:dp1734533961545065] 
            \draw  [draw opacity=0] (254.77,170.05) .. controls (254.81,165.97) and (251.88,162.6) .. (248.18,162.52) .. controls (244.44,162.43) and (241.33,165.7) .. (241.24,169.84) .. controls (241.23,169.9) and (241.23,169.95) .. (241.23,170.01) -- (248,170) -- cycle ; \draw   (254.77,170.05) .. controls (254.81,165.97) and (251.88,162.6) .. (248.18,162.52) .. controls (244.44,162.43) and (241.33,165.7) .. (241.24,169.84) .. controls (241.23,169.9) and (241.23,169.95) .. (241.23,170.01) ;  
            %Shape: Arc [id:dp1937528136867961] 
            \draw  [draw opacity=0] (317.24,170.3) .. controls (317.41,174.38) and (320.51,177.59) .. (324.21,177.49) .. controls (327.81,177.39) and (330.66,174.19) .. (330.76,170.26) -- (324,170) -- cycle ; \draw   (317.24,170.3) .. controls (317.41,174.38) and (320.51,177.59) .. (324.21,177.49) .. controls (327.81,177.39) and (330.66,174.19) .. (330.76,170.26) ;  
            % Text Node
            \draw (236,178.9) node [anchor=north west][inner sep=0.75pt]  [font=\footnotesize]  {$-\omega _{\vec{p}}$};
            % Text Node
            \draw (317.73,179.4) node [anchor=north west][inner sep=0.75pt]  [font=\footnotesize]  {$\omega _{\vec{p}}$};
        \end{tikzpicture}
        \caption{$x^0<y^0$的围道}
        \label{fig:q1f4a}
    \end{subfigure}
    \hfill
    \begin{subfigure}[b]{0.45\textwidth}
        \centering
        \begin{tikzpicture}[x=0.75pt,y=0.75pt,yscale=-.7,xscale=.7]
            %Shape: Axis 2D [id:dp2814236532414889] 
            \draw  (108,169.55) -- (470,169.55)(285.38,21.5) -- (285.38,303.5) (463,164.55) -- (470,169.55) -- (463,174.55) (280.38,28.5) -- (285.38,21.5) -- (290.38,28.5)  ;
            %Flowchart: Connector [id:dp1477182047051221] 
            \draw  [fill={rgb, 255:red, 0; green, 0; blue, 0 }  ,fill opacity=1 ] (246,170) .. controls (246,168.9) and (246.9,168) .. (248,168) .. controls (249.1,168) and (250,168.9) .. (250,170) .. controls (250,171.1) and (249.1,172) .. (248,172) .. controls (246.9,172) and (246,171.1) .. (246,170) -- cycle ;
            %Flowchart: Connector [id:dp1451620126132538] 
            \draw  [fill={rgb, 255:red, 0; green, 0; blue, 0 }  ,fill opacity=1 ] (322,170) .. controls (322,168.9) and (322.9,168) .. (324,168) .. controls (325.1,168) and (326,168.9) .. (326,170) .. controls (326,171.1) and (325.1,172) .. (324,172) .. controls (322.9,172) and (322,171.1) .. (322,170) -- cycle ;
            %Straight Lines [id:da554770276238623] 
            \draw    (137,169.75) -- (203.5,169.27) ;
            \draw [shift={(206.5,169.25)}, rotate = 179.59] [fill={rgb, 255:red, 0; green, 0; blue, 0 }  ][line width=0.08]  [draw opacity=0] (8.93,-4.29) -- (0,0) -- (8.93,4.29) -- cycle    ;
            %Straight Lines [id:da2352474286511015] 
            \draw    (374.5,170.25) -- (380.5,169.92) ;
            \draw [shift={(383.5,169.75)}, rotate = 176.82] [fill={rgb, 255:red, 0; green, 0; blue, 0 }  ][line width=0.08]  [draw opacity=0] (8.93,-4.29) -- (0,0) -- (8.93,4.29) -- cycle    ;
            %Shape: Arc [id:dp2545574686176101] 
            \draw  [draw opacity=0] (422.32,172.01) .. controls (421.58,245.76) and (361.71,305.47) .. (287.72,305.77) .. controls (213.34,306.07) and (152.76,246.23) .. (152.03,171.96) -- (287.17,170.62) -- cycle ; \draw   (422.32,172.01) .. controls (421.58,245.76) and (361.71,305.47) .. (287.72,305.77) .. controls (213.34,306.07) and (152.76,246.23) .. (152.03,171.96) ;  
            %Straight Lines [id:da07283350213642492] 
            \draw    (361.5,283.75) -- (357.02,286.63) ;
            \draw [shift={(354.5,288.25)}, rotate = 327.26] [fill={rgb, 255:red, 0; green, 0; blue, 0 }  ][line width=0.08]  [draw opacity=0] (8.93,-4.29) -- (0,0) -- (8.93,4.29) -- cycle    ;
            %Shape: Arc [id:dp1734533961545065] 
            \draw  [draw opacity=0] (254.77,170.05) .. controls (254.81,165.97) and (251.88,162.6) .. (248.18,162.52) .. controls (244.44,162.43) and (241.33,165.7) .. (241.24,169.84) .. controls (241.23,169.9) and (241.23,169.95) .. (241.23,170.01) -- (248,170) -- cycle ; \draw   (254.77,170.05) .. controls (254.81,165.97) and (251.88,162.6) .. (248.18,162.52) .. controls (244.44,162.43) and (241.33,165.7) .. (241.24,169.84) .. controls (241.23,169.9) and (241.23,169.95) .. (241.23,170.01) ;  
            %Shape: Arc [id:dp1937528136867961] 
            \draw  [draw opacity=0] (317.24,170.3) .. controls (317.41,174.38) and (320.51,177.59) .. (324.21,177.49) .. controls (327.81,177.39) and (330.66,174.19) .. (330.76,170.26) -- (324,170) -- cycle ; \draw   (317.24,170.3) .. controls (317.41,174.38) and (320.51,177.59) .. (324.21,177.49) .. controls (327.81,177.39) and (330.66,174.19) .. (330.76,170.26) ;  
            % Text Node
            \draw (236,178.9) node [anchor=north west][inner sep=0.75pt]  [font=\footnotesize]  {$-\omega _{\vec{p}}$};
            % Text Node
            \draw (317.73,179.4) node [anchor=north west][inner sep=0.75pt]  [font=\footnotesize]  {$\omega _{\vec{p}}$};
        \end{tikzpicture}
        \caption{$x^0>y^0$的围道}
        \label{fig:q1f4b}
    \end{subfigure}
    \caption{$\widetilde{D}_F(x-y)$围道示意图}
    \label{fig:q2f4}
\end{figure}

此外, 我们还有两种取围道的方式, 由此定义两种关联函数, 也就是所谓的Feynman传播子.
% \newpage
\begin{theorem}[Feynman传播子$D_F(x-y)$]
    取围道$C$如图\ref{fig:q2f3}
    \begin{equation}
        D_F(x-y)\equiv\int\dddd p \frac i{p^2-m^2}\exp{-ip(x-y)}\label{q2eq5}
    \end{equation}
    或者等价于取一个无穷小正数$\epsilon$:
    \begin{equation}
        D_F(x-y)\equiv\lim_{\epsilon\rightarrow0^+}\int\dddd p \frac i{p^2-m^2+i\epsilon}\exp{-ip(x-y)}
    \end{equation}
\end{theorem}
\begin{theorem}[共轭Feynman传播子$\widetilde{D}_F(x-y)$]
    取围道$C$如图\ref{fig:q2f4}
    \begin{equation}
        \widetilde D_F(x-y)\equiv\int\dddd p \frac i{p^2-m^2}\exp{-ip(x-y)}\label{q2eq6}
    \end{equation}
    或者等价于取一个无穷小正数$\epsilon$:
    \begin{equation}
        \widetilde D_F(x-y)\equiv\lim_{\epsilon\rightarrow0^+}\int\dddd p \frac i{p^2-m^2-i\epsilon}\exp{-ip(x-y)}
    \end{equation}
\end{theorem}

不难发现, 对于$x^0<y^0$或者$x^0>y^0$, 这俩关联函数都存在一个pole, 并且可以计算发现:
\begin{align}
    D_F(x-y)&=\Theta(x^0-y^0)D(x-y)+\Theta(y^0-x^0)D(y-x)\\
    &=\Theta(x^0-y^0)\braket{0|\phi(x)\phi(y)|0}+\Theta(y^0-x^0)\braket{0|\phi(y)\phi(x)|0}\\
    &=\braket{0|\mathcal T\phi(x)\phi(y)|0}
\end{align}
\begin{align}
    \widetilde D_F(x-y)&=-\Theta(x^0-y^0)D(x-y)-\Theta(y^0-x^0)D(y-x)\\
    &=-\Theta(x^0-y^0)\braket{0|\phi(x)\phi(y)|0}-\Theta(y^0-x^0)\braket{0|\phi(y)\phi(x)|0}\\
    &=-\braket{0|\mathcal T\phi(x)\phi(y)|0}
\end{align}
其中$\mathcal T$为时间顺序算符
\begin{definition}[时间顺序算符$\mathcal T$]
    对于一串$\phi(x_1,x_2,\cdots,x_n)$的算符, 时间顺序算符即起到排序作用, 将时间上在后面发生的算符放在左边, 时间上在前面发生的算符放在右边, 相当于一个冒泡排序.
\end{definition}

\kaishu 注意: 从\eqref{q2eq1}到\eqref{q2eq2}只有在一定条件下取正确的围道才成立, 并不能认为是一个任何条件下都恒等的关系. 它的作用在于启发性得导出传播子$\frac i{p^2-m^2}$, 并由此根据不同的围道取法定义出四种不同的关联函数. 因此\eqref{q2eq3}、\eqref{q2eq4}、\eqref{q2eq5}、\eqref{q2eq6}并不能视为是从\eqref{q2eq2}直接推导得到的.\songti

\begin{theorem}[关联函数的性质]
    这四个关联函数$D_R(x-y)$, $D_A(x-y)$, $D_F(x-y)$, $\widetilde D_F(x-y)$都是EoM的Green函数, 即:
    \begin{equation}
        \begin{split}
            &(\Box+m^2)D_R(x-y)=-i\delta^4(x-y)\\
            &(\Box+m^2)D_A(x-y)=-i\delta^4(x-y)\\
            &(\Box+m^2)D_F(x-y)=-i\delta^4(x-y)\\
            &(\Box+m^2)\widetilde D_F(x-y)=-i\delta^4(x-y)
        \end{split}
    \end{equation}

    而$D(x-y)$则没有这个性质.
\end{theorem}
\begin{proof}
    记这四种中的某个关联函数为$D_X(x-y)$
    我们从定义出发
    \begin{equation}
        D_X(x-y)=\int\dddd p \frac i{p^2-m^2}\exp{-ip(x-y)}
    \end{equation}

    因此
    \begin{equation}
        \Box D_X(x-y)=\int\dddd p \frac i{p^2-m^2}(-p^2)\exp{-ip(x-y)}
    \end{equation}
    \begin{equation}
        m^2 D_X(x-y)=\int\dddd p \frac i{p^2-m^2}(m^2)\exp{-ip(x-y)}
    \end{equation}
    直接相加, 利用\eqref{matheq1}即得:
    \begin{equation}
        (\Box+m^2)D_X(x-y)=-i\int\dddd p\exp{-ip(x-y)}=-i\delta^4(x-y)
    \end{equation}

    而对于$D(x-y)$
    \begin{align}
        (\Box+m^2)D(x-y)&=\int\ld p(p^2+m^2)\exp{-ip(x-y)}\\
        &=m^2\int\ddd p\frac1{\omega_{\vec p}}\exp{-ip(x-y)}\neq-i\delta^4(x-y)
    \end{align}
\end{proof}
\kaishu 讨论: 这些传播子在经典场论中中对应的就是EoM的Green函数, 这就说明了它们的物理意义正是标量粒子的传播. 而$\dddd p$则暗示着在qft中它的传播是off-shell的, 存在不满足色散关系的所谓"内线"粒子. 反之, 若考虑满足色散关系的实粒子, 则我们需要在$\dddd p$后加上$\Theta(p^0)\delta(p^2-m^2)$, 或者使用三维体元$\ld p$以使其on-shell.\songti

\subsection{Wick定理}
\begin{definition}[$\mathcal N$算符]
    $\mathcal N$算符可以将输入它的一串$\a{p}$, $\a{p}^\dagger$强行变成$\a{p}^\dagger$在前$\a{p}$在后的顺序. 例如:
    \begin{equation}
        \mathcal N\left\{\a{p}\a{p}^\dagger\a{q}\a{k}^\dagger\right\}=\a{p}^\dagger\a{k}^\dagger\a{q}\a{p}
    \end{equation}
\end{definition}
\begin{definition}[Wick收缩(Wick contract)]
    \begin{equation}
        \overline{\phi(x)\phi(y)}\equiv D_F(x-y)
    \end{equation}
\end{definition}

\begin{theorem}[Wick定理]\label{wicksTheorem}
    \begin{equation}
        \mathcal T\left[\phi(x_1)\phi(x_2)\cdots\phi(x_n)\right]=\mathcal N\left\{\phi(x_1)\phi(x_2)\cdots\phi(x_n)+\text{所有的Wick收缩}\right\}
    \end{equation}
    其中, 所谓的"所有的Wick"收缩指, 任意的一对两两收缩、任意的两对两两收缩、任意的三对两两收缩等等等\\
    比如对于$n=4$
    \begin{equation}
        \begin{split}
            \mathcal T[\phi(x_1)&\phi(x_2)\phi(x_3)\phi(x_4)]
            =\mathcal N\{\phi(x_1)\phi(x_2)\phi(x_3)\phi(x_4)\\
            &+\overline{\phi(x_1)\phi(x_2)}\phi(x_3)\phi(x_4)+\overline{\phi(x_1)\phi(x_3)}\phi(x_2)\phi(x_4)\\
            &+\overline{\phi(x_1)\phi(x_4)}\phi(x_2)\phi(x_3)+\overline{\phi(x_2)\phi(x_3)}\phi(x_1)\phi(x_4)\\
            &+\overline{\phi(x_2)\phi(x_4)}\phi(x_1)\phi(x_3)+\overline{\phi(x_3)\phi(x_4)}\phi(x_1)\phi(x_2)\\
            &+\overline{\phi(x_1)\phi(x_2)}\cdot\overline{\phi(x_3)\phi(x_4)}+\overline{\phi(x_1)\phi(x_3)}\cdot\overline{\phi(x_2)\phi(x_4)}\\
            &+\overline{\phi(x_1)\phi(x_4)}\cdot\overline{\phi(x_2)\phi(x_3)}\}
        \end{split}
    \end{equation}
\end{theorem}
\begin{proof}
    \begin{lemma}\label{lemma:N_commute}
        $\mathcal N$算符可以与$[,]$换序:
        \begin{equation}
            [\a{p}, N(\cdots)]=N([\a{p}, \cdots])
        \end{equation}
    \end{lemma}
    \begin{lemma}\label{lemma:psi_commute}
        对于$x^0>y^0$
        \begin{equation}
            [\psi(x), \phi(y)]=[\psi(x), \psi^\dagger(y)]=D(x-y)=D_F(x-y)
        \end{equation}
    \end{lemma}
    不妨假设$x_1^0\geq x_2^0 \geq x_3^0\geq\cdots x_n^0$
    首先, 对于$n=2$:
    \begin{equation}
        \mathcal T[\phi(x_1)\phi(x_2)]=\mathcal N\{\phi(x_1)\phi(x_2)+\overline{\phi(x_1)\phi(x_2)}\}
    \end{equation}
    显然成立.

    考虑数学归纳法, 若对$n=k-1$成立, 则对$n=k$:
    \begin{align*}
        &\mathcal T\left[\phi(x_1)\phi(x_2)\cdots\phi(x_k)\right]=\phi(x_1)\left[\phi(x_1)\phi(x_2)\cdots\phi(x_k)\right]\\
        &=(\psi(x_1)+\psi^\dagger(x_1))\mathcal N\left\{\phi(x_2)\phi(x_3)\cdots\phi(x_n)+\text{所有的}(k-1)\text{Wick收缩}\right\}\\
        &=\mathcal N\left\{\psi^\dagger(x_1)(\phi(x_2)\phi(x_3)\cdots\phi(x_n)+\text{所有的}(k-1)\text{Wick收缩})\right\}\\
        &~~~~+\mathcal N\left\{(\phi(x_2)\phi(x_3)\cdots\phi(x_n)+\text{所有的}(k-1)\text{Wick收缩})\psi(x_1)\right\}+[\psi(x_1), \mathcal N\{\cdots\}]\\
        &=\mathcal N\left\{\phi(x_1)\phi(x_2)\phi(x_3)\cdots\phi(x_n)+\text{所有的}(k-1)\text{Wick收缩}\right\}+\mathcal N\{[\psi(x_1), \cdots]\}\\
        &=\mathcal N\left\{\phi(x_1)\phi(x_2)\phi(x_3)\cdots\phi(x_n)+\text{所有的}(k-1)\text{Wick收缩}\right\}\\
        &~~~~+\mathcal N\{\text{与}\psi(x_1)\text{收缩后的所有}(k-1)\text{Wick收缩}\}
    \end{align*}

    最终我们合并得到:
    \begin{equation}
        \mathcal T\left[\phi(x_1)\phi(x_2)\cdots\phi(x_k)\right]=\mathcal N\left\{\phi(x_1)\phi(x_2)\phi(x_3)\cdots\phi(x_n)+\text{所有的}k\text{Wick收缩}\right\}
    \end{equation}
    于是根据归纳公理我们证明得到了Wick定理.

    \kaishu 最后补充一点, 对于Wick定理的证明, 我们并没有利用到任何Free Field的性质, 因此Wick定理也是可以适用于接下来的Interaction Theory的. \songti
\end{proof}

\newpage

\section{打开相互作用: $\phi^3$理论}

在之前我们研究的都是线性没有相互作用的场论, Free Theory. 但是在自然界中, 总是存在粒子与粒子间的相互作用的, 这反映在Lagrangian上就是非二次项的出现. 这是得我们的方程不再是线性的波动方程, 不能再和我们二次量子化中的标准流程一样进行简正模的分解来获得最终结果了(悲). 为了能够将我们的理论拓展到相互作用上, Richard Feynman利用微扰展开, 通过Feynman规则\&Feynman图, 给予了我们的理论通过渐近展开处理相互作用的能力.

而在本节中, 我们将考虑标量场的$\phi^3$理论, 即
\begin{equation}
    \mathcal L=\mathcal L_0+\mathcal L_{int}=\frac12\partial_\mu\phi\partial^\mu\phi-\frac12m^2\phi^2+\frac g{3!}\phi^3
\end{equation}
其中$\mathcal L_{int}=\frac g{3!}\phi^3$为微扰的相互作用项.

\subsection{Feynman黄金规则}
Feynman黄金规则, 在于给予S矩阵元实际的物理观测效应, 也就是利用S矩阵得到物理的散射截面、衰变率等可观测量.

而所谓S算符, 其实也就是时间演化算符, 它和两个粒子态的缩并就是S矩阵的元素$\braket{f|S|i}$, 它的物理意义就是描述了初态$\ket i$经过时间演化, 处于终态$\ket f$的概率.
\begin{definition}[S算符]
    \begin{equation}
        S=\exp{\int_{-\infty}^{+\infty} iHdt}
    \end{equation}
\end{definition}

于是
\begin{equation}
    \braket{f, t=+\infty|i, t=-\infty}=\braket{f|S|i}=S_{fi}
\end{equation}

其中终态$\ket{f}$, 初态$\ket{i}$都为渐近自由态(Asymptotic State).

\begin{definition}[渐近自由态(Asymptotic State)]
    在$t=+\infty, -\infty$时在无穷远处没有互相作用的粒子即为渐近自由态. 因而渐近自由态是on-shell的, 并且可以利用Free Theory处理, 直接将$a^\dagger_{\vec p}$作用于$\ket0$即可.
\end{definition}

然后对于微扰理论来说, $S$应当非常接近$1$, 互相作用项都是微扰, 因此
\begin{equation}
    S=1+i\mathcal T
\end{equation}
(请不要和编时算符$\mathcal T$)混淆.

并且定义散射振幅$\mathcal M$
\begin{definition}[散射振幅$\mathcal M$]
    \begin{equation}
        i\mathcal T_{fi}=\braket{f|i\mathcal T|i}=i\dpi4\delta^4\left(\sum p\right) \mathcal M_{if}
    \end{equation}
\end{definition}

于是
\begin{equation}
    \braket{f|S|i}=\braket{f|(1+i\mathcal T)|i}=i\dpi4\delta^4\left(\sum p\right)\mathcal M_{if}
\end{equation}

\begin{definition}[散射截面$\sigma$]
    考虑$2\rightarrow n$散射, 对于某个终态, 其微分散射截面定义为
    \begin{equation}
        \Phi\d\sigma=\frac NT
    \end{equation}
    其中, $\Phi=\sum_v nv$为粒子通量(flux), $N$为经过时间$T$后以此状态出射的粒子数

    并且对于单色粒子流$\Phi=\frac{N_{inc}v}{\mathcal V}$, 其中$\mathcal V$为系统体积
\end{definition}

\begin{definition}\label{ch4defP}
    到某个散射态的概率
    \begin{equation}
        \d P=\frac N{N_{inc}}=\frac{\left|\braket{f|S|i}\right|^2}{\braket{f|f}\braket{i|i}}
    \end{equation}
    其中$N_{inc}$为入射粒子数.
\end{definition}

根据定义\ref{ch4defP}我们发现, 公式左边有$\d$而右边没有, 非常不方便且丑陋, 于是我们利用这个trick:
\begin{equation}
    \delta^3(0)\d^3p=1
\end{equation}

于是
\begin{theorem}
    \begin{equation}
        \d P=\frac{\left|\braket{f|S|i}\right|^2}{\braket{f|f}\braket{i|i}}=\frac{\left|\braket{f|S|i}\right|^2}{\braket{f|f}\braket{i|i}}\prod_{i\in f} \delta^3(0)\d^3p_i=\frac{\left|\braket{f|S|i}\right|^2}{\braket{f|f}\braket{i|i}}\prod_{i\in f} \frac{\mathcal V}{\dpi3}\d^3p_i
    \end{equation}
\end{theorem}

其中, 利用了式\eqref{matheq1}:
\begin{equation}
    \delta^3(0)=\frac1{\dpi3}\int \d^3x=\frac{\mathcal V}{\dpi3}
\end{equation}

同样利用式\eqref{matheq1}我们有, 
\begin{equation}
    \braket{f|f}=\prod_{i\in f} 2E_{i}\delta^3(0)=\prod_{i\in f} 2E_{i}\mathcal V
\end{equation}

于是, 我们有
\begin{align}
    \d P&=\frac{\left|\braket{f|S|i}\right|^2}{\braket{f|f}\braket{i|i}}\prod_{i\in f} \frac{\mathcal V}{\dpi3}\d^3p_i\\
    &=|\mathcal M\dpi4\delta^4(\Sigma p)|^2\frac1{2E_1\mathcal V}\frac1{2E_2\mathcal V}\prod_{i\in f}\frac1{2E_f\mathcal V} \frac{\mathcal V}{\dpi3}\d^3p_i\\
    &=|\mathcal M\dpi4\delta^4(\Sigma p)|^2\frac1{2E_1\mathcal V}\frac1{2E_2\mathcal V}\prod_{i\in f}\frac{1}{2E_f\dpi3}\d^3p_i\\
    &=\dpi4\delta^4(\Sigma p)\mathcal VT|\mathcal M|^2\frac1{2E_1\mathcal V}\frac1{2E_2\mathcal V}\prod_{i\in f}\frac{1}{2E_f\dpi3}\d^3p_i
\end{align}

根据
\begin{equation}
    \d\sigma=\frac{N}{T\Phi}=\frac{N\mathcal V}{TN_{inc}v}=\frac{\mathcal V}{T}\d P
\end{equation}

最后我们得到散射截面的Feynman黄金规则
\begin{theorem}[散射截面的Feynman黄金规则]
    \begin{equation}
        \d\sigma=\frac{|\mathcal M|^2}{4E_1E_2|\vec v_1-\vec v_2|}\d\Pi_{LIPS}
    \end{equation}
    其中$\d\Pi_{LIPS}$为Lorentz Invariance Phase Space, 
    \begin{equation}
        \d\Pi_{LIPS}=\dpi4\delta^4(\Sigma p)\prod_{i\in f}\lips p
    \end{equation}
\end{theorem}

类似地, 我们定义衰变率
\begin{definition}[衰变率$\Gamma$]
    考虑$1\rightarrow n$散射, 对于某个终态, 其微分衰变率定义为
    \begin{equation}
        \d\Gamma=\frac{\d P}T
    \end{equation}
\end{definition}

同样对于$1\rightarrow n$散射, 我们可以计算得到
\begin{equation}
    \d P=T|\mathcal M|^2\frac1{2E}\d\Pi_{LIPS}
\end{equation}

最后得到衰变率的Feynman黄金规则
\begin{theorem}[衰变率的Feynman黄金规则]
    \begin{equation}
        \d\Gamma=\frac{|\mathcal M|^2}{2E}\d\Pi_{LIPS}
    \end{equation}
\end{theorem}

\begin{example}[$2\rightarrow2$散射]
    在质心系下考虑$p_1+p_2\rightarrow p_3+p_4$散射, 则
    \begin{align}
        \d\sigma&=\int_{p_3}\frac1{4E_1E_2|\vec v_1-\vec v_2|}\dpi4\delta^4|\mathcal M|^2\lips{p_3}\lips{p_4}\\
        &=\int_{p_4}\frac1{4E_1E_2|\vec v_1-\vec v_2|}2\pi\delta(E_1+E_2-E_3-E_4)|\mathcal M|^2 \frac{p_4^2\d p_4\d\Omega}{\dpi3 4E_3E_4}
    \end{align}
    设$E_1+E_2=E_{CM}$, 并且根据
    \begin{equation}
        \delta(E_{CM}-E_3-E_4)=\delta(p_4-p_{40})\frac{E_3}{p_3}\frac{E_4}{p_4}=\delta(p_4-p_{40})\frac{E_3E_4}{p_4^2}
    \end{equation}

    我们有:
    \begin{align}
        \left(\frac{\d\sigma}{\d\Omega}\right)_{CM}=\frac1{64E_1E_2}\frac{|\mathcal M|^2}{|\vec v_1-\vec v_2|}
    \end{align}

    再根据
    \begin{equation}
        |\vec v_1-\vec v_2|=\frac{|\vec p_1|}{E_1}+\frac{|\vec p_2|}{E_1}=|\vec p_i|\frac{E_{CM}}{E_1E_2}
    \end{equation}

    于是
    \begin{equation}
        \left(\frac{\d\sigma}{\d\Omega}\right)_{CM}=\frac{|\mathcal M|^2}{64\pi^2E_{CM}|\vec p_i|}
    \end{equation}
\end{example}

\begin{example}[$1\rightarrow2$衰变]
    在质心系下考虑质量为$M$的粒子衰变为质量分别为$m_1, m_2$的粒子, $p\rightarrow p_1+p_2$.
    \begin{align}
        \d\Gamma&=\frac1{2M}\frac{\d^3p_1}{\dpi3}\frac1{2E_1}\frac{\d^3p_2}{\dpi3}\frac1{2E_2}\notag\\
        &\quad\quad\times|\mathcal M|^2\dpi3\delta^3(\vec p_1+\vec p_2)2\pi\delta(M-E_1-E_2)
    \end{align}
    于是
    \begin{align}
        \Gamma&=\frac{1}{2M}\int\frac{\d^3p_1}{\dpi3}\frac1{2E_1}\frac{\d^3p_2}{\dpi3}\frac1{2E_2}\notag\\
        &\quad\quad\times|\mathcal M|^2\dpi3\delta^3(\vec p_1+\vec p_2)2\pi\delta(M-E_1-E_2)\\
        &=\frac{1}{2M}\int\frac{\d^3p}{\dpi3}\frac{1}{4E^2}|\mathcal M|^2 2\pi\delta(M-2E)\\
        &=\frac{1}{32\pi^2M}\int\frac{p^2\d\Omega\d p}{E^2}|\mathcal M|^2\delta(M-2E)\\
        &=\frac{1}{32\pi^2M}\int\frac{p^2\d\Omega}{E^2}|\mathcal M|^2\delta(M-2E)\frac{E}{2p}\\
        &=\frac{|\mathcal M|^2}{8\pi M^2}|\vec p|
    \end{align}

    其中, 
    \begin{equation}
        \sqrt{\vec p^2+m_1^2}+\sqrt{\vec p^2+m_2^2}=M
    \end{equation}
\end{example}

得到黄金规则之后, 那么我们接下来的任务就是计算散射振幅! 而在此之前, 我们需要两个重要的定理来作为预备.

\subsection{Lehmann-Symanzik-Zimmermann公式}
有了Feynman黄金规则, 我们下一步就在于计算散射振幅$\mathcal M$. 而本节的核心就在于将位置空间的关联函数与动量空间中的散射振幅连接起来, 得出LSZ公式:
\begin{theorem}[LSZ公式]\label{LSZformula}
    \begin{equation}
        \braket{f, +\infty|i, -\infty}=\int\d^4x_1 i\exp{-ip_1x_1}(\Box_1+m^2)\cdots\int\d^4x_n i\exp{ip_nx_n}(\Box_n+m^2)\braket{\phi_1\cdots\phi_n}
    \end{equation}
    其中, 对入射态用$\int i\exp{-ip_1x_1}(\Box+m^2) $, 而对出射态用$\int i\exp{ip_1x_1}(\Box+m^2)$
\end{theorem}

为此, 我们尝试证明引理
\begin{lemma}
    \begin{equation}
        i\int\d^4x\exp{ipx}(\Box+m^2)\phi=\sqrt{2\omega_{\vec p}}[\a p(+\infty)-\a p(-\infty)]
    \end{equation}
\end{lemma}
\begin{proof}
    将$\Box$展开并分部积分有:
    \begin{align}
        i\int\d^4x\exp{ipx}(\Box+m^2)\phi&=i\int\d^4x\exp{ipx}(\partial_t^2-\nabla^2+m^2)\phi\\
        &=i\int\d^4x\exp{ipx}(\partial_t^2+\vec p^2+m^2)\phi\\
        &=i\int\d^4x\exp{ipx}(\partial_t^2+\om {p}^2)\phi
    \end{align}

    注意到, 如果是free theory, 所有粒子都是on-shell的, 积分式就为$0$. 但是对于interating theory, 事情就没这么简单了. 我们希望能类似free theory那样处理, 又该怎么办呢? 我们可以设法让这个积分只和$t=\pm\infty$的时候相关, 那这样子粒子之间相距无穷远, 互相之间没有interaction, 那么自然就可以退化到free theory, 利用free theory的产生湮灭算符来处理了.

    于是我们尝试使用分布积分:
    \begin{align}
        i\int\d^4\exp{ipx}\partial_t^2\phi&=-i\int\d^4x\partial_t\exp{ipx}\partial_t\phi+i\int\d^4x\partial_t\left[\exp{ipx}\partial_t\phi\right]\\
        &=i\int\d^4x\partial_t^2\exp{ipx}\phi-i\int\d^4x\partial_t\left[\partial_t\exp{ipx}\phi\right]+i\int\d^4x\partial_t\left[\exp{ipx}\partial_t\phi\right]\\
        &=i\int\d^4x(-\om{p}^2)\exp{ipx}\phi-i\int\d^4x\partial_t\left[\partial_t\exp{ipx}\phi\right]+i\int\d^4x\partial_t\left[\exp{ipx}\partial_t\phi\right]
    \end{align}

    那么代回到原式中, 第一项正和我们在free theory中一样, 与$+\om{p}^2$抵消了, 于是剩下
    \begin{equation}
        i\int\d^4x\exp{ipx}(\Box+m^2)\phi=-i\int\d^4x\partial_t\left[\partial_t\exp{ipx}\phi\right]+i\int\d^4x\partial_t\left[\exp{ipx}\partial_t\phi\right]
    \end{equation}

    将对t的积分专门拎出来, 我们有:
    \begin{align}
        i\int\d^4x\exp{ipx}(\Box+m^2)\phi=\left.i\int\d^3x\exp{ipx}\left(\pi-i\om p\phi\right)\right|_{t=-\infty}^{t=+\infty}
    \end{align}

    因为内部的式子是在$t=\pm\infty$时求的, 所以我们可以直接利用free theory的$\phi, \pi$(即式\eqref{ch4freephi}, \eqref{ch4freepi})进行计算, 最后我们可以得到
    \begin{equation}
        i\int\d^4x\exp{ipx}(\Box+m^2)\phi=\sqrt{2\om p}\left[\a p(+\infty)-\a p(-\infty)\right]
    \end{equation}
\end{proof}

将这个引理取共轭我们有:
\begin{equation}
    -i\int\d^4x\exp{-ipx}(\Box+m^2)\phi=\sqrt{2\om p}\left[\a{p}^\dagger(+\infty)-\a{p}^\dagger(-\infty)\right]
\end{equation}

然后我们考虑计算$\braket{f, t=-\infty|i, t=+\infty}$, 
\begin{align}
    \braket{f, +\infty|i, -\infty}&=\braket{\Omega|\prod_{p\in f}\sqrt{2\om p}\a{p}(+\infty)\prod_{p\in i}\sqrt{2\om p}\a{p}^\dagger(-\infty)|\Omega}\\
    &=\braket{\Omega|\mathcal T\left\{{\prod_{p\in f}\sqrt{2\om p}\a{p}(+\infty)\prod_{p\in i}\sqrt{2\om p}\a{p}^\dagger(-\infty)}\right\}|\Omega}
\end{align}

插入时序算符$\mathcal T$后, 我们发现, 我们再添加$\a{p}(+\infty)$, $\a{p}^\dagger(-\infty)$是不会影响结果的, 因为$\mathcal T$会将它们分别置于最左边和最右边, 然后和真空态缩并直接得到0. 于是,
\begin{align}
    \braket{f, +\infty|i, -\infty}&=\bra{\Omega}\mathcal T\prod_{p\in f}\sqrt{2\om p}\left[\a{p}(+\infty)-\a{p}(-\infty)\right]\notag\\
    &\quad\prod_{p\in i}\sqrt{2\om p}\left[\a{p}^\dagger(-\infty)-\a{p}^\dagger(+\infty)\right]\ket{\Omega}\\
    &=\braket{\Omega}\mathcal T\prod i\int\d^4x_f\exp{ip_fx_f}(\Box_f+m^2)\phi_f\notag\\
    &\quad\prod i\int\d^4x_i\exp{-ip_ix_i}(\Box_i+m^2)\phi_i\ket{\Omega}
\end{align}

于是我们最终得到了LSZ公式:
\begin{equation}
    \braket{f, +\infty|i, -\infty}=\int\d^4x_1 i\exp{-ip_1x_1}(\Box_1+m^2)\cdots\int\d^4x_n i\exp{ip_nx_n}(\Box_n+m^2)\braket{\phi_1\cdots\phi_n}
\end{equation}

\subsection{Schwinger-Dyson定理}
本节的核心在于通过为接下来微扰展开计算空间关联函数继而得到位置空间的Feynman规则做准备.
\begin{theorem}[Schwinger-Dyson定理]\label{SchwingerDysonTheorem}
    \begin{equation}
        (\Box_x+m^2)\braket{\phi_x\phi_1\cdots\phi_n}=\braket{(\Box_x+m^2)\phi_x\phi_1\cdots\phi_n}-i\sum_j\delta_{xj}\braket{\phi_1\cdots\phi_{j-1}\phi_{j+1}\cdots\phi_n}
    \end{equation}
\end{theorem}
\begin{proof}
    首先考虑$\partial_t^2\mathcal T\left\{\phi_x\phi_1\cdots\phi_n\right\}$. 我们不妨假设$\phi_1, \phi_2\cdots\phi_n$是已经按先后顺序排列好的, 即$t_1\geq t_2\geq\cdots\geq t_n$.
    
    那么
    \begin{align}
        \mathcal T\left\{\phi_x\phi_1\cdots\phi_n\right\}&=(\phi_x\phi_1\cdots\phi_n)[\Theta(t-t_1)\Theta(t-t_2)\cdots\Theta(t-t_n)]\notag\\
        &\quad+(\phi_1\phi_x\cdots\phi_n)[\Theta(t_1-t)\Theta(t-t_2)\cdots\Theta(t-t_n)]+\cdots
    \end{align}

    然后
    \begin{align}
        \partial_t\mathcal T\left\{\phi_x\phi_1\cdots\phi_n\right\}&=\mathcal T\left\{\partial_t\phi_x\phi_1\cdots\phi_n\right\}\notag\\
        &\quad+(\phi_x\phi_1\cdots\phi_n)[\delta(t-t_1)\Theta(t-t_2)\cdots\Theta(t-t_n)]\notag\\
        &\quad-(\phi_1\phi_x\cdots\phi_n)[\delta(t_1-t)\Theta(t-t_2)\cdots\Theta(t-t_n)]+\cdots\\
        &=\mathcal T\left\{\partial_t\phi_x\phi_1\cdots\phi_n\right\}\notag\\
        &\quad+([\phi_x, \phi_1]\cdots\phi_n)[\delta(t-t_1)\Theta(t-t_2)\cdots\Theta(t-t_n)]+\cdots\\
        &=\mathcal T\left\{\partial_t\phi_x\phi_1\cdots\phi_n\right\}\\
        &=\mathcal T\left\{\pi_x\phi_1\cdots\phi_n\right\}
    \end{align}

    再次求导
    \begin{align}
        \partial_t^2\mathcal T\left\{\phi_x\phi_1\cdots\phi_n\right\}&=\mathcal T\left\{\partial_t^2\phi_x\phi_1\cdots\phi_n\right\}\notag\\
        &\quad+(\pi_x\phi_1\cdots\phi_n)[\delta(t-t_1)\Theta(t-t_2)\cdots\Theta(t-t_n)]\notag\\
        &\quad-(\phi_1\pi_x\cdots\phi_n)[\delta(t_1-t)\Theta(t-t_2)\cdots\Theta(t-t_n)]+\cdots\\
        &=\mathcal T\left\{\partial_t^2\phi_x\phi_1\cdots\phi_n\right\}\notag\\
        &\quad+([\pi_x, \phi_1]\cdots\phi_n)[\delta(t-t_1)\Theta(t-t_2)\cdots\Theta(t-t_n)]+\cdots\\
        &=\mathcal T\left\{\partial_t^2\phi_x\phi_1\cdots\phi_n\right\}\notag\\
        &\quad-i\delta^3(\vec x-\vec x_1)\delta(t-t_1)(\cdots\phi_n)[\Theta(t-t_2)\cdots\Theta(t-t_n)]+\cdots\\
        &=\mathcal T\left\{\partial_t^2\phi_x\phi_1\cdots\phi_n\right\}\notag\\
        &\quad-i\delta_{x1}(\cdots\phi_n)[\Theta(t-t_2)\cdots\Theta(t-t_n)]+\cdots\\
        &=\mathcal T\left\{\partial_t^2\phi_x\phi_1\cdots\phi_n\right\}-i\sum_j \delta_{xj}\mathcal T\left\{\phi_1\cdots\phi_{j-1}\phi_{j+1}\cdots\phi_n\right\}
    \end{align}

    然后由于$\nabla^2, m^2$都作用不到只含t的$\Theta$上, 因此可以直接提入$\braket{\phi_x\phi_1\cdots\phi_n}$内, 于是我们最终得到
    \begin{equation}
        (\Box_x+m^2)\braket{\phi_x\phi_1\cdots\phi_n}=\braket{(\Box_x+m^2)\phi_x\phi_1\cdots\phi_n}-i\sum_j\delta_{xj}\braket{\phi_1\cdots\phi_{j-1}\phi_{j+1}\cdots\phi_n}
    \end{equation}
\end{proof}

\subsection{Feynman规则, 启动!}
目前, 所有的工具以及准备都已经摆在我们面前了, 我们还有什么理由停下来? \textbf{Feynman规则, 启动!} (此时配合经典门酱表情)
\subsubsection{位形空间的Feynman规则}
我们以$\braket{\phi_1\phi_2}$为例, 继而导出位置空间的Feynman规则.
\begin{align}
    \braket{\phi_1\phi_2}&=\int\d^4x\delta_{1x}\braket{\phi_x\phi_2}=\int\d^4xi(\Box_x+m^2)D_{1x}\braket{\phi_x\phi_2}\\
    &=\int\d^4xiD_{1x}(\Box_x+m^2)\braket{\phi_x\phi_2}\\
    &=\int\d^4x\frac{ig}2 D_{1x}\braket{\phi_x^2\phi_2}-i\delta_{x2}\cdot iD_{1x}\\
    &=D_{12}+\frac{ig}2\int\d^4xD_{1x}\braket{\phi_x^2\phi_2}
\end{align}

而依葫芦画瓢我们有
\begin{align}
    \braket{\phi_x^2\phi_2}&=\int\d^4 iD_{y2}(\Box_y+m^2)\braket{\phi_x^2\phi_y}\\
    &=\frac{ig}2\int\d^4y D_{2y}\braket{\phi_x^2\phi_y^2}-2i\int\d^4y \delta_{xy}iD_{y2}\braket{\phi_x}\\
    &=\frac{ig}2\int\d^4y D_{2y}\braket{\phi_x^2\phi_y^2}+2D_{x2}\braket{\phi_x}
\end{align}

梅开三度
\begin{align}
    \braket{\phi_x}&=\int\d^4y iD_{yx}(\Box_y+m^2)\braket{\phi_y}\\
    &=\frac{ig}2\int\d^4y D_{yx}\braket{\phi_y^2}
\end{align}

于是最后整理得
\begin{equation}
    \braket{\phi_1\phi_2}=D_{12}-\left(\frac g2\right)^2\int \d^4x\d^4y(2D_{1x}D_{xy}D_{xy}D_{y2}+D_{1x}D_{xx}D_{yy}D_{y2}+2D_{1x}D_{xy}D_{yy}D_{x2})
\end{equation}

我们可以用这个fancy的图来表示这些二阶贡献, 这就是Feynman图:
\begin{figure}[htbp!]
    \begin{tikzpicture}
        \draw (-0.5,0) -- (1,0) ;
        \draw (2,0) -- (3.5,0);
        \filldraw[black] (1,0) circle (1pt) node[below] {$x$} ;
        \filldraw[black] (-0.5,0) circle (1pt)node[below] {$x_1$} ;
        \filldraw[black] (2,0) circle (1pt) node[below] {$y$} ;
        \filldraw[black] (3.5,0) circle (1pt)node[below] {$x_2$}  ;
        \draw (1,0)..controls(0.4,1.2)and(1.6,1.2)..(1,0);
        \draw (2,0)..controls(1.4,1.2)and(2.6,1.2)..(2,0);
        \draw (4.2,0)--(4.5,0);
        \draw (4.35,0.15)--(4.35,-0.15);
        \draw (5.2,0)--(6.7,0);
        \draw (7.7,0)--(9.2,0);
        \filldraw[black] (6.7,0) circle (1pt);% node[below] {$x$} ;
        \filldraw[black] (6.5,0)  node[below] {$x$} ;
        \filldraw[black] (5.2,0) circle (1pt)node[below] {$x_1$} ;
        \filldraw[black] (7.7,0) circle (1pt);
        \filldraw[black] (7.9,0) node[below] {$y$} ;
        \filldraw[black] (9.2,0) circle (1pt)node[below] {$x_2$}  ;
        \draw (6.7,0) arc (180:0:0.5 and 0.5);
        \draw (6.7,0) arc (180:360:0.5 and 0.5);
        \draw (9.9,0)--(10.2,0);
        \draw(10.05,0.15)--(10.05,-0.15);
        \draw (10.9,0)--(12.4,0);
        \draw (12.4,0)--(12.4,1);
        \filldraw[black] (12.4,0) circle (1pt) node[below] {$x$} ;
        \filldraw[black] (10.9,0) circle (1pt) node[below] {$x_1$} ;
        \filldraw[black] (12.4,1) circle (1pt) node[left] {$y$} ;
        \filldraw[black] (13.9,0) circle (1pt)node[below] {$x_2$}  ;
        \draw (12.4,0)--(13.9,0);
        \draw (12.4,1)..controls(13.6,1.6)and(13.6,0.4)..(12.4,1);
    \end{tikzpicture}
\end{figure}

然后我们可以写下$\phi^3$理论在位形空间的Feynman规则
\begin{important}
    \begin{enumerate}
        \item 标出所有外点
        \item 标出所有内点, 一个内点写一个$\frac{ig}{3!}\int\d^4 x$
        \item 对所有的内线(位形空间Feynman图只有内线), 写下Feynman传播子$D_{ij}$, 其中$i, j$为内线连接的两点
        \item 乘以对称数目
    \end{enumerate}
\end{important}

对称数目的计算我们可以以图\ref{fig:scalarFeymanEx}为例来讨论\cite{TreeSymmetryFactor}.
\begin{figure}[htbp!]
    \centering
    \begin{tikzpicture}
        \draw (10.9,0)--(12.4,0);
        \draw (12.4,0)--(12.4,1);
        \filldraw[black] (12.4,0) circle (1pt) node[below] {$x$} ;
        \filldraw[black] (10.9,0) circle (1pt) node[below] {$x_1$} ;
        \filldraw[black] (12.4,1) circle (1pt) node[left] {$y$} ;
        \filldraw[black] (13.9,0) circle (1pt)node[below] {$x_2$}  ;
        \draw (12.4,0)--(13.9,0);
        \draw (12.4,1)..controls(13.6,1.6)and(13.6,0.4)..(12.4,1);
    \end{tikzpicture}
    \caption{一张Feynman图的例子}
    \label{fig:scalarFeymanEx}
\end{figure}

具体来说, 我们写下
\begin{equation}
    \braket{\phi_x\phi_x\phi_x\phi_y\phi_y\phi_y}
\end{equation}

$x$有$A_3^2=6$种选择来分别连接$x_1, x_2$, 然后$y$有$3$种选择连接$x$, 于是总共的对称数目为$18$, 乘以$\frac1{(3!)^2}$也就是$\frac12$, 与我们前述的计算结果保持一致!

那么如何理解Schwartz上\cite{schwartzSymmetryFactor}除以对称因子的做法呢? 这本质上和我们这里介绍的乘以对称数目的方法其实是一样的. 我们考虑一个更复杂的例子($\phi^4$理论)来详细阐述这个问题, Feynman图如图\ref{fig:scalarFeymanEx2}
\begin{figure}[htbp!]
    \centering
    \includegraphics[width=0.5\textwidth]{image/sec5fey1.png}
    \caption{一个更复杂点的Feynman图例子}
    \label{fig:scalarFeymanEx2}
\end{figure}

首先考虑顶点a, 将其分为三堆, 一堆一个连接1, 一堆一个连接b, 再一堆两个自己连接自己, 可以得到$C_4^1C_3^1C_2^2=\frac{4!}{2}$种分配方法.

然后再考虑顶点b, 同样将其分为三堆, 一堆一个连接a, 一堆一个连接2, 一堆两个连接c, 可以得到$C_4^1C_3^1C_2^2=\frac{4!}2$种分配方法.

最后考虑顶点c, 将其分为两堆, 一堆两个连接b, 一堆两个自己连接自己, 并且考虑到从b到c过来的线我们已经在$C_2^2$中除以了$2!$的对称因子, 所以可以认为它有两个不同的出口$b1, b2$, 然后c中选择两个按顺序分别与$b1, b2$连接, 于是有$A_4^2C_2^2=\frac{4!}2$种分配方法.

综上, 与$\frac1{(4!)^3}$相抵消,总共剩下三个$C_n^m$贡献出来的对称因子$\frac18$.

从这里可以看出, Schwartz中所称的对称因子其实也就是我们在对某一个顶点$i$的脚分堆的过程中, 对于连接顶点$j\neq i$并有$n>1$个脚的堆, $n$个脚之间没有区别所带来的需要除去的因子$n!$. 但是需要注意, 由于这一堆在顶点$i$已经除以对称因子过了, 所以我们在顶点$j$连接顶点$i$的那一堆脚中不需要再除以对称因子$n!$了. 因此, 总的来说, 一对各有$n$个脚的 从$i$到$j$的脚的堆和从$j$到$i$的脚的堆, 总共贡献$n!$的对称因子.

细心的读者可以注意到, 在这里我们没有讨论顶点$j=i$也就是自己连接自己的情况. 这个问题比较微妙, 需要我们特殊考虑, 我们可以通过下面这个例子(如图\ref{fig:scalarFeymanEx3})来进行进一步的讨论.
\begin{figure}[htbp!]
    \centering
    \includegraphics[width=0.5\textwidth]{image/sec5fey2.png}
    \caption{一个更复杂点的Feynman图例子}
    \label{fig:scalarFeymanEx3}
\end{figure}

我们对顶点$a$进行分堆, 那么我们可以分出两堆, 每堆各两个腿, 每堆的两个腿相互连接, 由于这两堆是完全等价的, 所以说我们还需要额外地除以一个对称因子$2!$来去掉多余的计数. 于是我们有总数目$\frac{C_4^2C_2^2}2=\frac{4!}{2\cdot2\cdot2}=\frac{4!}{8}$, 也就是为$8$的对称因子. 

从这里我们可以看出来, 对于有$2n$(因为是两个脚相互连接, 因此一定是偶数个自己连接自己的脚)个自己连接自己的脚的顶点, 其对称因子应当为$n!2^n$. 可以看到, 比连接不同顶点的堆多贡献了一个$2^n$的因子.

综上所述, 我们同样可以再写出Schwartz版的Feynman规则$\phi^3$理论在位形空间的Feynman规则
\begin{important}
    \begin{enumerate}
        \item 标出所有外点
        \item 标出所有内点, 一个内点写一个$ig \int\d^4 x$
        \item 对所有的内线, 写下Feynman传播子$D_{ij}$, 其中$i, j$为内线连接的两点
        \item 除以对称因子
    \end{enumerate}
\end{important}

而以如下方式进行对称因子的计算:
\begin{important}
    \begin{enumerate}
        \item 一对各有$n$个脚的 从$i$到$j$的脚的堆和从$j$到$i$的脚的堆($i\neq j$), 总共贡献$n!$的对称因子
        \item $2n$(因为是两个脚相互连接, 因此一定是偶数个自己连接自己的脚)个自己连接自己的脚的顶点, 总共贡献$n!2^n$的对称因子
        \item 将所有的对称因子相乘即总的对称因子
    \end{enumerate}
\end{important}

\kaishu 一点感悟: 最大感受就是高中数学学得最烂的排列组合还在追我= =. 写这段时, 脑子里又浮现了高中数学课上香香反复告诫我们做排列组合题, 先分堆再计算的情景(真的好喜欢好怀念我的高中数学老师!). 花了好多时间完全弄明白这个问题, 又花了很多时间在这里详细讨论这个问题并把它写清楚...希望能够解决大家学习Symmetry Factor中遇到的问题吧. \songti

\subsubsection{动量空间的Feynman规则}
虽然已经有了位形空间的Feynman规则, 但是它并不能直接给我们散射振幅$\mathcal M$, 因此我们需要通过某种方式将动量空间与位形空间连接起来, 从而处理在动量空间中的问题. 而这个工具我们已经在提出过了, 即定理\ref{LSZformula}: LSZ公式. 所以, 我们直接动手吧!

首先, 利用分部积分, 我们不难得到LSZ公式的等价形式
\begin{equation}
    \braket{f, +\infty|i, -\infty}=\int\d^4x_1 -i\exp{-ip_1x_1}(p_1^2-m^2)\cdots\int\d^4x_n -i\exp{ip_nx_n}(p_n^2-m^2)\braket{\phi_1\cdots\phi_n}
\end{equation}

代入我们上节的结果应当有:
\begin{align}
    \braket{f, +\infty|i, -\infty}&=\int\d^4x_1 -i\exp{-ip_1x_1}(p_1^2-m^2)\cdots\int\d^4x_n -i\exp{ip_nx_n}(p_n^2-m^2)\notag\\
    &\quad\times\cdots\int\frac{\d^4p_1'}{\dpi4}\frac{i}{{p'}_1^2-m^2+i\epsilon}\exp{-ip_1'(x_1-x)}\cdots
\end{align}

对第一个$\d^4x_1$积分可以产生一个$\delta$函数并干掉和外点$x_1$连接的内线的传播子, 从而将其变成外线

于是有
\begin{align}
    \braket{f, +\infty|i, -\infty}&=\int \d^4x\exp{-i(p_1+q_1+\cdots)x}\int\d^4y\exp{-i(p_1+q_1+\cdots)y}\cdots\\
    &=\int\frac i{q^2-m^2+i\epsilon}\cdots\dpi4\delta^4(\sum p)\cdots\frac{\d^4 q}{\dpi4}\cdots
\end{align}
其中$x, y\cdots$为没有被LSZ干掉的内点.

然后对$x, y\cdots$再积分, 即可获得$\delta$函数$\dpi4\delta^4(p_1+q_1+\cdots)$

于是我们可以得到动量空间的Feynman规则\cite{griffthsPPFeynman}
\begin{important}
    \begin{enumerate}
        \item 标出外线$i, f$
        \item 对所有内点写下耦合常数$ig$, 以及能动量守恒$\dpi4\delta^4(\sum p)$
        \item 对所有内线写下$\int\frac{d^4q}{\dpi4}\frac i{q^2-m^2+i\epsilon}$
        \item 将最终结果扣去一个总的能动量守恒$\dpi4\delta^4(\sum p)$以及一个多余的$i$, 即得到散射振幅$\mathcal M_{if}$
    \end{enumerate}
\end{important}

并且我们可以发现, 对于非联通图, 我们一定会产生对于外线动量$p$, 类似于$\delta^4(p)$的结果, 而外线动量是on-shell的, 不能做到为$0$, 因此LSZ还会kill掉全部的非联通图贡献.

\subsection{从拉氏密度直接到动量空间Feynman规则}
接下来我们讨论一下如何从$\mathcal L$中直接得到动量空间Feynman规则.

我们的核心问题在于如何获得顶点, 传播子是通过free field的性质就可以直接得到的. 而关于顶点, 类似于上节对对称数目的讨论, 使用直接除以对称因子的方法, 我们需要将直接出现在$\mathcal L$中的系数分别乘以$n!$, 其中$n$为某个场$\phi$乘的次数, 并且顶点会与$n$个场$\phi$传播子相连接. 而如果是使用乘以对称数目的方式, 则内点是直接乘以$\mathcal L$中原始的系数, 然后对称数目通过类似的方式求. 

这段论述可能比较抽象, 我们可以通过两个具体的例子来考虑:

\begin{example}[$\phi^4$理论]
    \begin{equation}
        \mathcal L=\frac12(\partial_\mu\phi)^2-\frac12m^2\phi^2+\frac g{4!}\phi^4
    \end{equation}
    除以对称因子法:
    \begin{enumerate}
        \item 顶点因子: $ig\dpi4\delta^4(p_1+p_2-p_3-p_4)$
        \item $\phi$传播子: $\int\frac{d^4q}{\dpi4}\frac i{q^2-m^2+i\epsilon}$
        \item 一个顶点连接四个$\phi$传播子
    \end{enumerate}
    
    乘以对称数目法:
    \begin{enumerate}
        \item 顶点: $\frac{ig}{4!}\dpi4\delta^4(p_1+p_2-p_3-p_4)$
        \item $\phi$传播子: $\int\frac{d^4q}{\dpi4}\frac i{q^2-m^2+i\epsilon}$
        \item 一个顶点连接四个$\phi$传播子
    \end{enumerate}
    其中, 数对称数目时, 一个内点$x$对应$\braket{\phi_x\phi_x\phi_x\phi_x}$
\end{example}

\begin{example}[$\phi, \Phi$粒子耦合]
    \begin{equation}
        \mathcal L=\frac12(\partial_\mu\Phi)^2-\frac12M^2\Phi^2+\frac12(\partial_\mu\phi)^2-\frac12m^2\phi^2-\frac{\mu}{2!}\Phi\phi\phi
    \end{equation}
    除以对称因子法:
    \begin{enumerate}
        \item 顶点: $-i\mu\dpi4\delta^4(p_1+p_2-p_3)$
        \item $\Phi$传播子: $\int\frac{d^4q}{\dpi4}\frac i{q^2-M^2+i\epsilon}$
        \item $\phi$传播子: $\int\frac{d^4q}{\dpi4}\frac i{q^2-m^2+i\epsilon}$
        \item 一个顶点连接一个$\Phi$传播子和两个$\phi$传播子
    \end{enumerate}

    乘以对称数目法:
    \begin{enumerate}
        \item 顶点: $-\frac{i\mu}{2!}\dpi4\delta^4(p_1+p_2-p_3)$
        \item $\Phi$传播子: $\int\frac{d^4q}{\dpi4}\frac i{q^2-M^2+i\epsilon}$
        \item $\phi$传播子: $\int\frac{d^4q}{\dpi4}\frac i{q^2-m^2+i\epsilon}$
        \item 一个顶点连接一个$\Phi$传播子和两个$\phi$传播子
    \end{enumerate}
    其中, 数对称数目时, 一个内点$x$对应$\braket{\Phi_x\phi_x\phi_x}$
\end{example}


\newpage
\section{路径积分}
在本节我们介绍另外一种导出Feynman规则的方法, 也就是Feynman所提出的路径积分方法. 需要注意, 本节我们的推导基于Heisenberger绘景.

\subsection{单粒子路径积分}
我们首先考虑经典量子力学中的路径积分表述. 所谓路径积分, 就是考虑粒子在所有可能传播路径的贡献, 然后将它们加起来. 这听起来是一个疯狂的想法, 但是我们在经典物理中可以找到一个简单的Motivation: 双缝干涉实验. 我们知道基尔霍夫衍射公式, 它将边界态的所有点发出的光线求和, 从而得到光场分布, 如果我们再进一步, 考虑光线的所有贡献路径, 那么就是路径积分.

考虑传播子$\braket{x_f, t_f|x_i, t_i}$, 在Schodinger绘景下, 可以表示为
\begin{equation}
    \braket{x_f, t_f|x_i, t_i}=\bra{x_f}\exp{-i\int_{t_i}^{t_f}\d tH(t)}\ket{x_i}
\end{equation}
.

我们将这个积分过程进行拆分, 分为$N+1$个部分, 有
\begin{equation}
    \braket{x_f, t_f|x_i, t_i}=\bra{x_f}\exp{-i\Delta tH(t_n)}\exp{-i\Delta tH(t_{n-1})}\cdots\exp{-i\Delta tH(t_1)}\exp{-i\Delta tH(t_i)}\ket{x_i}
\end{equation}
然后插入恒等算符
\begin{equation}
    \int\dx\ket{x}\bra{x}=1
\end{equation}
得到
\begin{align}
    \braket{x_f, t_f|x_i, t_i}&=\int\dx_n\dx_{n-1}\cdots\dx_1\bra{x_f}\exp{-i\Delta tH(t_n)}ket{x_n}\notag\\
    &\;\bra{x_n}\exp{-i\Delta tH(t_{n-1})}\ket{x_{n-1}}\bra{x_{n-1}}\cdots\exp{-i\Delta tH(t_1)}\ket{x_{1}}\bra{x_{1}}\exp{-i\Delta tH(t_i)}\ket{x_i}
\end{align}

然后尝试计算
\begin{align}
    \bra{x_{j+1}}\exp{-i\Delta tH(t_j)}\ket{x_j}&=\int\frac{\d p}{2\pi}\braket{x_{j+1}|p}\bra p\exp{-i\Delta t(\frac{{\hat p}^2}{2m}+V(\hat x, t_j))}\ket{x_j}
\end{align}
因为$\Delta t\approx0$, 
\begin{equation}
    \exp{-i\Delta t(\frac{{\hat p}^2}{2m}+V(\hat x, t_j))}\approx1-i\Delta t(\frac{{\hat p}^2}{2m}+V(\hat x, t))
\end{equation}
所以
\begin{align}
    \bra p\exp{-i\Delta t(\frac{{\hat p}^2}{2m}+V(\hat x, t_j))}\ket{x_j}&\approx \bra p(1-i\Delta t(\frac{{\hat p}^2}{2m}+V(\hat x, t)))\ket{x_j}\\
    &=\bra p(1-i\Delta t(\frac{p^2}{2m}+V(x_j, t_j)))\ket{x_j}\\
    &\approx\bra p\exp{-i\Delta t(\frac{p^2}{2m}+V(x_j, t_j))}\ket{x_j}
\end{align}
再代入
\begin{equation}
    \braket{x|p}=\exp{ipx}
\end{equation}
于是
\begin{align}
    \bra{x_{j+1}}\exp{-i\Delta tH(t_j)}\ket{x_j}&=\int\frac{\d p}{2\pi}\exp{-i\frac{p^2}{2m}\Delta t+ip(x_{j+1}-x_j)\Delta t}\exp{-iV(x_j,t_j)\Delta t}\\
\end{align}
再根据Gaussian积分公式
\begin{equation}
    \int_{-\infty}^{+\infty}\dx\exp{-Ax^2+Bx}=\sqrt{\frac\pi A}\exp{\frac{B^2}{4A}}
\end{equation}
我们得到最终结果
\begin{align}
    \bra p\exp{-i\Delta t(\frac{{\hat p}^2}{2m}+V(\hat x, t_j))}\ket{x_j}&=\sqrt{\frac m{2\pi i\Delta t}}\exp{i\Delta t(\frac1{2m}(\frac{x_{j+1}-x_j}{\Delta t})^2-V(x_j, t_j))}\\
    &=\sqrt{\frac m{2\pi i\Delta t}}\exp{i\Delta tL(x_{j+1}, x_j, t_j)}
\end{align}
.

所以我们得出结论
\begin{align}
    \braket{x_f, t_f|x_i, t_i}&=\lim_{N\to\infty}\left(\frac m{2\pi i\Delta t}\right)^{\frac{N+1}2}\int \dx_n\dx_{n-1}\cdots\dx_1\exp{i\sum_{j=0}^N\Delta tL(x_{j+1}, x_j, t_j)}\\
    &=\lim_{N\to\infty}\left(\frac m{2\pi i\Delta t}\right)^{\frac{N+1}2}\int \dx_n\dx_{n-1}\cdots\dx_1\exp{iS[x]}\\
    &=\int \mathcal Dx\exp{iS[x]}
\end{align}
.

\subsection{场论路径积分}
为了能够应用QM中的路径积分, 我们首先对QM与QFT中的基本元素有如下类比
\begin{table}[!htbp]
    \centering
    \begin{tabular}{c|c}
        QM & QFT\\
        \hline
        $\hat x\ket x=x\ket x$ & $\hat\phi(x)\ket\phi=\phi(x)\ket\phi$\\
        $\hat p\ket p=p\ket p$ & $\hat\pi(x)\ket\pi=\pi(x)\ket\pi$\\
        $\braket{\vec p|\vec x}=\exp{-i\vec p\cdot\vec x}$ & $\braket{\pi|\phi}=\exp{-i\int\d^3x\pi(x)\phi(x)}$\\
        $\braket{x_f, t_f|x_i, t_i}$ & $\braket{\Omega, t_f|\Omega, t_i}$
    \end{tabular}
    \caption{QM与QFT中路径积分需要用到的基本元素类比}
\end{table}
.

设拉氏量
\begin{equation}
    L=\int\d^3x\left(\frac12(\partial_t\phi)^2-V(\phi)\right)
\end{equation}
做Legdren变换于是可以得到Hamiltonian
\begin{equation}
    H=\int\d^3x\left(\frac12\pi^2+V(\phi)\right)
\end{equation}
.

然后我们对真空振幅进行完全类似的操作, 可以得到
\begin{equation}
    \braket{\Omega, t_f|\Omega, t_i}=\bra\Omega\exp{-i\Delta tH(t_n)}\exp{-i\Delta tH(t_{n-1})}\cdots\exp{-i\Delta tH(t_1)}\exp{-i\Delta tH(t_i)}\ket\Omega
\end{equation}
插入积分, 但是需要注意, 这里我们是直接对一个时间切片$t_j$下的场$\phi(\vec x, t_j)$空间分布的枚举, 这个枚举本身就是一个路径积分$\int\mathcal D\phi_j$, 所以我们有
\begin{align}
    \braket{\Omega, t_f|\Omega, t_i}&=N\int\mathcal D\phi_n\mathcal D\phi_{n-1}\cdots\mathcal D\phi_1\bra\Omega\exp{-i\Delta tH(t_n)}\ket{\phi_n}\notag\\
    &\;\bra{\phi_n}\exp{-i\Delta tH(t_{n-1})}\ket{\phi_{n-1}}\cdots\bra{\phi_1}\exp{-i\Delta tH(t_i)}\ket\Omega
\end{align}
其中$N$为一个归一化系数.

完全类似地, 我们考虑计算$\braket{\phi_{j+1}|\exp{-i\Delta tH(t_j)}|\phi_j}$的结果, 继续插入一个单位算子
\begin{equation}
    N\int\mathcal D\pi_j\ket{\pi_j}\bra{\pi_j}=1
\end{equation}
于是
\begin{equation}
    \braket{\phi_{j+1}|\exp{-i\Delta tH(t_j)}|\phi_j}=\int\mathcal D\pi_j\braket{\phi_{j+1}|\pi_j}\bra{\pi_j}\exp{-i\Delta tH(t_j)}\ket{\phi_j}
\end{equation}
这还是一个Gaussian积分, 我们可以计算得到
\begin{equation}
    \int\mathcal D\pi_j\braket{\phi_{j+1}|\pi_j}\bra{\pi_j}\exp{-i\Delta tH(t_j)}\ket{\phi_j}=N\exp{i\Delta t\left\{\int\d^3x\left(\frac{(\phi_{j+1}(x)-\phi_j(x))}{2\Delta t^2}-V(\phi_j)\right)\right\}}
\end{equation}
其中$N$还是某一归一化系数.

于是我们可以得到完全类似QM中的路径积分表达式
\begin{equation}
    \braket{\Omega, t_f|\Omega, t_i}=N\int\mathcal D\phi\exp{iS[\phi]}
\end{equation}

然后我们考虑这个式子的含义
\begin{equation}
    N\int\mathcal D\phi\exp{iS[\phi]}\phi(\vec x_j, t_j)
\end{equation}
其中$\vec x_j, t_j$为某一个时空点.

如果我们返回到路径积分的表达式中, 这也就是在切片$t_j$中插入一个$\phi(\vec x_j, t_j)$(如我们用红色高亮标出的部分)
\begin{align}
    N\int\mathcal D\phi\exp{iS[\phi]}\phi(\vec x_j, t_j)&=N\int\mathcal D\phi_n\mathcal D\phi_{n-1}\cdots\mathcal D\phi_1\notag\\
    &\;\bra\Omega\exp{-i\Delta tH(t_n)}\ket{\phi_n}\bra{\phi_n}\exp{-i\Delta tH(t_{n-1})}\ket{\phi_{n-1}}\notag\\
    &\;\cdots\textcolor{red}{\bra{\phi_{j+1}}\exp{-i\Delta tH(t_{j})}\phi_j(\vec x_j, t_j)\ket{\phi_{j}}}\cdots\bra{\phi_1}\exp{-i\Delta tH(t_i)}\ket\Omega
\end{align}
而这个等于
\begin{equation}
    \bra{\phi_{j+1}}\exp{-i\Delta tH(t_{j})}\phi(\vec x_j)\ket{\phi_{j}}
\end{equation}
还原到整体的表达式中即
\begin{align}
    N\int\mathcal D\phi\exp{iS[\phi]}\phi(\vec x_j, t_j)&=N\int\mathcal D\phi_n\mathcal D\phi_{n-1}\cdots\mathcal D\phi_1\notag\\
    &\;\bra\Omega\exp{-i\Delta tH(t_n)}\ket{\phi_n}\bra{\phi_n}\exp{-i\Delta tH(t_{n-1})}\ket{\phi_{n-1}}\notag\\
    &\;\cdots\textcolor{red}{\bra{\phi_{j+1}}\exp{-i\Delta tH(t_{j})}\phi(\vec x_j)\ket{\phi_{j}}}\cdots\bra{\phi_1}\exp{-i\Delta tH(t_i)}\ket\Omega\\
    &=\bra\Omega\exp{-i\Delta tH(t_n)}\cdots\exp{-i\Delta tH(t_{j})}\phi(\vec x_j)\exp{-i\Delta tH(t_{j-1})}\cdots\exp{-i\Delta tH(t_i)}\ket\Omega\\
    &=\bra\Omega\exp{-i(t_f-t_j)H}\phi(\vec x_j)\exp{-i(t_j-t_i)H}\ket\Omega\\
    &=\bra\Omega\exp{-it_fH}\left(\exp{it_jH}\phi(\vec x_j)\exp{-it_jH}\right)\exp{it_iH}\ket\Omega\\
    &=\braket{\Omega, t_f|\phi(x_j)|\Omega, t_i}
\end{align}

如果是插入两个$\phi$, 即
\begin{equation}
    N\int\mathcal D\phi\exp{iS[\phi]}\phi(\vec x_i, t_i)\phi(\vec x_j, t_j)
\end{equation}
. 与前面同理, 我们可以预料也就是在我们上面的式子中以先后顺序插入$\phi(\vec x_i), \phi(\vec x_j)$, 这样转换到最后算子的表达式中, 应当有结果
\begin{equation}
    N\int\mathcal D\phi\exp{iS[\phi]}\phi(\vec x_i, t_i)\phi(\vec x_j, t_j)=\braket{\Omega, t_f|\mathcal T\{\phi(x_i)\phi(x_j)\}|\Omega, t_i}
\end{equation}

而考虑归一化系数, 因为我们需要
\begin{equation}
    N\int\mathcal D\phi\exp{iS[\phi]}=\braket{\Omega, t_f|\Omega, t_i}=1
\end{equation}
所以
\begin{equation}
    N=\frac1{\int\mathcal D\phi\exp{iS[\phi]}}
\end{equation}

于是, 我们可以得到结论
\begin{equation}
    \braket{\Omega, t_f|\mathcal T\{\phi(x_1)\phi(x_2)\cdots\phi(x_n)\}|\Omega, t_i}=\frac{\int\mathcal D\phi\exp{iS[\phi]}\phi(x_1)\phi(x_2)\cdots\phi(x_n)}{\int\mathcal D\phi\exp{iS[\phi]}}
\end{equation}
或者记为简写形式
\begin{equation}
    \braket{\phi_1\phi_2\cdots\phi_n}=\frac{\int\mathcal D\phi\exp{iS[\phi]}\phi_1\phi_2\cdots\phi_n}{\int\mathcal D\phi\exp{iS[\phi]}}
\end{equation}
这个式子与我们通过Dyson级数得到的式\eqref{corrleation-dyson-series}非常相似, 待会我们将会看到, 它们的具体展开形式是完全一样的.

所以现在我们知道通过计算路径积分, 可以直接得到真空期望. 但是留这给我们一个问题:
\begin{equation}
    \int\mathcal D\phi\exp{iS[\phi]}\phi_1\phi_2\cdots\phi_n=?
\end{equation}

我们可以从统计物理中获取灵感. 这个路径积分的式子与统计物理中计算物理量期望的形式极像:
\begin{equation}
    \braket{Q}=\frac{\int\d Q\exp{-\beta H}Q}{\int\d Q\exp{-\beta H}}
\end{equation}
在统计物理中我们可以引入配分函数
\begin{equation}
    Z=\int\d Q\exp{-\beta H'}
\end{equation}
其中
\begin{equation}
    H'=H+JQ
\end{equation}
于是可以写为
\begin{equation}
    \braket{Q}=-\frac1Z\frac{\partial Z}{\partial J}\bigg|_{J=0}
\end{equation}

那么类似地, 我们也给我们的路径积分引入配分函数
\begin{equation}
    Z[J]=\int\mathcal D\phi\exp{iS[\phi]+i\int\d^4xJ(x)\phi(x)}
\end{equation}
其中$J$为一个外加的源项, $Z$是关于$J$的泛函. 那么我们同样可以写出
\begin{equation}
    \braket{\phi_1\phi_2\cdots\phi_n}=\frac{(-i)^n}{Z[0]}\frac{\delta^n Z[J]}{\delta J(x_1)\delta J(x_2)\cdots\delta J(x_n)}\bigg|_{J=0}
\end{equation}

那么剩下的问题就是如何计算配分函数. 一般来说, 配分函数并不能解析计算, 我们需要引入微扰展开. 但是对于自由场, 我们是可以将其解析算出的.

比如考虑实自由场
\begin{equation}
    \mathcal L=\frac12(\partial_\mu\phi)(\partial^\mu\phi)-\frac12m^2\phi^2=-\frac12\phi(\Box+m^2)\phi-\frac12m^2\phi^2
\end{equation}

我们定义内积
\begin{equation}
    \braket{f, g}:=\int\d^4x f(x)g(x)
\end{equation}
.
定义算子
\begin{equation}
    K:=(\Box+m^2)
\end{equation}
. 不难发现它在我们定义的内积以及边界条件($\vec x\to\infty, \phi\to\infty, t\to\pm\infty, \partial_t\phi\to0$)下, 是自伴随的
\begin{equation}
    \braket{f, Kg}=\braket{Kf, g}
\end{equation}

那么$\rm{exp}$中的指数就可以写为
\begin{equation}
    S[\phi]+\int\d^4xJ(x)\phi(x)=-\frac12\braket{\phi, K\phi}+\braket{J, \phi}
\end{equation}

考虑配平方
\begin{align}
    \braket{\phi-K^{-1}J, K(\phi-K^{-1}J)}&=\braket{\phi-K^{-1}J, K\phi-J}\\
    &=\braket{\phi, K\phi}+\braket{K^{-1}J, J}-\braket{K^{-1}J, K\phi}-\braket{\phi, J}\\
    &=\braket{\phi, K\phi}+\braket{K^{-1}J, J}-\braket{KK^{-1}J, \phi}-\braket{\phi, J}\\
    &=\braket{\phi, K\phi}-2\braket{\phi, J}+\braket{K^{-1}J, J}
\end{align}
于是我们发现
\begin{equation}
    S[\phi]+\int\d^4xJ(x)\phi(x)=-\frac12\braket{\phi-K^{-1}J, K(\phi-K^{-1}J)}+\frac12\braket{K^{-1}J, J}
\end{equation}

于是我们可以将配分函数写为
\begin{align}
    Z[J]&=\int\mathcal D\phi\exp{i\left(-\frac12\braket{\phi-K^{-1}J, K(\phi-K^{-1}J)}+\frac12\braket{K^{-1}J, J}\right)}\\
    &=\exp{\frac i2\braket{K^{-1}J, J}}\int\mathcal D\phi\exp{i\left(-\frac12\braket{\phi-K^{-1}J, K(\phi-K^{-1}J)}\right)}
\end{align}
做积分换元
\begin{equation}
    \phi\to\phi+K^{-1}J
\end{equation}
则有
\begin{align}
    Z[J]&=\exp{\frac i2\braket{K^{-1}J, J}}\int\mathcal D\phi\exp{i\left(-\frac12\braket{\phi, K\phi}\right)}\\
    &=Z[0]\exp{\frac i2\braket{K^{-1}J, J}}
\end{align}

而根据
\begin{equation}
    (\Box+m^2)D_F(x-y)=-i\delta^4(x-y)
\end{equation}
得
\begin{equation}
    (K^{-1}J)(x)=\int\d^4y iD_F(x-y)J(y)
\end{equation}
所以
\begin{align}
    Z[J]&=Z[0]\exp{-\frac12\int\d^4x\d^4y J(x)D_F(x-y)J(y)}
\end{align}

由于配分函数是无所谓前面的常数系数的, 所以最终我们可以有结论
\begin{equation}
    Z[J]=\exp{-\frac12\int\d^4x\d^4y J(x)D_F(x-y)J(y)}
\end{equation}

我们可以验证对$Z[J]$取变分确实有我们预期的结果
\begin{equation}
    \braket{\phi_1\phi_2}=(-i)^2\frac1{Z[0]}\frac{\delta^2 Z[J]}{\delta J(x_1)\delta J(x_2)}\bigg|_{J=0}=D_F(x_1-x_2)
\end{equation}

为了与接下来的相互作用理论做区分, 我们记自由场的配分函数$Z[J]$为$Z_0[J]$. 

然后我们继续考虑$\phi^3$理论, 即相互作用拉氏量
\begin{equation}
    \mathcal L=-\frac12\phi(\Box+m^2)\phi+\frac{g}3!\phi^3
\end{equation}
我们可以写出配分函数
\begin{align}
    Z[J]&=\int\mathcal D\phi \exp{i\int d^4x[-\frac12\phi(\Box+m^2)\phi+J\phi+\frac g{3!}\phi^3]}\\
    &=\int\mathcal D\phi \exp{i\int d^4x[-\frac12\phi(\Box+m^2)\phi+J\phi]}\exp{\frac{ig}{3!}\int d^4x\phi^3}
\end{align}
, 然后发现这个配分函数不能解析求解. 但是若$g\ll1$, 相互作用项可以做微扰展开, 即
\begin{equation}
    \exp{\frac{ig}{3!}\int d^4x\phi^3}\approx1+\frac{ig}{3!}\int\d^4x\phi^3(x)+\left(\frac{ig}{3!}\right)^2\int\d^4x\d^4y\phi^3(x)\phi^3(y)+\cdots
\end{equation}
于是可以微扰地计算配分函数
\begin{align}
    Z[J]&\approx\int\mathcal D\phi \exp{i\int d^4x[-\frac12\phi(\Box+m^2)\phi+J\phi]}\left(1+{\frac{ig}{3!}}^{}\int\d^4x\phi^3(x)\right.\notag\\
    &\left.\;+\left(\frac{ig}{3!}\right)^2\int\d^4x\d^4y\phi^3(x)\phi^3(y)+\cdots\right)
\end{align}
. 可以发现, 微扰的相互作用配分函数就是自由配分函数以及对自由配分函数的配分的和, 即
\begin{equation}
    Z[J]\approx Z_0[J]+\frac{ig}{3!}(-i)\int\d^4x\frac{\delta^3 Z_0}{\delta J(x)^3}+\left(\frac{ig}{3!}\right)^2(-i)^2\int\d^4x\d^4y\frac{\delta^6 Z_0}{\delta J(x)^3 \delta J(y)^3}+\cdots
\end{equation}
.

于是我们可以计算$n$点关联函数
\begin{align}
    \braket{\phi_1\phi_2\cdots\phi_n}&=\frac{(-i)^n}{Z[J]}\frac{\delta^n Z[J]}{\delta J(x_1)\delta J(x_2)\cdots\delta J(x_n)}\bigg|_{J=0}\\
    &=\frac{Z_0}{Z}\frac{(-i)^n}{Z[J]}\frac{\delta^n Z_0[J]}{\delta J(x_1)\delta J(x_2)\cdots\delta J(x_n)}\bigg|_{J=0}\\
    &=\frac{Z_0}{Z}\left[\braket{\phi_1\phi_2\cdots\phi_n}_0+\frac{ig}{3!}\int\d^4x\braket{\phi_x^3\phi_1\phi_2\cdots\phi_n}_0\right.\notag\\
    &\;\left.+\left(\frac{ig}{3!}\right)^2\int\d^4x\d^4y\braket{\phi_x^3\phi_y^3\phi_1\phi_2\cdots\phi_n}_0+\cdots\right]
\end{align}
其中$\braket{\cdots}_0$表示在自由场理论下的真空期望值
\begin{equation}
    \braket{\phi_1\phi_2\cdots\phi_n}_0:=\braket{0|\mathcal T\{\phi_1\phi_2\cdots\phi_n\}|0}
\end{equation}
. 我们发现, 对于相互作用理论的微扰展开, $n$点关联函数即自由理论的关联函数以及其更高阶关联函数的积分的和.

剩下的任务就是计算
\begin{align}
    \frac{Z}{Z_0}&=1+\frac{ig}{3!}\left(\frac{-i}{Z_0}\right)\int\d^4x\frac{\delta^3 Z_0}{\delta J(x)^3}+\left(\frac{ig}{3!}\right)^2\left(\frac{(-i)^2}{Z_0}\right)\int\d^4x\d^4y\frac{\delta^6 Z_0}{\delta J(x)^3 \delta J(y)^3}+\cdots\\
    &=\braket{0|0}_0+\frac{ig}{3!}\int\d^4x\braket{\phi_x^3}_0+\left(\frac{ig}{3!}\right)^2\int\d^4x\d^4y\braket{\phi_x^3\phi_y^3}_0+\cdots
\end{align}

我们可以用Feynman图来表示这些项, 于是有:
\begin{align}
    \frac{Z}{Z_0} = 1 + 
    % --- 第一个图 (Dumbbell) ---
    % 修改说明:
    % 1. inline=(a.center): 强制让顶点 a 的中心对齐文字基线
    % 2. baseline=-2.5pt: 这是一个微调参数,让图在等号的高度上视觉居中(因为基线实际上是在文字底部)
    \feynmandiagram [small, horizontal=a to b, inline=(a.center), baseline=-0.5ex] {
        a -- b,
        % 左边的圈:调整角度使其向左凸出
        a -- [loop, min distance=1cm, out=135, in=-135] a,
        % 右边的圈:调整角度使其向右凸出
        b -- [loop, min distance=1cm, out=45, in=-45] b,
    };
    + 
    % --- 第二个图 (Sunset) ---
    % 这里的 inline=(a.center) 同样用于确保两端点与加号对齐
    \feynmandiagram [small, horizontal=a to b, inline=(a.center), baseline=-0.5ex] {
        a -- b,
        a -- [half left] b,
        a -- [half right] b,
    };
    + \cdots
\end{align}
, 从而将多点关联函数写为(在这里我们以两点关联函数为例)
\begin{align}
    \langle \phi_1 \phi_2 \rangle = \frac{
        % --- 分子 (Numerator) ---
        \displaystyle
        % 1. 自由传播子 (Free Propagator)
        \feynmandiagram [my_diagram] { a -- b };
        + 
        % 2. 蝌蚪图 (Tadpole)
        \feynmandiagram [my_diagram] {
            a -- b -- c,
            b -- d [above=1.4cm of b], 
            d -- [loop, min distance=0.8cm] d,
        };
        + 
        % 3. 自能图/日落插入 (Self-Energy)
        \feynmandiagram [my_diagram] {
            a -- b -- [half left] c -- d,
            c -- [half left] b,
        };
        + 
        % 4. 离连图 1: 传播子 * 哑铃图 (Disconnected Dumbbell)
        \feynmandiagram [my_diagram] { a -- b };
        \; % 添加一点间距表示乘积
        \feynmandiagram [my_diagram] {
            a -- b,
            a -- [loop, min distance=0.6cm, out=135, in=-135] a,
            b -- [loop, min distance=0.6cm, out=45, in=-45] b,
            a --[draw=none] z --[draw=none] b, % 辅助对齐
        };
        + 
        % 5. 离连图 2: 传播子 * 日落图 (Disconnected Sunset)
        \feynmandiagram [my_diagram] { a -- b };
        \; 
        \feynmandiagram [my_diagram] {
            a -- b,
            a -- [half left] b,
            a -- [half right] b,
        };
        + \cdots
    }{
        % --- 分母 (Denominator) ---
        \displaystyle
        1 + 
        % 分母图 1: 日落图 (Sunset)
        \feynmandiagram [my_diagram] {
            a -- b,
            a -- [half left] b,
            a -- [half right] b,
        };
        + 
        % 分母图 2: 哑铃图 (Dumbbell)
        \feynmandiagram [my_diagram] {
            a -- b,
            a -- [loop, min distance=0.6cm, out=135, in=-135] a,
            b -- [loop, min distance=0.6cm, out=45, in=-45] b,
            a --[draw=none] z --[draw=none] b,
        };
        + \cdots
    }
\end{align}
注意到, 如果我们将分母展开, 我们可以将所有与真空图乘积的图抵消掉, 于是有最终结果
\begin{align}
    \langle \phi_1 \phi_2 \rangle = \feynmandiagram [my_diagram] { a -- b };
        + 
        \feynmandiagram [my_diagram] {
            a -- b -- c,
            b -- d [above=1.4cm of b], 
            d -- [loop, min distance=0.8cm] d,
        };
        + 
        \feynmandiagram [my_diagram] {
            a -- b -- [half left] c -- d,
            c -- [half left] b,
        };
        + 
        \feynmandiagram [my_diagram] {
            a -- b -- [loop, min distance=0.6cm, out=45, in=-45] b,
        };
        \;
        \feynmandiagram [my_diagram] {
            a -- b,
            a -- [loop, min distance=0.6cm, out=135, in=-135] a,
        };
        +\cdots
\end{align}
注意到, 如果我们将分母展开, 我们可以将所有与真空图乘积的图抵消掉, 于是有最终结果
\begin{align}
    \braket{\phi_1\phi_2} = \feynmandiagram [my_diagram] { a -- b };
        + 
        \feynmandiagram [my_diagram] {
            a -- b -- c,
            b -- d [above=1.4cm of b], 
            d -- [loop, min distance=0.8cm] d,
        };
        + 
        \feynmandiagram [my_diagram] {
            a -- b -- [half left] c -- d,
            c -- [half left] b,
        };
        + 
        \feynmandiagram [my_diagram] {
            a -- b -- [loop, min distance=0.6cm, out=45, in=-45] b,
        };
        \;
        \feynmandiagram [my_diagram] {
            a -- b,
            a -- [loop, min distance=0.6cm, out=135, in=-135] a,
        };
        +\cdots
\end{align}

可以发现, 在这里我们的具体展开与推导都和Dyson级数的推导方式完全一样. 至此为止, 三种处理相互作用体系的微扰方法: Schwinger-Dyson定理, Dyson级数以及路径积分我们都介绍完毕了, 从最终的结果上看, 它们都殊途同归, 得到了完全一致的结果.

\subsection{路径积分在统计物理中的应用}
在上节中我们看到, 路径积分具有和和统计物理配分函数完全类似的形式, 更确切地说, 其实从本质上来说, 对于相互作用体系, 统计物理的微扰处理方法和路径积分的做法完全是一样的, 只是将$iS$替换为了$-\beta H$罢了, 本质上都是对一个场的路径积分. 于是在这里, 我想给出两个统计物理中的例子\cite{SchroederClusterExpansion}\cite{LiuChuanIsing}, 用来说明路径积分在统计物理中处理相互作用体系的作用, 并给出完全类似的Feynman规则.
\subsubsection{弱相互作用气体}

\subsubsection{铁磁Ising模型的Landau自由能模型}

\newpage
\section{电磁场}
\subsection{重电磁场(Massive)}
我们有拉格朗日量
\begin{equation}
    \mathcal L=-\frac14 F_{\mu\nu}F^{\mu\nu}+\frac12 m^2 A_\mu A^\mu
\end{equation}
其中$A^\mu=(\phi, \vec A)\Rightarrow A_\mu=(\phi, -\vec A)$

于是我们得到EoM:
\begin{equation}
    \partial_\mu F^{\mu\nu}+m^2 A^\nu=0
\end{equation}

将$\partial_\nu$作用到EoM, 并且因为$F^{\mu\nu}$反称, $\partial_\mu\partial_\nu F^{\mu\nu}=0$, 我们得到:
\begin{equation}
    m^2\partial_\nu A^\nu=0
\end{equation}

于是我们可以得到Proca方程
\begin{equation}
    (\partial^2+m^2)A^\nu=0
\end{equation}

正则共轭
\begin{equation}
    \Pi^{\mu\nu}=\pa{\mathcal L}{(\partial_\mu A_\nu)}=-F^{\mu\nu}=\begin{bmatrix}
        0 & E^1 & E^2 & E^3\\
        -E^1 & 0 & -B^3 & B^2\\
        -E^2 & B^3 & 0 & -B^1\\
        -E^3 & -B^2 & B^1 & 0
    \end{bmatrix}
\end{equation}

即$\Pi^{00}=0$, $\Pi^{0i}=E^i$

然后我们有
\begin{equation}
    \mathcal L=-\frac12(\vec B^2-\vec E^2)+\frac12m^2A_{\mu}A^{\mu}
\end{equation}

做Legendre变换, 得到哈密顿密度
\begin{align}
    \mathcal H&=\Pi^0_{~~i}\dot A^i-\mathcal L=-\Pi^{0i}\partial_t A^i-\mathcal L\\
    &=\vec E\cdot\partial_t\vec A+\frac12 \vec  B^2-\frac12 \vec  E^2-\frac12m^2A_{\mu}A^{\mu}\\
    &=-\vec E\cdot\partial_t\vec A+\frac12 \vec B^2-\frac12 \vec  E^2-\frac12m^2\phi^2+\frac12m^2\vec A^2
\end{align}

根据
\begin{equation}
    \vec E\cdot\nabla\phi=\nabla\cdot(\phi\vec E)-\phi\nabla\cdot\vec E
\end{equation}
以及EoM
\begin{equation}
    \nabla\cdot\vec E=-m^2\phi
\end{equation}
我们最终得到
\begin{equation}
    \mathcal H=\frac12\vec B^2+\frac12\vec E^2+\frac1{2m^2}(\nabla\cdot\vec E)^2+\frac12m^2\vec A^2
\end{equation}

然后我们尝试量子化, 利用正则量子化关系
\begin{equation}
    [A^i_{\vec x}, \Pi^{0j}_{\vec y}]=[A^i_{\vec x}, E^j_{\vec y}]=i\delta^{ij}\delta^3(\vec x-\vec y)=-ig^{ij}\delta^3(\vec x-\vec y)
\end{equation}

我们做Fourier变换, 得到
\begin{align}
    &A^\mu=\int\ldsq{p}\sum_{\lambda=1}^3\left(\epsilon_\lambda^\nu a_{\lambda\vec p}\exp{-ipx}+\epsilon_\lambda^{\mu*}a^\dagger_{\lambda\vec p}\exp{ipx}\right)\\
    &E^\mu=\int\ldsq{p}\sum_{\lambda=1}^3\left((p^\mu\epsilon_\lambda^{0*}-p^0\epsilon_\lambda^{\mu*})a^\dagger_{\lambda\vec p}\exp{ipx}-(p^\mu\epsilon^0_\lambda-p^0\epsilon_\lambda^\mu)a_{\lambda\vec p}\exp{-ipx}\right)
\end{align}

于是有对易子:
\begin{equation}
    [a_{\lambda\vec p}, a^\dagger_{\lambda'\vec p'}]=\dpi3\delta^3(\vec p-\vec p')\delta_{\lambda\lambda'}
\end{equation}

此外, 我们还可以得到
\begin{align}
    &\nabla\cdot\vec E=-m^2\int\ldsq p \sum_\lambda\left(\epsilon_\lambda^0a^\dagger_{\lambda\vec p}\exp{ipx}+\epsilon_\lambda^0a_{\lambda\vec p}\exp{-ipx}\right)\\
    &\vec B=\nabla\times\vec A=\int\ldsq p i\sum_\lambda\left((\vec p\times\vec\epsilon_\lambda)a_{\lambda\vec p}^\dagger\exp{ipx}-(\vec p\times\vec\epsilon_\lambda)a_{\lambda\vec p}\exp{-ipx}\right)
\end{align}

于是经过艰苦卓绝的爆算, 我们得到
\begin{align}
    H&=\int\d^3x\frac12(E^2+B^2+m^2A^2+\frac1{m^2}(\nabla\cdot\vec E)^2)\\
    &=\frac12\int\ld p\sum_\lambda\sum_{\lambda'}(\cdots)\\
    &=\frac12\int\ld p\sum_\lambda\sum_{\lambda'}(-m^2\epsilon_\lambda^0\epsilon_{\lambda'}^0+\omega^2\delta_{\lambda\lambda'}-m^2\epsilon^0_{\lambda}\epsilon^0_{\lambda'}\notag\\
    &\quad\quad+\vec p^2\delta_{\lambda\lambda'}+2m^2\epsilon_{\lambda'}^0\epsilon_\lambda^0+m^2\delta_{\lambda\lambda'})(a^\dagger_{\lambda\vec p}a_{\lambda'\vec p}+a_{\lambda'\vec p}a^\dagger_{\lambda\vec p})\\
    &=\int\ddd p\om p\sum_\lambda\left(a^\dagger_{\lambda\vec p}a_{\lambda\vec p}+\frac12\mathcal V\right)
\end{align}

(感兴趣的可以见图\ref{fig:massiveEDQ}的手动具体计算过程)

\begin{figure}[htbp!]
    \centering
    \includegraphics[width=0.8\textwidth]{image/massiveEM1.jpg}
    \includegraphics[width=0.8\textwidth]{image/massiveEM2.jpg}
    \caption{手算过程}
    \label{fig:massiveEDQ}
\end{figure}

写出$H$后, 我们尝试通过计算其Green函数继而得到重电磁场传播子. 

在傅里叶空间, 我们有方程:
\begin{equation}
    (-p^2+m^2)g_{\mu\nu}\tilde{A}^\mu=\tilde{J}_\nu
\end{equation}

于是
\begin{equation}
    \tilde{A}^\mu=\frac{-g^{\mu\nu}}{p^2-m^2} \tilde{J}_\nu=\frac{-g^{\mu\nu}}{p^2-m^2}\int d^4y J_\nu\exp{ipy}
\end{equation}

最终我们得到:
\begin{equation}
    A^\mu=\int\dddd p\tilde{A}^\mu\exp{-ipx}=\int\d^4y\left(\int\dddd p\frac{-g^{\mu\nu}}{p^2-m^2}\exp{-ip(x-y)}\right)J_\nu
\end{equation}

于是有Green函数:
\begin{equation}
    G_{\mu\nu}(x-y)=\int\dddd p\frac{-g^{\mu\nu}}{p^2-m^2}\exp{-ip(x-y)}
\end{equation}

因此传播子为:
\begin{equation}
    \frac{-ig_{\mu\nu}}{p^2-m^2+i\epsilon}
\end{equation}

\subsection{电磁场(Massless)}
为了得到电磁场的二次量子化结果, 自然的想法就是对电磁场取$m\rightarrow0$的结果. 然而这会面临一个问题: 电磁场的$A^\mu$具有规范不变性, 即$A^\mu\rightarrow A^\mu+\partial^\mu\Lambda$不改变其物理意义, 进过规范变换$\partial_\mu A^\mu$也不一定为0. 这使得我们对电磁场的描述存在冗余自由度, 这的一个直接结果就是电磁场只有两个极化方向而不是重电磁场的三个. 

于是, 为了能够正确得处理自由度, 消除冗余, 我们引入Lorenz规范:
\begin{equation}
    \nabla_\mu A^\mu=0
\end{equation}

但是故事并没有结束, 我们仍然可以通过满足$\partial^2\Lambda=0$的$\Lambda$来进行规范, 因此我们可以进一步地取$\partial_0\Lambda=-A_0=-\varphi$, 从而使得$\varphi=0$.

这样Lorenz规范就退化成了Coulomb规范:
\begin{equation}
    \nabla\cdot\vec A=0
\end{equation}

引入正则量子化条件:
\begin{equation}
    [a_{\vec pr}, a^\dagger_{\vec qs}]=\dpi3\delta_{rs}\delta^3(\vec p-\vec q)
\end{equation}

于是我们就可以二次量子化$\vec A$了
\begin{equation}
    \vec A=\int\ldsq p\sum_{r=1}^2\left(\vec\epsilon_ra_{\vec pr}\exp{-ipx}+\vec\epsilon_r^*a_{\vec pr}^\dagger\exp{ipx}\right)
\end{equation}

从而有
\begin{equation}
    \vec E=-i\int\ddd p\sqrt{\frac{\om p}2}\sum_{r=1}^2\left(\vec\epsilon_ra_{\vec pr}\exp{-ipx}-\vec\epsilon_r^*a_{\vec pr}^\dagger\exp{ipx}\right)
\end{equation}

然后有对易子
\begin{align}
    [A^i_{\vec x}, E^j_{\vec y}]&=i\int\ddd p\exp{i\vec p\cdot(\vec x-\vec y)}\sum_r\epsilon_r^i(\vec p)\epsilon^j_r(\vec p)\\
    &=i\int\ddd p\exp{i\vec p\cdot(\vec x-\vec y)}(\delta^{ij}-\frac{p^ip^j}{\vec p^2})\\
    &=i\delta^3_{\bf{tr}}(\vec x-\vec y)
\end{align}

\kaishu 这里我们可能会有疑问, 为什么这不能按照我们一般的正则量子化的方法, 让
\begin{equation}
    [A^i_{\vec x}, E^j_{\vec y}]=i\delta^{ij}\delta^3(\vec x-\vec y)
\end{equation}

这是因为根据我们的Coulomb规范, $\nabla\cdot\vec A=0$, 因此
\begin{equation}
    [\partial_i A^i, E^j]=0
\end{equation}

然而代入上面的对易关系, 我们会发现
\begin{equation}
    [\partial_iA^i_{\vec x}, E^j_{\vec y}]=i\partial^j\delta^3(\vec x-\vec y)\neq0
\end{equation}

这个正则量子化条件是不自洽的! 这里的原因还是因为无质量的电磁场存在规范冗余, 导致我们丢失了一个"物理的"极化方向.

而由$a, a^\dagger$写出的正则量子化条件, 我们可以验证它可以保证
\begin{align}
    [\partial_i A^i, E^j]&=i\ddd p\exp{i\vec p\cdot(\vec x-\vec y)}ip^i(\delta^{ij}-\frac{p^ip^j}{\vec p^2})\\
    &=-\ddd p\exp{i\vec p\cdot(\vec x-\vec y)}(p^j-\frac{\vec p^2 p^j}{\vec p^2})\\
    &=0
\end{align}
\songti

接着下一个任务就是给出光子的传播子了. 我们从EoM开始:
\begin{equation}
    (g_{\mu\nu}\partial^2-\partial_\mu\partial_\nu)A^\nu=J_\mu
\end{equation}

换到傅里叶空间, 我们有
\begin{equation}
    (-p^2g_{\mu\nu}+p_\mu p_\nu)\tilde A^\nu=\tilde J_\mu
\end{equation}

似乎只要求出$-p^2g_{\mu\nu}+p_\mu p_\nu$的逆矩阵就好了...?

然而不难发现, $-p^2g_{\mu\nu}+p_\mu p_\nu$是奇异的: $g_\mu\nu-\frac{p_\mu p_\nu}{p^2}$就是在度规张量在类光面上的诱导度规, 它的秩仅有2, 根本不可能找到逆.

一个自然的想法是, 引入一个参数$\xi$来使得它有逆, 即:
\begin{equation}
    (-p^2g_{\mu\nu}+p_\mu p_\nu)\to(-p^2g_{\mu\nu}+(1-\frac1\xi)p_\mu p_\nu)
\end{equation}

这样我们就能对$(-p^2g_{\mu\nu}+(1-\frac1\xi)p_\mu p_\nu)$求逆了:
\begin{equation}
    -\frac{g^{\mu\lambda}+(\xi-1)p^\mu p^\lambda/p^2}{p^2}(-p^2g_{\lambda\nu}+(1-\frac1\xi)p_\lambda p_\nu)
\end{equation}

也就是说我们想要EoM变为:
\begin{equation}
    (g_{\mu\nu}\partial^2-(1-\frac1\xi)\partial_\mu\partial_\nu)A^\nu=J_\mu
\end{equation}

\kaishu
这里我仍然不理解为什么规范变换可以改变EoM...
\songti

为了得到这样的EoM, 我们可以在拉氏量中加入一个规范项:
\begin{equation}
    \mathcal L=-\frac14F^2-J_\mu A^\mu-\frac1{2\xi}(\partial_\mu A^\mu)^2
\end{equation}

这样子就能得到要求的EoM, 并且注意到, 做规范变换$A'^\mu=A^\mu+\partial_\mu$, 其中
\begin{equation}
    \partial^2\partial_\nu\Lambda=\frac{\xi'-\xi}{\xi}\partial_\nu\partial_\mu A^\mu
\end{equation}
或者也可以写成
\begin{equation}
    \partial^2\partial_\nu\Lambda=\frac{\xi'-\xi}{\xi'}\partial_\nu\partial_\mu A'^\mu
\end{equation}
$A'^\mu$的EoM就变成:
\begin{equation}
    (g_{\mu\nu}\partial^2+(\frac1{\xi'}-1)\partial_\mu\partial_\nu)A'^\mu=J_\nu
\end{equation}

由此可见, $\xi$的不同取值其实就对应不同的规范, 并且Lorenz规范就是$\xi=0$: 这时$\frac1\xi$变为无穷大, 为了满足EoM, 我们必须要求$\partial_\mu A^\mu=0$

于是, 这样我们就可以得到Green函数:
\begin{equation}
    G_{\mu\nu}(x-y)=\int\dddd p\left(-\frac{g_{\mu\nu}+(\xi-1)p_\mu p_\nu/p^2}{p^2}\right)\exp{-ip(x-y)}
\end{equation}

从而得到光子传播子:
\begin{equation}
    \frac{-i(g_{\mu\nu}+(\xi-1)p_\mu p_\nu/p^2)}{p^2+i\epsilon}
\end{equation}

所以说, 对于Lorenz规范, $\xi=0$, 传播子为:
\begin{equation}
    \frac{-i(g_{\mu\nu}-p_\mu p_\nu/p^2)}{p^2+i\epsilon}
\end{equation}

在这里我们介绍一个更为常见并且简单的规范: Feynman规范, 它的$\xi=1$, 于是在Feynman规范下, 传播子为:
\begin{equation}
    \frac{-ig_{\mu\nu}}{p^2+i\epsilon}
\end{equation}

\newpage
\section{标量QED}
\subsection{规范变换}
我们知道, 复标量场$\psi$有$U(1)$对称性, 这是一个全局变换. 但是在一个局域的$U(1)$变换$\psi\to\psi\exp{i\alpha}$, 即我们新定义的一个规范变换下并不具有不变性. 但我们希望能够通过某些构造使其有这一不变性.

于是我们让$A_\mu$作为联络, 新定义对$\psi$的微分算符:
\begin{equation}
    D_\mu=\partial_\mu+ieA_\mu
\end{equation}

并且在$U(1)$规范变换下, $A_\mu, \psi$以如下方式变换:
\begin{align}
    &A'_\mu=A_\mu-\frac1e\partial_\mu\alpha\\
    &\psi'=\psi\exp{i\alpha}
\end{align}

不难发现
\begin{equation}
    D'_\mu\psi'=(\partial_\mu+ieA_\mu-i\partial_\mu\alpha)(\psi\exp{i\alpha})=\exp{i\alpha}(\partial_\mu+ieA_\mu)\psi=\exp{i\alpha}D_\mu\psi
\end{equation}

所以如果构造一个电磁场与标量场$\phi$耦合的拉氏密度
\begin{equation}
    \mathcal L=-\frac14 F^2+D_\mu\psi(D^\mu\psi)^*-m^2\psi\psi^*
\end{equation}
,则其有$U(1)$规范不变性.

我们可以给他写为更显式的形式
\begin{equation}
    \mathcal L=-\frac14F^2+\partial_\mu\psi\partial^\mu\psi^*-m^2\psi\psi^*+ieA_\mu(\psi\partial^\mu\psi^*-\psi^*\partial^\mu\psi-ie\psi\psi^*A^\mu)
\end{equation}

我们可以写出EoM
\begin{align}
    \begin{cases}
        &(\partial^2+m^2)\psi=-2ieA_\mu\partial^\mu\psi+e^2\psi A_\mu A^\mu\\
        &\partial_\mu F^{\mu\nu}=ie(\psi\partial^\nu\psi^*-\psi^*\partial^\nu\psi-2e\psi\psi^*A^\nu)
    \end{cases}
\end{align}

于是我们可以得到电流:
\begin{equation}
    J^\nu=ie(\psi\partial^\nu\psi^*-\psi^*\partial^\nu\psi-2e\psi\psi^*A^\nu)
\end{equation}

并且, 如果我们再次做全局$U(1)$变换, 我们可以再次得到这一守恒流:
\begin{align}
    J^\nu&=\pa{\mathcal L}{\partial_\mu\psi}\frac{\delta\psi}{\delta\alpha}+\pa{\mathcal L}{\partial_\mu\psi^*}\frac{\delta\psi^*}{\delta\alpha}\\
    &=i(\psi\partial^\nu\psi^*-\psi^*\partial^\nu\psi-2e\psi\psi^*A^\nu)
\end{align}

于是, 就从对规范不变性的追求中, 我们得到了标量QED.

\subsection{标量QED的Feynman规则}

\newpage
\section{旋量}
\subsection{Lorentz群的性质\label{Lorentz}}
我们首先考虑对矢量的Lorentz变换.
\begin{definition}[Lorentz变换]
    Lorentz变换为一种保内积的变换:
    \begin{equation}
        \bar x^\mu=\Lambda^\mu_{~~\nu}x^\nu
    \end{equation}
    使得
    \begin{equation}
        \bar x^\mu\bar x_\mu=x^\mu x_\mu
    \end{equation}
\end{definition}
\begin{theorem}[Lorentz变换的性质]\label{theorem:lorentz_property}
    \begin{equation}
        \Lambda^\mu_{~~\sigma}g_{\mu\nu}\Lambda^\nu_{~~\rho}=g_{\sigma\rho}
    \end{equation}
\end{theorem}
\begin{proof}
    \begin{equation}
        \bar{x}^2=g_{\mu\nu}\bar{x}^\mu\bar x^\nu=x^\sigma(\Lambda^\mu_{~~\sigma}g_{\mu\nu}\Lambda^\nu_{~~\rho})x^\rho=x^\sigma g_{\sigma\rho} x^\rho
    \end{equation}
\end{proof}
于是我们可以有如下推论:
\begin{theorem}[Lorentz变换的逆矩阵]\label{theorem:Lorentz_inverse}
    \begin{equation}
        (\Lambda^{-1})^{\mu}_{~~\nu}=\Lambda_\nu^{~~\mu}
    \end{equation}
\end{theorem}
\begin{proof}
    由\ref{theorem:lorentz_property}可得
    \begin{equation}
        g^{\rho\beta}\Lambda^\mu_{~~\rho}\Lambda^\nu_{~~\sigma}g_{\mu\nu}=g_{\rho\sigma}g^{\rho\beta}=\delta^\beta_{~~\sigma}
    \end{equation}
    即:
    \begin{equation}
        \Lambda_\nu^{~~\beta}\Lambda^\nu_{~~\sigma}=\delta^\beta_{~~\sigma}
    \end{equation}
    于是可以得证.
\end{proof}
\begin{definition}[$\delta\omega^\mu_{~~\nu}$]\label{def:deltaomega}
    对于无穷小Lorentz变换$\Lambda^\mu_{~~\nu}$, 定义
    \begin{equation}
        \Lambda^\mu_{~~\nu}=\delta^\mu_{~~\nu}+\delta\omega^\mu_{~~\nu}
    \end{equation}
\end{definition}
通过\eqref{theorem:lorentz_property}可以发现$\delta\omega_{\mu\nu}$是反称的. 并且我们还可以进一步写成矩阵形式\:
\begin{equation}
    \delta\omega^\mu_{~~\nu}=\begin{bmatrix}
        0 & v^1 & v^2 & v^3 \\
        v^1 & 0 & \theta^3 & -\theta^2 \\
        v^2 & -\theta^3 & 0 & \theta^1 \\
        v^3 & \theta^2 & -\theta^1 & 0
    \end{bmatrix}
\end{equation}
其中, $v^i$为参考系间的相对速度, $\theta^i$为沿着$i$轴旋转的角度. 或者写为降指标后的结果
\begin{equation}
    \delta\omega_{\mu\nu}=\begin{bmatrix}
        0 & v^1 & v^2 & v^3 \\
        -v^1 & 0 & -\theta^3 & \theta^2 \\
        -v^2 & \theta^3 & 0 & -\theta^1\\
        -v^3 & -\theta^2 & \theta^1 & 0
    \end{bmatrix}
\end{equation}

于是我们还可以进一步写出Lorentz变换的显式形式
\begin{align}\label{lorentz-for-vector}
    \begin{cases}
        \delta x^0=\beta^i x^i\\
        \delta x^i=\beta^i x^0-\epsilon_{ijk}\theta^j x^k
    \end{cases}
\end{align}
\begin{theorem}[$\delta\omega_{\mu\nu}$的性质]
    \begin{equation}
        \delta\omega_{\mu\nu}=\delta\omega_{[\mu\nu]}
    \end{equation}
\end{theorem}

然后我们想要把Lorentz变换的操作抽象化, 一般化, 将其提升到群表示论的高度, 于是我们有一个用抽象的Lorentz变换参数$\omega_{\mu\nu}$表示的对某一个对象进行的Lorentz变换, 这成为Lorentz群的一个表示.
\begin{definition}[无穷小Lorentz群变换的表示$U(\Lambda)$]
    \begin{equation}
        U(\mathbf 1+\delta\omega)=1-\frac i2\delta\omega_{\mu\nu}M^{\mu\nu}
    \end{equation}
    其中,$M^{\mu\nu}=M^{[\mu\nu]}$, 是某一个算符
\end{definition}
以此我们有有限Lorentz群变换的表示
\begin{equation}
    U(\omega)=\exp{-\frac i2\omega_{\mu\nu}M^{\mu\nu}}
\end{equation}

\begin{theorem}[结合律]\label{theorem:U_combine}
    作为群表示, 我们要求$U$满足:
    \begin{equation}
        U(\Lambda\Lambda')=U(\Lambda)U(\Lambda')
    \end{equation}
\end{theorem}
根据定理\ref{theorem:Lorentz_inverse}, 定理\ref{theorem:U_combine}, 定义\ref{def:deltaomega}, 我们要求$U(\Lambda^{-1}\Lambda'\Lambda)=U(\Lambda^{-1})U(\Lambda')U(\Lambda)$, 于是有\ref{theorem:UMU}:
\begin{theorem}\label{theorem:UMU}
    \begin{equation}
        U^{-1}_\Lambda M^{\mu\nu}U_\Lambda=\Lambda^\mu_{~~\rho}\Lambda^\nu_{~~\sigma}M^{\rho\sigma}
    \end{equation}
\end{theorem}
\begin{proof}
    \begin{equation}\label{2eq1}
        U_\Lambda^{-1}U_{\Lambda'}U_\Lambda=1-\frac i2\delta{\omega'}_{\mu\nu}U_\Lambda^{-1}M^{\mu\nu}U_\Lambda
    \end{equation}
    \begin{equation}
        U(\Lambda^{-1}\Lambda'\Lambda)=U(1+\Lambda^{-1}\omega'\Lambda)=1-\frac i2(\Lambda^{-1}\delta\omega'\Lambda)_{\mu\nu}M^{\mu\nu}
    \end{equation}
    计算$(\Lambda^{-1}\delta\omega'\Lambda)^{\mu}_{~~\nu}$
    \begin{equation}
        (\Lambda^{-1}\delta\omega'\Lambda)^{\mu}_{~~\nu}=\Lambda_\sigma^{~~\mu}\delta{\omega'}^\sigma_{~~\rho}\Lambda^\rho_{~~\nu}
    \end{equation}
    于是
    \begin{equation}
        (\Lambda^{-1}\delta\omega'\Lambda)_{\mu\nu}=\Lambda^{\sigma}_{~~\mu}\delta{\omega'}_{\sigma\rho}\Lambda^\rho_{~~\nu}
    \end{equation}
    因此
    \begin{equation}\label{2eq2}
        U(\Lambda^{-1}\Lambda'\Lambda)=U(1+\Lambda^{-1}\omega'\Lambda)=1-\frac i2\Lambda^{\sigma}_{~~\mu}\delta{\omega'}_{\sigma\rho}\Lambda^\rho_{~~\nu}M^{\mu\nu}
    \end{equation}
    将\eqref{2eq1}与\eqref{2eq2}取等我们有
    \begin{equation}
        \delta{\omega'}_{\rho\sigma}U_\Lambda^{-1}M^{\rho\sigma}U_\Lambda=\Lambda^{\sigma}_{~~\mu}\delta{\omega'}_{\sigma\rho}\Lambda^\rho_{~~\nu}M^{\mu\nu}
    \end{equation}
    于是
    \begin{equation}
        U^{-1}_\Lambda M^{\mu\nu}U_\Lambda=\Lambda^\mu_{~~\rho}\Lambda^\nu_{~~\sigma}M^{\rho\sigma}
    \end{equation}
\end{proof}
进一步展开我们可以得到
\begin{theorem}[$M^{\mu\nu}$的对易子]\label{M-commutator}
    \begin{equation}
        [M^{\mu\nu}, M^{\rho\sigma}]=i(-g^{\mu\rho}M^{\nu\sigma}-g^{\sigma\nu}M^{\mu\rho}+g^{\mu\sigma}M^{\nu\rho}+g^{\rho\nu}M^{\mu\sigma})
    \end{equation}
\end{theorem}
\begin{proof}
    展开
    \begin{equation}
        (1+\frac i 2\delta \omega_{\alpha\beta}M^{\alpha\beta})M^{\mu\nu}(1-\frac i 2\delta \omega_{\rho\sigma}M^{\rho\sigma})=(\delta^\mu_{~~\rho}+\delta\omega^\mu_{~~\rho})(\delta^\nu_{~~\sigma}+\delta\omega^\nu_{~~\sigma})M^{\rho\sigma}
    \end{equation}
    化简整理得到
    \begin{equation}
        \frac i2\delta\omega_{\rho\sigma}[M^{\rho\sigma}, M^{\mu\nu}]=\delta\omega_{\rho\sigma}(M^{\mu\sigma}g^{\rho\nu}-M^{\rho\nu}g^{\mu\sigma})
    \end{equation}
    于是
    \begin{equation}\label{eq3}
        [M^{\mu\nu}, M^{\rho\sigma}]=2i(g^{\mu\sigma}M^{\nu\rho}+g^{\rho\nu}M^{\mu\sigma})+A^{\mu\nu\rho\sigma}
    \end{equation}
    其中$A^{\mu\nu\rho\sigma}=A^{\nu\mu\rho\sigma}$, $A^{\mu\nu\rho\sigma}=A^{\mu\nu\sigma\rho}$.\\
    交换$\mu$, $\nu$:
    \begin{equation}\label{eq4}
        [M^{\nu\mu}, M^{\rho\sigma}]=2i(g^{\nu\sigma}M^{\mu\rho}+g^{\rho\mu}M^{\nu\sigma})+A^{\mu\nu\rho\sigma}
    \end{equation}
    注意到$M^{\mu\nu}$反称, \eqref{eq3}+\eqref{eq4}得到:
    \begin{equation}
        A^{\mu\nu\rho\sigma}=-i(g^{\mu\sigma}M^{\nu\rho}+g^{\nu\rho}M^{\mu\sigma}+g^{\mu\sigma}M^{\nu\sigma}+g^{\nu\sigma}M^{\mu\rho})
    \end{equation}
    于是可得
    \begin{important}
        \begin{equation}
            [M^{\mu\nu}, M^{\rho\sigma}]=i(-g^{\mu\rho}M^{\nu\sigma}-g^{\sigma\nu}M^{\mu\rho}+g^{\mu\sigma}M^{\nu\rho}+g^{\rho\nu}M^{\mu\sigma})
        \end{equation}
    \end{important}
\end{proof}
\begin{definition}[Lorentz群生成元]\label{lorentz-generators}
    \begin{equation}
        J^i =\frac12\epsilon_{ijk}M^{jk}\Rightarrow M^{ij}=\epsilon_{ijk}J^k
    \end{equation}
    \begin{equation}
        K^i=M^{i0}
    \end{equation}
\end{definition}
写为矩阵形式就是
\begin{equation}
    M^{\mu\nu}=\begin{bmatrix}
        0 & -K^1 & -K^2 & -K^3\\
        K^1 & 0 & J^3 & -J^2\\
        K^2 & -J^3 & 0 & J^1\\
        K^3 & J^2 & -J^1 & 0
    \end{bmatrix}
\end{equation}

然后我们可以将无穷小Lorentz群表示写为
\begin{equation}
    1+i\theta^iJ^i+i\beta^iK^i
\end{equation}
有限大的写为
\begin{equation}
    \exp{i\theta^iJ^i+i\beta^iK^i}
\end{equation}

然后我们有
\begin{theorem}[Lorentz群生成元的对易关系]\label{lorentz-commutator}
    \begin{equation}
        [J^i, J^j]=i\epsilon_{ijk}J^k
    \end{equation}
    \begin{equation}
        [J^i, K^j]=i\epsilon_{ijk}K^k
    \end{equation}
    \begin{equation}
        [K^i, K^j]=-i\epsilon_{ijk}J^k
    \end{equation}
\end{theorem}
注意, 在这里我们都认为是具体指标的计算, 因此不关心上下标的问题: 矢量就是上标, 体元就是下标, 从而避免(+---)度规三维部分升降指标会多出负号的恼人特性.

\begin{proof}
    \begin{align}
        [J^i, J^j]&=\frac14\epsilon_{iml}\epsilon_{jnp}M^{ml}M^{np}\\
        &=\frac14\epsilon_{iml}\epsilon_{jnp}\cdot 2i(g^{mp}M^{ln}+g^{lm}M^{mp})\\
        &=-\frac i2\epsilon_{iml}\epsilon_{jnp}(\delta_{mp}M^{ln}+\delta_{ln}M^{mp})\\
        &=-i\epsilon_{mli}\epsilon_{mjn}M^{ln}\\
        &=-i(\delta_{lj}\delta_{in}-\delta_{ln}\delta_{ij})M^{ln}\\
        &=iM^{ij}\\
        &=i\epsilon_{ijk}J^k\\
        [J^i, K^j]&=\frac12\epsilon_{imn}\cdot2i(g^{m0}M^{nj}+g^{nj}M^{m0})\\
        &=i\epsilon_{imn}(-\delta_{nj})M^{m0}\\
        &=i\epsilon_{ijk}M^{k0}\\
        &=i\epsilon_{ijk}K^k
    \end{align}
    \begin{align}
        [K^i, K^j]&=[M^{i0}, M^{j0}]\\
        &=-i g^{00}M^{ij}\\
        &=-iM^{ij}\\
        &=-i\epsilon_{ijk}J^k
    \end{align}
\end{proof}
\subsubsection{分解Lorentz群}
正如我们在定义\ref{lorentz-generators}中看到的那样, 洛伦兹群有六个生成元, 从而将Lorentz变换表示为:
\begin{equation}
    \Lambda=\exp{i\theta_i J^i+i\beta_i K^i}
\end{equation}

并且定理\ref{lorentz-commutator}, 有对易关系
\begin{equation}
    [J_i, J_j]=i\epsilon_{ijk}J^k
\end{equation}
\begin{equation}
    [J_i, K_j]=i\epsilon_{ijk}K^k
\end{equation}
\begin{equation}
    [K_i, K_j]=-i\epsilon_{ijk}J^k
\end{equation}

或者写为一个张量形式
\begin{equation}
    M^{\mu\nu}=\begin{bmatrix}
        0 & -K^1 & -K^2 & -K^3\\
        K^1 & 0 & J^3 & -J^2\\
        K^2 & -J^3 & 0 & J^1\\
        K^3 & J^2 & -J^1 & 0
    \end{bmatrix}
\end{equation}

并且根据定理\ref{M-commutator}我们有:
\begin{equation}
    [M^{\mu\nu}, M^{\rho\sigma}]=i(-g^{\mu\rho}M^{\nu\sigma}-g^{\sigma\nu}M^{\mu\rho}+g^{\mu\sigma}M^{\nu\rho}+g^{\rho\nu}M^{\mu\sigma})
\end{equation}

于是
\begin{equation}
    \Lambda=\exp{-\frac i2\omega_{\mu\nu}M^{\mu\nu}}
\end{equation}

我们设
\begin{equation}
    J_i^{\pm}=\frac12(J_i\pm iK_i)
\end{equation}

也就是:
\begin{align}
    \begin{cases}
        & \vec J=\vec J^++\vec J^-\\
        & \vec K=i(\vec J^--\vec J^+)
    \end{cases}
\end{align}

于是有对易关系
\begin{align}
    \begin{cases}
        & [J^+_i, J^+_j]=i\epsilon_{ijk}J^{+k} \\
        & [J^-_i, J^-_j]=i\epsilon_{ijk}J^{-k} \\
        & [J^+_i, J^-_j]=0
    \end{cases}
\end{align}
我们发现,$\vec J^\pm$是解耦的, 而它们分别满足$\mathfrak{su}(2)$的Lie代数关系!

于是我们得到结论
\begin{important}
    \begin{equation}
        \mathfrak{so}(1, 3)=\mathfrak{su}(2)\otimes\mathfrak{su}(2)
    \end{equation}
\end{important}

因此, 我们可以将Lorentz群的不可约表示用两个半整数$(m, n)$表示, 分别代表两个$\mathfrak{su}(2)$部分的角量子数. 

于是我们发现不可约表示$(m, n)$的维度为$(2m+1)(2n+1)$.

\subsubsection{不可约表示}
\begin{example}[$(0, 0)$型]
    其维度为1, 并且其是Lorentz不变的. 因此我们指出$(0, 0)$型就是标量.
\end{example}
\begin{example}[$(\frac12,\frac12)$型]
    其维度为4, 并且我们有生成元
    \begin{equation}
        J^{+i}=J^{-i}=\frac{\sigma^i}2
    \end{equation}

    于是我们有
    \begin{equation}
        \vec J=\frac12\vec\sigma\otimes1+1\otimes\frac12\vec\sigma, \vec K=i\left(1\otimes\frac12\vec\sigma-\frac12\vec\sigma\otimes1\right)
    \end{equation}
    (千万不要直接将两个$\frac12\vec\sigma$相加, 因为它们是分别作用到不同的旋量部分的)

    考虑两旋量$\xi, \eta$, 我们用矩阵$\xi\eta^T\sigma^2$表示这两旋量的张量积.(为什么这里这么奇怪地在最后插入一个$\sigma^2$? 原因在于只有$\sigma^{T2}=-\sigma^2\neq\sigma^2$, 如果没有这个$\sigma^2$就会使得接下来的变换规则非常奇怪)

    而$\vec J, \vec K$对其的作用为
    \begin{align}
        & \vec J(\xi\eta^T)=\frac12(\sigma\xi\eta^T\sigma^2+\xi\eta^T\sigma^T\sigma^2)\\
        & \vec K(\xi\eta^T)=\frac i2(-\sigma\xi\eta^T\sigma^2+\xi\eta^T\sigma^T\sigma^2)
    \end{align}

    需要注意到, 这里出现了$\vec\sigma^T$, 而
    \begin{align}
       &\sigma^{iT}=\sigma^i, i=0,1,3\\
       &\sigma^{2T}=-\sigma^2
    \end{align}

    我们设
    \begin{equation}
        \xi\eta^T\sigma^2=V^\mu\bar\sigma_\mu
    \end{equation}
    (注意这里, 因为单纯$\xi\eta^T\sigma^2$的秩为$1$, 我们要表示任意的旋量其实需要多个$\xi\eta^T\sigma^2$做线性组合. 但是因为线性性使得对于单个$\xi\eta^T\sigma^2$成立的对于它们的线性组合式子也仍然成立, 所以这里出于书写简便性的考虑我们就不妨写一个$\xi\eta^T\sigma^2$来表示$V^\mu\bar\sigma_\mu$)

    计算可以发现
    \begin{align}
        J^2(V^\mu\bar\sigma_\mu)&=\frac12(\sigma^2V^\mu\bar\sigma_\mu+V^\mu\bar\sigma_\mu\sigma^2\sigma^{2T}\sigma^2)\\
        &=\frac12V^\mu(\sigma^2\bar\sigma_\mu-\bar\sigma_\mu\sigma^2)\\
        &=i\bar\sigma_j\epsilon_{j2i}V^i
    \end{align}
    并且对于$k\neq2$, 我们不难计算得到
    \begin{align}
        J^k(V^\mu\bar\sigma_\mu)=i\bar\sigma_j\epsilon_{jki}V^i
    \end{align}
    整理结果有, 对于$k=1,2,3$
    \begin{align}
        J^k(V^\mu\bar\sigma_\mu)=i\bar\sigma_j\epsilon_{jki}V^i
    \end{align}

    同理我们有
    \begin{equation}
        K^i(V^\mu\bar\sigma_\mu)=\frac i2(-\sigma^kV^\mu\bar\sigma_\mu+V^\mu\bar\sigma_\mu\sigma^2\sigma^{kT}\sigma^2)
    \end{equation}
    于是我们可以计算得到
    \begin{equation}
        K^k(V^\mu\bar\sigma_\mu)=-i(V^k\bar\sigma_0+V^0\bar\sigma_k)
    \end{equation}

    所以我们发现, 对于$V^\mu\bar\sigma_\mu$做无穷小Lorentz变换有
    \begin{align}
        \delta(V^\mu\bar\sigma_\mu)&=\Lambda_{\theta^i, \beta^i}(V^\mu\bar\sigma_\mu)-V^\mu\bar\sigma_\mu\\
        &=i\theta^kJ^k(V^\mu\bar\sigma_\mu)+i\beta^kK^k(V^\mu\bar\sigma_mu)\\
        &=-\bar\sigma_j\epsilon_{jki}\theta^kV^i+\beta^k(V^k\bar\sigma_0+V^0\bar\sigma_k)
    \end{align}

    对照\eqref{lorentz-for-vector}这符合Lorentz群作用下矢量的变换, 因此我们指出, $(\frac12, \frac12)$其实代表的就是4矢量.
\end{example}
\begin{example}[$(0,\frac12)$型-右手Weyl旋量]
    其维度为2. 于是我们可以将其记为$\psi_R$, 是一个二维列向量. 根据生成元
    \begin{align}
        \begin{cases}
            &\vec J^-=0\\
            &\vec J^+=\frac{\sigma}2
        \end{cases}
    \end{align}
    我们有得到Lorentz变换的群作用
    \begin{equation}
        \psi_R=\exp{\frac12(i\theta^i\sigma^i+\beta^i\sigma^i)}\psi_R        
    \end{equation}
    对无穷小Lorentz变换有
    \begin{equation}
        \delta\psi_R=\frac12(i\theta^j+\beta^j)\sigma^j\psi_R
    \end{equation}

    我们计算发现
    \begin{equation}
        \delta(\psi_R^\dagger\psi_R)=\beta^i\psi_R^\dagger\sigma^i\psi_R
    \end{equation}
    \begin{equation}
        \delta(\psi_R^\dagger\sigma^i\psi_R)=-\epsilon_{ijk}\theta^j\psi_R^\dagger\sigma^k\psi_R+\beta^i\psi_R^\dagger\psi_R
    \end{equation}
    如果我们将它们组合为$(\psi_R^\dagger\psi_R, \psi_R^\dagger\sigma^i\psi_R)^T$, 可以发现这正是矢量的Lorentz变换形式. 因此我们发现了一个4矢量
    \begin{equation}
        \psi_R^\dagger\sigma^\mu\psi_R
    \end{equation}, 其中
    \begin{equation}
        \sigma^0=\begin{pmatrix}
            1 & 0 \\
            0 & 1
        \end{pmatrix}
    \end{equation}

    从这里我们也可以体会到所谓"旋量是矢量开平方根"的说法的道理: 我们将两个旋量组合在一起, 就能乘出一个矢量.
\end{example}
\begin{example}[$(\frac12,0)$型-左手Weyl旋量]
    其维度为2. 于是我们可以将其记为$\psi_L$, 是一个二维列向量. 生成元为
    \begin{align}
        \begin{cases}
            &\vec J^-=\frac{\sigma}2\\
            &\vec J^+=0\
        \end{cases}
    \end{align}
    我们有得到Lorentz变换的群作用
    \begin{equation}
        \psi_L=\exp{\frac12(i\theta^i\sigma^i-\beta^i\sigma^i)}\psi_R        
    \end{equation}
    对无穷小Lorentz变换有
    \begin{equation}
        \delta\psi_L\frac12(i\theta^j-\beta^j)\sigma^j\psi_R
    \end{equation}

    我们定义$\bar\sigma^\mu=(1, -\sigma^i)^T$, 并计算发现
    \begin{equation}
        \delta(\psi_L^\dagger\psi_L)=-\beta^i\psi_L^\dagger\sigma^i\psi_L=\beta^i\psi_L^\dagger\bar\sigma^i\psi_L
    \end{equation}
    \begin{equation}
        \delta(\psi_L^\dagger\bar\sigma^i\psi_L)=-\sigma_{ijk}\theta^j\psi_L^\dagger\sigma^k\psi_L+\beta^i\psi_L^\dagger\psi_L
    \end{equation}

    于是我们发现
    \begin{equation}
        \psi_L^\dagger\bar\sigma^\mu\psi_L
    \end{equation}
    是一个4矢量
\end{example}

进一步地, 我们还可以发现
\begin{equation}
    \delta(\psi_L^\dagger\psi_R)=\delta(\psi_R^\dagger\psi_L)=0
\end{equation}
于是$\psi_L^\dagger\psi_R, \psi_R^\dagger\psi_L$是Lorentz标量.

再计算
\begin{equation}
    \delta(\psi^\dagger_R\sigma^\mu\partial_\mu\psi_R)
\end{equation}
根据
\begin{equation}
    \delta(\partial_\mu)=\partial_\mu'-\partial_\mu
\end{equation}
即
\begin{align}
    &\delta(\partial_0)=-\beta^i\partial_i\\
    &\delta(\partial_i)=-\beta^i\partial_0+\epsilon_{kji}\theta^j\partial_k
\end{align}
我们有
\begin{equation}
    \delta(\psi^\dagger_R\sigma^\mu\partial_\mu\psi_R)=0
\end{equation}
这个结论对于$\psi^\dagger_L\bar\sigma^\mu\partial_\mu\psi_L$同样成立.

所以我们发现
\begin{equation}
    \psi^\dagger_R\sigma^\mu\partial_\mu\psi_R, \psi^\dagger_L\bar\sigma^\mu\partial_\mu\psi_L
\end{equation}
是Lorentz标量.

\subsection{Dirac旋量}
利用上一节中我们组合出来的标量, 我们可以构造一个Lagrangian
\begin{equation}
    \mathcal L=i\psi^\dagger_R\sigma^\mu\partial_\mu\psi_R+i\psi^\dagger_R\sigma^\mu\partial_\mu\psi_R-m(\psi_R^\dagger\psi_L+\psi_L^\dagger\psi_R)
\end{equation}

我们将$\psi_L, \psi_R$拼到一起
\begin{equation}
    \psi=\begin{pmatrix}
        \psi_L\\
        \psi_R
    \end{pmatrix}
\end{equation}
然后构造
\begin{equation}
    \gamma^\mu=\begin{pmatrix}
        0 & \sigma^\mu\\
        \bar\sigma^\mu & 0
    \end{pmatrix}
\end{equation}
定义
\begin{equation}
    \bar\psi=\psi^\dagger\gamma^0=(\psi_R^\dagger,\psi_L^\dagger)
\end{equation}

就有
\begin{equation}
    \mathcal L=\bar\psi(i\slashed\partial-m)\psi
\end{equation}
其中
\begin{equation}
    \slashed\partial=\gamma^\mu\partial^\mu
\end{equation}

并且从矩阵形式我们注意到
\begin{equation}
    \{\gamma^\mu, \gamma^\nu\}=2g^{\mu\nu}
\end{equation}
以及
\begin{equation}\label{gamma-conj-idx}
    \gamma^{0\dagger}=\gamma^0, \gamma^{i\dagger}=-\gamma^i
\end{equation}
利用
\begin{equation}
    \gamma^0\gamma^0=g^{00}=1, \gamma^i\gamma0=-\gamma^0\gamma^i
\end{equation}
我们可以将式\eqref{gamma-conj-idx}写为更紧凑的形式
\begin{equation}
    \gamma^{\mu\dagger}=\gamma^0\gamma^\mu\gamma^0
\end{equation}

在下一节\ref{clifford}中我们将会看到一般化的对于$\gamma^\mu$性质的讨论.

接着我们继续考虑Dirac旋量, 我们可以从Lagrangian中得到EoM:
\begin{equation}
    (i\slashed\partial-m)\psi=0
\end{equation}

这是一个一阶的PDE, 似乎与预期中的Klein-Gordan方程不符? 我们可以将其左乘$(i\slashed\partial+m)$, 得到
\begin{equation}
    (i\slashed\partial+m)(i\slashed\partial-m)\psi=(-\partial^2-m^2)\psi=0
\end{equation}
于是我们发现, 将其解耦为二阶PDE后, 它仍然是满足Klein-Gordan方程的, 从而具有我们所预期的色散关系
\begin{equation}
    \omega^2=\vec p^2+m^2
\end{equation}

关于EoM的讨论我们见\ref{2ndq-dirac}节, 在那我们将会详细地讨论Dirac方程的解, 并将其二次量子化.

然后我们尝试获得Dirac旋量的Lorentz变换及其生成元. 我们首先直接考虑$\psi$的变换:
\begin{align}
    \delta\psi&=\begin{pmatrix}
        \delta\psi_L\\
        \delta\psi_R
    \end{pmatrix}=\frac i2\theta^i\begin{pmatrix}
        \sigma^i & 0\\
        0 & \sigma^i
    \end{pmatrix}\psi+\frac12v^i\begin{pmatrix}
        \bar\sigma^i & 0\\
        0 & \sigma^i
    \end{pmatrix}\psi
\end{align}
考虑到
\begin{equation}
    [\gamma^j, \gamma^k]=-[\sigma^j, \sigma^k]\begin{pmatrix}
        1 & 0\\
        0 & 1
    \end{pmatrix}=-i\epsilon_{jkl}\sigma^l\begin{pmatrix}
        1 & 0\\
        0 & 1
    \end{pmatrix}
\end{equation}
即
\begin{equation}
    \epsilon_{ijk}[\gamma^j, \gamma^k]=-2i\sigma^i\begin{pmatrix}
        1 & 0\\
        0 & 1
    \end{pmatrix}
\end{equation}
还有
\begin{align}
    \gamma^0\gamma^i=\begin{pmatrix}
        \bar\sigma^i & 0\\
        0 & \sigma^i
    \end{pmatrix}, \gamma^i\gamma^0=\begin{pmatrix}
        -\bar\sigma^i & 0\\
        0 & -\sigma^i
    \end{pmatrix}
\end{align}
即
\begin{equation}
    [\gamma^i, \gamma^0]=-2\begin{pmatrix}
        \bar\sigma^i & 0\\
        0 & \sigma^i
    \end{pmatrix}
\end{equation}
于是
\begin{align}
    \delta\psi&=i\epsilon_{ijk}\theta^i(\frac i4[\gamma^j, \gamma^k])\psi+iv^i(\frac i4[\gamma^i, \gamma^0])\\
    &=-\frac i2\omega_{\mu\nu}(\frac i4[\gamma^\mu, \gamma^\nu])
\end{align}

所以我们发现, 
\begin{definition}[Dirac旋量生成元]
    \begin{equation}
        S^{\mu\nu}=\frac i4[\gamma^\mu, \gamma^\nu]
    \end{equation}
\end{definition}
对Dirac旋量, Lorentz变换为
\begin{equation}
    \psi\to\Lambda_s\psi, \Lambda_s=\exp{-\frac i2\omega_{\mu\nu}S^{\mu\nu}}
\end{equation}
并且有旋转与Boost生成元
\begin{align}
    & J^{i}=\frac12\epsilon_{ijk}S^{jk}=\frac i8\epsilon_{ijk}[\gamma^j, \gamma^k]\\
    & K^i=S^{i0}=\frac i4[\gamma^i, \gamma^0]
\end{align}
\begin{equation}
    \psi\to\Lambda_s\psi, \Lambda_s=\exp{i\vec\theta\cdot\vec J+i\vec v\cdot\vec K}
\end{equation}

并且我们可以验证$S^{\mu\nu}$满足定理\ref{M-commutator}: 首先计算对易子
\begin{equation}\label{S-gamma-commutator}
    [S^{\mu\nu}, \gamma^\rho]=i(\gamma^\mu g^{\nu\rho}-\gamma^\nu g^{\mu\rho})
\end{equation}
所以
\begin{align}
    [S^{\mu\nu}, S_{\rho\sigma}]&=\frac i4\left([S^{\mu\nu}, \gamma^\rho\gamma^\sigma]-[S^{\mu\nu}, \gamma^\sigma\gamma^\rho]\right)\\
    &=\frac i4\left(\gamma^\rho[S^{\mu\nu}, \gamma^\sigma][S^{\mu\nu}, \gamma^\rho]\gamma^\sigma-\gamma^\sigma[S^{\mu\nu}, \gamma^\rho]-[S^{\mu\nu}, \gamma^\sigma]\gamma^\rho\right)
\end{align}
于是可得
\begin{equation}
    [S^{\mu\nu}, S^{\rho\sigma}]=i(-g^{\mu\rho}S^{\nu\sigma}-g^{\sigma\nu}S^{\mu\rho}+g^{\mu\sigma}S^{\nu\rho}+g^{\rho\nu}S^{\mu\sigma})
\end{equation}

\subsection{Clifford代数\label{clifford}}
上一节中, 我们定义出来的$\gamma^\mu$有其独特的代数结构, 本节我们将它抽象出来, 提升到代数的角度研究$\gamma^\mu$.
\begin{definition}[Clifford代数]
    Clifford代数即满足
    \begin{equation}
        \{\gamma^\mu, \gamma^\nu\}=2g^{\mu\nu}
    \end{equation}
    \begin{equation}
        \gamma^{\mu\dagger}=\gamma^0\gamma^\mu\gamma^0
    \end{equation}
    的代数.
\end{definition}

Spinor的代数空间是一个$4\times4$的复矩阵线性空间, 所以它应当有16个基矢. 我们可以将单位阵$1$以及$\gamma^\mu$作为其中的5个基矢, 然后利用它们的乘积得到剩下的12个基矢. 首先是我们定义一个新的基矢
\begin{equation}
    \gamma^5=i\gamma^0\gamma^1\gamma^2\gamma^3.
\end{equation}
可以验证它具有这些性质
\begin{align}
    \left(\gamma^5\right)^2=1\\
    \gamma^{5\dagger}=\gamma^5\\
    \rm{Tr}(\gamma^5)=0\\
    \{\gamma^5, \gamma^\mu\}=0
\end{align}
再将$\gamma^5$与$\gamma^\mu$做乘法得到4个基矢$\gamma^5\gamma^\mu$(由于$\gamma^5$和$\gamma^\mu$反对易, 所以我们只需要任选一个$\gamma^5$在左或者在右的定义即可), 最后加上根据$[\gamma^\mu, \gamma^\nu]$定义出来的6个基矢$S^{\mu\nu}=\frac i4[\gamma^\mu, \gamma^\nu]$, 就得到了这个线性空间的全部16个基矢. 它们之间的所有乘法都可以用它们的线性组合表示. 比如说
\begin{equation}
    \gamma^\mu\gamma^\nu=\frac12\left(\{\gamma^\mu, \gamma^\nu\}+[\gamma^\mu, \gamma^\nu]\right)=g^{\mu\nu}-2iS^{\mu\nu}
\end{equation}

% 我们通过如下表格归纳这16个基矢中的一些性质
% \begin{table}[!htbp]
%     \centering
%     \begin{tabular}{c|ccc}
%             & ${}^2$ & ${}^\dagger$ & $\rm{Tr}$ \\
%         \hline
%         $1$ &  $1$    &    $1$       &    $4$      \\
%         $\gamma^0$ &  $1$     &    $\gamma^0$       &    $0$    \\
%         $\gamma^i$ &  $-1$     &   $-\gamma^i$        &    $0$      \\
%         $\gamma^5$ &   $1$   &   $\gamma^5$        &     $0$    \\
%         % $\gamma^5\gamma^\mu$ &        &              &            \\
%         % $S^{\mu\nu}$ &        &              &            \\ 
%     \end{tabular}
%     \caption{Clifford空间基矢的性质}
% \end{table}

我们用如下表格归纳一些基矢之间的对易与反对易关系
\begin{table}[!htbp]
    \centering
    \begin{tabular}{c|ccccc}
                             & $1$                   & $\gamma^\mu$          & $\gamma^5$         & $\gamma^5\gamma^\mu$ & $S^{\mu\nu}$\\
        \hline
        $1$                  & $\{2\}$                   & $\{2\gamma^\mu\}$ & $\{2\gamma^5\}$     & $\{2\gamma^5\gamma^\mu\}$ & $\{2S^{\mu\nu}\}$ \\
        $\gamma^\nu$         & $\{2\gamma^\nu\}$         & $\{2g^{\mu\nu}\}$ & $\{0\}$             & $[2\gamma^5g^{\mu\nu}]$    &                   \\
        $\gamma^5$           & $\{2\gamma^5\}$           & $\{0\}$           & $\{2\}$            &  $\{0\}$                     &  $[0]$    \\
        $\gamma^5\gamma^\nu$ & $\{2\gamma^5\gamma^\nu\}$ &  $[-2\gamma^5g^{\mu\nu}]$            & $\{0\}$            &                              &               \\
        $S^{\rho\sigma}$     & $\{2S^{\rho\sigma}\}$     &                  &    $[0]$               &                           & %$[-i(-g^{\mu\rho}S^{\nu\sigma}-g^{\sigma\nu}S^{\mu\rho}+g^{\mu\sigma}S^{\nu\rho}+g^{\rho\nu}S^{\mu\sigma})]$
    \end{tabular}
    \caption{Cliffod空间基矢(反)对易关系}
\end{table}
其中$[]$内的内容表示对易关系(行在前列在后), $\{\}$内的内容表示是反对易关系.

然后我们可以首先研究这些基矢在$O(1, 3)$变换下的性质.

\begin{theorem}
    \begin{equation}
        \Lambda_s^{-1}\gamma^\mu\Lambda_s=(\Lambda_v)^\mu_{~~\nu}\gamma^\nu
    \end{equation}
    \begin{equation}
        \Lambda_s^{-1}\gamma^5\gamma^\mu\Lambda_s=(\Lambda_v)^\mu_{~~\nu}\gamma^5\gamma^\nu
    \end{equation}
    其中$\Lambda_v$即$\omega$对应的对矢量的Lorentz变换矩阵.
\end{theorem}
\begin{proof}
    考虑无穷小Lorentz变换.
    \begin{equation}
        \Lambda_s=(1-\frac12i\omega_{\alpha\beta}S^{\alpha\beta}), \Lambda_s^{-1}=(1+\frac12i\omega_{\alpha\beta}S^{\alpha\beta})
    \end{equation}
    于是
    \begin{align}
        \Lambda_s^{-1}\gamma^\mu\Lambda_s&=(1+\frac i2\omega_{\alpha\beta}S^{\alpha\beta})\gamma^\mu(1-\frac i2\omega_{\alpha\beta}S^{\alpha\beta})\\
        &=\gamma^\mu+\frac i2\omega_{\alpha\beta}[S^{\alpha\beta}, \gamma^\mu]
    \end{align}

    利用式\eqref{S-gamma-commutator}我们有
    \begin{align}
        \Lambda_s^{-1}\gamma^\mu\Lambda_s&=\gamma^\mu-\frac 12\omega_{\alpha\beta}(\gamma^\alpha g^{\beta\mu}-\gamma^\beta g^{\alpha\mu})\\
        &=\gamma^\mu+\frac12\omega_{\beta\alpha}g^{\beta\mu}\gamma^\alpha+\frac12\theta_{\alpha\beta}g^{\alpha\mu}\gamma^\beta\\
        &=\gamma^\mu+\omega^\mu_{~~\nu}\gamma^\nu\\
        &=(\delta^\mu_{~~\nu}+\omega^\mu_{~~\nu})\gamma^\nu\\
        &=(\Lambda_v)^\mu_{~~\nu}\gamma^\nu
    \end{align}
    其中
    \begin{equation}
        (\Lambda_v)^\mu_{~~\nu}=\delta^\mu_{~~\nu}+\omega^\mu_{~~\nu}
    \end{equation}
    即矢量的Lorentz变换.\\
\end{proof}

\begin{theorem}
    \begin{equation}
        \gamma^0\gamma^0\gamma^0=\gamma^0, \gamma^0\gamma^i\gamma^0=-\gamma^i
    \end{equation}
    \begin{equation}
        \gamma^0\gamma^5\gamma^0=-\gamma^5
    \end{equation}
\end{theorem}

\begin{theorem}
    \begin{equation}
        \gamma^0\Lambda_s^\dagger\gamma^0=\Lambda_s^{-1}
    \end{equation}
\end{theorem}
\begin{proof}
    我们求生成元的共轭
    \begin{align}
        S^{\mu\nu\dagger}&=-\frac i4(\gamma^\mu\gamma^\nu-\gamma^\nu\gamma^\mu)^\dagger\\
        &=-\frac i4(\gamma^{\nu\dagger}\gamma^{\mu\dagger}-\gamma^{\mu\dagger}\gamma^{\nu\dagger})\\
        &=\frac i4(\gamma^0\gamma^\mu\gamma^0\gamma^0\gamma^\nu\gamma^0-\gamma^0\gamma^\nu\gamma^0\gamma^0\gamma^\mu\gamma^0)\\
        &=\gamma^0S^{\mu\nu}\gamma^0
    \end{align}

    然后由于$\gamma^0\gamma^0=1$, 所以$\gamma^0$可以从$\Lambda_s^\dagger$的两边拎入指数中:
    \begin{equation}
        \gamma^0\Lambda_s^\dagger\gamma^0=\gamma^0\exp{\frac i2\omega_{\mu\nu}S^{\mu\nu\dagger}}\gamma^0=\exp{\frac i2\omega_{\mu\nu}\gamma^0S^{\mu\nu\dagger}\gamma^0}=\exp{\frac i2\omega_{\mu\nu}S^{\mu\nu}}=\Lambda_s^{-1}
    \end{equation}
\end{proof}

\begin{theorem}
    $\bar\psi\psi$是Lorentz标量, $\bar\psi\gamma^5\psi$是Lorentz赝标量, $\bar\psi\gamma^\mu\psi$是Loretnz矢量, $\bar\psi\gamma^5\gamma^\mu\psi$是Loretnz赝矢量, $\bar\psi\gamma^\mu\gamma^\nu\psi$是Lorentz张量.
\end{theorem}
\begin{proof}
    \begin{align}
        \bar\psi\psi=\psi^\dagger\gamma^0\psi\to \psi^\dagger\Lambda_s^\dagger\gamma^0\Lambda_s\psi=\psi^\dagger\gamma^0\Lambda_s^{-1}\Lambda_s\psi=\psi^\dagger\gamma^0\psi=\bar\psi\psi
    \end{align}
    \begin{align}
        \bar\psi\gamma^\mu\psi=\psi^\dagger\gamma^0\gamma^\mu\psi&\to\psi^\dagger\Lambda_s^\dagger\gamma^0\gamma^\mu\Lambda_s\psi=\psi^\dagger\Lambda_s^{-1}\gamma^\mu\Lambda_s\psi\notag\\
        &=\psi^\dagger\gamma^0(\Lambda_v)^\mu_{~~\nu}\psi=(\Lambda_v)^\mu_{~~\nu}\gamma^\nu\bar\psi\gamma^\mu\psi
    \end{align}
    \begin{align}
        \bar\psi\gamma^\mu\gamma^\nu\psi=\psi^\dagger\gamma^0\gamma^\mu\gamma^\nu\psi&\to\psi^\dagger\Lambda_s^\dagger\gamma^0\gamma^\mu\gamma^\nu\Lambda_s\psi=\psi^\dagger\Lambda_s^{-1}\gamma^\mu\Lambda_s\Lambda_s^{-1}\gamma^\nu\Lambda_s\psi\notag\\
        &=\psi^\dagger\gamma^0(\Lambda_v)^\mu_{~~\alpha}\gamma^\alpha(\Lambda_v)^\nu_{~~\beta}\gamma^\beta\psi=(\Lambda_v)^\mu_{~~\alpha}(\Lambda_v)^\nu_{~~\beta}\bar\psi\gamma^\alpha\gamma^\beta\psi
    \end{align}
\end{proof}

\begin{definition}[Slash]
    对于矢量$A^\mu$, 其slash
    \begin{equation}
        \slashed A\equiv \gamma^\mu A_\mu
    \end{equation}
\end{definition}

关于$\gamma$矩阵乘积的迹, 我们还有如下定理\cite{griffthsClifford}
\begin{theorem}[$\gamma$乘积迹定理]\label{gamma-mutiply-trace-them}
    对于奇数个$\gamma$乘积
    \begin{equation}
        \rm{Tr}(\gamma^{\mu_1}\gamma^{\mu_2}\cdots\gamma^{\mu_{2n-1}})=0
    \end{equation}
    对于偶数个$\gamma$乘积, 我们有
    \begin{align}
        &\rm{Tr}(\gamma^\mu\gamma^\nu)=4g^{\mu\nu}\\
        &\rm{Tr}(\gamma^\mu\gamma^\nu\gamma^\lambda\gamma^\sigma)=4(g^{\mu\nu}g^{\lambda\sigma}+g^{\mu\sigma}g^{\lambda\nu}-g^{\mu\lambda}g^{\nu\sigma})\\
        &\cdots\notag
    \end{align}
\end{theorem}

利用定理\ref{gamma-mutiply-trace-them}, 我们可以有如下结论
\begin{theorem}
    \begin{align}
        &\rm{Tr}(\slashed p)=0\\
        &\rm{Tr}(\slashed p\slashed q)=4pq\\
        &\rm{Tr}(\slashed p\slashed q\slashed k)=0\\
        &\rm{Tr}(\slashed p_1\slashed p_2\slashed p_3\slashed p_4)=4\left[(p_1p_2)(p_3p_4)+(p_1p_4)(p_3p_2)-(p_1p_3)(p_2p_4)\right]\\
        &\cdots\notag
    \end{align}
\end{theorem}

特别地, 如果$\gamma$乘积后最外侧的左右两个$\gamma$指标缩并, 即如$\gamma^\mu\gamma^\nu\gamma_\mu$, 我们有
\begin{theorem}[$\gamma$缩并定理]\label{gamma-contraction-them}
    \begin{align}
        &\gamma_\mu\gamma^\mu=4\\
        &\gamma_\mu\gamma^\nu\gamma^\mu=-2\gamma^\nu\\
        &\gamma_\mu\gamma^\nu\gamma^\lambda\gamma^\mu=4g^{\nu\lambda}\\
        &\gamma_\mu\gamma^{\nu}\gamma^\lambda\gamma^\sigma\gamma^\mu=-2\gamma^\sigma\gamma^\lambda\gamma^\nu\label{gamma-contraction-last-eq}\\
        &\cdots\notag
    \end{align}
\end{theorem}
\begin{proof}
    作为示例, 我们仅给出式\eqref{gamma-contraction-last-eq}的证明:
    \begin{align}
        \gamma^\mu\gamma^\sigma\gamma^\nu\gamma^\rho\gamma_\mu&=\gamma^\mu\gamma^\sigma\gamma^\nu(2\delta^\rho_{~~\mu}-\gamma_\mu\gamma^\rho)\\
        &=\cdots\notag\\
        &=2\gamma^\rho\gamma^\sigma\gamma^\nu-2\gamma^\nu\gamma^\sigma\gamma^\rho-2\gamma^\sigma\gamma^\nu\gamma^\rho\\
        &=2\gamma^\rho\gamma^\sigma\gamma^\nu-4g^{\nu\sigma}\gamma^\rho\\
        &=-2\gamma^\rho\gamma^\nu\gamma^\sigma
    \end{align}
\end{proof}

利用定理\ref{gamma-contraction-them}我们可以得到如下结论
\begin{theorem}
    \begin{align}
        &\slashed p\slashed p=p^2\\
        &\gamma^\mu\gamma_\mu=4\\
        &\gamma^\mu\slashed p\gamma_\mu=-2\slashed p\\
        &\gamma^\mu\slashed p\slashed q\gamma_\mu=4pq\\
        &\gamma^\mu\slashed p\slashed q\slashed k\gamma_\mu=-2\slashed k\slashed q\slashed p
    \end{align}
\end{theorem}

特别地, 还有
\begin{theorem}
    \begin{equation}
        \slashed p\slashed p=p^2
    \end{equation}
\end{theorem}
\begin{proof}
    \begin{align}
        \slashed p\slashed p&=\gamma^\mu\gamma^\nu p_{\mu}p_\nu=\frac12(\gamma^\mu\gamma^\nu+\gamma^\nu\gamma^\mu)p_\mu p_\nu=g^{\mu\nu}p_{\mu}p_{\nu}=p^2
    \end{align}
\end{proof}
需要注意, $\slashed p\slashed q\neq pq$.

对于$\slashed A$我们有如下结论\cite{sredinicki-ugammau}
\begin{theorem}\label{gamma-p-contraction}
    \begin{equation}
        \gamma^\mu\slashed p=p^\mu-2iS^{\mu\nu}p_\nu
    \end{equation}
    \begin{equation}
        \slashed p\gamma^\mu=p^\mu+2iS^{\mu\nu}p_\nu
    \end{equation}
\end{theorem}
\begin{proof}
    \begin{align}
        \gamma^\mu\slashed p&=\frac12\{\gamma^\mu, \gamma^\nu\}p_\nu+\frac12[\gamma^\mu, \gamma^\nu]p_\nu\\
        &=p^\mu-2iS^{\mu\nu}p_\nu
    \end{align}
    \begin{align}
        \slashed p\gamma^\mu&=\frac12\{\gamma^\mu, \gamma^\nu\}p_\nu+\frac12[\gamma^\nu, \gamma^\mu]p_\nu\\
        &=p^\mu+2iS^{\mu\nu}p_\nu
    \end{align}
\end{proof}

\subsection{Dirac旋量的二次量子化\label{2ndq-dirac}}
\subsubsection{旋量运动方程}
首先是第一步, 解EoM. 设在Weyl基底下一个一般的解为$(\psi_L \psi_R)^T$, 于是EoM可以写为
\begin{equation}
    \begin{pmatrix}
        -m & i\sigma^\mu\partial_\mu\\
        i\bar\sigma^\mu\partial_\mu & -m
    \end{pmatrix}\begin{pmatrix}
        \psi_L\\
        \psi_R
    \end{pmatrix}=0
\end{equation}

在动量空间中有
\begin{align}
    &\sigma^\mu p_\mu\psi_R=(E-\sigma\cdot\vec p)\psi_R=m\psi_L\\
    &\bar\sigma^\mu p_\mu\psi_L=(E+\sigma\cdot\vec p)\psi_L=m\psi_R
\end{align}

对于零质量Fermion, 这个方程是解耦的:
\begin{align}
    &\sigma^\mu p_\mu\psi_R=(E-\vec\sigma\cdot\vec p)\psi_R=0\\
    &\bar\sigma^\mu p_\mu\psi_L=(E+\vec\sigma\cdot\vec p)\psi_L=0
\end{align}

\begin{definition}[螺旋度]
    \begin{equation}
        H=\frac{\vec\sigma\cdot\vec p}{|p|}
    \end{equation}
\end{definition}

可以发现
\begin{align}
    & H\psi_R=\psi_R\\
    & H\psi_L=-\psi_L\\
\end{align}

可见, $\psi_R, \psi_L$分别是螺旋度的本征矢. 这里螺旋度的物理意义就是, 自旋指向和运动方向的夹角. 在这里我们发现, Weyl旋量的左右手其实分别就是自旋方向和运动方向分别是左手螺旋和右手螺旋的关系.

我们一般认为中微子的$m\approx0$, 所以中微子是具有固定的手性. 由于我们世界Parity的破缺, 导致自然界中其实基本上只存在左手中微子.

\subsubsection{旋量的极化}
一般性的讨论结束, 我们开始从EoM中寻求可以拿来正则量子化的解. 设正能解$\psi=u^s\exp{-ipx}$, 负能解$\psi=v^s\exp{ipx}$

对于$u^s$
\begin{equation}
    (\slashed p-m)u^s=0
\end{equation}
在Weyl基底中即
\begin{equation}
    \begin{pmatrix}
        -m & p^\mu\sigma_\mu\\
        p^\mu\bar\sigma_\mu & -m
    \end{pmatrix}u^s=0
\end{equation}

根据
\begin{align}
    (\vec p\cdot\vec\sigma)^2&=p^ip^j\sigma^i\sigma^j=p^ip^j(\delta_{ij}+i\epsilon_{ijk}\sigma^k)=p^ip^i=\vec p^2
\end{align}
然后
\begin{align}
    (p\cdot\sigma)(p\cdot\bar\sigma)&=(p^0\sigma^0-\vec p\cdot\vec\sigma)(p^0\sigma^0+\vec p\cdot\vec\sigma)\\
    &=(p^0)^2-(\vec p\cdot\sigma)^2=(p^0)^2-(\vec p)^2=p^2=m^2
\end{align}
即(都是算数平方根, 取正根, 这里没有负根的情况)
\begin{equation}
    \sqrt{(p\cdot\sigma)(p\cdot\bar\sigma)}=\sqrt{m^2}=m
\end{equation}
我们可以计算验证
\begin{align}
    \begin{pmatrix}
        -m & p^\mu\sigma_\mu\\
        p^\mu\bar\sigma_\mu & -m
    \end{pmatrix}\begin{pmatrix}
        \sqrt{p\sigma}\zeta_s\\
        \sqrt{p\bar\sigma}\zeta_s
    \end{pmatrix}&=\begin{pmatrix}
        -m\sqrt{p\sigma}\zeta_s+p\sigma\sqrt{p\bar\sigma}\zeta_s\\
        p\bar\sigma\sqrt{p\sigma}\zeta_s-m\sqrt{p\bar\sigma}\zeta_s
    \end{pmatrix}\\
    &=\begin{pmatrix}
        \sqrt{p\sigma}\left(-m+\sqrt{(p\bar\sigma)(p\sigma)}\right)\zeta_s\\
        \sqrt{p\bar\sigma}\left(\sqrt{(p\sigma)(p\bar\sigma)}-m\right)\zeta_s
    \end{pmatrix}=0
\end{align}

所以
\begin{equation}
    u^s=\begin{pmatrix}
        \sqrt{p\sigma}\zeta_s\\
        \sqrt{p\bar\sigma}\zeta_s
    \end{pmatrix}
\end{equation}
是EoM的解.

对于$v^s$同理, 它满足
\begin{equation}
    (\slashed p+m)v^s=0
\end{equation}

可以验证
\begin{equation}
    v^s=\begin{pmatrix}
        \sqrt{p\sigma}\eta_s\\
        -\sqrt{p\bar\sigma}\eta_s
    \end{pmatrix}
\end{equation}
是满足EoM的解.

然后我们取正交基底张成$u^s, v^s$的解空间, 即我们取$\zeta_1, \zeta_2$以及$\eta_1, \eta_2$满足
\begin{equation}
    \zeta_r^\dagger\zeta_s=\delta_{rs}, \eta_r^\dagger\eta_s=\delta_{rs}
\end{equation}
然后取$\zeta_s, \eta_s$分别代入$u^s, v^s$得到$u^s, v^s$的解空间的基底.

我们可以验证正交性
\begin{equation}
    \bar u^ru^s=\begin{pmatrix}
        \sqrt{p\bar\sigma}\zeta_r^\dagger & \sqrt{p\sigma}\zeta_r^\dagger
    \end{pmatrix}\begin{pmatrix}
        \sqrt{p\sigma}\zeta_s\\
        \sqrt{p\bar\sigma}\zeta_s
    \end{pmatrix}=m\delta^{rs}+m\delta^{rs}=2m\delta^{rs}
\end{equation}
\begin{equation}
    \bar v^rv^s=\begin{pmatrix}
        -\sqrt{p\bar\sigma}\eta_r^\dagger & \sqrt{p\sigma}\eta_r^\dagger
    \end{pmatrix}\begin{pmatrix}
        \sqrt{p\sigma}\eta_s\\
        -\sqrt{p\bar\sigma}\eta_s
    \end{pmatrix}=-m\delta^{rs}-m\delta^{rs}=-2m\delta^{rs}
\end{equation}
\begin{equation}
    \bar u^rv^s=\begin{pmatrix}
        \sqrt{p\bar\sigma}\zeta_r^\dagger & \sqrt{p\sigma}\zeta_r^\dagger
    \end{pmatrix}\begin{pmatrix}
        \sqrt{p\sigma}\eta_s\\
        -\sqrt{p\bar\sigma}\eta_s
    \end{pmatrix}=0
\end{equation}

于是
\begin{theorem}
    \begin{equation}
        \bar u^ru^s=-\bar v^rv^s=2m\delta_{rs}, \bar u^rv^s=\bar v^ru^s=0
    \end{equation}
\end{theorem}

\begin{theorem}
    \begin{equation}
        \sum_s u^s\bar u^s=\slashed p+m
    \end{equation}
    \begin{equation}
        \sum_s v^s\bar v^s=\slashed p-m
    \end{equation}
\end{theorem}
\begin{proof}
    根据$\zeta_1, \zeta_2$是$\mathbb{C}^2$上的完备正交基底, 所以
    \begin{equation}
        \sum_s \zeta_s\zeta_s^\dagger=1
    \end{equation}

    于是
    \begin{align}
        \sum_s u^s\bar u^s&=\sum_s\begin{pmatrix}
            \sqrt{p\sigma}\zeta_s\\
            \sqrt{p\bar\sigma}\zeta_s
        \end{pmatrix}\begin{pmatrix}
            \sqrt{p\bar\sigma}\zeta_s^\dagger & \sqrt{p\sigma}\zeta_s^\dagger
        \end{pmatrix}\notag\\
        &=\sum_s\begin{pmatrix}
            \sqrt{(p\sigma)(p\bar\sigma)}\zeta_s\zeta_s^\dagger & p\sigma\zeta_s\zeta_s^\dagger\\
            p\bar\sigma\zeta_s\zeta_s^\dagger & \sqrt{(p\sigma)(p\bar\sigma)}\zeta_s\zeta_s^\dagger
        \end{pmatrix}\notag\\
        &=\begin{pmatrix}
            m & p\sigma\\
            p\bar\sigma & m
        \end{pmatrix}=p\gamma+m=\slashed p+m
    \end{align}
    \begin{align}
        \sum_s v^s\bar v^s&=\sum_s\begin{pmatrix}
            \sqrt{p\sigma}\eta_s\\
            -\sqrt{p\bar\sigma}\eta_s
        \end{pmatrix}\begin{pmatrix}
            -\sqrt{p\bar\sigma}\eta_s^\dagger & \sqrt{p\sigma}\eta_s^\dagger
        \end{pmatrix}\notag\\
        &=\sum_s\begin{pmatrix}
            -\sqrt{(p\sigma)(p\bar\sigma)}\eta_s\eta_s^\dagger & p\sigma\eta_s\eta_s^\dagger\\
            p\bar\sigma\eta_s\eta_s^\dagger & -\sqrt{(p\sigma)(p\bar\sigma)}\eta_s\eta_s^\dagger
        \end{pmatrix}\notag\\
        &=\begin{pmatrix}
            -m & p\sigma\\
            p\bar\sigma & -m
        \end{pmatrix}=p\gamma-m=\slashed p-m
    \end{align}
\end{proof}

\begin{theorem}\label{ugammau-contraction}
    对于$u_s, v_s$我们有\cite{sredinicki-ugammau}:
    \begin{equation}
        2m\bar u_{s'}(\vec p')\gamma^\mu u_s(\vec p)=\bar u_{s'}(\vec p')\left[(p'+p)^\mu-2iS^{\mu\nu}(p'-p)_\nu\right]u_s(\vec p)
    \end{equation}
    \begin{equation}
        -2m\bar v_{s'}(\vec p')\gamma^\mu v_s(\vec p)=\bar v_{s'}(\vec p')\left[(p'+p)^\mu-2iS^{\mu\nu}(p'-p)_\nu\right]v_s(\vec p)
    \end{equation}
    \begin{equation}
        2m\bar u_{s'}(\vec p')\gamma^\mu v_s(\vec p)=\bar u_{s'}(\vec p')\left[(p'-p)^\mu-2iS^{\mu\nu}(p'+p)_\nu\right]v_s(\vec p)
    \end{equation}
    \begin{equation}
        -2m\bar v_{s'}(\vec p')\gamma^\mu u_s(\vec p)=\bar v_{s'}(\vec p')\left[(p'-p)^\mu-2iS^{\mu\nu}(p'+p)_\nu\right]u_s(\vec p)
    \end{equation}
\end{theorem}
\begin{proof}
    \begin{align}
        2m\bar u_{s'}(\vec p')\gamma^\mu u_s(\vec p)&=\bar u_{s'}(\vec p')\slashed p'\gamma^\mu u_s(\vec p)+\bar u_{s'}(\vec p')\gamma^\mu\slashed pu_s(\vec p)\\
        &=\bar u_{s'}(\vec p')\left(\slashed p'\gamma^\mu+\gamma^\mu\slashed p\right)u_s(\vec p)\\
        &=\bar u_{s'}(\vec p')\left[(p'+p)^\mu-2iS^{\mu\nu}(p'-p)_\nu\right]u_s(\vec p)
    \end{align}
    其中, 第三个等号利用定理\ref{gamma-p-contraction}. 而对于后三个等式的证明同理, 在此不赘述.
\end{proof}

对定理\ref{ugammau-contraction}取$p'=p$, 就有
\begin{equation}
    \bar u^s(\vec p)\gamma^\mu u_{s'}(\vec p)=2p^\mu\delta_{ss'}\label{gamma-p-p-contraction}
\end{equation}
\begin{equation}
    \bar v^s(\vec p)\gamma^\mu v_{s'}(\vec p)=2p^\mu\delta_{ss'}
\end{equation}
不过需要注意$\bar u^s(\vec p)\gamma^\mu v_{s'}(\vec p)\neq0$:
\begin{equation}
    2m\bar u^s(\vec p)\gamma^\mu v_{s'}(\vec p)=\bar u^s\slashed p\gamma^\mu v_{s'}-\bar u^s\gamma^\mu\slashed pv_{s'}=\bar u^sp_\nu[\gamma^\nu, \gamma^\mu]v_{s'}\neq0
\end{equation}
. 但是对于$\vec p'=-\vec p$, 我们根据此定理有推论
\begin{theorem}\label{ubar-gamma-v}
    \begin{equation}
        \bar u_{s'}(\vec p)\gamma^0 v_s(-\vec p)=\bar v_{s'}(\vec p)\gamma^0 u_s(-\vec p)=0
    \end{equation}
\end{theorem}

\subsubsection{反对易的二次量子化与关联函数}
关于$u, v$性质的讨论告一段落. 接下来由于涉及到对易子的问题, 为了能够区分对易子的乘法顺序以及做乘法的方式(内积还是外积), 我们对$\gamma, \psi$引入指标:
\begin{align}
    &\psi\to\psi_A\\
    &\bar\psi\to\bar\psi^A\\
    &\gamma^\mu\to\gamma^{\mu~~B}_{~~A}\\
    &\slashed p\to\slashed p_A^{~~B}
\end{align}

那么
\begin{align}
    &\bar\psi\psi\to\bar\psi^A\psi_A=\psi_A\bar\psi^A\\
    &\psi\bar\psi\to\psi_A\bar\psi^B=\bar\psi^B\psi_A
\end{align}

首先对$\mathcal L$做Legdren变换, 注意到$\mathcal L$中只有$\partial_0\psi$而没有$\partial_0\psi^\dagger$, 因此我们只需要定义
\begin{equation}
    \pi=\pa{\mathcal L}{\partial_0\psi}=i\bar\psi\gamma^0=i\psi^\dagger
\end{equation}
而不需要定义共轭动量(注意$\pi^{(\dagger)}$是指$\psi^\dagger$的正则动量, 而不是正则动量$\pi$的共轭, 即$\pi^{(\dagger)}\neq\pi^\dagger$)
\begin{equation}
    \pi^{(\dagger)}=\pa{\mathcal L}{\partial_0\psi^\dagger}=0
\end{equation}
, 从而
\begin{align}
    \mathcal H&=\pi\partial_0\psi-\mathcal L=\bar\psi(m-i\gamma^i\partial_i)\psi.
\end{align}

引入$\a p, \b p$, 我们可以将Dirac场量子化
\begin{align}
    &\psi_A=\int\ldsq{p}\sum_s(u^s_A\a{p}^s\exp{-ipx}+v^{s}_A\b{p}^{s\dagger}\exp{ipx})\\
    &\bar\psi^A=\int\ldsq{p}\sum_s(\bar u^{sA}\a{p}^{s\dagger}\exp{ipx}+\bar v^{sA}\b{p}^{s}\exp{-ipx}).
\end{align}

我们可以计算得到
\begin{align}
    m\int\d^3x\bar\psi_x\psi_x&=\int\d^3x\int\ldsq p\sum_s\left(\bar u^s\a p^{s\dagger}\exp{ipx}+\bar v^s\b p^s\exp{-ipx}\right)\notag\\
    &\;\;\int\ldsq q\sum_{s'}\left(u^{s'}\a q^{s'}\exp{-iqx}+v^{s'}\b q^{s'\dagger}\exp{iqx}\right)\\
    &=\int\ddd p\frac{m^2}{\om p}\sum_s\left(\a p^\dagger\a p-\b p\b p^\dagger\right)
\end{align}
以及
\begin{align}
    \int\d^3x\bar\psi(-i\gamma^i\partial_i)\psi&=\int\d^3x\ldsq q\sum_s\left(\bar u^s\a q^{s\dagger}\exp{iqx}+\bar v^s\b q^s\exp{-iqx}\right)\notag\\
    &\;\;\int\ldsq p[\gamma^ip_i]\sum_{s'}\left(-u^{s'}\a p\exp{-ipx}+v^{s'}\b p^{s'\dagger}\exp{ipx}\right)\\
    &=\int\ld p\sum_{ss'}\left(-\bar u^s\gamma^ip_i u^{s'}\a p^{s\dagger}\a p^{s'}+\bar v^s\gamma^ip_iv^{s'}\b p^s\b p^{s'\dagger}\right)
\end{align}
利用式\eqref{gamma-p-p-contraction}, 我们有
\begin{align}
    \bar u^s\gamma^ip_i u^{s'}&=-2\vec p^2\delta_{ss'}\\
    \bar v^s\gamma^ip_i v^{s'}&=-2\vec p^2\delta_{ss'}
\end{align}
从而有
\begin{align}
    \int\d^3x\bar\psi(-i\gamma^i\partial_i)\psi&=\int\ddd p\frac{\vec p^2}{\om p}\sum_s\left(\a p^\dagger\a p-\b p\b p^\dagger\right)
\end{align}
利用这两式, 我们可以得到Hamiltonian
\begin{equation}
    H=\int\d^3\mathcal H=\int\ddd p\om p\sum_s\left(\a p^\dagger\a p-\b p\b p^\dagger\right)
\end{equation}

类似地, 我们利用正则对易关系, $\b p, \b p^\dagger$给对易过来, 似乎就可以得到最终结果了...?
\begin{equation}
    H=\int\d^3\mathcal H=\int\ddd p\om p\sum_s\left(\a p^\dagger\a p-\b p^\dagger\b p\right)
\end{equation}
但是这个式子有个巨大的问题: 非正定! 也就是说, 如果$\b p^\dagger$激发产生一个反粒子, 那么它的能量是负的! 而大自然会倾向于低能量的状态, 也就是这会导致Dirac旋量场会自发放出无穷大的能量! 这个在物理上显然是荒谬的. 而我们可以确信我们上述的计算过程是没有问题的, 那么只有两个地方的假设可能是错的: 1. Dirac场能用极化旋量和$\a p, \b p$表示. 2. 正则对易关系.

而第一个假设如果放弃, 我们会失去一切. 因此我们的首选做法是放弃第二个假设. 注意到, 如果我们引入反对易关系:
\begin{align}
    & \{\a p, \a q^\dagger\}=\dpi3\delta^3(\vec p-\vec q), \{\a b, \b q^\dagger\}=\dpi3\delta^3(\vec p-\vec q)\\
    & \{\a p, \a q\}=\{\a p^\dagger, \a q^\dagger\}=\{\b p, \b q\}=\{\b p^\dagger, \b q^\dagger\}=0
\end{align}

则Hamiltonian可以对易为
\begin{equation}
    H=\int\d^3\mathcal H=\int\ddd p\om p\sum_s\left(\a p^\dagger\a p+\b p^\dagger\b p+\mathcal V\right)
\end{equation}
这样就可以得到正确的结果.

反对易关系的引入, 还导致了在统计上与对易关系的玻色子的重大区别: Pauli不相容原理, 即
\begin{equation}
    \a p^\dagger\a p^\dagger\ket0=0.
\end{equation}
我们不能让两个电子处于完全一样的态, 这样的粒子称为费米子, 遵从Fermi-Dirac分布, 而满足对易关系的粒子称为玻色子, 遵从Bose-Einstein分布. 接下来我们还将从关联函数协变性的角度更深刻地看到费米子反对易性的必要性.

然后我们尝试计算两点关联函数, 首先有
\begin{align}
    \braket{0|\psi_A(y)\bar\psi^B(x)|0}&=\int\ddd{p}\frac{\slashed p_A^{~~B}+m}{2\om p}\exp{ip(x-y)}
\end{align}
\begin{align}
    \braket{0|\bar\psi^B(x)\psi_A(y)|0}&=\int\ddd{p}\frac{\slashed p_A^{~~B}-m}{2\om p}\exp{-ip(x-y)}
\end{align}

考虑Fermion的反交换性, 我们很自然地要求
\begin{equation}
    \mathcal T\left\{\psi_x\bar\psi_y\right\}=-\mathcal T\left\{\bar\psi_y\psi_x\right\}
\end{equation}
这需要我们定义对Dirac场的编时算符为
\begin{equation}
    \mathcal T\left\{\psi_A(y)\bar\psi^B(x)\right\}\equiv\Theta(y^0-x^0)\psi_A(y)\bar\psi^B(x)-\Theta(x^0-y^0)\bar\psi^B(x)\psi_A(y)
\end{equation}

这样我们经过类似\ref{2pt-real-scalar}节后半段的方法计算可以得到
\begin{align}
    \Theta(y^0-x^0)\psi_A(y)\bar\psi^B(x)&=\int\frac{i\d\omega}{2\pi}\frac{\exp{i\omega(t_x-t_y)}}{\omega+i\epsilon}\int\ld p[\slashed p+m]\exp{ip(x-y)}\\
    &=\int i\dddd p\frac{\slashed p-(p^0-\om p)\gamma^0+m}{2\om p(p^0-\om p+i\epsilon)}\exp{ip(x-y)}\label{dirac-spinor-2pt1}\\
    \mathcal T\left\{\bar\psi_y\psi_x\right\}&=\int i\dddd p\frac{-\slashed p+(p^0+\om p)\gamma^0-m}{2\om p(-p^0-\om p+i\epsilon)}\exp{ip(x-y)}\label{dirac-spinor-2pt2}
\end{align}

将\eqref{dirac-spinor-2pt1}, \eqref{dirac-spinor-2pt2}两式相减即可得Feynman传播子
\begin{equation}
    \braket{\psi_A(y)\bar\psi^B(x)}=\int\dddd p\frac{i(\slashed p+m)}{p^2-m^2+i\epsilon}\exp{ip(x-y)}
\end{equation}
不难看出这是协变的.

\kaishu
而如果我们仍然强加对易关系到编时算符上, 即定义
\begin{equation}
    \mathcal T\left\{\psi_A(y)\bar\psi^B(x)\right\}\equiv\Theta(y^0-x^0)\psi_A(y)\bar\psi^B(x)+\Theta(x^0-y^0)\bar\psi^B(x)\psi_A(y)
\end{equation}
则需要将\eqref{dirac-spinor-2pt1}, \eqref{dirac-spinor-2pt2}两式相加, 得到
\begin{equation}
    \braket{\psi_A(y)\bar\psi^B(x)}=\int i\dddd p\frac{\omega}{\om p}\frac{\slashed p-\frac{p^2-m^2}{\omega}\gamma^0+m}{p^2-m^2+i\epsilon}
\end{equation}
这又丑又完全不协变, 是一个灾难性的结果.
\songti

\newpage
\section{离散变换}
在本节我们尝试讨论对场的三种幺正离散变换, 即$C,P,T$变换. 所谓的$C$, 其实就是电荷共轭(Charge Conjugate), 就是指将粒子变为反粒子, 反转其的所有量子荷(比如电荷, 但还包括轻子数、重子数等所有性质). 而$P$就是大名鼎鼎的宇称(Parity), 就是指$\vec x\to-\vec x$的镜像变换. $T$则是时间反演(Time reversal), 也就是将对象的运动过程反演: $t\to-t$. 

一般性地讨论来说, 对于任意的离散变换算符$X$, 我们有如下定义
\begin{equation}
    X\psi(x)X^\dagger=\boldsymbol{\mathrm C}\psi(\Lambda_X\cdot x)
\end{equation}
其中$\boldsymbol{\mathrm X}$是对场量的变换算符, $\Lambda_X$是对坐标的变换, 而$X$是一个幺正算符
\begin{equation}
    XX^\dagger=1.
\end{equation}

\subsection{C变换}
$C$变换就是取反粒子, 也就是对场算符我们有如下要求
\begin{equation}
    C\psi(x)C^\dagger=\boldsymbol{\mathrm C}\psi^\dagger(x).
\end{equation}
这里取$\dagger$是因为正反粒子的相位旋转方向不同, 我们将场算符中的正反粒子反转必然需要取一个共轭. 并且其中$C$在幺正的基础上还满足
\begin{equation}
    C^2=1, 
\end{equation}


上面的要求等价于我们要求
\begin{align}
    & C\b{p}^\dagger\ket0=\a{p}^\dagger\ket 0\\
    & C\a{p}^\dagger\ket0=\b{p}^\dagger\ket 0
\end{align}

在$\b{p}^\dagger, \a{p}^\dagger$与$\ket 0$间插入$C^\dagger c$, 我们有
\begin{align}
    & C\b{p}^\dagger C^\dagger C\ket0=\a{p}^\dagger\ket 0\\
    & C\a{p}^\dagger C^\dagger C\ket0=\b{p}^\dagger\ket 0
\end{align}

这暗示我们
\begin{align}
    & C\b{p}^\dagger C^\dagger=\a{p}^\dagger\\
    & C\a{p}^\dagger C^\dagger=\b{p}^\dagger.
\end{align}

然后我们以实标量场、复标量场、Dirac旋量场为例讨论$C$的具体作用.
\begin{example}[实标量场的电荷共轭变换]
    由于实标量场只有一组产生湮灭算符, 它自己就是自己的反粒子, $C$变换对其不起作用, 于是我们有
    \begin{equation}
        C=1, \boldsymbol{\rm C}=1
    \end{equation}
    % 其中$\dagger$即为对场算符取$\dagger$
\end{example}
\begin{example}[复标量场的电荷共轭变换]
    对于复标量场, 我们也没有特殊的要求, 仍然有
    \begin{equation}
        \boldsymbol{\rm C}=1
    \end{equation}
    于是我们有结论
    \begin{align}
        &C\psi(x)C^\dagger=\int\ldsq p\left(\a p\exp{-ipx}+\b p^\dagger\exp{ipx}\right)=\psi^\dagger(x)
    \end{align}
    所以
    \begin{align}
        & C\b{p}^\dagger C^\dagger=\a{p}^\dagger\\
        & C\a{p}^\dagger C^\dagger=\b{p}^\dagger.
    \end{align}
\end{example}
\begin{example}[实矢量场的电荷共轭变换]
    由于实矢量场只有一组产生湮灭算符, 它自己就是自己的反粒子, $C$变换对其不起作用, 于是我们有
    \begin{equation}
        C=1, \boldsymbol{\rm C}=1
    \end{equation}
\end{example}
\begin{example}[Dirac旋量的电荷共轭变换]
    我们首先推导$\boldsymbol{\mathrm C}$. 因为它负责将正粒子变为反粒子, 而正粒子反粒子的与$\slashed p$乘积满足的条件是不一样的:
    \begin{align}
        (\slashed p-m)\boldsymbol{\mathrm C}u^{s}=0\\
        (\slashed p+m)\boldsymbol{\mathrm C}v^{s}=0.
    \end{align}
    所以说我们需要求一个$\boldsymbol{\mathrm C}$使得
    \begin{align}
        \boldsymbol{\mathrm C}v^{s*}=u^s\\
        \boldsymbol{\mathrm C}u^{s*}=v^s\label{uv-conguate-star}.
    \end{align}
    而这意味着
    \begin{align}
        (\slashed p-m)\boldsymbol{\mathrm C}v^{s*}=0\\
        (\slashed p+m)\boldsymbol{\mathrm C}u^{s*}=0.
    \end{align}
    又因为
    \begin{align}
        m\boldsymbol{\mathrm C}u^{s*}&=\boldsymbol{\mathrm C}(mu)^{s*}\\
        &=p_\mu\boldsymbol{\mathrm C}\gamma^{\mu*} u^{s*}
    \end{align}
    同时根据式\eqref{uv-conguate-star}, 我们有
    \begin{align}
        m\boldsymbol{\mathrm C}u^{s*}=mv^s=-p_\mu\gamma^\mu v^s=-p_\mu\gamma^\mu\boldsymbol{\mathrm C}u^{s*}
    \end{align}
    所以有
    \begin{equation}
        \boldsymbol{\mathrm C}\gamma^\mu\boldsymbol{\mathrm C}^\dagger=-\gamma^{\mu*}.
    \end{equation}

    又注意到, 
    \begin{align}
        \gamma^{\mu*}=\gamma^\mu, \mu=0,1,3\\
        \gamma^{2*}=-\gamma^2.
    \end{align}
    所以我们不难猜想, $\boldsymbol{\mathrm C}$应当与$\gamma^2$有关. 不难验证,
    \begin{equation}
        \boldsymbol{\mathrm C}=i\gamma^2
    \end{equation}

    所以说我们得到
    \begin{align}
        &C\psi_AC^\dagger=\int\ldsq{p}\sum_s(u^s_A\b{p}^s\exp{-ipx}+v^{s}_A\a{p}^{s\dagger}\exp{ipx})
    \end{align}
    这同样要求
    \begin{align}
        & C\b{p}^\dagger C^\dagger=\a{p}^\dagger\\
        & C\a{p}^\dagger C^\dagger=\b{p}^\dagger.
    \end{align}

    % 我们想要将其写为与$\psi^*$(注意, 这里如果是$\psi^\dagger$那么就变成行向量了, 结构不同, 而$\psi^*$则只是取共轭, 也就是将$a, b$取$\dagger$, 给$u, v$取$*$却不取$\dagger$)有关的东西, 那么我们就需要获得$u$与$v$之间的关系.
\end{example}

\subsection{P变换}\label{sec-parity}
$P$变换就是宇称变换, 即将场的三维部分反转
\begin{equation}
    \psi(t, \vec x)\to\psi(t, -\vec x), 
\end{equation}
也就是
\begin{equation}
    P\psi(t, \vec x)P^\dagger=\boldsymbol{\mathbf P}\psi(t, -\vec x), 
\end{equation}
其中$P$在幺正的基础上还满足
\begin{equation}
    P^2=1.
\end{equation}

\begin{example}[实标量场的宇称变换]
    对于标量来说(如果是赝标量那么就是$\boldsymbol{\mathbf P}=-1$了)
    \begin{equation}
        \boldsymbol{\mathbf P}=1
    \end{equation}
    我们发现这样要求
    \begin{align}
        &P\phi(x)P^\dagger=\int\ldsq p\left(\a p\exp{-i\omega t-i\vec p\cdot\vec x}+\a p^\dagger\exp{i\omega t+i\vec p\cdot\vec x}\right)
    \end{align}
    我们做换元$\vec p\to-\vec p$得到
    \begin{align}
        &P\psi(x)P^\dagger=\int\ldsq p\left(\a {-p}\exp{-ipx}+\a {-p}^\dagger\exp{ipx}\right)
    \end{align}
    所以说有结论
    \begin{equation}
        P\a p P^\dagger=\a{-p}
    \end{equation}
\end{example}
\begin{example}[复标量场的宇称变换]
    同样有
    \begin{equation}
        \boldsymbol{\mathbf P}=1
    \end{equation}
    我们发现这样要求
    \begin{align}
        &P\psi(x)P^\dagger=\int\ldsq p\left(\a p\exp{-i\om p t-i\vec p\cdot\vec x}+\b p^\dagger\exp{i\om p t+i\vec p\cdot\vec x}\right)
    \end{align}
    做换元$\vec p\to-\vec p$得到
    \begin{align}
        &P\psi(x)P^\dagger=\int\ldsq p\left(\a {-p}\exp{-ipx}+\b {-p}^\dagger\exp{ipx}\right)
    \end{align}
    所以说有结论
    \begin{align}
        P\a p P^\dagger=\a{-p}\\
        P\b p P^\dagger=\b{-p}
    \end{align}
\end{example}
\begin{example}[电磁场的宇称变换]
    考虑到Lorentz规范下, 电磁4矢势满足
    \begin{equation}
        \Box A^\mu=J^\mu.
    \end{equation}
    而$J^\mu$在$P$变换下满足
    \begin{equation}
        \boldsymbol{\mathbf P}J^\mu=\boldsymbol{\mathbf P}(\rho, \vec j)^T=(\rho, -\vec j)^T,
    \end{equation}
    这个性质也自然地被诱导给$A^\mu$, 因此$A^\mu$同样满足
    \begin{equation}
        \boldsymbol{\mathbf P}A^\mu=\boldsymbol{\mathbf P}(\varphi, \vec A)^T=(\varphi, -\vec A)^T.
    \end{equation}
    
    考虑到对于电磁波极化矢量$\epsilon_s^\mu$(在Lorentz规范下它是完全没有$t$分量的, 也就是$\epsilon^0=0$)我们有
    \begin{align}
        \boldsymbol{\mathbf P}\epsilon_1^\mu(\vec p)=-\epsilon_1^\mu(\vec p)=\epsilon_1^i(-\vec p)\\
        \boldsymbol{\mathbf P}\epsilon_2^\mu(\vec p)=-\epsilon_2^\mu(\vec p)=\epsilon_2^i(-\vec p)
    \end{align}
    从而得到
    \begin{align}
        P\vec A P^\dagger&=\int\ldsq p\sum_{r=1}^2\left(\vec\epsilon_r(-\vec p)a_{\vec pr}\exp{-i\om pt-i\vec p\cdot\vec x}+\vec\epsilon_r^*(-\vec p)a_{\vec pr}^\dagger\exp{i\om pt+i\vec p\cdot\vec x}\right)\\
        &=\int\ldsq p\sum_{r=1}^2\left(\vec\epsilon_r(\vec p)a_{-\vec pr}\exp{-ipx}+\vec\epsilon_r^*(\vec p)a_{-\vec pr}^\dagger\exp{ipx}\right)
    \end{align}
    所以说有结论
    \begin{align}
        P\a p P^\dagger=\a{-p}\\
        P\b p P^\dagger=\b{-p}
    \end{align}
\end{example}
\begin{example}[Dirac旋量的宇称变换]
    首先我们需要推导旋量在宇称变换下的矩阵. 因为宇称变换是$\vec x\to-\vec x$这同样代表着$\vec p\to-\vec p$, 所以说我们要求一$\boldsymbol{\mathbf P}$满足
    \begin{equation}
        \boldsymbol{\mathbf P}u_s(\vec p)=\eta_s u_s(-\vec p)
    \end{equation}
    其中$\eta_s$为某一相位常数.

    注意到
    \begin{align}
        \gamma^0u^s(\vec p)&=\gamma^0\begin{pmatrix}
            \sqrt{p\sigma}\zeta_s\\
            \sqrt{p\bar\sigma}\zeta_s
        \end{pmatrix}=\begin{pmatrix}
            \sqrt{p\bar\sigma}\zeta_s\\
            \sqrt{p\sigma}\zeta_s
        \end{pmatrix}=u^s(-\vec p)\\
        \gamma^0`v^s(\vec p)&=\gamma^0\begin{pmatrix}
            \sqrt{p\sigma}\zeta_s\\
            -\sqrt{p\bar\sigma}\zeta_s
        \end{pmatrix}=\begin{pmatrix}
            -\sqrt{p\bar\sigma}\zeta_s\\
            \sqrt{p\sigma}\zeta_s
        \end{pmatrix}=-v^s(-\vec p)
    \end{align}
    所以说
    \begin{equation}
        \boldsymbol{\mathbf P}=\gamma^0.
    \end{equation}

    根据定义我们直接有
    \begin{align}
        P\psi(t,\vec x)P^\dagger&=\boldsymbol{\mathbf P}\psi(t,-\vec x)\\
        &=\int\ldsq{p}\sum_s(u^s(-\vec p)\a{p}^s\exp{-i\om pt-i\vec p\cdot\vec x}-v^{s}(-\vec p)\b{p}^{s\dagger}\exp{i\om pt+i\vec p\cdot\vec x})
    \end{align}
    做$\vec p\to-\vec p$换元我们有
    \begin{align}
        P\psi(t,\vec x)P^\dagger&=\int\ldsq{p}\sum_s(u^s(\vec p)\a{-p}^s\exp{-ipx}-v^{s}(\vec p)\b{-p}^{s\dagger}\exp{ipx})
    \end{align}

    所以得到结论
    \begin{align}
        P\a p P^\dagger=\a{-p}\\
        P\b p P^\dagger=-\b{-p}
    \end{align}
\end{example}

\subsection{T变换}
$T$变换就是时间反演变换, 这个变换比较特殊, 因为它需要是反线性的: 在Schrödinger绘景下, 考虑态演化, 我们有
\begin{equation}
    (i\pa{}t-H)\ket{\psi(t)}=0.\label{time-reverse-schrodinger-eq}
\end{equation}
时间反演要求有
\begin{equation}
    T\ket{\psi(t)}=\ket{\psi(-t)}
\end{equation}
所以说
\begin{equation}
    (i\pa{}t+H)T\ket{\psi(t)}=0
\end{equation}
$H$不含时, 因此$T$可与$H$交换. 同样地, $T$可以与时间导数算子交换, 但如果$T$不是反线性, 就会有
\begin{equation}
    T\left((i\pa{}t+H)\ket{\psi(t)}\right)=0
\end{equation}
这和式\eqref{time-reverse-schrodinger-eq}是矛盾的. 但是如果$T$是反线性的, 即
\begin{equation}
    iT=-Ti
\end{equation}
那么我们就可以得到
\begin{equation}
    T\left((-i\pa{}t+H)\ket{\psi(t)}\right)=0.
\end{equation}
这个结果才是自洽的.

对于场算符, 我们要求$T$的表现形式为
\begin{equation}
    T\psi(t, \vec x)T^\dagger=\boldsymbol{\rm T}\phi(-t, \vec x).
\end{equation}

\subsection{CPT不变性}
如果$CPT$组合起来, 这是一个反线性算符, 并且对于复标量场我们容易验证
\begin{equation}
    CPT\psi(x)(CPT)^\dagger=\psi^\dagger(-x).
\end{equation}

\newpage
\section{旋量QED}
然后我们试图将Dirac旋量与电磁场耦合, 考虑Dirac旋量与电磁场的相互作用, 即标准的量子电动力学(QED).
\subsection{最小耦合}
同样考虑局部$U(1)$规范变换, 即
\begin{equation}
    \psi\to\exp{-i\alpha}\psi
\end{equation}
设
\begin{equation}
    \partial_\mu\to D_\mu=\partial_\mu+ieA_\mu
\end{equation}
让局域$U(1)$规范变换对$A_\mu$满足
\begin{equation}
    A_\mu\to A_\mu+\frac1e\partial_\mu\alpha
\end{equation}
则
\begin{equation}
    D_\mu(\exp{-i\alpha}\psi)=\exp{-i\alpha}D_\mu\psi
\end{equation}

于是根据最小耦合原理我们可以得到Lagrangian
\begin{equation}
    \mathcal L=\bar\psi(i\slashed D-m)\psi-\frac14F^2
\end{equation}
即
\begin{equation}
    \mathcal L=\bar\psi(i\partial_\mu-m)\psi-eA_\mu\bar\psi\gamma^\mu\psi
\end{equation}

于是我们发现电流项
\begin{equation}
    J^\mu=e\bar\psi\gamma^\mu\psi
\end{equation}

还有EoM
\begin{align}
    \begin{cases}
        &(i\slashed\partial-m)\psi=e\slashed A\psi\\
        &\partial_\mu F^{\mu\nu}=e\bar\psi\gamma^\nu\psi
    \end{cases}
\end{align}

\subsection{旋量LSZ公式}
我们首先有引理
\begin{lemma}
    \begin{equation}
        \int\d^4xi\exp{ipx}\bar u_s(m-i\slashed\partial)\psi=\sqrt{2\om p}(\a p^s(+\infty)-\a p^s(-\infty))
    \end{equation}
    \begin{equation}
        \int\d^4xi\exp{-ipx}\bar v_s(-m-i\slashed\partial)\psi=\sqrt{2\om p}(\b p^{\dagger s}(+\infty)-\b p^{\dagger s}(-\infty))
    \end{equation}
    \begin{equation}
        \int\d^4xi\exp{-ipx}\Tr\left[(-m-i\slashed\partial)u_s\bar\psi\right]=\sqrt{2\om p}(\a p^{\dagger s}(+\infty)-\a p^{\dagger s}(-\infty))
    \end{equation}
    \begin{equation}
        \int\d^4xi\exp{ipx}\Tr\left[(m-i\slashed\partial)v_s\bar\psi\right]=\sqrt{2\om p}(\b p^{\dagger s}(+\infty)-\b p^{\dagger s}(-\infty))
    \end{equation}
\end{lemma}
\begin{proof}
    \begin{align}
        \int\d^4xi\exp{ipx}\bar u_s(m-i\slashed\partial)\psi&=\int\d^4xi\exp{ipx}\bar u_s(m--i\gamma^i\partial_i-i\gamma^0\partial_0)\psi\\
        &=\int\d^4xi\exp{ipx}\bar u_s(m-\gamma^ip_i-i\gamma^0\partial_0)\psi=
    \end{align}
    根据
    \begin{equation}
        \partial_0(\exp{ipx}\bar u_s\gamma^0\psi)=ip_0\exp{ipx}\bar u_s\gamma^0\psi+\exp{ipx}\bar u_s\gamma^0\partial_0\psi
    \end{equation}
    我们有
    \begin{align}
        &\;\;\int\d^4xi\exp{ipx}\bar u_s(m-i\slashed\partial)\psi\\
        &=\int\d^4xi\exp{ipx}\bar u^s(m-\slashed p)\psi\notag\\
        &\;+\int\d^3x\exp{ipx}\bar u_s\gamma^0\int\ldsq q{}\sum_{s'}\left(u_{s'}\a q^s\exp{-iqx}+v_s\b q^{\dagger s}\exp{iqx}\right)\Big|_{-\infty}^{+\infty}\\
        &=\bar u_s\gamma^0\int\ldsq q{\dpi3}\sum_{s'}\left(u_{s'}\a q^s\delta^3(\vec p-\vec q)+v_s\b q^{\dagger s}\delta^3(\vec p+\vec q)\right)\Big|_{-\infty}^{+\infty}
    \end{align}
    而根据定理\ref{ugammau-contraction}我们知道
    \begin{equation}
        \bar u_s(\vec p)\gamma^0u_{s'}(\vec p)=2p^0\delta_{ss'}, \bar u_s(\vec p)\gamma^0v_{s'}(\vec p)=0
    \end{equation}
    从而有
    \begin{equation}
        \int\d^4xi\exp{ipx}\bar u_s(m-i\slashed\partial)\psi=\sqrt{2\om p}(\a p^s(+\infty)-\a p^s(-\infty))
    \end{equation}

    剩下三个式子证明同理.
\end{proof}

于是我们可以有最终结论
\begin{align}
    \braket{f, +\infty|i, -\infty}&=\int\d^4xi\exp{ip_1x_1}\bar u_s(m-i\slashed\partial_1)\textcolor{blue}{\int\d^4x_2i\exp{-ipx}\bar v_s(-m-i\slashed\partial_2)}\notag\\
    &\;\partial_{3\mu}\partial_{4\nu}\braket{\psi_1\psi_2...\bar\psi_3\bar\psi_4}\notag\\
    &\;\int\d^4xi\exp{-ip_3x_3}(-m-i\gamma^\mu)u_s\textcolor{blue}{\int\d^4xi\exp{ip_4x_4}(m-i\gamma^\nu)v_s}
\end{align}
其中, 第一行第一个为出射正粒子, 第一行第二个为入射反粒子, 第三行第一个为入射正粒子, 第三行第四个为出射反粒子.

\subsection{旋量QED的Feynman规则}
我们可以不难从Lagrangian中读出Feynman规则
\begin{enumerate}
    \item Dirac旋量传播子
    $$
        \begin{tikzpicture}[baseline=(current bounding box.center)]
            \begin{feynman}
                \vertex (a) {\(p\)};
                \vertex [right=2cm of a] (b);
                \diagram* {
                    (a) -- [fermion, momentum'=\(p\)] (b),
                };
            \end{feynman}
        \end{tikzpicture}
        = \frac{i(\slashed{p} + m)}{p^2 - m^2 + i\epsilon}
    $$
    \item 光子传播子
    $$
        \begin{tikzpicture}[baseline=(current bounding box.center)]
            \begin{feynman}
                \vertex (a) {\(\mu\)};
                \vertex [right=2cm of a] (b) {\(\nu\)};
                \diagram* {
                    (a) -- [photon, momentum'=\(p\)] (b),
                };
            \end{feynman}
        \end{tikzpicture}
        = \frac{-i g_{\mu\nu}}{p^2 + i\epsilon} \quad (\text{Feynman 规范})
    $$
    \item 旋量-光子三点顶角
    $$
    \begin{tikzpicture}[baseline=(current bounding box.center)]
        \begin{feynman}
            \vertex (a);
            \vertex [above right=1.5cm of a] (b);
            \vertex [below right=1.5cm of a] (c);
            \vertex [left=1.5cm of a] (d);
            \diagram* {
                (b) -- [fermion] (a) -- [fermion] (c),
                (a) -- [photon] (d),
            };
        \end{feynman}
    \end{tikzpicture}
    = -ie\gamma^\mu
    $$
\end{enumerate}

对于每一个过程的外线,都有对应的因子:
\begin{itemize}
    \item 入射电子(${e^-}$): $u(p, s)$
    \item 出射电子 (${e^-}$): $\bar{u}(p, s)$
    \item 入射正电子 (${e^+}$): $\bar{v}(p, s)$
    \item 出射正电子 (${e^+}$): $v(p, s)$
    \item 入射光子 (${\gamma}$): $\epsilon^\mu(p, \lambda)$
    \item 出射光子 (${\gamma}$): $\epsilon^{\mu*}(p, \lambda)$
\end{itemize}

需要注意, 由于Fermion场的反对易性, 对于一些构型我们会有额外的$-1$因子, 比如Fermion环. 比较保险的方法是将图还原到Dyson级数的缩并中具体地检查正负号.

\subsection{树图阶散射计算}
接下来我们开始计算几个简单的例子来熟悉QED的计算, 并获得它们的散射界面. 在计算之前, 我们首先需要考虑一个问题: 在实际实验中, 我们是并不知道粒子的自旋设置的, 或者说它们的自旋方向是完全随机的, 而我们的实验只能测量随机方向自旋的总效应: 对于入射态, 我们需要对不同自旋方向的$|\mathcal M|^2$取平均; 对于出射态, 因为不同的自旋出射都贡献到总的散射截面中, 因此我们需要直接求和. 比如考虑$n\to m$散射, 我们就有
\begin{equation}
    |\mathcal M_{aver}|=\frac1{2^n}\sum_{i_n, f_m}|\mathcal M_{i_n, f_m}|^2
\end{equation}
\begin{example}[Bhabha散射]
    Bhabha散射即电子-正电子散射:
    $$ e^-+e^+\to e^-+e^+. $$

    我们在树图阶考虑, 这有两个通道的贡献
    \begin{align}
        \begin{tikzpicture}[baseline=(current bounding box.center)]
            \begin{feynman}
                \coordinate (i1) at (-1, 1.5);
                \coordinate (f1) at (1, 1.5);
                \vertex (v1) at (0, 0.5); % 上侧顶点
                \coordinate (i2) at (-1, -1.5);
                \coordinate (f2) at (1, -1.5);
                \vertex (v2) at (0, -0.5);  % 下侧顶点
                \diagram* {
                    (i1) -- [fermion] (v1) -- [fermion] (f1),
                    (v1) -- [photon] (v2), % 交换的玻色子
                    (f2) -- [fermion] (v2) -- [fermion] (i2),
                };
            \end{feynman}
        \end{tikzpicture}&=-ie^2\bar u_3\gamma^\mu u_1\bar v_2\gamma_\mu v_4\frac1{t}
    \end{align}
    \begin{align}
        \begin{tikzpicture}[baseline=(current bounding box.center)]
            \begin{feynman}
                \coordinate (i1) at (-1.5, 1);
                \coordinate (i2) at (-1.5, -1);
                \coordinate (f1) at (1.5, 1);
                \coordinate (f2) at (1.5, -1);
                \vertex (v1) at (-0.5, 0); % 左侧顶点
                \vertex (v2) at (0.5, 0);  % 右侧顶点
                \diagram* {
                    (i1) -- [fermion] (v1),
                    (v1) -- [fermion] (i2),
                    (v1) -- [photon] (v2), % 交换的玻色子
                    (v2) -- [fermion] (f1),
                    (f2) -- [fermion] (v2),
                };
            \end{feynman}
        \end{tikzpicture}&=ie^2\bar v_2\gamma^\mu u_1\bar u_3\gamma_\mu v_4\frac1{s}
    \end{align}
    所以我们可以得到树图阶的总振幅
    \begin{align}
        \mathcal M_{total}=e^2\left(\frac{-\bar v_2\gamma^\mu u_1\bar u_3\gamma_\mu v_4}{s}+\frac{\bar u_3\gamma^\mu u_1\bar v_2\gamma_\mu v_4}{t}\right).
    \end{align}

    为了简化我们的计算, 我们假设$p^2\gg m^2$, 因此$m^2$可以被忽略.

    然后计算其模长平方
    \begin{align}
        |\mathcal M_{aver}|^2&=\frac14\sum|\mathcal M|^2\\
        &=\frac{e^4}4\sum\left[-\frac{\bar u_1\gamma^\mu v_2\bar v_4\gamma_\mu u_3}{s}+\frac{\bar u_1\gamma^\mu u_3\bar v_4\gamma_\mu v_2}{t}\right]\notag\\
        &\;\;\left[\frac{-\bar v_2\gamma^\mu u_1\bar u_3\gamma_\mu v_4}{s}+\frac{\bar u_3\gamma^\mu u_1\bar v_2\gamma_\mu v_4}{t}\right].
    \end{align}
    这是一个相当相当复杂的算式, 我们耐心地一项一项展开计算: 首先是$t^2$项
    \begin{align}
        \bar u_1\gamma^\mu v_2\bar v_4\gamma_\mu u_3\bar v_2\gamma^\mu u_1\bar u_3\gamma_\mu v_4&=\rm{Tr}\left[\gamma^\mu\slashed p_2\gamma^\nu\slashed p_1\right]\rm{Tr}\left[\gamma_\mu\slashed p_3\gamma_\nu\slashed p_4\right]\\
        &=32\left[(p_2\cdot p_3)(p1\cdot p_4)+(p_2\cdot p_4)(p_1\cdot p_3)\right]\\
        &=32\left[(m^2-\frac u2)^2+(m^2-\frac t2)^2\right]\\
        &\approx8(u^2+t^2).
    \end{align}
    然后是$t, s$交叉项, 利用定理\ref{gamma-contraction-them}, 我们有
    \begin{align}
        \rm{Tr}\left[\bar u_1\gamma^\mu v_2\gamma^\nu v_4\bar v_4\gamma_\mu u_3\gamma_\nu u_1\right]=-8\rm{Tr}(\sl p_4\sl p_1)(p_2\cdot p_3)=-32(p_1\cdot p_4)(p_2\cdot p_3)
    \end{align}
    最后是$s^2$项, 我们有
    \begin{align}
        \bar u_1\gamma^\mu v_3\bar v_4\gamma_\mu u_2\bar v_3\gamma^\mu u_1\bar u_2\gamma_\mu v_4&=\rm{Tr}\left[\gamma^\mu\slashed p_3\gamma^\nu\slashed p_1\right]\rm{Tr}\left[\gamma_\mu\slashed p_2\gamma_\nu\slashed p_4\right]\\
        &=32\left[(p_2\cdot p_3)(p1\cdot p_4)+(p_3\cdot p_4)(p_1\cdot p_2)\right]\\
        &=32\left[(m^2-\frac u2)^2+(\frac s2-m^2)^2\right]\\
        &\approx8(u^2+s^2).
    \end{align}
    于是最终我们有结论
    \begin{align}
        |\mathcal M_{aver}|&=2e^4\left(\frac{u^2+t^2}{s^2}+2\frac{u^2}{st}+\frac{u^2+s^2}{t^2}\right)\\
        &=2e^4\left((\frac us+\frac ut)^2+\frac{t^2}{s^2}+\frac{s^2}{t^2}\right).
    \end{align}

    根据Feynman黄金规则, 我们利用例\ref{ex-2-2-scattering}中的结果, 在质心系($|\vec p_f|=|\vec p_i|$)下我们有
    \begin{equation}
        \left(\frac{\d\sigma}{\d\Omega}\right)_{CM}=\frac{|\mathcal M|^2}{64\pi^2E_{CM}^2}
    \end{equation}
    最后代入我们计算得到的$|\mathcal M_{aver}|$, 就有结果
    \begin{align}
        \left(\frac{\d\sigma}{\d\Omega}\right)_{CM}&=\frac{\alpha^2}{2E_{CM}|\vec p_i|}\left((\frac us+\frac ut)^2+\frac{t^2}{s^2}+\frac{s^2}{t^2}\right)\\
        &=\frac{\alpha^2}{4E^2}\left((\frac us+\frac ut)^2+\frac{t^2}{s^2}+\frac{s^2}{t^2}\right).
    \end{align}

    然后我们尝试进一步化简, 将其写为偏转角$\theta$的函数. 我们设
    \begin{align}
        &p_1=(E, E, 0, 0), p_2=(E, -E, 0, 0)\\
        &p_3=(E, E\cos\theta, E\sin\theta, 0), p_4=(E, -E\cos\theta, -E\sin\theta, 0)
    \end{align}
    从而计算得到
    \begin{align}
        s&\approx4E^2\\
        t&\approx-4E^2\sin^2(\theta/2)\\
        u&\approx-4E^2\cos^2(\theta/2)
    \end{align}
    因此有最终结论
    \begin{equation}
        \left(\frac{\d\sigma}{\d\Omega}\right)_{CM}=\frac{\alpha^2}{4E^2}\left(\left(-\cos^2(\theta/2)+\cot^2(\theta/2)\right)^2+\sin^4(\theta/2)+\frac1{\sin^4(\theta/2)}\right)
    \end{equation}
    
    可以发现t通道对散射截面的贡献在$\theta\to0$的时候是发散的, 这是因为光子传播子的长程性质, 与我们在经典力学中的计算情况一致. 而总的计算结果则是在经典计算上加入了对正负电子湮灭然后产生的这一量子修正.
\end{example}
\begin{example}[Møller散射]
    Møller散射即电子-电子散射:
    $$ e^-e^-\to e^-e^-. $$

    我们在树图阶计算, 有t通道与u通道的贡献
    \begin{align}
        \begin{tikzpicture}[baseline=(current bounding box.center)]
            \begin{feynman}
                \coordinate (i1) at (-1, 1.5);
                \coordinate (f1) at (1, 1.5);
                \vertex (v1) at (0, 0.5); % 上侧顶点
                \coordinate (i2) at (-1, -1.5);
                \coordinate (f2) at (1, -1.5);
                \vertex (v2) at (0, -0.5);  % 下侧顶点
                \diagram* {
                    (i1) -- [fermion] (v1) -- [fermion] (f1),
                    (v1) -- [photon] (v2), % 交换的玻色子
                    (i2) -- [fermion] (v2) -- [fermion] (f2),
                };
            \end{feynman}
        \end{tikzpicture}&=-ie^2\bar u_3\gamma^\mu u_1\bar u_4\gamma_\mu u_2\frac1{t}
    \end{align}
    \begin{align}
        \begin{tikzpicture}[baseline=(current bounding box.center)]
            \begin{feynman}
                \coordinate (i1) at (-1, 1.5);
                \coordinate (f1) at (1, 1.5);
                \vertex (v1) at (0, 0.5); % 上侧顶点
                \coordinate (i2) at (-1, -1.5);
                \coordinate (f2) at (1, -1.5);
                \vertex (v2) at (0, -0.5);  % 下侧顶点
                \diagram* {
                    (i1) -- [fermion] (v1) -- [fermion] (f2),
                    (v1) -- [photon] (v2), % 交换的玻色子
                    (i2) -- [fermion] (v2) -- [fermion] (f1),
                };
            \end{feynman}
        \end{tikzpicture}&=ie^2\bar u_3\gamma^\mu u_2\bar u_4\gamma_\mu u_1\frac1{u}
    \end{align}
    所以我们可以得到树图阶的总振幅
    \begin{align}
        \mathcal M_{total}=e^2\left(\frac{-\bar u_3\gamma^\mu u_1\bar u_4\gamma_\mu u_2}{t}+\frac{\bar u_3\gamma^\mu u_2\bar u_4\gamma_\mu u_1}{u}\right).
    \end{align}

    然后计算其模长平方
    \begin{align}
        |\mathcal M_{aver}|^2&=2e^4\left(\frac{(s-2m)^2+(t-2m)^2+4m^2u}{u^2}\right.\notag\\
        &\;\;\left.+\frac{(s-2m)^2+(u-2m)^2+4m^2t}{t^2}+2\frac{(s-2m^2)(2-6m^2)}{tu}\right)
    \end{align}
\end{example}
\begin{example}[Campton散射]
    Campton散射即
    $$\gamma+e^-\to\gamma+e^-.$$
    在Campton的原始实验中, 他将单色X光打在石墨上, X射线发生了散射, 并且存在不同散射角度不同光频的分布. 因此我们的散射发生在电子静止系中, 并且是考虑光子的散射. 设光子的散射角为$\theta$, 则根据能动量守恒我们很容易得到散射光$\omega'$与入射光$\omega$之间的关系
    \begin{equation}
        \frac m{\omega'}-\frac m\omega=1-\cos\theta.
    \end{equation}
    或者说
    \begin{equation}
        \omega'=\frac{m\omega}{m+(1-\cos\theta)\omega}
    \end{equation}

    然后我们考虑计算散射截面, 设入射电子动量$p_1$, 光子$p_2$; 出射电子动量$p_3$, 光子$p_4$, 计算$s,u$通道树图
    \begin{align}
        % --- s-channel ---
        i\mathcal{M}_s &= 
        \begin{tikzpicture}[baseline=(v1.base)]
            \begin{feynman}
                \vertex (v1);
                \vertex [above left=1.5cm of v1] (i1);
                \vertex [below left=1.5cm of v1] (i2);
                \vertex [right=1.5cm of v1] (v2);
                \vertex [above right=1.5cm of v2] (f1);
                \vertex [below right=1.5cm of v2] (f2);
                \diagram* {
                    (i1) -- [fermion] (v1),
                    (i2) -- [photon] (v1),
                    (v1) -- [fermion] (v2),
                    (v2) -- [fermion] (f1),
                    (v2) -- [photon] (f2),
                };
            \end{feynman}
        \end{tikzpicture}&=-ie^2\bar u_3\gamma^\nu\frac{\slashed{p_1}+\slashed{p_2}+m}{s-m^2}\gamma^\mu u_1\epsilon_{2\mu}\epsilon_{4\nu}^*
    \end{align}
    \begin{align}
        i\mathcal{M}_u &= 
        \begin{tikzpicture}[baseline=(v1.base)]
            \begin{feynman}
                \vertex (v1);
                \vertex [below=1cm of v1] (v2);
                \vertex [above left=1.5cm of v1] (ie);
                \vertex [above right=1.5cm of v1] (ogamma);
                \vertex [below left=1.5cm of v2] (igamma);
                \vertex [below right=1.5cm of v2] (oe);
                \diagram* {
                    (ie) -- [fermion] (v1),
                    (v1) -- [fermion] (v2),
                    (v2) -- [fermion] (oe),
                    (v1) -- [photon] (ogamma),
                    (v2) -- [photon] (igamma),
                };
            \end{feynman}
        \end{tikzpicture}&=-ie^2\bar u_3\gamma^\mu\frac{\slashed{p_1}-\slashed{p_4}+m}{u-m^2}\gamma^\nu u_1\epsilon_{4\nu}^*\epsilon_{2\mu}
    \end{align}
    得到总平均散射振幅
    \begin{align}
        |\mathcal M_{aver}|^2&=\frac14\sum|\mathcal M|^2\\
        &=\frac{e^4}{4}\sum\epsilon_{4\mu}^*\epsilon_{2\nu}\epsilon_{4\alpha}\epsilon_{2\beta}^*\left(\frac{\bar u_3\gamma^\nu(\slashed{p_1}+\slashed{p_2}+m)\gamma^\mu u_1}{s-m^2}+\frac{\bar u_3\gamma^\mu(\slashed{p_1}-\slashed{p_4}+m)\gamma^\nu u_1}{u-m^2}\right)\notag\\
        &\;\;\;\;\;\;\left(\frac{\bar u_1\gamma^\alpha(\slashed{p_1}+\slashed{p_2}+m)\gamma^\beta u_3}{s-m^2}+\frac{\bar u_1\gamma^\beta(\slashed{p_1}-\slashed{p_4}+m)\gamma^\alpha u_3}{u-m^2}\right)
    \end{align}
    利用Ward恒等式做替换:
    \begin{equation}
        \sum_i \epsilon_{2\mu}\epsilon_{2\nu}^*\to-g_{\mu\nu}
    \end{equation}
    利用s,t,u变量
    \begin{align}
        &s=m^2-2p_1p_2=m^2-2p_3p_4\\
        &t=m^2-2p_1p_3=-2p_2p_4\\
        &u=m^2-2p_1p_4=m^2-2p_2p_3
    \end{align}
    做代换, 经过艰苦卓绝的计算我们可以得到
    \begin{align}
        |\mathcal M_{aver}|^2&=2e^4\left(\frac{m^4-2u+m^2(3s+u)}{(s-m^2)^2}+\frac{m^4-2u+m^2(3u+s)}{(u-m^2)^2}\right.\notag\\
        &\;\;\;\;\left.+2m^2\frac{2m^2+s+u}{(s-m^2)(u-m^2)}\right)
    \end{align}

    然后我们取电子静止系具体计算s, u. 设
    \begin{align}
        p_1=(m,0,0,0), p_2=(\omega,\omega,0,0), p_4=(\omega',\omega'\cos\theta,\omega'\sin\theta)
    \end{align}
    则我们有
    \begin{align}
        s=m^2+2m\omega, u=m^2-2m\omega'
    \end{align}
    然后代入$s,u$得到
    \begin{equation}
        |\mathcal M_{aver}|^2=2e^4\left(\frac{\omega'}{\omega}+\frac{\omega}{\omega'}-\sin^2\theta\right)
    \end{equation}

    然后在电子静止系中推导散射截面和$|\mathcal M|^2$的关系:
    \begin{align}
        \d\sigma&=\frac{|\mathcal M|^2}{4E_1E_2|\vec v_1-\vec v_2|}\dpi4\delta^4\lips{p_1}\lips{p_2}\\
        &=\frac{|\mathcal M|^2}{8m\omega\sin\theta/2}\frac{1}{4E'\omega'}2\pi\delta(m+\omega-E'-\omega')\frac{\omega'^2\d\omega'\d\Omega}{\dpi3}
    \end{align}
    根据
    \begin{equation}
        E'=\sqrt{(\omega'^2+\omega^2-2\omega\omega'\cos\theta)+m^2}
    \end{equation}
    所以
    \begin{equation}
        \de{E'}{\omega'}+\de{\omega'}{\omega'}=\frac{\omega'+E'-\omega\cos\theta}{E'}=\frac{m+(1-\cos\theta)\omega}{E'}
    \end{equation}
    从而
    \begin{equation}
        \delta(m+\omega-E'-\omega')=\frac{E'}{m+\omega(1-\cos\theta)}\delta(\omega'-\frac{m\omega}{m+(1-\cos\theta)\omega})
    \end{equation}
    于是可以化简得到
    \begin{align}
        \de{\sigma}{\Omega}&=\frac{e^4}{64\pi^2m^2}\frac{\omega'^2}{\omega^2}\left(\frac{\omega'}{\omega}-\sin^2\theta\right)
    \end{align}
    其中
    \begin{equation}
        \frac{\omega'}{\omega}=\frac{m}{m+(1-\cos\theta)\omega}.
    \end{equation}
    此即Klein-Nishina公式, 这是Thomson微分界面的QED修正.
\end{example}
\begin{example}[Rutherford散射(半量子化)]
    Rutherford散射即电子与原子核的散射, 由于原子核是一个由质子中子复合形成的一个复杂束缚态, 而质子和中子又是由夸克组成的复杂束缚态. 如果我们采取直接全部量子化计算的方式, 过程将会非常复杂. 于是作为简化近似, 我们计算最低阶的散射截面时可以不将电磁场量子化, 仅仅量子化Dirac场, 保留经典的电磁场矢势$A_\mu$, 这样子就可以直接唯像地将原子核处理为一个点源了, 从而有相互作用Hamiltonian
    \begin{equation}
        H_I=\int\d^3x e\bar\psi\gamma^\mu\psi A_\mu.
    \end{equation}
    
    我们首先尝试导出这一半量子化场论的Feynman规则. 我们计算两点关联函数
    \begin{align}
        \braket{\psi_1\bar\psi_2}&=\frac{\bra0\mathcal T\exp{-i\int\d^4x\bar\psi\gamma^\mu\psi A_\mu}\psi_1\bar\psi_2\ket0}{\braket{0|\mathcal T\exp{-i\int\d^4x\bar\psi\gamma^\mu\psi A_\mu}|0}}\\
        &=\int\dddd p\frac{i(\slashed p+m)}{p^2-m^2}\exp{ip(x_2-x_1)}\notag\\
        &\;-ie\int\d^4xA_\mu\int\dddd p\frac{i(\slashed p+m)}{p^2-m^2}\exp{-ip(x_1-x)}\gamma^\mu\int\dddd q\frac{i(\slashed q+m)}{q^2-m^2}\exp{-iq(x-x_2)}
    \end{align}
    利用旋量LSZ公式, 我们有
    \begin{align}
        \braket{p'|i\mathcal T|p}=\braket{f, +\infty|i, -\infty}&=-ie\int\d^4xA_\mu\exp{i(p'-p)x}\bar u_{s'}(p')\gamma^\mu u_s(p)\\
        &=-ie\bar u_{s'}(p')\gamma^\mu u_s(p)\tilde{A_\mu}(p'-p).
    \end{align}

    对于原子核来说, 取原子核静系, $A_\mu$是不含时的, 因此$A_\mu$的傅里叶变换$\tilde A_\mu(p)$含有一个delta函数, 因此我们设
    \begin{equation}
        \braket{p'|i\mathcal T|p}=i\mathcal M(2\pi)\delta(E_i-E_f).
    \end{equation}
    这样就有如下的Feynman规则来计算$\mathcal M$
    \begin{align}
        \begin{tikzpicture}[baseline=(current bounding box.center)]
            \begin{feynman}
                \vertex (a) at (0, 0);
                \coordinate (s) at (1, 0);
                \coordinate (f1) at (-1, -1);
                \coordinate (f2) at (-1, 1);
                \diagram* {
                    a -- [photon] s[crossed dot],
                    f1 -- [fermion] a -- [fermion] f2,
                };
            \end{feynman}
        \end{tikzpicture}&=-ie\gamma^\mu A_\mu
    \end{align}

    回到散射截面最原始的定义\ref{cross-section-def}
    \begin{equation}
        \frac Nv v_i T\d\sigma=N\d P\Rightarrow \d\sigma=\frac1{v_i}\frac VT\d P
    \end{equation}
    再根据
    \begin{equation}
        V\ddd{p_f}=1
    \end{equation}
    我们可以得到
    \begin{align}
        \d P&=\frac{|\braket{f|i}|^2}{\braket{f|f}\braket{i|i}}\\
        &=\frac{|\mathcal M|^2 T^2}{2E_f\cdot2E_iV^2}V\ddd{p_f}
    \end{align}
    将$\d P$代入$\d\sigma$, 我们就有散射截面的表达式
    \begin{align}
        \d\sigma&=\frac{1}{v_i}\frac VT\frac{|\mathcal M|^2 T^2}{2E_f\cdot2E_iV^2}V\ddd{p_f}\\
        &=\frac{1}{v_i}\frac{|\mathcal M|^2}{4E_iE_f}2\pi\delta(E_f-E_i)\ddd{p_f}
    \end{align}

    在原子核静止系中, 我们有
    \begin{equation}
        A^0=\frac{Ze}{4\pi r}
    \end{equation}
    做傅里叶变换有
    \begin{equation}
        \tilde A^0(\vec k)=-\frac{Ze^2}{k^2}.
    \end{equation}

    代入就有
    \begin{equation}
        \d\sigma=\frac1{4E_i^2v_i}\ddd{p_f}2\pi\delta(E_f-E_i)\frac12\sum\bar u(\vec p_f)\gamma^0u(\vec p_i)\bar u(\vec p_i)\gamma^0u(\vec p_f)\frac{Z^2e^4}{(\vec p_f-\vec p_i)^4}
    \end{equation}
    然后计算
    \begin{align}
        \sum\bar u(\vec p_f)\gamma^0u(\vec p_i)\bar u(\vec p_i)\gamma^0u(\vec p_f)&=\sum\rm{Tr}\left[\gamma^0(\slashed p_f+m)\gamma^0(\slashed p_i+m)\right]\\
        &=\rm{Tr}\left[m^2+m\slashed p_i+m\slashed p_f+\gamma^0\gamma^0\slashed p_f\slashed p_i\right]\\
        &=\rm{Tr}\left[m^2\right]+\rm{Tr}\left[\gamma^\mu\gamma^0\gamma^\nu\gamma^0\right]p_{i\mu}p_{f\nu}\\
        &=4m^2-4p_i\cdot p_f+8p_i^0p_f^0
    \end{align}
    从而有
    \begin{equation}
        \d\sigma=\frac1{4E_i^2v_i}\ddd{p_f}2\pi\delta(E_f-E_i)(2m^2-2p_i\cdot p_f+4p_i^0p_f^0)\frac{Z^2e^4}{(\vec p_f-\vec p_i)^4}.
    \end{equation}

    取非相对论极限
    \begin{equation}
    2m^2-2p_i\cdot p_f+4p_i^0p_f^0\approx 4m^2, E_i\approx m
    \end{equation}
    我们得到
    \begin{align}
        \frac{\d\sigma}{\d\Omega}&=\frac{1}{4m^2v_i}\frac{p_i^2}{\dpi2}\frac{m}{p_i}4m^2\frac{Z^2e^4}{p_i^4(1-\cos\theta)^2}\\
        &=\frac{Z^2e^4}{16\pi^2m^2v_i^4}\frac1{(1-\cos\theta)^2}\\
        &=\frac{Z^2\alpha^2}{4m^2v_i^4\sin^4(\theta/2)}
    \end{align}
    这个结果和经典计算完全一致, 即Rutherford公式.

    而考虑相对论的情况
    \begin{align}
        m^2-p_i\cdot p_f+2p_i^0p_f^0&=m^2+E_i^2+\vec p_i\cdot\vec p_f\\
        &=m^2(1+\gamma^2+\gamma^2\beta^2\cos\theta)\\
        &=2m^2\gamma^2(1-\beta^2\sin^2(\theta/2))
    \end{align}
    于是我们可以得到Motte公式
    \begin{align}
        \frac{\d\sigma}{\d\Omega}&=\frac1{4E_i^2v_i}\frac{p_i^2}{\dpi2}\frac{E_i}{p_i}2(m^2-p_i\cdot p_f+2p_i^0p_f^0)\frac{Z^2e^4}{4p_i^2(1-\cos\theta)^2}\\
        &=\frac1{4E_i^2v_i}\frac{p_i^2}{\dpi2}\frac{E_i}{p_i}2(2m^2\gamma^2(1-\beta^2\sin^2(\theta/2)))\frac{Z^2e^4}{4p_i^2(1-\cos\theta)^2}\\
        &=\frac{Z^2\alpha^2}{4|\vec p|^2\beta^2\sin^4(\theta/2)}(1-\beta^2\sin^2(\theta/2)).
    \end{align}
\end{example}

\newpage
\section{重整化}
在QFT中我们经常遇到传播子发散的情况, 为了解决这个问题, 我们引入重整化步骤, 将无穷大扫到地板下, 从而得到有限的可观测量预测.

\newpage
\appendix
\section{Clebsch-Gordan系数表}
\begin{figure}[!htbp]
    \centering
    \includegraphics[page=1,width=1.0\textwidth]{clebsch-gordan.pdf}
    \caption{Clebsch-Gordan系数表}\label{cgcoeff}
\end{figure}

\section{数学公式参考}
\subsection{$\Gamma$函数}
\begin{definition}[$\Gamma$函数]
    \begin{equation}
        \Gamma(z) = \int_0^\infty t^{z-1}e^{-t}dt
    \end{equation}
\end{definition}
通过分部积分我们可以证明
\begin{theorem}[$\Gamma$函数递推关系]
    \begin{equation}
        \Gamma(z+1) = z\Gamma(z)
    \end{equation}
\end{theorem}
从而有推论
\begin{theorem}
    \begin{equation}
        \Gamma(n+1)=n!, \forall n\in\mathbb N.
    \end{equation}
\end{theorem}

我们还可以计算得到
\begin{equation}
    \Gamma(\frac12)=\sqrt\pi.
\end{equation}

我们还有$\Gamma$函数的渐进形式
\begin{theorem}
    当$n\to+\infty$, 
    \begin{equation}
        \Gamma(n+1)\approx n^n\exp{-n}\sqrt{2\pi n}
    \end{equation}
\end{theorem}
\begin{proof}
    对于$n\to+\infty$, 
    \begin{align}
        \Gamma(n+1)=\int_0^{+\infty} x^n\exp{-x}\d x
    \end{align}
    我们使用Gauss积分\cite{SchroederGamma}估计这个形式, 得到
    \begin{equation}
        x^n\exp{-x}\approx nn^n\exp{-n}\exp{-y^2/2n}
    \end{equation}
    其中
    \begin{equation}
        y=x-n.
    \end{equation}
    于是可以得到最终结果
    \begin{equation}
        \Gamma(n+1)\approx n^n\exp{-n}\sqrt{2\pi n}
    \end{equation}
\end{proof}

\begin{definition}
    \begin{equation}
        \psi(z):=\de{}{z}\ln\Gamma(z)
    \end{equation}
\end{definition}
\begin{theorem}
    \begin{equation}
        \psi(z+1)=\frac1z+\psi(z)
    \end{equation}
\end{theorem}
\begin{definition}[Euler-Mascheroni常数]
    \begin{equation}
        \gamma=-\psi(1)\approx0.5772156649\cdots
    \end{equation}
\end{definition}

\begin{theorem}
    当$x\to0$, 
    \begin{equation}
        \Gamma(z)\approx \frac 1z-\gamma,
    \end{equation}
\end{theorem}
\begin{proof}
    \begin{align}
        \ln\Gamma(z+1)\approx\ln\Gamma(1)+z\psi(1)=z\psi(1)=-\gamma z
    \end{align}
    于是
    \begin{equation}
        \Gamma(z)=\frac{\Gamma(z+1)}{z}\approx\frac{\exp{-\gamma z}}{z}\approx\frac{1-\gamma z}{z}=\frac 1z-\gamma
    \end{equation}
\end{proof}

\subsection{B函数}
\begin{definition}
    \begin{equation}
        B(x,y)=\int_0^1 t^{x-1}(1-t)^{y-1}\d t
    \end{equation}
    其中$\rm{Re}(x), \rm{Re}(y)>0$
\end{definition}
\begin{theorem}
    \begin{equation}
        B(x,y)=B(y,x)
    \end{equation}
\end{theorem}
\begin{proof}
    做换元$t'=1-t$, 则有
    \begin{align}
        B(x,y)&=-\int_1^0 (1-t')^{x-1}t'^{y-1}\d t'\\
        &=\int_0^1(1-t')^{x-1}t'^{y-1}\d t'
    \end{align}
\end{proof}
\begin{theorem}
    \begin{equation}
        B(x,y)=\frac{\Gamma(x)\Gamma(y)}{\Gamma(x+y)}
    \end{equation}
\end{theorem}
\begin{proof}
    考虑
    \begin{align}
        \Gamma(x)\Gamma(y)&=\left(\int_0^{+\infty}u^{x-1}\exp{-u}\d u\right)\left(\int_0^{+\infty}u^{y-1}\exp{-v}\d v\right)\\
        &=\int_0^{+\infty}\int_0^{+\infty}u^{x-1}v^{y-1}\exp{-(u+v)}\d u\d v
    \end{align}
    做换元$u=st, v=s(1-t)$, 有
    \begin{align}
        \Gamma(x)\Gamma(y)&=\int_0^1\d t\int_0^{+\infty}s\d s\exp{-s}\left(st\right)^{x-1}\left(s(1-t)\right)\\
        &=\int_0^1\d t t^{x-1}(1-t)^{y-1}\int_0^{+\infty}\d ss^{x+y-1}\exp{-s}\\
        &=B(x,y)\Gamma(x+y)
    \end{align}
\end{proof}
\begin{theorem}\label{beta-equiv}
    \begin{equation}
        B(x,y)=\int_0^{+\infty}\frac{t^{x-1}}{(1+t)^{x+y}}\d t
    \end{equation}
\end{theorem}
\begin{proof}
    对原始的积分定义换元, 设$t=\frac{t'}{t'+1}$, 于是有
    \begin{align}
        B(x,y)&=\int_0^1 t^{x-1}(1-t)^{y-1}\d t=\int_0^{+\infty} \left(\frac{t'}{t'+1}\right)^{x-1}{\frac1{1+t'}}^{y-1}\frac{\d t'}{(1+t')^2}\\
        &=\int_0^{+\infty}\frac{t'^{x-1}}{(1+t')^{x+y}}\d t'
    \end{align}
\end{proof}

\begin{theorem}
    \begin{equation}
        B(x,y)=2\int_0^{\frac\pi2}\sin^{2x-1}\theta\cos^{2y-1}\theta\d\theta
    \end{equation}
\end{theorem}
\begin{proof}
    对于原始定义, 设$t=\sin^2\theta$, 则
    \begin{align}
        B(x,y)&=\int_0^1 t^{x-1}(1-t)^{y-1}\d t\\
        &=\int_0^{\frac\pi2}\sin^{2(x-1)}\theta\cos^{2(y-1)}\theta\cdot2\sin\theta\cos\theta\d\theta\\
        &=2\int_0^{\frac\pi2}\sin^{2x-1}\theta\cos^{2y-1}\theta\d\theta
    \end{align}
\end{proof}
代入$x=\frac{n+1}2, y=\frac12$我们有推论
\begin{theorem}\label{sin-n-integral}
    \begin{equation}
        \int_0^{\pi}\sin^n\theta\d\theta=\frac{\sqrt\pi\Gamma(\frac n2+\frac 12)}{\Gamma(\frac n2+1)}
    \end{equation}
\end{theorem}

\begin{theorem}
    \begin{equation}
        B(1,x)=\frac 1x
    \end{equation}
\end{theorem}
\begin{theorem}
    \begin{equation}
        B(x,1-x)=\frac{\pi}{\sin(n\pi)}
    \end{equation}
\end{theorem}
\begin{proof}
    根据定理\ref{beta-equiv}, 我们使用等价形式
    \begin{equation}
        B(x,1-x)=\int_0^{+\infty}\frac{t^{x-1}}{1+t}\d t.
    \end{equation}
    考虑围道如图\ref{B-x-1-x-contour}(其中正实轴为$\ln z$的割线), 我们需要计算的积分即$\Gamma_{up}$部分, 即
    \begin{figure}
        \centering
        \begin{tikzpicture}[scale=1.5,
            >=Stealth,
            thick,
            decoration={markings,
                mark=at position 0.5 with {\arrow{>}}},
            % 定义波浪线风格
            snake_cut/.style={decorate, decoration={snake, amplitude=0.5mm, segment length=2mm, post length=1mm}}
        ]

            % 参数:R(大半径,现在用得更大), \epsilon(小半径)
            \def\R{3.0}
            \def\epsilon{0.1}

            % 1. 绘制坐标轴
            \draw[->] (-2.7, 0) -- (\R+0.5, 0) node[right] {Re($z$)};
            \draw[->] (0, -2.5) -- (0, 2.5) node[above] {Im($z$)};
            \node at (0, 0) [circle, fill, inner sep=1.5pt] {}; % 原点 0

            % 2. 绘制极点 (红点)
            \node at (-1, 0) [circle, fill=red, inner sep=2pt, label={[red]above: $z=-1$}] {}; % 极点 -1

            % 3. 绘制分支切割(割线,波浪线)
            % 我们用波浪线代替实轴,从 epsilon 之外一直延伸
            \draw[snake_cut, blue!70] (0, 0) -- (\R, 0);

            % --- 绘制锁孔路径 ---

            % 4. 绘制路径 Gamma_up (上侧) - 从 R 到 epsilon (积分方向向左)
            \draw[postaction={decorate}] (0, 0.1) -- (\R, 0.1);
            \node[above] at (2, 0.1) {$\Gamma_{up}$};

            % 5. 绘制路径 gamma_epsilon (小圆弧) - 绕过原点 (逆时针)
            \draw[] (0, 0.1) arc (90:270:0.1);
            % \node[right] at (\epsilon, 0) {$\gamma_\epsilon$};

            % 6. 绘制路径 Gamma_down (下侧) - 从 epsilon 到 R (积分方向向右)
            \draw[postaction={decorate}] (\R, -0.1) -- (0, -0.1);
            \node[below] at (2, -0.1) {$\Gamma_{down}$};
            % 调整下侧箭头方向,确保是从 R 到 epsilon 的逆向
            % 由于 postaction 是沿着路径方向画箭头,如果路径是从 \epsilon 到 \R,则需要注意。
            % 修正:标准 Keyhole Contour 是逆时针的,所以下侧是从 R 到 epsilon,但方向是左。
            % 这里的 \draw 实际上是 \epsilon -> \R,我们标记方向是逆时针的。
            % 为了符合闭合路径的逆时针方向,\Gamma_{down} 应该从 R 到 \epsilon,方向向左。

            % 7. 绘制路径 Gamma_R (大圆弧) - 封闭路径 (逆时针)
            % 确保大圆弧的尺寸与 \R 匹配
            \draw[postaction={decorate}] (\R, 0.1) arc (1:357: \R); % 稍微夸张地画一个大弧
            \node[above right] at (2.4, 1.8) {$\Gamma_{C}$};
        \end{tikzpicture}
        \caption{$B(x,1-x)$围道示意图}\label{B-x-1-x-contour}
    \end{figure}
    \begin{equation}
        B(x,1-x)=\int_{\Gamma_{up}}\frac{z^{x-1}}{z+1}\d z=\int_{\Gamma_{up}}\frac{\exp{(x-1)\ln z}}{z+1}\d z
    \end{equation}
    不难发现, $\Gamma_{down}$部分的$\ln z$比$\Gamma_{up}$多$2\pi i$的相位, 于是有
    \begin{equation}
        \int_{\Gamma_{down}}\frac{\exp{(x-1)\ln z}}{z+1}\d z=-\int_{\Gamma_{up}}\frac{\exp{(x-1)(\ln z+2\pi i)}}{z+1}\d z=-\exp{2\pi xi}B(x,1-x)
    \end{equation}
    再利用大圆弧定理, 
    \begin{equation}
        \int_{\Gamma_{C}}\frac{\exp{(x-1)\ln z}}{z+1}\d z=0
    \end{equation}
    于是有
    \begin{equation}
        \left(\int_{\Gamma_{up}}+\int_{\Gamma_{down}}+\int_{\Gamma_{C}}\right)\frac{\exp{(x-1)\ln z}}{z+1}\d z=(1-\exp{2\pi xi})B(x,1-x)
    \end{equation}
    利用留数定理, 左边的积分正是
    \begin{equation}
        2\pi i\rm{Res}(\frac{\exp{(x-1)\ln z}}{z+1})=2\pi i\exp{(x-1)\pi i}=-2\pi i\exp{\pi xi}
    \end{equation}

    整理有
    \begin{equation}
        B(x,1-x)=\frac{\pi}{\sin\pi x}
    \end{equation}
\end{proof}

\subsection{$n$维超球}
\begin{definition}[n维超球]
    在$n$维欧式空间中, 半径维$r$的$n$维超球即
    \begin{equation}
        \left\{(x_1,x_2, \cdots x_n)|\sum_{i=1}^n x_i^2\leq r^2\right\}
    \end{equation}
\end{definition}
\begin{theorem}[$n$维超球面角]
    \begin{equation}
        \Omega_n=\frac{2\pi^{n/2}}{\Gamma(n/2)}
    \end{equation}
\end{theorem}
\begin{proof}
    我们利用递推法, 有
    \begin{align}
        \Omega_n=\int_0^\pi\Omega_{n-1}\sin^{n-2}\theta\d\theta
    \end{align}
    根据定理\ref{sin-n-integral}, 我们有
    \begin{equation}
        \int_0^\pi\sin^{n-2}\theta\d\theta=\frac{\sqrt\pi\Gamma(\frac n2-\frac 12)}{\Gamma(\frac n2)}
    \end{equation}
    于是有
    \begin{align}
        \Omega_n=\frac{\sqrt\pi\Gamma(\frac n2-\frac 12)}{\Gamma(\frac n2)}\Omega_{n-1}
    \end{align}
    利用
    \begin{equation}
        \Omega_1=2
    \end{equation}
    所以有
    \begin{align}
        \Omega_n&=\frac{\sqrt\pi\Gamma(\frac{n-1}2)}{\Gamma(\frac n2)}\frac{\sqrt\pi\Gamma(\frac{n-2}2)}{\Gamma(\frac {n-1}2)}\cdots\frac{\sqrt\pi\Gamma(\frac{n-(n-1)}2)}{\Gamma(\frac {n+1-(n-1)}2)}\cdot2\\
        &=\frac{2\pi^{n/2}}{\Gamma(n/2)}
    \end{align}
\end{proof}

\begin{theorem}[n维超球体积]
    半径维$r$的$n$维超球体积
    \begin{equation}
        V=\frac{\Omega_n}{n}r^n=\frac{2\pi^{n/2}}{n\Gamma(n/2)}r^n
    \end{equation}
\end{theorem}
\begin{proof}
    \begin{align}
        V&=\int_0^r\Omega_n r^{n-1}\d r=\frac{\Omega_n}{n}r^n
    \end{align}
\end{proof}

\newpage
\nocite{*}
\printbibliography[heading=bibintoc, title=\ebibname]

% 致谢
\newpage
\section*{致谢}
\addcontentsline{toc}{section}{致谢}
非常无比感谢我最喜欢的钱禺琛劳师无微不至的答疑解惑以及阅读评价, 平均每天帮我解决两个问题! 感谢张宇泰学长与黄海洋学长的答疑解惑和纠错、徐浩翔同学的讨论以及谭骐忠同学的纠错.

\end{document}