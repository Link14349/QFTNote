\section{矢量场}
\subsection{有质量矢量场(Massive)}
我们有拉格朗日量
\begin{equation}
    \mathcal L=-\frac14 F_{\mu\nu}F^{\mu\nu}+\frac12 m^2 A_\mu A^\mu
\end{equation}
其中$A^\mu=(\phi, \vec A)\Rightarrow A_\mu=(\phi, -\vec A)$

于是我们得到EoM:
\begin{equation}
    \partial_\mu F^{\mu\nu}+m^2 A^\nu=0
\end{equation}

将$\partial_\nu$作用到EoM, 并且因为$F^{\mu\nu}$反称, $\partial_\mu\partial_\nu F^{\mu\nu}=0$, 我们得到:
\begin{equation}
    m^2\partial_\nu A^\nu=0
\end{equation}

于是我们可以得到Proca方程
\begin{equation}
    (\partial^2+m^2)A^\nu=0
\end{equation}

正则共轭
\begin{equation}
    \Pi^{\mu\nu}=\pa{\mathcal L}{(\partial_\mu A_\nu)}=-F^{\mu\nu}=\begin{bmatrix}
        0 & E^1 & E^2 & E^3\\
        -E^1 & 0 & B^3 & -B^2\\
        -E^2 & -B^3 & 0 & B^1\\
        -E^3 & B^2 & -B^1 & 0
    \end{bmatrix}
\end{equation}

即$\Pi^{00}=0$, $\Pi^{0i}=E^i$

然后我们有
\begin{equation}
    \mathcal L=-\frac12(\vec B^2-\vec E^2)+\frac12m^2A_{\mu}A^{\mu}
\end{equation}

做Legendre变换, 得到哈密顿密度
\begin{align}
    \mathcal H&=\Pi^0_{~~i}\dot A^i-\mathcal L=-\Pi^{0i}\partial_t A^i-\mathcal L\\
    &=-\vec E\cdot\partial_t\vec A+\frac12 \vec  B^2-\frac12 \vec  E^2-\frac12m^2A_{\mu}A^{\mu}\\
    &=-\vec E\cdot\partial_t\vec A+\frac12 \vec B^2-\frac12 \vec  E^2-\frac12m^2\varphi^2+\frac12m^2\vec A^2
\end{align}

根据
\begin{equation}
    \vec E\cdot\nabla\varphi=\nabla\cdot(\varphi\vec E)-\varphi\nabla\cdot\vec E
\end{equation}
以及EoM
\begin{equation}
    \nabla\cdot\vec E=-m^2\varphi
\end{equation}
我们最终得到
\begin{equation}
    \mathcal H=\frac12\vec B^2+\frac12\vec E^2+\frac1{2m^2}(\nabla\cdot\vec E)^2+\frac12m^2\vec A^2
\end{equation}

然后我们尝试量子化, 利用正则量子化关系
\begin{equation}
    [A^i_{\vec x}, \Pi^{0j}_{\vec y}]=[A^i_{\vec x}, E^j_{\vec y}]=i\delta^{ij}\delta^3(\vec x-\vec y)=-ig^{ij}\delta^3(\vec x-\vec y)
\end{equation}

我们做Fourier变换, 得到
\begin{align}
    &A^\mu=\int\ldsq{p}\sum_{\lambda=1}^3\left(\epsilon_\lambda^\mu a_{\lambda\vec p}\exp{-ipx}+\epsilon_\lambda^{\mu*}a^\dagger_{\lambda\vec p}\exp{ipx}\right)\\
    &E^\mu=\int\ldsq{p}\sum_{\lambda=1}^3\left((p^\mu\epsilon_\lambda^{0*}-p^0\epsilon_\lambda^{\mu*})a^\dagger_{\lambda\vec p}\exp{ipx}-(p^\mu\epsilon^0_\lambda-p^0\epsilon_\lambda^\mu)a_{\lambda\vec p}\exp{-ipx}\right)
\end{align}

于是有对易子:
\begin{equation}
    [a_{\lambda\vec p}, a^\dagger_{\lambda'\vec p'}]=\dpi3\delta^3(\vec p-\vec p')\delta_{\lambda\lambda'}
\end{equation}

此外, 我们还可以得到
\begin{align}
    &\nabla\cdot\vec E=-m^2\int\ldsq p \sum_\lambda\left(\epsilon_\lambda^0a^\dagger_{\lambda\vec p}\exp{ipx}+\epsilon_\lambda^0a_{\lambda\vec p}\exp{-ipx}\right)\\
    &\vec B=\nabla\times\vec A=\int\ldsq p i\sum_\lambda\left((\vec p\times\vec\epsilon_\lambda)a_{\lambda\vec p}^\dagger\exp{ipx}-(\vec p\times\vec\epsilon_\lambda)a_{\lambda\vec p}\exp{-ipx}\right)
\end{align}

于是经过艰苦卓绝的爆算, 我们得到
\begin{align}
    H&=\int\d^3x\frac12(E^2+B^2+m^2A^2+\frac1{m^2}(\nabla\cdot\vec E)^2)\\
    &=\frac12\int\ld p\sum_\lambda\sum_{\lambda'}(\cdots)\\
    &=\frac12\int\ld p\sum_\lambda\sum_{\lambda'}(-m^2\epsilon_\lambda^0\epsilon_{\lambda'}^0+\omega^2\delta_{\lambda\lambda'}-m^2\epsilon^0_{\lambda}\epsilon^0_{\lambda'}\notag\\
    &\quad\quad+\vec p^2\delta_{\lambda\lambda'}+2m^2\epsilon_{\lambda'}^0\epsilon_\lambda^0+m^2\delta_{\lambda\lambda'})(a^\dagger_{\lambda\vec p}a_{\lambda'\vec p}+a_{\lambda'\vec p}a^\dagger_{\lambda\vec p})\\
    &=\int\ddd p\om p\sum_\lambda\left(a^\dagger_{\lambda\vec p}a_{\lambda\vec p}+\frac12\mathcal V\right)
\end{align}

% (感兴趣的可以见图\ref{fig:massiveEDQ}的手动具体计算过程)

% \begin{figure}[htbp!]
%     \centering
%     \includegraphics[width=0.8\textwidth]{image/massiveEM1.jpg}
%     \includegraphics[width=0.8\textwidth]{image/massiveEM2.jpg}
%     \caption{手算过程}
%     \label{fig:massiveEDQ}
% \end{figure}

然后我们计算它的传播子, 根据
\begin{align}
    \braket{0|A^\mu(x)A^\nu(y)|0}&=\int\ld p\exp{-ip(x-y)}\left(\sum_\lambda^3\epsilon^\mu_\lambda\epsilon^\nu_\lambda\right)\\
    &=\int\ld p\exp{-ip(x-y)}\left(-g^{\mu\nu}+p^\mu p^\nu/p^2\right)
\end{align}
注意到, 此时$p$还是on shell的, 因此$p^2=m^2$, 从而有
\begin{equation}
    \braket{0|A^\mu(x)A^\nu(y)|0}=\int\ld p\exp{-ip(x-y)}\left(-g^{\mu\nu}+p^\mu p^\nu/m^2\right)
\end{equation}
从而不难计算得到
\begin{equation}
    \braket{0|\mathcal TA_\mu(x)A_\nu(y)|0}=\int\dddd p\exp{-ip(x-y)}\frac{-i(g_\mu\nu-p_\mu p_\nu/m^2)}{p^2-m^2+i\epsilon}.
\end{equation}
因此传播子为:
\begin{equation}
    \frac{-i(g_{\mu\nu}-p_\mu p_\nu/m^2)}{p^2-m^2+i\epsilon}
\end{equation}

\subsection{无质量的规范矢量场: 电磁场(Massless)}
为了得到电磁场的二次量子化结果, 自然的想法就是对电磁场取$m\rightarrow0$的结果. 然而这会面临一个问题: 电磁场的$A^\mu$具有规范不变性, 即$A^\mu\rightarrow A^\mu+\partial^\mu\Lambda$不改变其物理意义, 进过规范变换$\partial_\mu A^\mu$也不一定为0. 这使得我们对电磁场的描述存在冗余自由度, 这的一个直接结果就是电磁场只有两个极化方向而不是重电磁场的三个. 

\subsubsection{Lorentz规范}
于是, 为了能够正确得处理自由度, 消除冗余, 我们引入Lorenz规范:
\begin{equation}
    \nabla_\mu A^\mu=0
\end{equation}

但是故事并没有结束, 我们仍然可以通过满足$\partial^2\Lambda=0$的$\Lambda$来进行规范, 因此我们可以进一步地取$\partial_0\Lambda=-A_0=-\varphi$, 从而使得$\varphi=0$.

这样Lorenz规范就退化成了Coulomb规范:
\begin{equation}
    \nabla\cdot\vec A=0
\end{equation}

引入正则量子化条件:
\begin{equation}
    [a_{\vec pr}, a^\dagger_{\vec qs}]=\dpi3\delta_{rs}\delta^3(\vec p-\vec q)
\end{equation}

于是我们就可以二次量子化$\vec A$了
\begin{equation}
    \vec A=\int\ldsq p\sum_{r=1}^2\left(\vec\epsilon_ra_{\vec pr}\exp{-ipx}+\vec\epsilon_r^*a_{\vec pr}^\dagger\exp{ipx}\right)
\end{equation}

从而有
\begin{equation}
    \vec E=-i\int\ddd p\sqrt{\frac{\om p}2}\sum_{r=1}^2\left(\vec\epsilon_ra_{\vec pr}\exp{-ipx}-\vec\epsilon_r^*a_{\vec pr}^\dagger\exp{ipx}\right)
\end{equation}

然后有对易子
\begin{align}
    [A^i_{\vec x}, E^j_{\vec y}]&=i\int\ddd p\exp{i\vec p\cdot(\vec x-\vec y)}\sum_r\epsilon_r^i(\vec p)\epsilon^j_r(\vec p)\\
    &=i\int\ddd p\exp{i\vec p\cdot(\vec x-\vec y)}(\delta^{ij}-\frac{p^ip^j}{\vec p^2})\\
    &=i\delta^3_{\bf{tr}}(\vec x-\vec y)
\end{align}

\kaishu 这里我们可能会有疑问, 为什么这不能按照我们一般的正则量子化的方法, 让
\begin{equation}
    [A^i_{\vec x}, E^j_{\vec y}]=i\delta^{ij}\delta^3(\vec x-\vec y)
\end{equation}

这是因为根据我们的Coulomb规范, $\nabla\cdot\vec A=0$, 因此
\begin{equation}
    [\partial_i A^i, E^j]=0
\end{equation}

然而代入上面的对易关系, 我们会发现
\begin{equation}
    [\partial_iA^i_{\vec x}, E^j_{\vec y}]=i\partial^j\delta^3(\vec x-\vec y)\neq0
\end{equation}

这个正则量子化条件是不自洽的! 这里的原因还是因为无质量的电磁场存在规范冗余, 导致我们丢失了一个"物理的"极化方向.

而由$a, a^\dagger$写出的正则量子化条件, 我们可以验证它可以保证
\begin{align}
    [\partial_i A^i, E^j]&=i\ddd p\exp{i\vec p\cdot(\vec x-\vec y)}ip^i(\delta^{ij}-\frac{p^ip^j}{\vec p^2})\\
    &=-\ddd p\exp{i\vec p\cdot(\vec x-\vec y)}(p^j-\frac{\vec p^2 p^j}{\vec p^2})\\
    &=0
\end{align}
\songti

接着下一个任务就是给出光子的传播子了. 如果不做规范选取, 我们从EoM开始:
\begin{equation}
    (g_{\mu\nu}\partial^2-\partial_\mu\partial_\nu)A^\nu=J_\mu
\end{equation}

换到傅里叶空间, 我们有
\begin{equation}
    (-p^2g_{\mu\nu}+p_\mu p_\nu)\tilde A^\nu=\tilde J_\mu
\end{equation}

似乎只要求出$-p^2g_{\mu\nu}+p_\mu p_\nu$的逆矩阵就好了...?

然而不难发现, $-p^2g_{\mu\nu}+p_\mu p_\nu$是奇异的: $g_\mu\nu-\frac{p_\mu p_\nu}{p^2}$就是在度规张量在类光面上的诱导度规, 它的秩仅有2, 根本不可能找到逆.

这个原因在于电磁场具有规范冗余, 而规范冗余会导致这个矩阵奇异. 因此我们需要做规范的选取. 最直接的想法就是我们在Lorentz规范下计算传播子. 那么我们首先需要计算
\begin{equation}
    \braket{0|A_i(x)A_j(y)|0}=\int\ld p\sum_s\epsilon_{si}(\vec p)\epsilon_{sj}(\vec p)\exp{-ip(x-y)}, 
\end{equation}
根据$\epsilon_i p^i=0$, 我们有
\begin{equation}
    \sum_s\epsilon_{si}(\vec p)\epsilon_{sj}(\vec p)=\delta_{ij}+\frac{p_ip_j}{\vec p^2}=-g_{ij}++\frac{p_ip_j}{\vec p^2}, 
\end{equation}
于是
\begin{equation}
    \braket{0|A_i(x)A_j(y)|0}=\int\ld p\left(g_{ij}+\frac{p_ip_j}{\vec p^2}\right)\exp{-ip(x-y)}.
\end{equation}
所以
\begin{equation}
    \Theta(t_x-t_y)\braket{0|A_i(x)A_j(y)|0}=\int i\dddd p\frac{\exp{-ip(x-y)}}{2|\vec p|(\omega-|\vec p|+i\epsilon)}\left(-g_{ij}+\frac{p_ip_j}{|\vec p|^2}\right)
\end{equation}
于是有Feynman传播子(但是并非下小节的Feynman规范的传播子)
\begin{align}
    \braket{0|\mathcal TA_i(x)A_j(y)|0}&=\int\dddd p\frac{-i(g_{ij}-p_i p_j/|\vec p|^2)}{p^2+i\epsilon}\exp{-ip(x-y)}
\end{align}
, 并且还有
\begin{align}
    \braket{0|\mathcal TA_0(x)A_\mu(y)|0}=\braket{0|\mathcal TA_\mu(x)A_0(y)|0}=0.
\end{align}

这个传播子实在有点丑, 而且很不方便. 这一问题来源于Lorentz规范下$A^0$并没有动力学, 这导致我们的传播子没有$00, 0i, i0$元素. 同时电磁场是没有纵模的, 只有横模的两个极化方向, 这导致传播子中需要将纵模剔除, 这导致$g_{ij}+p_ip_j/|\vec p|^2$中第二项的出现. 尽管理论上来说我们完全可以在Lorentz规范下进行计算, 但一般情况下我们并不希望使用这么复杂的传播子, 因此我们试图寻找一种新的规范固定方式, 使得其传播子拥有一个简洁的形式.

\subsubsection{$\xi$规范}
我们希望我们的传播子能够更加漂亮且方便, 我们引入一个所谓的$\xi$规范来进行规范固定(注意, 规范固定Gauge Fixing和规范条件Gauge Condition并不是一个概念, 我们上面的Lorentz规范是一个规范条件, 它用一个额外的方程强行约束了$A$, 但是接下来我们会看到, 规范固定采取一种不同的做法). 我们的Motivation是引入一个参数$\xi$来使得它有逆, 即:
\begin{equation}
    (-p^2g_{\mu\nu}+p_\mu p_\nu)\to(-p^2g_{\mu\nu}+(1-\frac1\xi)p_\mu p_\nu)
\end{equation}

这样我们就能对$(-p^2g_{\mu\nu}+(1-\frac1\xi)p_\mu p_\nu)$求逆了:
\begin{equation}
    -\frac{g^{\mu\lambda}+(\xi-1)p^\mu p^\lambda/p^2}{p^2}(-p^2g_{\lambda\nu}+(1-\frac1\xi)p_\lambda p_\nu)
\end{equation}

也就是说我们想要EoM变为:
\begin{equation}
    (g_{\mu\nu}\partial^2-(1-\frac1\xi)\partial_\mu\partial_\nu)A^\nu=J_\mu
\end{equation}

为了得到这样的EoM, 我们可以在拉氏量中加入一个规范项:
\begin{equation}
    \mathcal L=-\frac14F^2-J_\mu A^\mu-\frac1{2\xi}(\partial_\mu A^\mu)^2, 
\end{equation}
从而得到我们想要的EoM. 并且注意到, 做规范变换$A'^\mu=A^\mu+\partial^\mu\Lambda$, 其中
\begin{equation}
    \partial^2\partial_\nu\Lambda=\frac{\xi'-\xi}{\xi}\partial_\nu\partial_\mu A^\mu
\end{equation}
或者写成
\begin{equation}
    \partial^2\partial_\nu\Lambda=\frac{\xi'-\xi}{\xi'}\partial_\nu\partial_\mu A'^\mu
\end{equation}
$A'^\mu$的EoM就变成:
\begin{equation}
    (g_{\mu\nu}\partial^2+(\frac1{\xi'}-1)\partial_\mu\partial_\nu)A'^\mu=J_\nu
\end{equation}
这说明$\xi$的不同取值其实就对应不同的规范.

\kaishu
这里我们会质疑, 规范变换不应当改变EoM, 为什么$\xi$规范会改变EoM? 我们可以证明, $\xi$规范带来的额外非物理贡献在涉及计算具体物理量, 比如$S$矩阵时是可以严格消掉的. (具体证明待续)

进一步的理解就是, 这里的规范固定后的场已经不是我们经典意义下的电磁场了: 显而易见, 它的动力学和经典电磁场完全不同, 它并不是Maxwell方程所描述的电磁场. 我们回到经典情况下考虑, 这个$\xi$"规范"下电磁场的EoM为
\begin{align}
    \nabla\cdot\vec E=\rho-\frac1\xi\partial_t\left(\partial_t A^0+\nabla\cdot\vec A\right)\\
    \nabla\times\vec B-\partial_t\vec E=\vec J+\frac1\xi\nabla\left(\partial_t A^0+\nabla\cdot\vec A\right)
\end{align}
无源的两个Maxwell仍然满足:
\begin{align}
    \nabla\times\vec E+\partial_t\vec B=0\\
    \nabla\cdot\vec B=0
\end{align}
粒子的受力仍然为
\begin{equation}
    \vec F=q(\vec E+\vec v\times\vec B).
\end{equation}

有源的两个方程组的修改导致$\vec E, \vec B$显然完全不是经典电磁场的值. 所以说经过这么一个规范固定过程, 我们相当于找到了一个并不满足经典电磁场运动学, 但是能够保持$S$矩阵不变的"辅助场", 在某个确定$\xi$的Lagrangian下, 它就是一个不具有规范不变性的场: 没有Lorentz规范之类的规范条件强行约束. 它不具有规范冗余, 单纯的一个EoM
\begin{equation}
    (g_{\mu\nu}\partial^2-(1-\frac1\xi)\partial_\mu\partial_\nu)A^\nu=0
\end{equation}
就足以描述它的全部动力学性质.
\songti

% 我们将EoM改写为
% \begin{equation}
%     \partial_\mu F^{\mu\nu}+\frac1\xi\partial^\nu\partial_\mu A^\mu=0,
% \end{equation}
% 利用$F^{\mu\nu}$的反称性, 对等式施加$\partial_\nu$, 我们有
% \begin{equation}
%     (\partial^2)\partial_\mu A^\mu=0.
% \end{equation}
% 在动量空间可以写为
% \begin{equation}
%     p^2 p\cdot A=0, 
% \end{equation}

% 由此可见, $\xi$的不同取值其实就对应不同的规范, 在不同规范下传播子是不同的:
% % 并且Lorenz规范就是$\xi=0$: 这时$\frac1\xi$变为无穷大, 为了满足EoM, 我们必须要求$\partial_\mu A^\mu=0$
我们直接写出Green函数
\begin{equation}
    G_{\mu\nu}(x-y)=\int\dddd p\left(-\frac{g_{\mu\nu}+(\xi-1)p_\mu p_\nu/p^2}{p^2}\right)\exp{-ip(x-y)}
\end{equation}
从而得到光子传播子:
\begin{equation}
    \frac{-i(g_{\mu\nu}+(\xi-1)p_\mu p_\nu/p^2)}{p^2+i\epsilon}
\end{equation}
取$\xi=1$, 就是Feynman规范, 得到其传播子
\begin{equation}
    \frac{-ig_{\mu\nu}}{p^2+i\epsilon}
\end{equation}

\newpage
\section{标量QED}
\subsection{规范变换}
我们知道, 复标量场$\psi$有$U(1)$对称性, 这是一个全局变换. 但是在一个局域的$U(1)$变换$\psi\to\psi\exp{i\alpha}$, 即我们新定义的一个规范变换下并不具有不变性. 但我们希望能够通过某些构造使其有这一不变性.

于是我们让$A_\mu$作为联络, 新定义对$\psi$的微分算符:
\begin{equation}
    D_\mu=\partial_\mu+ieA_\mu
\end{equation}

并且在$U(1)$规范变换下, $A_\mu, \psi$以如下方式变换:
\begin{align}
    &A'_\mu=A_\mu-\frac1e\partial_\mu\alpha\\
    &\psi'=\psi\exp{i\alpha}
\end{align}

不难发现
\begin{equation}
    D'_\mu\psi'=(\partial_\mu+ieA_\mu-i\partial_\mu\alpha)(\psi\exp{i\alpha})=\exp{i\alpha}(\partial_\mu+ieA_\mu)\psi=\exp{i\alpha}D_\mu\psi
\end{equation}

所以如果构造一个电磁场与标量场$\phi$耦合的拉氏密度
\begin{equation}
    \mathcal L=-\frac14 F^2+D_\mu\psi(D^\mu\psi)^*-m^2\psi\psi^*
\end{equation}
,则其有$U(1)$规范不变性.

我们可以给他写为更显式的形式
\begin{equation}
    \mathcal L=-\frac14F^2+\partial_\mu\psi\partial^\mu\psi^*-m^2\psi\psi^*+ieA_\mu(\psi\partial^\mu\psi^*-\psi^*\partial^\mu\psi-ie\psi\psi^*A^\mu)
\end{equation}

我们可以写出EoM
\begin{align}
    \begin{cases}
        &(\partial^2+m^2)\psi=-2ieA_\mu\partial^\mu\psi+e^2\psi A_\mu A^\mu\\
        &\partial_\mu F^{\mu\nu}=ie(\psi\partial^\nu\psi^*-\psi^*\partial^\nu\psi-2ie\psi\psi^*A^\nu)
    \end{cases}
\end{align}

于是我们可以得到电流:
\begin{equation}
    J^\nu=ie(\psi\partial^\nu\psi^*-\psi^*\partial^\nu\psi-2ie\psi\psi^*A^\nu)
\end{equation}

并且, 如果我们再次做全局$U(1)$变换, 我们可以再次得到这一守恒流:
\begin{align}
    J^\nu&=\pa{\mathcal L}{\partial_\mu\psi}\frac{\delta\psi}{\delta\alpha}+\pa{\mathcal L}{\partial_\mu\psi^*}\frac{\delta\psi^*}{\delta\alpha}\\
    &=i(\psi\partial^\nu\psi^*-\psi^*\partial^\nu\psi-2e\psi\psi^*A^\nu)
\end{align}

于是, 就从对规范不变性的追求中, 我们得到了标量QED. 

\subsection{标量QED的Feynman规则}
我们不难写出Dyson级数
\begin{align}
    \braket{0|\mathcal T\exp{-i\int\d^4x H_I'}|0}&=\braket{0|\mathcal T\exp{-e\int\d^4xA_\mu\left(\psi\partial^\mu\psi^\dagger-\psi^\dagger\partial^\mu\psi-ie\psi^\dagger\psi A^\mu\right)}|0}\\
    &=1-e\int\d^4xA_\mu\left(\psi\partial^\mu\psi^\dagger-\psi^\dagger\partial^\mu\psi-ie\psi^\dagger\psi A^\mu\right)+\cdots
\end{align}

这里主要的难点在于这个$\partial_\mu\psi, \partial_\mu\psi^\dagger$怎么处理. 我们可以首先将$\partial_\mu$提出缩并外, 考虑到
\begin{equation}
    \braket{0|\mathcal T\psi_x\psi_y^\dagger|0}=\int\dddd p\frac{i}{p^2-m^2+i\epsilon}\exp{-ip(x-y)}
\end{equation}
所以说$\partial_\mu\psi$就相当于一个$-ip_\mu\psi$, $\partial_\mu\psi^\dagger$就相当于$ip_\mu\psi^\dagger$, 我们可以将$p_\mu$给放在相互作用顶点中, 于是我们可以总结Feynman规则
\begin{enumerate}
    \item 复标量传播子
    $$
        \begin{tikzpicture}[baseline=(current bounding box.center)]
            \begin{feynman}
                \vertex (a);
                \vertex [right=2cm of a] (b);
                \diagram* {
                    % dashed, arrow=... 添加了方向箭头
                    (a) -- [scalar arrow, momentum'=\(p\)] (b),
                };
            \end{feynman}
        \end{tikzpicture}
        = \frac{i}{p^2 - m^2 + i\epsilon}
    $$
    \item 光子传播子
    $$
        \begin{tikzpicture}[baseline=(current bounding box.center)]
            \begin{feynman}
                \vertex (a) {\(\mu\)};
                \vertex [right=2cm of a] (b) {\(\nu\)};
                \diagram* {
                    (a) -- [photon, momentum'=\(p\)] (b),
                };
            \end{feynman}
        \end{tikzpicture}
        = \frac{-i g_{\mu\nu}}{p^2 + i\epsilon}
    $$
    \item 标量-光子三点顶角
    $$
        \begin{tikzpicture}[baseline=(current bounding box.center)]
            \begin{feynman}
                \vertex (a);
                \vertex [above right=1.5cm of a] (b); % 出射标量 p'
                \vertex [below right=1.5cm of a] (c); % 入射标量 p
                \vertex [left=1.5cm of a] (d) {\(\mu\)}; % 光子
                \diagram* {
                    % 注意箭头的方向:从 c 到 a,从 a 到 b
                    (c) -- [scalar arrow, momentum=\(p\)] (a),
                    (a) -- [scalar arrow, momentum'=\(p'\)] (b),
                    (a) -- [photon] (d),
                };
            \end{feynman}
        \end{tikzpicture}
        = -ie(p+p')^\mu
    $$
    \item 标量-双光子四点顶角
    $$
        \begin{tikzpicture}[baseline=(current bounding box.center)]
            \begin{feynman}
                \vertex (a);
                \vertex [above left=1.5cm of a] (b) {\(\mu\)};
                \vertex [below left=1.5cm of a] (c) {\(\nu\)};
                \vertex [above right=1.5cm of a] (d);
                \vertex [below right=1.5cm of a] (e);
                \diagram* {
                    (b) -- [photon] (a),
                    (c) -- [photon] (a),
                    % 标量线箭头流向一致 (例如,从 e 到 a,从 a 到 d)
                    (e) -- [scalar arrow] (a), 
                    (a) -- [scalar arrow] (d),
                };
            \end{feynman}
        \end{tikzpicture}
        = 2ie^2 g^{\mu\nu}
    $$
\end{enumerate}

对于所有外线都有对应的因子:
\begin{itemize}
    \item 入射光子 (${\gamma}$): $\epsilon^\mu(p, \lambda)$
    \item 出射光子 (${\gamma}$): $\epsilon^{\mu*}(p, \lambda)$
    \item 其他复标量粒子: $1$
\end{itemize}

这里有一个suble的点就是有关三点顶角中$p^\mu$还是$-p^\mu$的问题, 我们在这里标动量方向的时候都是保持和粒子流的方向一致的, 这样子导致$p, p'$都是正号. 但是如果我们的$p$在图中标的方向和粒子流相反, 也就是说我们在我们的复标量传播子
$$
    \braket{0|\mathcal \psi_1\psi_2^\dagger|0}=\int\dddd p\frac{i}{p^2-m^2+i\epsilon}\exp{-ip(x_1-x_2)}
$$
中做了个$p\to-p$换元, 得到
$$
    \braket{0|\mathcal \psi_1\psi_2^\dagger|0}=\int\dddd p\frac{i}{p^2-m^2+i\epsilon}\exp{ip(x_1-x_2)}
$$
这会导致$\partial_\mu$的作用多了个$-$号, 于是就从$+ip_\mu$变成了$-ip_\mu$, 从而顶角变成
\begin{equation}
    -ie(p-p')^\mu
\end{equation}

然后我们可以计算几个散射的例子作为尝试.
\begin{example}[pion散射]
    $\pi^+, \pi^-$是一对spin0的复标量粒子, 我们考虑散射
    $$
        \pi^+(p)\pi^-(p')\to\pi^+(k)\pi^-(k').
    $$
    计算到树图阶, 不难发现散射有s-channel以及t-channel, 我们直接计算有
    \begin{align}
        \begin{tikzpicture}[baseline=(current bounding box.center)]
            \begin{feynman}
                \vertex (a);
                \vertex [right=1.5cm of a] (b);
                \vertex [above left=1.5cm of a] (ipion);
                \vertex [below left=1.5cm of a] (iantipion);
                \vertex [above right=1.5cm of b] (opion);
                \vertex [below right=1.5cm of b] (oantipion);
                \diagram* {
                    (a) -- [photon] (b),
                    (ipion) -- [scalar arrow] (a),
                    (a) -- [scalar arrow] (iantipion),
                    (b) -- [scalar arrow] (opion),
                    (oantipion) -- [scalar arrow] (b),
                };
            \end{feynman}
        \end{tikzpicture}&=(-ie)^2(p-p')^\mu(k-k')^\nu\frac{-ig_\mu\nu}{(p+p')^2}=ie^2\frac{u-t}{s}
    \end{align}
    \begin{align}
        \begin{tikzpicture}[baseline=(current bounding box.center)]
            \begin{feynman}
                \vertex (a);
                \vertex [below=1.5cm of a] (b);
                \vertex [above left=1.5cm of a] (ipion);
                \vertex [above right=1.5cm of a] (opion);
                \vertex [below left=1.5cm of b] (iantipion);
                \vertex [below right=1.5cm of b] (oantipion);
                \diagram* {
                    (a) -- [photon] (b),
                    (ipion) -- [scalar arrow] (a),
                    (a) -- [scalar arrow] (opion),
                    (b) -- [scalar arrow] (oantipion),
                    (iantipion) -- [scalar arrow] (b),
                };
            \end{feynman}
        \end{tikzpicture}&=-(-ie)^2(p+k)^\mu(p'+k')^\nu\frac{-ig_\mu\nu}{(p-k')^2}=ie^2\frac{u-s}{t}
    \end{align}
    于是有总振幅
    \begin{equation}
        \mathcal M=e^2\left(\frac{u-t}s+\frac{u-s}t\right)
    \end{equation}
    从而计算得到质心系中的散射截面
    \begin{equation}
        \left(\frac{\d\sigma}{\d\Omega}\right)_{CM}=\frac{e^4}{64\pi^2E_{CM}|\vec p_i|}\left(\frac{u-t}s+\frac{u-s}t\right)^2
    \end{equation}
\end{example}
