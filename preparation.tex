\section{绪论}
\subsection{Why QFT?}
QFT is a framework for incorporating an infinite number of quantum degrees of
freedom, arranged in spacetime, with the interactions constrained by locality, symmetry,
unitarity, and causality. And it is the language of the laws of nature.\cite{qft:lecture1}
\begin{itemize}
    \item Locality: interactions occur at single points in spacetime
    \item Symmetry: the laws of physics are invariant under certain transformations
    \item Unitarity: the total probability of all possible outcomes of a quantum event is 1
    \item Causality: cause precedes effect, and information cannot travel faster than light
\end{itemize}
\subsection{符号约定}
\begin{itemize}
    \item 采用自然Heaviside-Lorentz单位制: $\hbar = c = \mu_0 = \epsilon_0 = 1$
    \item 度规: $g_{\mu\nu} = \mathrm{diag}(1,-1,-1,-1)$
    \item 爱因斯坦求和约定: 重复指标隐含求和, 并且希腊字母如$\mu,\nu,\lambda\cdots$取0,1,2,3, 拉丁字母如$i,j,k\cdots$取1,2,3
    \item 在不影响歧义的情况下, 对于标量场$\phi$, 其偏导$\mu$分量记为$\phi_\mu=\partial_\mu\phi$
    \item 在不影响歧义的情况下, $p_\mu x^\mu$类的缩并可简记为$px$
    \item $\phi_\mathbf x\equiv\phi(\mathbf x)$
    \item $\phi_i\equiv\phi(x_i)$
    \item $\delta_{xy}\equiv\delta^4(x-y)$
    \item $\delta_{ij}\equiv\delta^4(x_i-x_j)$
    \item $D_{xy}\equiv D_F(x-y)$
    \item $D_{ij}\equiv D_F(x_i-x_j)$
    \item $\braket{\phi_1\phi_2\cdots\phi_n}\equiv\braket{\Omega|\mathcal T\phi_1\phi_2\cdots\phi_n|\Omega}$
\end{itemize}
\begin{equation}
    \Theta(x)=\begin{cases}
        1, & x>0\\
        0, & x<0
    \end{cases}
\end{equation}
单位元电荷
\begin{equation}
    e=-|e|
\end{equation}
精细结构常数
\begin{equation}
    \alpha=\frac{e^2}{4\pi}\approx \frac{1}{137}
\end{equation}
作用量
\begin{equation}
    S = \int \mathcal{L} \mathrm d^4x=\int (-m\mathrm d\tau-qA_\mu\mathrm dx^\mu)-\int \mathrm d^4x \frac14 F_{\mu\nu}F^{\mu\nu}
\end{equation}
其中,$F_{\mu\nu}$为电磁场张量
\begin{equation}
    F_{\mu\nu} = \mathrm dA_{\mu\nu}
\end{equation}
$\mathrm d$为外微分算符,$A_\mu$为电磁四势.\\
展开为分量形式有: 
\begin{subequations}
    \begin{align}
        F_{\mu\nu} &= \begin{bmatrix}
            0 & E_x & E_y & E_z \\
            -E_x & 0 & -B_z & B_y \\
            -E_y & B_z & 0 & -B_x \\
            -E_z & -B_y & B_x & 0
        \end{bmatrix}\\
        F^{\mu\nu} &= \begin{bmatrix}
            0 & -E_x & -E_y & -E_z \\
            E_x & 0 & -B_z & B_y \\
            E_y & B_z & 0 & -B_x \\
            E_z & -B_y & B_x & 0
        \end{bmatrix}
    \end{align}
\end{subequations}
以及有粒子动力学方程:
\begin{equation}
    m\frac{\mathrm d^2x^\mu}{\mathrm d\tau^2}=qF^\mu_{~~\nu}\frac{\mathrm dx^\nu}{\mathrm d\tau}
\end{equation}
还有Maxwell方程组的协变形式:
\begin{equation}
    \begin{cases}
        \partial_\mu F^{\mu\nu} = J^\nu\\
        \partial_{[\mu} F_{\nu\lambda]} = 0
    \end{cases}
\end{equation}
化为矢量方程组:
\begin{equation}
    \begin{cases}
        \nabla \cdot \mathbf{E} &= \rho \\
        \nabla \cdot \mathbf{B} &= 0 \\
        \nabla \times \mathbf{E} &= -\frac{\partial \mathbf{B}}{\partial t} \\
        \nabla \times \mathbf{B} &= \mathbf{J} + \frac{\partial \mathbf{E}}{\partial t}
    \end{cases}
\end{equation}
spin$1/2$ Pauli矩阵:
\begin{equation}
    \sigma^0=\begin{pmatrix}
        1 & 0 \\
        0 & 1
    \end{pmatrix}
\end{equation}
\begin{equation}
    \sigma^1=\begin{pmatrix}
        0 & 1 \\
        1 & 0
    \end{pmatrix}
\end{equation}
\begin{equation}
    \sigma^2=\begin{pmatrix}
        0 & -i \\
        i & 0
    \end{pmatrix}
\end{equation}
\begin{equation}
    \sigma^3=\begin{pmatrix}
        1 & 0 \\
        0 & -1
    \end{pmatrix}
\end{equation}

\begin{equation}
    [\frac{\sigma^i}2, \frac{\sigma^j}2]=i\epsilon_{ijk}\frac{\sigma^k}2
\end{equation}
\begin{equation}
    \{\frac{\sigma^i}2, \frac{\sigma^j}2\}=\frac12\delta_{ij}
\end{equation}
\begin{equation}
    \sigma^i\sigma^j=\delta_{ij}+i\epsilon_{ijk}\sigma^k
\end{equation}
spin$1$ Pauli矩阵
\begin{equation}
    S^1=\frac{1}{\sqrt{2}}\begin{pmatrix}
        0 & 1 & 0\\
        1 & 0 & 1\\
        0 & 1 & 0
    \end{pmatrix}
\end{equation}
\begin{equation}
    S^2=\frac{1}{\sqrt{2}}\begin{pmatrix}
        0 & -i & 0\\
        i & 0 & -i\\
        0 & i & 0
    \end{pmatrix}
\end{equation}
\begin{equation}
    S^3=\begin{pmatrix}
        1 & 0 & 0\\
        0 & 0 & 0\\
        0 & 0 & -1
    \end{pmatrix}
\end{equation}
\begin{equation}
    [S^i, S^j]=i\epsilon_{ijk}S^k
\end{equation}
Lorentz群变换(见\ref{Lorentz}节)
\begin{equation}
    \Lambda=\exp{-\frac i2\omega_{\mu\nu}M^{\mu\nu}}
\end{equation}
其中
\begin{equation}
    \delta\omega_{\mu\nu}=\begin{bmatrix}
        0 & v^1 & v^2 & v^3 \\
        -v^1 & 0 & -\theta^3 & \theta^2 \\
        -v^2 & \theta^3 & 0 & -\theta^1\\
        -v^3 & -\theta^2 & \theta^1 & 0
    \end{bmatrix}
\end{equation}
Lorentz群生成元
\begin{align}
    & J^i=\frac12\epsilon_{ijk}M^{jk}\\
    & K^i=M^{i0}
\end{align}
\begin{equation}
    M^{\mu\nu}=\begin{bmatrix}
        0 & -K^1 & -K^2 & -K^3\\
        K^1 & 0 & J^3 & -J^2\\
        K^2 & -J^3 & 0 & J^1\\
        K^3 & J^2 & -J^1 & 0
    \end{bmatrix}
\end{equation}


\newpage
\section{数学基础}
\subsection{泛函}
\begin{definition}[泛函]
    泛函$f: \mathscr{F} \to \mathbb{R}$是定义在某个函数空间上的映射, 全体泛函组成的集合记为$\mathcal{F}$
\end{definition}
\begin{example}
    设$\mathcal{F}$是定义在区间$[a,b]$上的所有实值连续函数的空间,则
    \begin{equation}
        f[\phi] = \int_a^b \phi(x) \, \mathrm{d}x
    \end{equation}
    是$\mathcal{F}$上的一个泛函
\end{example}

\subsection{$\delta$函数}
\begin{definition}[$\delta$函数]
    $\delta(x)$是满足
    \begin{equation}
        \int f(x) \delta(x) \mathrm dx = f(0)
    \end{equation}
    的函数
\end{definition}
\begin{theorem}
    \begin{equation}
        \int \mathrm dx \mathrm e^{ikx}=2\pi \delta(k)\label{matheq1}
    \end{equation}
\end{theorem}
\begin{theorem}
    \begin{equation}
        \int_a^b f(x) \delta(g(x)) \mathrm dx = \sum_{i\in \{i~|~g(x_i)=0, x_i\in[a, b]\}}\frac{f(x_i)}{|g'(x_i)|}= \sum_{i\in \{i~|~g(x_i)=0\}}\frac{f(x_i)}{|g'(x_i)|}\Theta(b-x)\Theta(x-a)
    \end{equation}
\end{theorem}

\subsection{变分}
\begin{definition}[变分]
    设泛函$f: \mathcal{F} \to \mathbb{R}$\\
    其变分
    \begin{equation}
        \frac{\delta f}{\delta \phi(x_0)} = \lim_{\epsilon \to 0} \frac{f[\phi(x) + \epsilon \delta(x-x_0)] - f[\phi]}{\epsilon}
    \end{equation}
\end{definition}
\begin{example}
    设泛函
    \begin{equation}
        f[\phi] = \phi
    \end{equation}
    则
    \begin{equation}
        \frac{\delta f}{\delta \phi(x_0)} = \delta(x-x_0)
    \end{equation}
\end{example}
\begin{example}
    设泛函
    \begin{equation}
        f[\phi] = g(\phi), \text{其中} g: \mathbb{R} \to \mathbb{R}
    \end{equation}
    则
    \begin{equation}
        \frac{\delta f}{\delta \phi(x_0)} = \frac{\partial g}{\partial \phi} \delta(x-x_0)
    \end{equation}
\end{example}
\begin{example}
    设泛函
    \begin{equation}
        f[\phi] = \int g(\phi) \mathrm dx, \text{其中} g: \mathbb{R} \to \mathbb{R}
    \end{equation}
    则
    \begin{equation}
        \frac{\delta f}{\delta \phi(x_0)} = \left.\frac{\partial g}{\partial \phi}\right|_{\phi(x_0)}
    \end{equation}
\end{example}
\begin{example}
    设泛函
    \begin{equation}
        f[\phi] = \int g(\phi, \nabla\phi) \mathrm d^3x, \text{其中} g: \mathbb{R} \times \mathbb{R}^3 \to \mathbb{R}
    \end{equation}
    则
    \begin{equation}
        \frac{\delta f}{\delta \phi(x_0)} = \left.\left(\frac{\partial g}{\partial \phi} - \partial_i\frac{\partial g}{\partial \phi_i}\right)\right|_{\nabla\phi(x_0)}
    \end{equation}
\end{example}
参考: \cite{functionals}

% \subsection{群\&群表示论}

\newpage
\section{狭义相对论动力学技巧}
一些常用技巧\cite{griffthsPPSRTrick}:
\begin{enumerate}
    \item $\vec v=\frac{\vec p}{E}$
    \item 使用四矢量以及不变量点积
    \item 使用质心系简化计算
\end{enumerate}
\begin{definition}[Mandelstam变量]
    对于以$m_1, p_1$, $m_2, p_2$的粒子入射, 以$m_3, p_3$ $m_4, p_4$的粒子出射的动力学系统
    \begin{equation}
        s=(p_1+p_2)^2=(p_3+p_4)^2, \quad t=(p_1-p_3)^2=(p_2-p_4)^2, \quad u=(p_1-p_4)^2=(p_2-p_3)^2
    \end{equation}
\end{definition}
\begin{theorem}[s+t+u守恒]
    \begin{equation}
        s+t+u=m_1^2+m_2^2+m_3^2+m_4^2
    \end{equation}
\end{theorem}
\begin{proof}
    我们取质心系, 可以写出$p_1, p_2, p_3, p_4$
    \begin{align}
        p_1=(E_1, \vec p), \quad p_2=(E_2, -\vec p)\\
        p_3=(E_3, \vec k), \quad p_4=(E_4, -\vec k)
    \end{align}

    于是
    \begin{equation}
        s=(E_1+E_2)^2, \quad t=(E_3-E_1)^2-(\vec k-\vec p)^2, \quad u=(E_4-E_1)^2-(\vec k+\vec p)^2
    \end{equation}

    计算可得:
    \begin{align}
        s+t+u&=E_1^2+E_2^2+E_3^2+E_4^2-2\vec p^2-2\vec k^2+2E_1^2+2E_1E_2-2E_1E_3-2E_1E_4\\
        &=m_1^2+m_2^2+m_3^2+m_4^2+2E_1(E_1+E_2-E_3-E_4)\\
        &=m_1^2+m_2^2+m_3^2+m_4^2
    \end{align}
\end{proof}

\begin{theorem}[质心系中$m_1$的能量]
    \begin{equation}
        E^{\mathbf{CM}}_1=\frac{s+m_1^2-m_2^2}{2\sqrt s}
    \end{equation}
\end{theorem}
\begin{theorem}[实验室系($m_2$静止)中$m_1$的能量]
    \begin{equation}
        E^{\mathbf{lab}}_1=\frac{s-m_1^2-m_2^2}{2m_2}
    \end{equation}
\end{theorem}
\begin{theorem}[质心系总能量]
    \begin{equation}
        E^{\mathbf{CM}}_{\mathbf{TOT}}=\sqrt s
    \end{equation}
\end{theorem}
